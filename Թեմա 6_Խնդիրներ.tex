\bigskip
\bigskip
\subsubsection*{Խնդիրներ և հարցեր թեմա 6-ի վերաբերյալ}

\begin{enumerate}[label=\thesection.\arabic*.]
% 6.1
\item Իրական թվերի սովորական տոպոլոգիայում գտեք անվերջ քանակով փակ ենթաբազմությունների որևէ ընտանիք, որի տարրերի միավորումը փակ չէ։

% 6.2
\item Ճի՞շտ է արդյոք հետևյալ պնդումը․ ցանկացած տոպոլոգիական տարածությու\-նում (բացառությամբ դիսկրետ տարածությունների) գոյություն ունի անվերջ քանակով փակ ենթաբազմությունների ընտանիք, որի տարրերի միավորումը փակ չէ։

\begin{hint}
Դիտարկեք $\left\{ [x,1);\;  x > 0 \right\}$ ընտանիքը $(\mathbb{R}, \mapsto)$ տարածությունում։
\end{hint}

% 6.3
\item Թվային ուղղի սովորական տոպոլոգիայում գտեք $A$ ենթաբազմության փա\-կույթը, եթե
\begin{itemize}
    \item[ա)] $A$-ն ամբողջ թվերի $\Z$ ենթաբազմությունն է,
    \item[բ)] $ A = \left\{ \dfrac{(-1)^{n}}{n}; \; n \in \mathbb{N} \right\} $,
    \item[գ)] $A$-ն բոլոր ռացիոնալ թվերի ենթաբազմությունն է,
    \item[դ)] $A$-ն բոլոր իռացիոնալ թվերի ենթաբազմությունն է։
\end{itemize}

% 6.4
\item Թվային ուղղի հաշվելի լրացումների տոպոլոգիայում գտեք $A$ ենթաբազմու\-թյան փակույթը նախորդ խնդրում թվարկված դեպքերում։

% 6.5
\item Դիտարկենք $\mathbb{R}$ թվային ուղղի ենթաբազմությունների $\Phi$ ընտանիքը՝ կազմված $\varnothing$, $\mathbb{R}$ և բոլոր $(-r, r)$, $r > 0$ ենթաբազմություններից։ Ապացուցեք, որ $\Phi$-ն որոշում է տոպոլոգիա $\mathbb{R}$-ի վրա։ Ամեն մի $r \in \mathbb{R}$ դեպքում գտեք $X(r)= (-\infty, -r) \cup (r, +\infty)$ 
ենթաբազմության ներքինը, արտաքինը, եզրը և փակույթը։

% 6.6
\item Դիտարկենք $0$ սկզբնակետով $\mathbb{R}^2$ կոորդինատային հարթության ենթաբազմու\-թյունների $\Psi$ ընտանիքը՝ կազմված $\varnothing$, $\mathbb{R}^2$ և $0$ կենտրոնով բոլոր $\mathcal{D}(r) = \{(x,y) \mid x^2 + y^2 < r^2; \; r>0\}$ շրջաններից։ Ապացուցեք, որ $\Psi$-ն որոշում է տոպոլոգիա $\mathbb{R}^2$-ի վրա (կոչվում է \textbf{համակենտրոն տոպոլոգիա})։ Ապացուցեք, որ ամեն մի $r>0$ դեպքում $A(r)=\mathbb{R}^2 \setminus  \mathcal{D}(r)$ ենթաբազմության ներքինը, արտաքինը, եզրը և փակույթը հետևյալն են՝
\[
\inter A(r)=\varnothing, \quad \exter A(r) = \mathcal{D}(r), \quad \partial A(r) = A(r), \quad \widebar{A(r)} = A(r):
\]

% 6.7
\item Ապացուցեք, որ $X$ տոպոլոգիական տարածության կամայական $A$ ենթաբազ\-մու\-թյան և $X \setminus A$ ենթաբազմության եզրերը նույնն են՝ $\partial A = \partial (X \setminus A)$:

% 6.8
\item Ապացուցեք, որ տոպոլոգիական տարածության կամայական երկու չհատվող փակ ենթաբազմություններ չունեն որևէ ընդհանուր եզրային կետ։

% 6.9
\item Ապացուցեք․ եթե $X$ տոպոլոգիական տարածության $Y$ ենթաբազմությունն այնպիսին է, որ $Y \subset F \subset X$, որտեղ $F$-ը փակ ենթաբազմություն է, ապա $\widebar{Y} \subset F$։

% 6.10
\item Ապացուցեք․ $X$ տոպոլոգիական տարածության $A$ ենթաբազմությունը փակ է այն և միայն այն դեպքում, երբ $\partial A \subset A$։

% 6.11
\item Ապացուցեք, որ $X$ տոպոլոգիական տարածության $Y$ ենթաբազմության $\partial Y$ եզրը դատարկ ենթաբազմություն է այն և միայն այն դեպքում, երբ $Y$-ը բաց է և փակ է։

% 6.12
\item Ապացուցեք․ տոպոլոգիական տարածության կամայական երկու չհատվող բաց ենթաբազմություններից յուրաքանչյուրի փակումը չի հատվում մյուս ենթա\-բազ\-մության հետ։
\end{enumerate}

