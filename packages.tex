\documentclass[12pt]{extarticle}

\usepackage{subfiles}
\usepackage{import}

% Font-եր
\usepackage{fontspec}
\setmainfont{CMUSerif}[
    Path=./fonts/,
    Scale=1,
    Extension = .ttf,
    UprightFont=*-Roman,
    BoldFont=*-Bold,
    ItalicFont=*-Italic,
    % BoldItalicFont=*Z
    ]

\usepackage{geometry}
 \geometry{
 a4paper,
 total={155mm,257mm},
 left=25mm,
 top=20mm,
 }

% էջի կառուցվածք
%\usepackage[a5paper,left=1.5cm,right=1.75cm,top=1cm,bottom=2cm,bindingoffset=0cm]{geometry}
%\usepackage[a5paper, textwidth=5in, textheight=7in]{geometry}

% Վերնագրերը մեջտեղից
% \usepackage[center]{titlesec}
% \titleformat
% {\chapter} % command
% [display] % shape
% {\bfseries\small\itshape}
% {} % label
% {} % sep
% {} % before-code
% [] % after-code
\usepackage{titlesec}
\titleformat
{\section} % command
[display] % shape
{\center \bfseries \large} % format
{\center Թեմա \thesection} % label
%{\center Թեմա. ի՞նչ է տոպոլոգիան} % label
{0.5ex}{} % sep

% \usepackage{titlesec}
% \titleformat
% {\section} % command
% [] % shape
% {\bfseries \large} % format
% {\center Թեմա \thesection} % label
% {1cm}{} % sep

\titleformat{\subsection}[block]{\large\bfseries\filcenter}{}{1em}{}

\usepackage{graphicx}
\graphicspath{ {./images/} }

\usepackage{amsfonts,amssymb,amsthm,mathtools} % AMS
\usepackage{amsmath}
\usepackage{ stmaryrd } % mapsfrom <-|
\usepackage{upgreek}
%\usepackage{icomma} % "խելացի" ստորակե, $0,2$ --- թիվ, $0, 2$ --- թվարկում
\usepackage{mathrsfs} % Սիրուն տառատեսակ
\usepackage{nicefrac}

% գույներ, գծագրեր
\usepackage{tikz}
\usepackage{tikz-cd}
\usepackage{tikz-layers}
\usetikzlibrary{quotes,angles}
\usetikzlibrary{positioning}

\usepackage{graphicx,calc}
\newlength\myheight
\newlength\mydepth
\settototalheight\myheight{Xygp}
\settodepth\mydepth{Xygp}
\setlength\fboxsep{0pt}
\newcommand*\inlinegraphics[1]{%
  \settototalheight\myheight{Xygp}%
  \settodepth\mydepth{Xygp}%
  \raisebox{-\mydepth}{{#1}}%
}

\usepackage{caption}
\usepackage{subcaption}

\usepackage{xcolor}
\usepackage{relsize}  % mathsmaller

% հիպերհղում
\usepackage[hidelinks, colorlinks=true,linkcolor=blue]{hyperref}
\usepackage{faktor}
\usepackage{multicol}
\usepackage{xparse}

\makeatletter
\def\thmhead@plain#1#2#3{%
  \thmname{#1}\thmnumber{\@ifnotempty{#1}{ }\@upn{#2}}%
  \thmnote{ {\the\thm@notefont#3}}}
\let\thmhead\thmhead@plain
\makeatother

\newtheoremstyle{main_theorem_style}% name of the style to be used
{}% measure of space to leave above the theorem. E.g.: 3pt
{}% measure of space to leave below the theorem. E.g.: 3pt
{}% name of font to use in the body of the theorem
{18pt} %measure of space to indent
{\bfseries}% name of head font
{:}% punctuation between head and body
{ }% space after theorem head; " " = normal interword space
{}

\theoremstyle{main_theorem_style}
\newtheorem{example}{Օրինակ}
\counterwithin*{example}{section}
\newtheorem{examples}{Օրինակներ}
\newtheorem{orinakner}{Օրինակներ}
\newtheorem*{nexamples}{Օրինակներ}
\newtheorem*{nexample}{Օրինակ}

\newtheorem{question}{Հարց}
\newtheorem*{nquestion}{Հարց}
\newtheorem{theorem}{Թեորեմ}
\newtheorem*{ntheorem}{Թեորեմ}
\counterwithin*{theorem}{section}
\newtheorem{hatkutyun}{Հատկություն}
\counterwithin*{hatkutyun}{section}

\newtheorem*{hetevanq1}{Հետևանքներ թեորեմ 1-ից}
\newtheorem*{hetevanq1_sa}{Հետևանք թեորեմ 1-ից}
\newtheorem*{hetevanq2_sa}{Հետևանք թեորեմ 2-ից}
\newtheorem*{hetevanq3_sa}{Հետևանք թեորեմ 3-ից}
\newtheorem*{hetevanq4_sa}{Հետևանք թեորեմ 4-ից}
\newtheorem*{hetevanq5_sa}{Հետևանք թեորեմ 5-ից}

\newtheorem*{hetevanq2}{Հետևանքներ թեորեմ 2-ից}
\newtheorem*{hetevanq3}{Հետևանքներ թեորեմ 3-ից}
\newtheorem*{hetevanq4}{Հետևանքներ թեորեմ 4-ից}
\newtheorem*{hetevanq5}{Հետևանքներ թեորեմ 5-ից}
\newtheorem*{hetevanq6}{Հետևանքներ թեորեմ 6-ից}
\newtheorem*{hetevanq7}{Հետևանք թեորեմ 7-ից}
\newtheorem*{hetevanqlind}{Հետևանք (Լինդելյոֆի) թեորեմից}
\newtheorem*{hetevanqax}{Հետևանք աքսիոմներից}
\newtheorem*{separabelex}{Սեպարաբել տարածությունների օրինակներ}
\newtheorem*{definition}{Սահմանում}
\newtheorem*{note}{Դիտողություն}
\newtheorem*{lemma}{Լեմմա}
\newtheorem*{hetevanq}{Հետևանք}
\newtheorem{hetevanq_counter}{Հետևանք}


\usepackage{setspace}

\renewcommand\qedsymbol{$\blacksquare$}
\renewcommand*{\proofname}{\hspace{18pt}\textbf{Ապացուցում:}}

\newtheoremstyle{theoremdd}% name of the style to be used
{}% measure of space to leave above the theorem. E.g.: 3pt
{}% measure of space to leave below the theorem. E.g.: 3pt
{}% name of font to use in the body of the theorem
{18pt} %measure of space to indent
{\bfseries}% name of head font
{}% punctuation between head and body
{ }% space after theorem head; " " = normal interword space
{\thmname{#1}.}

\theoremstyle{theoremdd}


%\newtheorem{example}{Օրինակ}[section]
%\renewcommand{\theexample}{\arabic{example}}
%\newenvironment{example} % this is the environment name for the input
% {\renewcommand{\qedsymbol}{$\triangle$}%
% %{\renewcommand{\qedsymbol}{$\blacksquare$}%
% %{\renewcommand{\qedsymbol}{$\qedhere$}%
% \pushQED{\qed}\begin{examplex}}
% {\popQED\end{examplex}}

% \newenvironment{example}
%   {\pushQED{\qed}\renewcommand{\qedsymbol}{$\triangle$}\examplex}
%   {\popQED\endexamplex}
% \newcommand{\orinakiverj}
% {\par \vspace{-1.7\baselineskip} 
% \qedhere}


\newtheorem*{hint}{Ցուցում}

\newtheorem*{equivdefinition}{Համարժեք սահմանում}



\newtheoremstyle{theoremddd}% name of the style to be used
{}% measure of space to leave above the theorem. E.g.: 3pt
{}% measure of space to leave below the theorem. E.g.: 3pt
{}% name of font to use in the body of the theorem
{18pt} %measure of space to indent
{\bfseries}% name of head font
{:}% punctuation between head and body
{ }% space after theorem head; " " = normal interword space
{\thmname{#1}.}

\theoremstyle{theoremddd}
 
 
\usepackage[inline]{enumitem}

% Սահմանումներ
\DeclareMathOperator{\tg}{tg}
\DeclareMathOperator{\ch}{ch}
\DeclareMathOperator{\sh}{sh}
\DeclareMathOperator{\ctg}{ctg}
\DeclareMathOperator{\arctg}{arctg}
\DeclareMathOperator{\arcctg}{arcctg}
\DeclareMathOperator{\sinhyp}{sh}
\DeclareMathOperator{\coshyp}{ch}
\DeclareMathOperator{\rank}{rank}
\DeclareMathOperator{\Aut}{Aut}
\DeclareMathOperator{\Ens}{Ens}
\DeclareMathOperator{\inter}{int}

\newcommand{\R}{\mathbb{R}}
\newcommand{\C}{\mathbb{C}}
\newcommand{\Q}{\mathbb{Q}}
\newcommand{\Z}{\mathbb{Z}}
\newcommand{\N}{\mathbb{N}}

\renewcommand{\d}{\mathrm{d}}

\let\ForAll\forall
\renewcommand{\forall}{\ForAll\,}
\let\Exists\exists
\renewcommand{\exists}{\Exists\,}

\newcommand{\diff}[2]{\frac{\mathrm{d} #1}{\mathrm{d} #2}}
\DeclareMathOperator{\id}{id}
\DeclareMathOperator{\cl}{cl}
\usepackage{dsfont}
\DeclareMathOperator{\nuynakan}{\mathds{1}}

\DeclarePairedDelimiter\abs{\lvert}{\rvert}

% Հավասարացնել հավասարումները ինչ-որ ձև
\newcommand{\migic}[1]{\makebox[25em][l]{#1}}
\newcommand{\alright}[1]{\makebox[15em][l]{#1}}


\newcommand\lagrangeprime[1]{^{%
\ifcase#1 \or\prime\or\prime\prime\or\prime\prime\prime\else\mathrm{\romannumeral #1}\fi}}
\newcommand{\RNum}[1]{\uppercase\expandafter{\romannumeral #1\relax}}

\newcommand*\mathbold[1]{\pdfliteral direct{2 Tr 0.25 w}#1\pdfliteral direct{0 Tr 0 w}} 
% \usepackage{titlesec}
% \titleformat{\subsection}
%   {\bfseries\scshape\Large}{\textsection \thesubsection}{1em}{}
  
\usepackage[normalem]{ulem} % [normalem] prevents the package from changing the default behavior of `\emph` to underline.

\newcommand{\red}[1]{\textcolor{red}{#1}}


\usepackage{indentfirst}

\makeatletter
\newenvironment{sqcases}{%
  \matrix@check\sqcases\env@sqcases
}{%
  \endarray\right.%
}
\def\env@sqcases{%
  \let\@ifnextchar\new@ifnextchar
  \left\lbrack
  \def\arraystretch{1.2}%
  \array{@{}l@{\quad}l@{}}%
}
\makeatother

\usepackage[sorting=nty,natbib,style=numeric,defernumbers]{biblatex}
%\usepackage{cite}
%\bibliographystyle{unsrt}
\addbibresource{grqer.bib}

\newtheoremstyle{theorem_attribute}% name of the style to be used
{}% measure of space to leave above the theorem. E.g.: 3pt
{0pt}% measure of space to leave below the theorem. E.g.: 3pt
{}% name of font to use in the body of the theorem
{18pt} %measure of space to indent
{\bfseries}% name of head font
{:}% punctuation between head and body
{ }% space after theorem head; " " = normal interword space
{}

\newtheoremstyle{theorem_statement}% name of the style to be used
{}% measure of space to leave above the theorem. E.g.: 3pt
{0pt}% measure of space to leave below the theorem. E.g.: 3pt
{}% name of font to use in the body of the theorem
{18pt} %measure of space to indent
{\bfseries}% name of head font
{.}% punctuation between head and body
{ }% space after theorem head; " " = normal interword space
{}

\theoremstyle{theorem_statement}
\newtheorem{statement}{Պնդում}
\newtheorem{hatkutun_counter}{Հատկություն}
\counterwithin*{theorem}{section}

\theoremstyle{theorem_statement}
\newtheorem{attribute}{Հ}
\counterwithin*{theorem}{section}


\makeatletter
\let\save@mathaccent\mathaccent
\newcommand*\if@single[3]{%
  \setbox0\hbox{${\mathaccent"0362{#1}}^H$}%
  \setbox2\hbox{${\mathaccent"0362{\kern0pt#1}}^H$}%
  \ifdim\ht0=\ht2 #3\else #2\fi
  }
%The bar will be moved to the right by a half of \macc@kerna, which is computed by amsmath:
\newcommand*\rel@kern[1]{\kern#1\dimexpr\macc@kerna}
%If there's a superscript following the bar, then no negative kern may follow the bar;
%an additional {} makes sure that the superscript is high enough in this case:
\newcommand*\widebar[1]{\@ifnextchar^{{\wide@bar{#1}{0}}}{\wide@bar{#1}{1}}}
%Use a separate algorithm for single symbols:
\newcommand*\wide@bar[2]{\if@single{#1}{\wide@bar@{#1}{#2}{1}}{\wide@bar@{#1}{#2}{2}}}
\newcommand*\wide@bar@[3]{%
  \begingroup
  \def\mathaccent##1##2{%
%Enable nesting of accents:
    \let\mathaccent\save@mathaccent
%If there's more than a single symbol, use the first character instead (see below):
    \if#32 \let\macc@nucleus\first@char \fi
%Determine the italic correction:
    \setbox\z@\hbox{$\macc@style{\macc@nucleus}_{}$}%
    \setbox\tw@\hbox{$\macc@style{\macc@nucleus}{}_{}$}%
    \dimen@\wd\tw@
    \advance\dimen@-\wd\z@
%Now \dimen@ is the italic correction of the symbol.
    \divide\dimen@ 3
    \@tempdima\wd\tw@
    \advance\@tempdima-\scriptspace
%Now \@tempdima is the width of the symbol.
    \divide\@tempdima 10
    \advance\dimen@-\@tempdima
%Now \dimen@ = (italic correction / 3) - (Breite / 10)
    \ifdim\dimen@>\z@ \dimen@0pt\fi
%The bar will be shortened in the case \dimen@<0 !
    \rel@kern{0.6}\kern-\dimen@
    \if#31
      \overline{\rel@kern{-0.6}\kern\dimen@\macc@nucleus\rel@kern{0.4}\kern\dimen@}%
      \advance\dimen@0.4\dimexpr\macc@kerna
%Place the combined final kern (-\dimen@) if it is >0 or if a superscript follows:
      \let\final@kern#2%
      \ifdim\dimen@<\z@ \let\final@kern1\fi
      \if\final@kern1 \kern-\dimen@\fi
    \else
      \overline{\rel@kern{-0.6}\kern\dimen@#1}%
    \fi
  }%
  \macc@depth\@ne
  \let\math@bgroup\@empty \let\math@egroup\macc@set@skewchar
  \mathsurround\z@ \frozen@everymath{\mathgroup\macc@group\relax}%
  \macc@set@skewchar\relax
  \let\mathaccentV\macc@nested@a
%The following initialises \macc@kerna and calls \mathaccent:
  \if#31
    \macc@nested@a\relax111{#1}%
  \else
%If the argument consists of more than one symbol, and if the first token is
%a letter, use that letter for the computations:
    \def\gobble@till@marker##1\endmarker{}%
    \futurelet\first@char\gobble@till@marker#1\endmarker
    \ifcat\noexpand\first@char A\else
      \def\first@char{}%
    \fi
    \macc@nested@a\relax111{\first@char}%
  \fi
  \endgroup
}
\makeatother

\usepackage{cutwin}


% որ \[\]-ի դեպքում շատ լայն տարածքներ չթողի
\newcommand{\densequation}{
\setlength{\belowdisplayskip}{0pt} \setlength{\belowdisplayshortskip}{0pt}
\setlength{\abovedisplayskip}{0pt} \setlength{\abovedisplayshortskip}{0pt}}




% drawing
\usepackage{float}
\usetikzlibrary{
  hobby,
  intersections,
  spath3,
  decorations.markings,
  arrows.meta,
}
\usepackage{xfp}
\usetikzlibrary{patterns}
\usetikzlibrary{calc}
% drawing finish

