\documentclass[./main.tex]{subfiles}
%\usepackage{mathabx}
% 9, 10 tegherob poxel a

\begin{document}
\onehalfspacing
\section{Հաջորդականությունների զուգամիտությունը տոպոլոգիական տարածություններում։ Տարածության տոպոլոգիայի նկարագրումը զուգամետ հաջորդա\-կանությունների տերմիններով։}\label{sec:9}

% \par Ինչպես արդեն նշել ենք, երկրաչափություն կամ մաթ․ անալիզ կառուցելու տե\-սանկ\-յու\-նից հատկապես կարևորվում են մետրական, մասնավորապես $\mathbb{R}^n$ էվկլիդյան տարածությունները։

\par Մաթեմատիկական անալիզի հիմնական շատ հասկացություններ սերտորեն կապ\-ված են թվային հաջորդականություն, ենթահաջորդականություն, թվային հաջոր\-դա\-կա\-նու\-թյան սահման հասկացությունների հետ։ Սկզբունքորեն հաջորդակա\-նութ\-յունների զուգամիտության տեսություն կարելի է զարգացնել ոչ միայն թվային ուղղի վրա, և ոչ միայն մետրիկային, այլև կամայական տոպոլոգիական տարածությունում։

% \par Մաթ․ անալիզն ունի երկու հիմնասյուն՝ իրական կամ կոմպլեքս թվերը իրենց անհրաժեշտ հանրահաշվական հատկություններով, և զուգամիտության կամ այսպես կոչված սահմանների տեսությունը։ Սկզբունքորեն դրանք մեկը մյուսից անկախ են և հաջորդականությունների զուգամիտության տեսություն կարելի է զարգացնել ոչ միայն մետրական, այլև կամայական տոպոլոգիական տարածությունում։

% \par Որպես դրա առաջին քայլ $\mathbb{R}$ թվային ուղղի նմանությամբ, ներմուծվում է հաջորդա\-կանություն, ենթահաջորդականություն և զուգամետ հաջորդականություն հասկացու\-թյունները կամայական տոպոլոգիական տարածություններում։

\par Եթե ունենք որևէ $X$ բազմություն, ապա \textbf{հաջորդականություն $\boldsymbol{X}$-ում} կոչվում է նրա տարրերի (բնական թվերով համարակալված) ամեն մի $\{x_n \mid n \in \mathbb{N} \} \subset X$ ենթաբազ\-մու\-թյուն։

\par Համարժեք սահմանում․ հաջորդականություն $X$-ում կոչվում է ցանկացած ${f:\mathbb{N}\rightarrow X}$ արտապատկերում։ Իրոք, ամեն մի $n \in \mathbb{N}$ թվի համար նշանակելով $f(n)\in X$ կետը $x_n$-ով, կստանանք $\{x_n \mid n \in \mathbb{N}\}$ հաջորդականություն նախկին իմաստով և հակառակը։

Երբեմն հաջորդականության նշանակման համար կօգտագործենք համառոտ գը\-րառում՝ $\{x_n\}$։
\par Դիցուք ունենք երկու՝ $\{x_n\}$ և $\{y_n\}$ հաջորդականություններ $X$ բազմությունում, այսինքն $f:\N\to X$ և $g:\N\to X$ արտապատկերումներ, և $x_n=f(n),\ y_n=g(n),\ n\in \Z$։ 

\begin{definition}
%Ասում են, որ $f:\mathbb{N}\rightarrow X$ հաջորդականությունը $g:\mathbb{N}\rightarrow X$ հաջորդականության ենթահաջորդականություն է, եթե գոյություն ունի $h:\mathbb{N}\rightarrow X$ արտապատկերում, որ $f(i)>f(j)$ ցանկացած $i>j$ դեպքում և $f=g\circ h$։
Ասում են, որ $\{y_n\}$ հաջորդականությունը $\{x_n\}$ \textbf{հաջորդականութ\-յան ենթահաջորդականություն} է, եթե գոյություն ունի $h:\mathbb{N}\rightarrow \N$ արտապատ\-կե\-րում, որ $h(i)>h(j)$ ցանկացած $i>j$ դեպքում և $g=f\circ h$։
\end{definition}

Սահմանումից հետևում է․ $y_n=g(n)=(f\circ h)(n)=f(h(n))=x_{h(n)}$, այսինքն $\{y_n\}$ ենթահաջորդականության յուրաքանչյուր $y_n$ անդամ $\{x_n\}$ հաջորդականության որևէ անդամ է։
\begin{definition}
$X$ բազմության $\{x_n\}$ հաջորդականությունը կոչվում է \textbf{ստացիոնար հաջորդականություն}, եթե ինչ-որ $m \in \mathbb{N}$ համարից սկսած նրա բոլոր անդամները նույնն են՝ $x_m = x_{m+1} = \cdots$
\end{definition}

\begin{definition}
Դիցուք ունենք $(X,\tau)$ տոպոլոգիական տարածություն և $\{x_n\} \subset X$ հաջորդականություն։ Ասում են, որ այն զուգամիտում է $a \in X$ կետին (գրառվում է $\lim⁡\{x_n\}=a$), եթե $a$ կետի ցանկացած $U$ շրջակայքի համար գոյություն ունի $n_0 \in \mathbb{N}$ թիվ, որ $x_n \in U$ ցանկացած $n>n_0$ թվի դեպքում։ Այս դեպքում ինքը՝ հաջորդականությունը կոչվում է \textbf{զուգամետ հաջորդականություն}, իսկ $a$-ն կոչվում է $\{x_n\}$ \textbf{հաջորդականության սահման}։
\end{definition}

\par Նկատենք, որ հաջորդականությունը կարող է զուգամետ լինել կամ չլինել (սահ\-ման ունենալ, կամ չունենալ), ինչպես նաև՝ զուգամետ հաջորդականության սահմանը կարող է միակը չլինել։

\begin{example}
ա) Ցանկացած տոպոլոգիական տարածությունում ամեն մի ստա\-ցիո\-նար հա\-ջոր\-դա\-կանություն զուգամետ հաջորդականություն է (հիմնավորե՛ք)։
\par բ) $(X, \textrm{դիսկր․})$ տարածությունում զուգամիտում են միայն ստացիո\-նար հաջոր\-դա\-կանությունները, ընդ որում սահմանը միակն է (հիմնավորե՛ք)։
\par գ)  $(X, \textrm{անտիդիսկր․})$-ում ցանկացած հաջորդականություն զուգամիտում է $X$-ի ցանկացած կետի (հիմնավորե՛ք)։

\par դ) Դիտարկենք որևէ $(X, \textrm{հաշվ․ լր․})$ տարածություն, որտեղ $X$-ը ոչ հաշվելի բազ\-մու\-թյուն է։ Այստեղ ևս զուգամիտում են միայն ստացիոնար հաջորդականություն\-ները։ Իրոք, ենթադրենք $\{x_n\}$-ը ստացիոնար չէ և գոյություն ունի $\lim \{x_n\}=a$։ Սա նշանակում է, որ $\{x_n\}$ հաջորդականությունը պարունակում է հաշվելի անվերջ թվով $a$ կետից տարբեր անդամներ։ Դիտարկենք $a$ կետի $U=X \setminus \{x_n \mid x_n \neq a,\ n\in \mathbb{N}\}$ բաց շրջա\-կայքը։ Ըստ մեր ենթադրության գոյություն ունի $n_0$ բնական թիվ, որ $x_n \in U$ բոլոր $n>n_0$ համարների դեպքում։ Իսկ դա հնարավոր է միայն այն դեպքում, երբ $x_n=a$ բոլոր $n>n_0$ համարների դեպքում (ինչո՞ւ)։ Ստացանք հակասություն։
\end{example}
Թվարկենք զուգամիտության մի քանի պարզ հատկություններ։
\begin{enumerate}
    \item Ցանկացած $\{x_n\} \in X$ զուգամետ հաջորդականության ամեն մի $\{y_n\}$ ենթահաջորդականություն նույն\-պես զուգամետ հաջորդականություն է և ունի նույն սահմանը (սահմանները), ինչը որ ունի $\{x_n\}$-ը (հիմնավորե՛ք)։
    \item Եթե $\{x_n\} \subset X$ հաջորդականությունը ունի $\lim \{x_n\} = a$ սահման, ապա $a \in \overline{\{x_n\}}$ (հիմնավորե՛ք):
    \item Ցանկացած ($\textrm{T}_2$) տարածությունում ամեն մի զուգամետ հաջորդա\-կա\-նու\-թյուն ունի միակ սահման։
    % \item Եթե $\lim\{x_n\}=a$, ապա $a\in \widebar{\{x_n\}}$։ (Ինչո՞ւ):
    % \label{$3$-րդ հատկությանը}
\end{enumerate}

\par Իրոք, ենթադրենք որևէ $X$ $T_2$-տարածությունում ինչ-որ $\{x_n\}$ հաջորդականություն ունի $\lim x_n = a$, $\lim x_n = b$, $a \neq b$ սահմաններ: Դիտարկենք $a$ և $b$ կետերի չհատվող որևէ $U$ և $V$ շրջակայքեր (\textbf{շարունակեք ապացույցը}):

\begin{theorem}
Կամայական $X$ մետրիկային տարածության ամեն մի $\{x_n\} \subset X$ հաջորդականության և $a \in X$ կետի համար հետևյալ երկու պնդումները համարժեք են․
\par ա) գոյություն ունի $\lim\{x_n\}=a$ սահման,
\par բ) ցանկացած $\mathcal{D}(a,r)$ բաց գնդի համար գոյություն ունի $n_0$ բնական թիվ, որ $x_n \in \mathcal{D}(a,r)$ երբ $n>n_0$։
\end{theorem}
\begin{proof}
 Անցումը ա)-ից բ)-ին ակնհայտ է։ Ցույց տանք բ) $\Rightarrow$ ա) անցումը։ Դիցուք $U$-ն $a$ կետի որևէ շրջակայք է։ Ըստ կետի շրջակայքի սահմանման, գոյություն ունի $a$ կետի $V$ բաց շրջակայք, որ $a \in V \subset U$։ Համաձայն նախորդ թեմայի \hyperlink{sec:8}{թեորեմ 3}-ի \red{ref}, գոյություն ունի $a$ կենտրոնով $\mathcal{D}(a,r)$ բաց գունդ, որ $\mathcal{D}(a,r) \subset V$։ Ըստ պայմանի, գոյություն ունի $n_0$ թիվ, որ $x_n \in \mathcal{D}(a,r)$, երբ $n>n_0$։ Հետևաբար $x_n \in U$, երբ $n>n_0$, ուրեմն $\lim\{x_n\} = a$։
\end{proof}

\par Քննարկենք հետևյալ հարցը․ ինչպես են միմյանց հետ կապված տարածութ\-յան տոպոլոգիան (այսինքն բաց և փակ ենթաբազմությունները) և տվյալ տարածութ\-յունում զուգամետ հաջորդականությունները։ Սկսենք բաց ենթաբազմություններից։

\begin{theorem}
    Դիցուք $X$ տոպոլոգիական տարածությունը բավարարում է հաշվելիության առաջին աքսիոմին։ Ապա $X$-ի որևէ $A$ ենթաբազմությունը բաց ենթաբազմություն է այն և միայն այն դեպքում, երբ $X$-ում ամեն մի հաջորդականություն, որը զուգամիտում է $A$-ի որևէ կետի, ընկած է $A$-ի մեջ սկսած ինչ-որ անդամից։
\end{theorem}
\label{thm:9.2}
\par Նախ ապացուցենք մի օժանդակ պնդում։
\begin{lemma}
Եթե $X$ տոպոլոգիական տարածությունը բավարարում է հաշվելիության առաջին աքսիո\-մին, ապա նրա ամեն մի $x$ կետի համար գոյություն ունի շրջակայքերի այնպիսի $\{ U_i(x)\}$ հաշվելի բազա, որ $U_{i+1}(x)\subset U_i(x)$ ցանկացած $i$ բնական թվի դեպքում։
%ինդեքսների $I$ բազմության ամեն մի $i,\, i+1$ տարրերի դեպքում։
\end{lemma}
\begin{proof}
Դիտարկենք $x$ կետի շրջակայքերի որևէ $\{ V_i(x)\}$ հաշ\-վե\-լի բազա և սահմանենք $x$-ի շրջակայքերի նոր՝ $\{ U_i(x)\}$ հաշ\-վե\-լի բազա, վերցնելով $U_1(x)=V_1(x)$ և $U_i(x)=\bigcap\limits_{k=1}^i V_k(x)$ ամեն մի $i$ ինդեքսի համար։ Պարզ է, որ միշտ $U_{i+1}(x)\subset U_i(x)$։ %ամեն մի $i,\, i+1\in I$ դեպքում։
\end{proof}
\renewcommand*{\proofname}{\hspace{18pt}\textbf{Թեորեմ 2-ի ապացուցում։}\nopunct}
\begin{proof}
    Պայմանի անհրաժեշտությունը անմիջապես հե\-տե\-վում է հաջորդա\-կա\-նու\-թյան զուգամիտության սահմանումից։ Ցույց տանք պայմանի բա\-վարարութ\-յունը հակասող ենթադրությամբ։ Դիցուք որևէ $A$ ենթաբազմության համար նշված պայմանը տեղի ունի, բայց $A$-ն բաց չէ։ Նշանակում է՝ գոյություն ունի $s\in A$ կետ, որը ներքին կետ չէ $A$-ի համար։ Ըստ լեմմայի, գոյություն ունի $s$ կետի շրջակայքերի $\{ U_i(s);\, i\in I \subset \N\}$ հաշվելի բազա, որ $U_i(s)\supset U_{i+1}(s)$ ամեն մի $i,\, i+1\in I$ դեպքում։ Քանի որ $s \not\in \inter A$, ուստի ամեն մի $U_n(s)$-ում կարող ենք ընտրել $x_n$ կետ, որ $x_n\not\in A$։ Պարզ է, որ ստացված $\{x_n\}$ հաջորդականությունը զուգամիտում է $s$ կետին, և ըստ թեորեմի պայմանի՝ նրա անդամները ինչ-որ համարից սկսած պետք է պատկանեն $A$-ին։ Բայց դա հակասում է նրան, որ $\{x_n\}\cap A=\varnothing$։\qedhere

    % $\{x_n\} \subset X$ հաջորդա\-կա\-նու\-թյուն և $s \in A $ կետ, որ $\lim{x_n}=s$, բայց ${\{x_n\}}$-ը ընկած չէ $A$-ի մեջ ոչ մի պահից սկսած։ Ուստի գոյություն ունի ${\{x_n\}}$-ի ${\{x_{n_b}\}}$ ենթահաջորդականություն, որ ${\{x_n\}} \subset {X \setminus A}$ և $\lim{x_n}=s$։ Այժմ, կիրառելով թեորեմի առաջին մասը ${\{x_n\}}$ հաջորդականության նկատմամբ, ստանում ենք, որ մի որոշ պահից սկսած նրա բոլոր անդամները պետք է գտնվեն $A$-ի մեջ, ինչը սակայն հակասում է նրան, որ ${\{x_n\}} \subset X \setminus A$։ Ուստի $A$-ն բաց ենթաբազմություն է։\qedhere
\end{proof}
\renewcommand*{\proofname}{\hspace{18pt}\textbf{Ապացուցում։}\nopunct}
\par Այժմ քննարկենք փակ ենթաբազմությունների դեպքը։ Այդ նպատակով վերադառ\-նանք վերը բերված \hyperref[հատկությանը $2$-րդ]{հատկությանը $2$-րդ}՝ \red{ref} եթե գոյություն ունի ${\lim\{x_n\} = s}$ սահման, ապա $s \in \widebar{\{x_n\}}$։ 

\par Այս հատկությունը ճիշտ է նաև հետևյալ ավելի ընդհանուր տեսքով․ Դիցուք ունենք $A \subset X$ որևէ ենթաբազմություն և $\{x_n\} \subset A$ հաջորդականություն։ Եթե գոյություն ունի $\lim{x_n}=s$ սահման, ապա $s \subset \widebar{A}$։ Այսինքն $A$ ենթաբազ\-մու\-թյան մեջ պարունակվող ցանկացած զուգամետ հաջորդականության ամեն մի սահման հպման կետ է $A$-ի համար (պարզունակ, սահմանումների մակարդակով կատարվող ապացույցը \textbf{թողնվում է ընթերցողին})։ 

\begin{theorem}
Դիցուք $X$ տարածությունը բավարարում է հաշվելիության առաջին աքսիոմին։ Ապա $X$-ի կամայական ոչ դատարկ $A$ ենթաբազմություն փակ է այն և միայն այն դեպքում, երբ $A$ ենթաբազմությունում պարունակվող ամեն մի զուգամետ հաջորդականության $\forall\hspace{-0.166em}$ սահման հպման կետ է $A$-ի համար։
\end{theorem}

Պայմանի անհրաժեշտության  մասին արդեն խոսել ենք։ Իսկ բավարարությունը ապացուցվում է ինչպես \hyperref[thm:9.2]{թեորեմ 2}-ում \red{ref}՝ դարձյալ լեմմայի օգնությամբ (մանրամասնե\-րը թողնում ենք ընթերցողին)։
% Ապացուցումը թողնում ենք ընթերցողին (ցուցում․ $X$-ի ամեն մի կետի համար գոյություն ունի նրա շրջակայքերի ${\{V_n\}}$ հաշվելի ընտանիք, որ $V_{n+1} \subset V_n, \forall n$-ի դեպքում)։
\par Հաջորդ հիմնական հարցը հետևյալն է՝ կարելի՞ է արդյոք տարածության տոպո\-լոգիան նկարագրել զուգամետ հաջորդականությունների և նրանց սահմանների միջոցով։ Նախ հստակեցնենք հարցադրումը։
\par Դիցուք ունենք ինչ-որ $X$ բազմություն, դիտարկենք մի $M$ բազմություն, որի տարրերը $(\{x_n\},s)$ տեսքի որոշ զույգեր են, որտեղ $\{x_n\}$-ը որևէ հաջորդականություն է $X$-ում, իսկ $s$-ը $X$-ի որևէ կետ է։
Պահանջենք, որ $M$-ը բավարարի հետևյալ երկու պայմաններին՝
\begin{enumerate}
    \item[ա)] Եթե $(\{x_n\},s) \in M$, ապա $\{x_n\}$-ի ցանկացած $\{y_n\}$ ենթահաջորդականության դեպքում $(\{y_n\},s) \in M$։
\item[բ)] ցանկացած $\{x_n\}$ ստացիոնար հաջորդականության դեպքում, եթե ինչ-որ ան\-դամից սկսած $x_m=x_{m+1}=\cdots=s$, ապա $(\{x_n\},s) \in M$։
\end{enumerate}
\begin{question}
Տվյալ $X$ և $M$ բազմությունների դեպքում գոյություն ունի՞ արդյոք միակ այնպիսի $\tau$ տոպոլոգիա $X$-ում, որ $(\{x_n\},s) \in M$ այն և միայն այն դեպքում, երբ գոյություն ունի $\lim{x_n}=s$ սահման $(X,\tau)$ տարածությունում։
\par Այս հարցի դրական պատասխանի դեպքում կասենք, որ $X$-ի $\tau$ տոպոլոգիան որոշվում է $M$ բազմությունով։
\end{question}
\begin{question}[(հարց 1-ի հակառակը)]
Ճի՞շտ է արդյոք, որ $X$ բազմության կամայական $\tau$ տոպոլոգիայի համար գոյություն ունի վերը նկարագրված $(\{x_n\},s)$ զույգերի $M$ բազմություն այնպես, որ $M$-ով որոշվող $X\textrm{-ի}$ տոպոլոգիան համընկնում է $\tau$-ի հետ։
\end{question}
\par 1 և 2 հարցերի դրական պատասխանների դեպքում կունենանք, որ տվյալ $X$ բազմության բոլոր տոպոլոգիաները կարող են լիովին նկարագրվել $X$-ում զուգամետ հաջորդականությունների տերմիններով։

\par Այժմ քննարկենք այդ հարցերը փակման գործողության տեսանկյունից։ Դիցուք $A$-ն $X$ տարածության սևեռված ենթաբազմություն է։ Նշանակենք $\widetilde{A}$ բոլոր $A$-ի մեջ պարունակվող զուգամետ հաջորդականությունների սահմանների բազմությունը։ Պարզ է, որ $\widetilde{A} \subset \widebar{A}$։ Ենթադրենք, որ իր հերթին $\widebar{A} \subset \widetilde{A}$, և հետևաբար $\widetilde{A} = \widebar{A}$։

\par Սա նշանակում է, որ $\widebar{A}$ փակումը համընկնում է $A$ ենթաբազմությունում պարու\-նակ\-վող բոլոր զուգամետ հաջորդականությունների սահմանների բազմության հետ։ Եթե ասվածը տեղի ունենա $X$-ի ցանկացած $A$ ենթաբազմության դեպքում, ապա $X$ տարածությունում ենթաբազմությունների փակման գործողությունը, ուստի և $X$-ի տոպոլոգիան լիովին կորոշվի $X$-ում զուգամետ հաջորդականությունների և նրանց սահմանների միջոցով (ևս մի հնարավորություն բազմության վրա տոպոլոգիա սահ\-մա\-նելու համար)։ Այն, որ դա հնարավոր է որոշ տարածությունների դեպքում, նախա\-պես ցույց տանք հասարակ օրինակով։

\par Վերցնենք որևէ $X$ բազմություն և նրանում զուգամետ հաջորդականություններ հայտարարենք միայն և միայն ստացիոնար հաջորդականությունները։ Եթե $\{x_n\}\textrm{-ը}$ մի այդպիսի հաջորդականություն է՝ $x_m = x_{m+1} = \cdots = s$, որոշ $m$ համարից սկսած, ապա սահմանենք $\lim\{x_n\} = a$։ Պարզ է, որ $\forall A \subset X$ ենթաբազմության դեպքում $\widetilde{A} = A$։ Արդյունքում ստանում ենք $A \mapsto \cl A$ գործողություն $X$-ում, որտեղ, վերցնելով $\cl A=\widetilde{A}$։ Հեշտությամբ ստուգվում է, որ այն Կուրատովսկու փակման գործողություն է (բավարարվում են K1-K4 աքսիոմները) և ստացված տոպոլոգիայում $\widebar{A}=\widetilde{A}=A$։ Նկատենք, որ ստաց\-վածը $X$-ի դիսկրետ տոպոլոգիան է (\textbf{հիմնավորե՛ք})։	

\begin{nquestion} Տեղի ունի՞ արդյոք նույնը ցանկացած տոպոլոգիական տարածության դեպքում։
Պատասխանը բացասական է, ցույց տանք օրինակով։
\end{nquestion}
\begin{example}
 Դիտարկենք որևէ $(X, \textrm{հաշվ․ լրաց․})$ տարածություն, որտեղ $X$-ը ոչ հաշվելի բազմություն է։ Սևեռենք որևէ $a \in X$ կետ և դիտարկենք $X$-ի $A = X \setminus \{a\}$ ենթաբազմությունը։ Հեշտ է տեսնել, որ $a \in \widebar{A}$։ Ցույց տանք, որ գոյություն չունի $\{x_n\} \subset A$ զուգամետ հաջորդականություն, որ $\lim\{x_n\}=a$։ Ենթադրենք հակառակը․ գոյություն ունի $\{x_n\} \subset A$ հաջորդականություն որ $\lim\{x_n\}=a$։ Պարզ է, որ $U= X \setminus \{x_n\}$ ենթաբազմությունը $a$ կետի բաց շրջակայք է։ Բայց $U \cap \{x_n\}= \varnothing$, ինչը հակասում է $\lim\{x_n\} = a$ պայմանին։
\end{example}
\par Այս «անհաջողության» պատճառն այն է, որ հաջորդականությունները հաշվելի բազմություններ են, ինչը թույլ չի տալիս ընդհանուր $(X,\tau)$ տոպոլոգիական տարածությունների դեպքում, հայտնաբերել ենթաբազմության բոլոր հպման կետերը հաջորդականության սահմանների տեսքով։

\par Այնուամենայնիվ ընդհանուր տոպոլոգիայում զարգացվում է զուգամիտության տեսություն այնպես, որ ցանկացած տոպոլոգիա լիովին նկարագրվում է զուգամի\-տու\-թյան տերմիններով։ Արվում է դա երկու (համարժեք) եղանակով, ընդհանրաց\-նելով հաջորդականություն և հաջորդականության սահման հասկացությունները։ Մի դեպքում ներմուծվում են \textbf{ֆիլտր} և \textbf{զուգամետ ֆիլտր} հասկացությունները, իսկ մյուս դեպքում՝ \textbf{ուղղվածություն} և \textbf{զուգամետ ուղղվածություն} հասկացությունները (մանրամասնությունները տե՛ս \cite{Calley} \red{empty?} գրքում)։

\red{chem hishum ditoxutyny vonc einq dnum}
Որոշ դասի տարածությունների համար վերոհիշյալ հարցի պատասխանը դրական է նաև սովորական հաջորդականությունների դեպքում։ Իրոք, տեղի ունի․

\begin{theorem}
Եթե $X$ տոպոլոգիական տարածությունը բավարարում է հաշվելիության առաջին աքսիոմին (մասնավորապես՝ մետրիկային տարածություն է), ապա ցանկացած $A \subset X$ ենթաբազմության համար $\widetilde{A} = \widebar{A}$։
\end{theorem}
\begin{proof}
Ցույց տանք, որ $\widebar{A}\subset \widetilde{A}$։ Դիցուք $a \in \widebar{A}$, ցույց տանք գոյություն ունի $\{x_n\} \subset A$ հաջորդականություն, որ $\lim\{x_n\}=a$։ Ըստ լեմմայի՝ $a$ կետի համար գոյություն ունի շրջակայքերի հաշվելի $\{U_n\}_{n\in \N}$ բազա, որ $U_{n+1}\subset U_n, \ n\in \N$։
% $V_i,\ i \in \mathbb{N}$ բազա։ Կառուցենք $a$ կետի շրջակայքերի նոր՝ $U_i,\ i \in \mathbb{N}$ հաշվելի բազա, վերցնելով $U_1 = V_1,\ U_2={V_1 \cap V_2}, \cdots,\\{U_n= \bigcap\limits_{k=1}^n V_k}$։ Պարզ է, որ $U_1 \supset U_2 \supset U_3 \supset \cdots$ և $U_n \subset V_n,\ n \in \mathbb{N}$։

\par Քանի որ $U_n \cap A \neq  \varnothing$, կարող ենք յուրաքանչյուր $U_n$-ում ընտրել որևէ $x_n \in A$ կետ։ Ստանում ենք $\{x_n\} \subset A$ հաջորդականություն, և ակնհայտ է, որ $\lim\{x_n\}=a$։ Ուստի $a\in \widetilde{A}$։ 
\end{proof}
\end{document}