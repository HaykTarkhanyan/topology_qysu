\documentclass[./main.tex]{subfiles}

\begin{document}
\onehalfspacing
\section*{Նախաբան}\label{sec:intro}
\par
Սույն մեթոդական ձեռնարկը նախատեսված է մաթեմատիկայի և մեխանիկայի ֆակուլտետի երկրորդ կուրսեցիների համար և նպատակ ունի դյուրացնելու «Տոպո\-լոգիա» ուսումնական դասընթացի յուրացումը։
\par Ձեռնարկում շարադրված նյութը պայմանականորեն բաժանել ենք 17 թեմաների (ըստ 1 կիսամյակում դասախոսությունների հնարավոր առավելագույն քանակի)։
\par Առաջին երեք թեմաները նվիրված են բազմությունների նաիվ (կանտորյան) տեսությանը։ Այստեղ, բացի քիչ թե շատ հեշտ յուրացվող՝ բազմությունների ուղիղ արտադրյալ, ֆակտոր-բազմություն, բազմությունների արտապատկերումներ հաս\-կա\-ցու\-թյուն\-ներից՝ քննարկվում է նաև անվերջության հասկացությունը վերջավոր և անվերջ բազմությունների հակադրման տեսքով, ինչպես նաև բազմությունների քանակական համեմատումը և բազմության տարրերի քանակություն հասկացու\-թյունը ընդհանուր դեպքում։
\par Հաջորդ վեց թեմաներում քննարկվում է բազմության վրա տոպոլոգիա սահմանե\-լու ֆունդամենտալ խնդիրը․ դիտարկվում են տոպոլոգիայի և տոպոլոգիական տա\-րա\-ծու\-թյան տարբեր (բայց համարժեք) սահմանումներ բաց (փակ) ենթա\-բազմու\-թյուն\-ների, կետերի շրջակայքերի, փակման գործողության, հաջորդականություն\-ների զուգամիտության տերմիններով։  Այստեղ ներմուծվում են նաև մետրական տարածություն և մետրական տոպոլոգիա կարևոր հասկացությունները, անջատելի\-ության $\textrm{T}_0, \textrm{T}_1, \textrm{T}_2$ աքսիոմները, հաշվելիության \RNum{1} և \RNum{2} աքսիոմները, սեպարաբել տարածության հասկացությունը։
\par Հաջորդ՝ 10-րդ և 11-րդ թեմաները նվիրված են տոպոլոգիական տարածու\-թյուն\-ների անընդհատ արտապատկերումներին, այդ թվում՝ հոմեոմորֆիզմ, հոմեոմորֆ տարածություններ, տոպոլոգիական ինվարիանտ ֆունդամենտալ հասկացու\-թյուն\-ներին։
\par Հաջորդ՝ 12, 13, 14 թեմաներում սահմանվում և պարզաբանվում են տոպո\-լոգիա\-կան տարածությունների երեք հիմնական կառույցներ՝ ենթատարածություն, ֆակտոր-տարածություն, տարածությունների ուղիղ արտադրյալ հասկացությունները։
\par Նախավերջին 15-րդ և 16-րդ թեմաները նվիրված են տոպոլոգիական տարածու\-թյուն\-ների երկու կարևոր դասերի՝ կապակցված և գծորեն կապակցված տարածու\-թյուն\-ներին։ Վերջին՝ 17-րդ թեման նվիրված է կոմպակտ տարածություններին, դաս\-ընթացն ավարտվում է էվկլիդյան $\R^n$ տարածությունների կոմպակտ ենթա\-բազմու\-թյուն\-ների նկարագրումով։
\par Ընդհանուր տոպոլոգիան իր խիստ վերացականության պատճառով վատ է յուրաց\-վում (հաճախ պարզապես չի յուրացվում) հատկապես հարակից ուսումնական առար\-կաների նախնական վատ պատրաստվածություն ունեցող ուսանողների կողմից։ Հաշվի առնելով սա՝ փորձել ենք որքան հնարավոր է ընդգծել տոպոլոգիայի բնույթը՝ որպես երկրաչափության բաժին, ինչպես նաև նրա կապերը մաթեմատիկական անալիզի հետ։
\par Քանի որ բացի շաբաթական մի դասախոսությունից՝ նախատեսված է նաև մի գործնական պարապմունք, հարմար գտանք դասընթացի սույն տեքստում խուսափել հատուկ ընտրված խնդիրներ հանձնարարելուց՝ նպատակ դնելով օրինակների քննարկ\-ման և հարցերի միջոցով հասնել հիմնական հասկացությունների ըմբռնմանը և յուրացմանը։ Դրան նպաստելու համար շատ դեպքերում ուսանողին առաջարկում ենք ինքնուրույն պատասխանել փոքր դետալների վերաբերյալ տեքստում առկա «ինչո՞ւ» կամ «հիմնավորե՛լ» հարցադրումներին։ 
\par Նշենք նաև, որ թեորեմներն ու օրինակները համարակալված են ըստ թեմաների։ Քանի որ հղումները շատ չեն, խուսափել ենք թեորեմների և օրինակների կրկնակի համարակալումից։
\par Ձեռնարկի վերջում բերված է գրականության ոչ մեծաթիվ ցանկ, որի միջոցով ընթերցողը կարող է լրացնել իր գիտելիքը և խորանալ տոպոլոգիայի առանձին բաժինների մեջ։
% \par Ձեռնարկի մեջ վրիպակներ կամ/և սխալներ գտնելու դեպքում խնդրում ենք ուղարկել դրանք mathmechsss@gmail.com էլեկտրոնային հասցեին։ Նախապես շնորհակալություն։

\end{document}