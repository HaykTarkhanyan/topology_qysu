% Ստուգված, Խնդիրներն էլ հետը 07.12
% 
% 

\documentclass[./main.tex]{subfiles}

\begin{document}
\onehalfspacing

\section{Թվային ուղիղը որպես տոպոլոգիական տարածության նախատիպ։ Տոպոլոգիա բազմության վրա, տոպոլոգիական տարածության հասկացությունը, տոպոլոգիաների համեմատումը։ Տոպոլոգիական տարածության այլընտրանքային սահմանում հիմնված կետի շրջակայք հասկացության վրա։}\label{sec:4}

\par Տոպոլոգիական երկրաչափության բաժին է, որի հիմքում ընկած են տոպոլոգիա\-կան տարածություն և տոպոլոգիական տարածությունների անընդհատ արտա\-պատ\-կերում հասկացությունները։ Եթե տարրական երկրաչափությունը ուսումնասիրում է երկրաչա\-փական պատկերների այն հատկությունները, որոնք պահպանվում են (ինվարիանտ են) իզոմետրիկ  ձևափոխությունների դեպքում (այսինքն այնպիսի ձևափոխությունների դեպքում, որոնք չեն փոխում կետերի միջև հեռավորություն\-նե\-րը), ապա տոպոլոգիան ուսումնասիրում է երկրաչափական պատկերների այն հատ\-կու\-թյունները, որոնք անփոփոխ են այդ պատկերների ցանկացած \textbf{անընդհատ} ձևափոխութ\-յունների դեպքում։ Ձևափոխությունը կոչվում է անընդհատ, եթե այն միմյանց բավականաչափ մոտիկ կետերը արտապատկերում է նորից բավականաչափ մոտիկների։

\par Մաթեմատիկայում հայտնի են մի շարք տարբեր երկրաչափություններ՝ էվկ\-լի\-դեսյան, աֆինական, պրոյեկտիվ, ոչ էվկլիդեսյան և այլն։ երկրաչափություններ։ Դրանցից յուրաքանչյուրի հիմքում ընկած է իրեն բնորոշ ձևափոխությունների խումբ։ Նկատենք, որ ձևափոխությունների այդ խմբերը, այս կամ այլ իմաստով լինելով իրարից տարբեր, ունեն մի ընդհանուր առանձնահատկություն՝ դրանց տար\-րերը (ձևափոխութ\-յունները) անընդհատ են։

\par Ուստի կարող ենք ասել, որ տոպոլոգիան բոլոր հնարավոր երկրաչափու\-թյուն\-ների ընդհանուր մասն է, և այդ իմաստով կարող է դիտվել որպես վերջնական և բոլորի հիմքում ընկած յուրահատուկ երկրաչափություն։

\par Ընդհանուր տոպոլոգիայում երկրաչափական պատկերները հանդես են գալիս որպես հատուկ կառուցվածքով, վերացական բնույթի կետերի բազմություններ, որոնց անվանում են \textbf{տոպոլոգիական տարածություններ}։ Տոպոլոգիական տարածություն հասկա\-ցու\-թյունը անցել է ձևավորման և կայացման երկար ուղի։ Սկզբնական շրջանում (1900-1930 թթ․) այն ունեցել է տարբեր անվանումներ՝ շրջակայքային տարածություն, փակման գործողութ\-յունով տարածություն, մետրիկային տարածություն։ Դրանց նվիրված էր երկու գլուխ Ֆ․ Հաուսդորֆի «Բազմությունների տեսություն» նշանա\-վոր գրքի 1914 և 1927 թվերի հրատարակումներում (որպես բազմությունների տեսու\-թյան ենթաբաժիններ)։ Եվ միայն 1930 թվին ամերիկացի մաթեմատիկոս Ս. Լեֆշեցը հրապարակեց գիրք «Տոպոլոգիա» (Topology) անվանումով, որտեղ տոպոլոգիան ներկայացված է որպես երկրաչա\-փություն հիմնված, և հետագայում համընդհանուր հավանության արժանացած, աքսիոմների համակարգի վրա։

\par Մինչև այդ աքսիոմները ներկայացնելը քննարկենք տոպոլոգիական տարածու\-թյան մի օրինակ, որը հանդիսացել է նախատիպ՝ տոպոլոգիական տարածություն ընդհանուր հասկացության համար։

\par Մաթեմատիկական անալիզի դասընթացում դիտարկվում են թվային ուղղի կե\-տե\-րին և ենթաբազ\-մություններին առնչվող մի շարք հասկացություններ՝ կետի $\varepsilon$-շրջակայք, ենթաբազմության ներքին կետ, հպման կետ, բաց ենթաբազմություն, փակ ենթաբազմություն, զուգամետ հաջորդականություն։ Այդ հասկացություններից որևէ մեկը կանվանենք \textbf{հիմնային հասկացություն}, եթե մյուսները կարող են սահ\-մանվել դրա միջոցով։

\par Ցույց տանք, որ \textbf{կետի $\varepsilon$-շրջակայքը հիմնային հասկացություն է}։ Հիշեցնենք, որ թվային ուղղի $x_0$ կետի $\varepsilon$-շրջակայք կոչվում է $|x-x_0|<\varepsilon$ (որտեղ $\varepsilon$-ը 0-ից մեծ թիվ է) պայմանին բավարարող բոլոր $x$ թվերի բազմությունը, կամ որ նույնն է՝ $(x_0-\varepsilon, x_0+\varepsilon)$ ինտերվալը։ Այնուհետև, որևէ $X \subset \R$ ենթաբազմության $x_0$ կետը կոչվում է $X$-ի ներքին կետ, եթե գոյություն ունի $x_0$-ի որևէ $\varepsilon$-շրջակայք, որն ամբողջությամբ ընկած է $X$-ում՝ $(x_0-\varepsilon,x_0+\varepsilon) \subset X$։ Օրինակ՝ $X=\{-2\} \cup (1,3] \cup [5,+\infty)$ ենթաբազմության համար ներքին կետեր են իր բոլոր կետերը բացառությամբ $-2,3,5$ կետերի (ինչո՞ւ)։

\par Ըստ սահմանման՝ թվային ուղղի որևէ ենթաբազմությունը կոչվում է բաց ենթաբազ\-մություն, եթե նրա բոլոր կետերը ներքին կետեր են։ Օրինակ՝ պարզ է, որ ամեն մի $(a,b)$ ինտերվալ բաց ենթաբազմություն է (հիմնավորե՛ք)։

\par Ենթաբազմության հպման կետ հասկացությունը նույնպես սահմանվում է կետի $\varepsilon$-շրջակայք հասկացությունով․ $x_0$ կետը կոչվում է $X \subset \R$ ենթաբազմության հպման կետ, եթե այդ կետի ցանկացած $\varepsilon$-շրջակայք ունի ոչ դատարկ հատում $X$-ի հետ։ Պարզ է, որ ցանկացած ենթաբազմության բոլոր կետերը հպման կետեր են իր համար։ Իսկ օրինակ՝ $X=\{\frac{1}{n},\, n \in \N\}\cup[2,4)\cup (5,+\infty)$ ենթաբազմության համար բացի սեփական կետերից հպման կետեր են նաև իրեն չպատկանող $0,4,5$ կետերը (հիմնավորե՛ք)։

\par Թվային ուղղի ենթաբազմությունը կոչվում է փակ ենթաբազմություն, եթե այն պարունակում է իր բոլոր հպման կետերը։ Հեշտ է ցույց տալ, որ փակ ենթաբազմություններ են թվային ուղղի բոլոր միկետանոց $\{a\}$, բոլոր $[a,b], (-\infty,a],[b,+\infty)$ տեսքերի ենթաբազմությունները, ինչպես նաև դրանց բոլոր վերջավոր միավորումները։ Ուստի, մեծ հաշվով, փակ ենթաբազմությունը նույնպես սահմանվում է կետի $\varepsilon$-շրջակայք հասկացությունով։

Վերջապես պարզ է նաև, որ զուգամետ հաջորդականություն հասկացությունը ևս սահմանվում է $\varepsilon$-շրջակայք հասկացությունով (հիմնավորե՛ք)։

\par Այսպիսով, կետի $\varepsilon$-շրջակայք հասկացությունը հիմնային հասկացություն է թվա\-յին ուղղի համար, և այս փաստը կարձանագրենք կարճ՝ ասելով, որ \textbf{կետի $\varepsilon$-շրջա\-կայք հասկացությունը որոշում է տոպոլոգիա թվային ուղղի վրա}։

\par Այժմ ցույց տանք, որ թվային ուղղի համար \textbf{հիմնային հասկացություն է նաև\linebreak բաց ենթաբազմություն հասկացությունը}։ Այդ նպատակով նախ ճշտենք թվային ուղղի բաց ենթաբազմություն հասկացությունը՝ առանց հիմնվելու կետի $\varepsilon$-շրջա\-կայք հաս\-կացության վրա։

\par Բաց բազմությունների ինտուիտիվ ընկալվող առաջին օրինակները $(a,b)$ տեսքի ինտերվալներն են։ Ի տարբերություն $[a,b], (a,b], [a,b)$ տեսքերի միջակայքերի, որոնք «ցանկապատված են» մի կամ երկու կողմից $a,b$ կետերով, $(a,b)$ ինտերվալները ցանկապատ չունեն և բաց են երկու կողմից։ Ուստի բնական կլինի թվային ուղղի որևէ $X$ ենթաբազմություն համարել նրա բաց ենթաբազմություն, եթե $X$-ի ոչմի կետ չի հանդիսանում ցանկապատ նրա համար։ Իսկ դա նշանակում է՝ գոյություն ունի որևէ $(a,b)$ ինտերվալ, որ $x_0 \in (a,b)$ և $(a,b)$-ն ընկած է $X$-ում։ Այստեղից որպես հետևանք ստանում ենք․ թվային ուղղի որևէ $X$ ենթաբազմություն բաց ենթաբազմություն է այն և միայն այն դեպքում, երբ այն կամ որևէ ինտերվալ է, կամ ներկայացվում է որպես որոշ քանակով ինտերվալների միավորում (հիմնավորե՛ք)։

\par Այժմ կարող ենք ընդլայնել նաև կետի $\varepsilon$-շրջակայք հասկացությունը, ներմու\-ծե\-լով նոր՝ \textbf{կետի շրջակայք} հասկացություն․ թվային ուղղի կամայական կետի շրջա\-կայք կանվանենք այդ կետը պարունակող ցանկացած բաց ենթաբազմություն։ Սա հնա\-րա\-վորություն է տալիս նորովի, բաց ենթաբազմության հիմքով, վերա\-սահ\-մանել նաև ենթաբազմության ներքին կետ, հպման կետ, փակ ենթաբազմություն, զուգա\-մետ հաջորդականություն հասկացությունները․ դրա համար բավական է հին սահ\-մա\-նում\-ների ձևակերպումներում ամենուրեք <<կետի $\varepsilon$-շրջակայքը>> բառակապակցությունը փոխարինել <<կետի շրջակայք>> բառակապակցությունով։

\par Այսպիսով, թվային ուղղի \textbf{բաց ենթաբազմություն հասկացությունը հիմնային է և նույնպես որոշում է տոպոլոգիա այդ ուղղի վրա}։

\par Նկատենք, որ ի տարբերություն կետի $\varepsilon$-շրջակայք հասկացության՝ թվային ուղղի բաց ենթաբազմություններն օժտված են պարզ ձևակերպվող և հեշտ ապացուցվող հետևյալ երեք հատկություններով՝
\begin{enumerate}
    \item ինքը՝ թվային ուղիղը բաց ենթաբազմություն է,
    \item ցանկացած (վերջավոր կամ անվերջ) քանակով բաց ենթաբազմությունների միավորումը բաց ենթաբազմություն է,
    \item վերջավոր քանակով ցանկացած բաց ենթաբազմությունների հատումը կամ դա\-տարկ է, կամ դարձյալ բաց ենթաբազմություն է։
\end{enumerate}

\par 3-րդ հատկությունում ենթաբազմությունների քանակի սահմանափակումը
պայմանավորված է նրանով, որ անվերջ քանակով բաց ենթաբազմությունների հատումը կարող է բաց չլինել: Օրինակ, ինտերվալների $\bigcap_{n \in N} (-\frac{1}{n}, \frac{1}{n})$ հատումը թվային ուղղի միկետանոց $\{0\}$ ենթաբազմություն է, որը բաց ենթաբազմություն չէ (հիմնավորե՞ք)։

\par Ելնելով որոշ տեխնիկական նկատառումներից՝ նպատակահարմար է դատարկ բազմությունը նույնպես համարել բաց ենթաբազմություն (դրան խոչընդոտող որևէ իրական հանգամանք չկա)։ Արդյունքում 1-ին և 3-րդ հատկությունները կարելի է վերաձևակերպել հետևյալ տեսքով․

\begin{enumerate}
    \item $\R$-ը և  $\varnothing$-ը բաց ենթաբազմություններ են,
    \item[3.] վերջավոր քանակով ցանկացած բաց ենթաբազմությունների հատումը բաց ենթաբազմություն է։
\end{enumerate}

Թվային ուղղի բաց ենթաբազմությունների այս երեք հատկությունները վերածվել են աքսիոմների ընդհանուր դեպքում, այսինքն՝ կամայական բազմություններում, աքսի\-ոմ\-ների միջոցով տոպոլոգիա սահմանելու համար։

\begin{note}
Նշենք, որ բաց բազմությունների 1-3 հատկություն ները բնորոշ են ոչ միայն թվային ուղղի, այլ նաև հարթության, եռաչափ տարածության և, ընդհանրապես $n$-չափականության $\R^n$ էվկլիդեսյան տարածությունների բաց ենթաբազմություններին:

\par Ասվածը պարզաբանենք հարթության օրինակով, չխորանալով դետալների մեջ: Այն դերը, որ կատարում են կետերի $\varepsilon$-շրջակայքերը թվային ուղղի բաց ենթաբազմություններ սահմանելիս, հարթության դեպքում կատարում են անեզր շրջանները: Հարթության որևէ $X$ ենթաբազմություն կոչվում է բաց ենթաբազմություն, եթե նրա ամեն մի $x^0=(x_1^0, x_2^0)$ կետի համար գոյություն ունի $x^0$ կենտրոնով, $\varepsilon_0 > 0$ շառավղով $\{x=(x_1, x_2); \quad (x_1 - x_1^0)^2 + (x_2 - x_2^0)^2 < \varepsilon^2\}$ անեզր շրջան, որն ամբողջովին ընկած է $X$-ում:

\par Որպես ամփոփում նշենք. առանձին վերցված ամեն մի դեպքում (ուղիղ, հարթություն և այլն) բաց ենթաբազմությունները սահմանվում են յուրովի: Բայց դրանք բոլորն էլ օժտված են միևնույն վերոհիշյալ
1-3 հատկություններով
\end{note}

\begin{note}
    Ինչպես գիտենք, էվկլիդեսյան երկրաչափության հիմքում ըն\-կած են երեք տիպի օբյեկտներ, որոնք կոչվում են «կետեր», «ուղիղներ», «հար\-թու\-թյուն\-ներ» և չեն սահմանվում։ Փոխարենը աքսիոմների տեսքով ձևակերպվում են փոխ\-հա\-րա\-բե\-րություններ այդ օբյեկտների միջև․ օրինակ՝ ամեն մի երկու տարբեր կե\-տե\-րով անցնում է ուղիղ, և այն միակն է, կամ՝ ուղղով և նրան չպատկանող կետով կարելի է (դրանց հարթության մեջ) տանել ոչ ավելի, քան մի ուղիղ, որը չի հատում տվյալ ուղղին (զուգահեռության աքսիոմ)։ Այնուհետև զուտ տրամաբանորեն դուրս են բերվում հետևություններ աքսիոմներից, սահմանվում են նոր հասկացություններ, բացահայտելով (ապացուցելով) այդ հասկացությունների որոշակի հատկություններ և այլն։
\end{note}

\par Նշենք, որ տվյալ երկրաչափության համար աքսիոմների համակարգը որպես կանոն միակը չէ։ Օրինակ՝ վերլուծական երկրաչափության դասընթացում (կոորդի\-նա\-տային մեթոդի միջոցով) կառուցվում է էվկիլդեսյան երկրաչափությանը համ\-արժեք երկրաչափություն, որի հիմքում ընկած են «կետ» և «վեկտոր» չսահմանվող հասկացու\-թյունները։ Այս դեպքում արդեն ուղիղն ու հարթությունը սահմանվում են կետ և վեկտոր հասկացու\-թյուններով։

\par Նույնը տեղի ունի նաև տոպոլոգիայի դեպքում․ բազմության վրա տոպոլոգիա և տոպոլոգիական տարածություն հասկացությունները կարող են սահմանվել նաև աքսիոմների այլ համակարգերով՝ հիմքում դնելով ոչ թե բաց ենթաբազմություն (չսահմանվող) հասկա\-ցությունը, այլ մեկ ուրիշ (կամ մի քանի) դարձյալ չսահմանվող հասկացություն, օրինակ՝ բազմության \textbf{կետի շրջակայք} հասկացությունը։ Ընդ որում, մի մոտեցման դեպքում տոպոլոգիայի հիմքում ընկած չսահմանվող հաս\-կա\-ցությունը կարող է արդեն սահմանվել մեկ այլ մոտեցման դեպքում։

\par Այժմ ընդհանուր դեպքում սահմանենք բազմության վրա տոպոլոգիա և տոպո\-լո\-գիա\-կան տարածություն հասկացությունները՝ հիմք ընդունելով բաց բազմություն չսահման\-վող հասկացությունը։

\begin{definition}
Ասում են, որ կամայական $X$ բազմության վրա տրված է \textbf{տոպոլոգիա}, եթե տրված է $X$-ի որոշ $U_i$ ենթաբազմությունների $\tau=\{U_i,\ i\in I\}$ ընտանիք, որը բավարարում է հետևյալ երեք պայմաններին (աքսիոմներին)․
\begin{enumerate}
    \item $X$-ը և դատարկ բազմությունը պետք է պատկանեն $\tau$-ին,
	\item $\tau$-ի ցանկացած քանակով տարրերի միավորումը պետք է պատկանի $\tau$-ին,
	\item $\tau$-ի ցանկացած վերջավոր քանակով տարրերի հատումը նույնպես պետք է պատկանի $\tau$-ին։
	\par Նկատենք, որ 3-րդ պայմանը կարող է փոխարինվել հետևյալ համարժեք պայ\-մա\-նով՝
	\item[3$'$.] $\tau$-ի ցանկացած երկու տարրերի հատումը նորից պետք է պատկանի $\tau$-ին։
\end{enumerate}
\end{definition}

% Այսպիսով տոպոլոգիայի աքսիոմների այս համակարգում ունենք միայն մի չսահ\-մանվող հասկացություն, այն է՝ բաց ենթաբազմություն հասկացությունը։

\begin{definition}
$X$ բազմությունը նրա վրա տրված $\tau=\{U_i\}_{i\in I}$ տոպոլոգիայի հետ միասին (այսինքն $(X, \tau)$ զույգը) կոչվում է \textbf{տոպոլոգիական տարածություն}։ $X$ բազմության $x\in X$ տարրերը կոչվում են տոպոլոգիական տարածության \textbf{կետեր}, իսկ $U_i \subset X$ \red{chem jokum inchi a nshel} ենթաբազմությունները (այսինքն $\tau$ տոպոլոգիայի տարրերը) կոչվում են տվյալ տոպոլոգիական տարածության \textbf{բաց ենթաբազմություններ}։
\end{definition}

\red{noric nuyny berel ? :)}

Ստորև բերվում են տոպոլոգիաների մի քանի օրինակներ, որոնցով գոյանում են համապատասխան անվանումներով տոպոլոգիական տարածություններ։

\begin{example}
Մեկից ավելի տարրեր պարունակող ամեն մի $X$ բազմության վրա կարելի է սահմանել առնվազն երկու տոպոլոգիա․
\begin{enumerate}
    \item[ա)] որպես $\tau$ վերցնենք $X$-ի բոլոր ենթաբազմությունների բազմությունը։
	Այս տո\-պո\-լոգիան կոչվում է \textbf{դիսկրետ տոպոլոգիա} $X$-ի վրա։
	\item[բ)] որպես $\tau$ վերցնենք միայն $X$-ից և դատարկ բազմությունից կազմված ընտա\-նիքը՝ $\tau=\{\varnothing, X\}$։ Այն կոչվում է \textbf{անտիդիսկրետ} տոպոլոգիա $X$-ի վրա։
    \par	Երկու դեպքում էլ 1-3 պայմանների ստուգումը դժվարություն չի հարուցում։
\end{enumerate}
\end{example}

Այսպիսով, \textbf{դիսկրետ տոպոլոգիական տարածությունում} (կնշանակենք \linebreak $(X;\textbf{դիսկր․})$) բաց ենթաբազմութ\-յուն\-ներ են համարվում $X$-ի բոլոր ենթաբազմու\-թյունները։ Իսկ \textbf{անտի\-դիսկրետ տոպո\-լոգիա\-կան տարածությունում} (կնշանակենք \linebreak $(X; \textbf{անտ․})$) բաց ենթաբազ\-մություններ են միայն $X$-ը և $\varnothing$-ը։


\begin{example}
Դիտարկենք երկու տարր պարունակող որևէ $X=\{x_1,x_2\}$ բազմու\-թյուն և  նրա ենթաբազմությունների  $\tau=\{ \varnothing,\{x_1\},\{x_1,x_2\}\}$ ընտանիքը։ Հեշտու\-թյամբ ստուգվում է, որ $\tau$-ն բավարարում է տոպոլոգիայի 1-3 աքսիոմներին։ Ստաց\-ված $(X,\tau)$ տոպոլոգիական տարածությունը կարճ կոչվում է \textbf{կապակցված կետա\-զույգ}։ Այս տարածությունում բաց ենթազնություններ են $\varnothing$, $X$ և $\{x_1\}$-ը, իսկ $\{x_2\}$ ենթաբազմությունը բաց ենթաբազմություն չէ։

\par Նկատենք, որ $\forall X=\{x_1,x_2\}$ բազմությունից ստացվում է ճիշտ և ճիշտ չորս տոպոլոգիական տարածություն․ դրանք են՝ դիսկրետ, անտիդիսկրետ տարածու\-թյուն\-ները և երկու կապակցված կետազույգերը (հիմնավորե՛ք, որ տվյալ $X$-ից այլ տոպոլոգիական տարածություն չի գոյանում)։
\end{example}


\begin{example}
$\R$ թվային ուղղի վրա դիտարկենք ենթաբազմությունների $\tau$ ընտա\-նիք կազմված $\varnothing$ բազմությունից, բոլոր $(a,b)$ ինտերվալներից և դրանց բոլոր հնա\-րա\-վոր միավորումներից։ Առաջին երկու պայմանները բավարարվում են անմիջա\-կա\-նորեն, ըստ $\tau$-ի սահմանման։ Ստուգենք $3$ պայմանը $3'$ տեսքով։ Եթե ունենք $\tau$ ընտանիքի որևէ երկու $U_1=\bigcup\limits_j{(a_j,b_j)}, \ j\in J$ և $U_2=\bigcup\limits_k {(c_k,d_k )},\ k\in K$ տարրեր (դրանք ինտերվալների միավորումներ են ըստ ինդեքսների $J$ և $K$ բազմությունների), ապա
\[ 
U_1 \cap U_2= \bigg(\mathsmaller{\bigcup}\limits_{j}{(a_j;b_j)} \bigg) \cap \bigg(\mathsmaller{\bigcup}\limits_{k}{(c_k;d_k )}\bigg) =\mathsmaller{\bigcup}\limits_{i,\, j} \left( (a_j;b_j ) \cap (c_k;d_k)\right)
\]
\par Պարզ է, որ ինտերվալների $(a_j,b_j)\cap (c_k,d_k)$ հատումը կամ դատարկ է, կամ էլ նորից ինտերվալ է։ Ուստի $U_1\cap U_2\in \tau$ ըստ $\tau$-ի սահմանման։
\par 
Այս տոպոլոգիան կոչվում է թվային ուղղի \textbf{սովորական տոպոլոգիա}։ 
\end{example}
\par Թվային ուղղի սովորական տոպոլոգիան շահեկանորեն առանձնանում է $\R$-ի մյուս բոլոր տոպոլոգիաներից։ Այդ տոպոլոգիայի հիմքով է կառուցվում մի փոփո\-խա\-կանի ֆունկցիաների մաթեմատիկական անալիզը։ Ստացված տոպոլոգիական տա\-րա\-ծությունը կնշանակենք $(\R; \textbf{սովոր․})$։ Այս տարածությունում բաց ենթաբազմու\-թյուն\-ներ են $\varnothing$-ը, բոլոր $(a,b)$ ինտերվալները և նրանց միավո\-րում\-ները։ 
\par Մասնավորապես բաց ենթաբազմություններ են նաև $\R$-ը և $(-\infty,a),\ (b, +\infty)$ տեսքերի ինտերվալները, իսկ մի կետանոց՝ $\{a\}$ տեսքի,  նաև $[a, b],\ (a, b]$,\linebreak $[a, b),\ (-\infty,a],\ [b,+\infty)$ տեսքերի ենթաբազմությունները, ինչպես նաև ռացիոնալ թվերի $\Q$, իռացիոնալ թվերի $I$ ենթաբազմությունները բաց ենթաբազմություններ չեն, քանի որ չեն կարող ներկայացվել որպես $(a,b)$ տեսքի ինտերվալների միավո\-րում\-ներ (հիմնավորե՛ք)։

\begin{example}
Դիցուք $X$-ը որևէ անվերջ բազմություն է։ Դիտարկենք նրա ենթա\-բազմությունների $\tau$ ընտանիքը՝ կազմված $X$-ից, $\varnothing$-ից և այն բոլոր $U\subset X$ ենթա\-բազմություններից, որոնց $X\setminus U$ լրացումը վերջավոր բազմություն է։ Ցույց տանք, որ $\tau$-ն որոշում է տոպոլոգիա X-ի վրա։ Ստուգենք 2-րդ պայմանը։ Դիցուք $U_j\in \tau,\ j\in J$, որտեղ $J$-ն ինդեքսների ինչ-որ բազմություն է։ Ըստ դե Մորգանի 1-ին բանաձևի՝ ${X\setminus \bigcup\limits_j U_j= \bigcap\limits_j (X\setminus U_j)}$։ Քանի որ $X\setminus U_j$ լրացումները վերջավոր ենթաբազ\-մու\-թյուն\-ներ են, ուստի նրանց հատումը նույնպես $X$-ի վերջավոր ենթաբազմություն է։ Հետե\-վա\-բար $\bigcup\limits_j U_j \in \tau$։ Ստուգենք 3-րդ պայմանը։ Դիցուք $U_{1}, U_{2},\dots, U_{k} \in \tau$։ Ըստ դե Մորգանի 2-րդ բանաձևի՝ $\bigcap\limits_{i=1}^k U_i$ ենթաբազմության $X\setminus \bigcap\limits_{i=1}^k U_i =\bigcup\limits_{i=1}^k (X\setminus U_i)$ լրա\-ցումը վերջավոր ենթաբազմություն է։ Ուստի $\bigcap\limits_{i=1}^k U_i\in \tau$։ Այս տոպոլոգիան կոչվում է \textbf{վերջավոր լրացումների տոպոլոգիա} $X$-ի վրա։
\end{example}


\begin{example}
Դիցուք $X$-ը ոչ հաշվելի որևէ բազմություն է։ Դիտարկենք նրա են\-թա\-բազմութ\-յուն\-նե\-րի $\tau$ ընտանիքը կազմված $X$-ից, $\varnothing$-ից և այն բոլոր $V$ են\-թա\-բազմություն\-նե\-րից, որոնց $X\setminus V$ լրացումը հաշվելի բազմություն է։ Ապա $\tau$-ն տոպո\-լոգիա է $X$-ի վրա (կոչվում է \textbf{հաշվելի լրացումների տոպոլոգիա})։ Ապացուցումը (թողնվում է ընթերցողին) կատարվում է նախորդ օրինակի նմանությամբ՝ հիմնվե\-լով հաշվելի բազմությունների միավորման և հատման հատկությունների վրա։
\end{example}

\par Կնշանակենք $(X,\textbf{վերջ․ լր․})$ և $(X, \textbf{հաշվ․ լր․})$ սիմվոլներով այն տոպոլոգիական տարածությունները, որոնք գոյանում են $X$ բազմության համապատասխանաբար վերջավոր լրացումների և հաշվելի լրացումների տոպոլոգիաներով։ Համեմատելով դրանք միմյանց հետ՝ նկատենք, որ իռացիոնալ թվերի բազմությունը բաց բազմու\-թյուն չէ $(\R, \textrm{վերջ․ լր․})$ տարածությունում, բայց բաց բազմություն է $(\R, \textrm{հաշվ․ լր․})$ տոպո\-լոգիա\-կան տարածությունում (ինչո՞ւ)։

\par Սահմանափակվելով առայժմ 1-5 օրինակներով՝ նկատենք նաև, որ եթե միև\-նույն \red{ba stegh inch poghel?} $X$ բազմության վրա ունենք որևէ երկու $\tau_1$ և $\tau_2$ տոպոլոգիա, ապա նրանց $\tau_1\cap \tau_2$ հատումը նորից տոպոլոգիա է $X$-ի վրա (հիմնավորե՛ք)։ Իսկ նրանց $\tau_1\cup\tau_2$ միավորումը կարող է չլինել տոպոլոգիա $X$-ի վրա (այդպիսի օրինակ կբերվի թեմա 5-ում)։ \\ \red{husam \\-y nor togh a tanum normal :) }
Միևնույն $X$ բազմության վրա տրված տոպոլոգիաների բազմության վրա սահմանվում է մասնակի կարգ՝ «թույլ է», «ուժեղ է» տեսքով։


\begin{definition}
Եթե $X$-ի վրա տրված են երկու՝ $\tau_1$ և $\tau_2$ տոպոլոգիա այնպես, որ $\tau_1\subset \tau_2$ ապա ասում են, որ $\tau_1$-ը \textbf{ուժեղ չէ} $\tau_2$-ից, կամ $\tau_2$-ը \textbf{թույլ չէ} $\tau_1$-ից։ Եթե $\tau_1\subset \tau_2$ և $\tau_1\neq \tau_2$, ապա ասում են, որ $\tau_1$-ը \textbf{թույլ է} $\tau_2$-ից ($\tau_2$-ը \textbf{ուժեղ է} $\tau_1$-ից)։
\end{definition}

Միևնույն $X$ բազմության վրա բոլոր տոպոլոգիաներից անտիդիսկրետ տոպոլո\-գիան ամենաթույլ, իսկ դիսկրետ տոպոլոգիան ամենաուժեղ տոպոլոգիաներն են։ Նկատենք, որ նույն $X$-ի վրա որևէ երկու տոպոլոգիաներ միշտ չէ, որ համեմատելի են։
\begin{example}
Եթե $X=\{x_1,x_2\}$, ապա $\tau_1=\{ \varnothing,\{x_1\},\{x_1,x_2\}\}$ և $\tau_2=\{\varnothing, \{x_2\}, \linebreak\{x_1,x_2\}\}$ տոպոլոգիաները համեմատելի չեն, քանի որ $\{x_1\} \in \tau_1$, բայց $\{x_1\}\not\in \tau_2$, ինչպես նաև $\{x_2\}\in \tau_2$, բայց $\{x_2\}\not\in  \tau_1$։
\end{example}

\begin{example}
Ցույց տանք, որ $\R$ թվային ուղղի սովորական տոպոլոգիան ուժեղ է $\R$-ի վերջավոր լրացումների տոպոլոգիայից։ Պարզ է, որ $(0, 1)$ ինտերվալը պատկա\-նում է դրանցից առաջինին և չի պատկանում երկրորդին։ Մյուս կողմից, եթե \linebreak $U\subset \R$ ենթաբազմությունը պատկանում է երկրորդին, ապա $\R \setminus U$ լրացումը վերջա\-վոր բազմություն է, ուստի կամ $U=\R$, կամ $\R \setminus U=\{x_1,x_2,\dots,x_n\}$: Երկրորդ դեպքում, համարելով $x_1<x_2<\dots<x_n$, կունենանք $U=(-\infty,x_1) \cup (x_1,x_2)\cup \dots \cup(x_{n-1},x_n) \cup(x_n,+\infty),$ և հետևաբար $U$-ն պատկանում է $\R$-ի սովորական տոպոլոգիային (ինչի՞ց է դա հետևում)։
\par Ընթերցողին առաջարկում ենք ցույց տալ (որպես օգտակար խնդիր), որ թվային ուղղի հաշվելի լրացումների տոպոլոգիան ուժեղ է վերջավոր լրացումների տոպո\-լո\-գիայից, բայց համեմատելի չէ թվային ուղղի սովորական տոպոլոգիայի հետ։
\end{example}

%Էջ 44 Օրինակ 7ից հետո 
Այժմ ցույց տանք, որ հնարավոր է ներմուծել \textbf{կետի շրջակայք} հասկացություն բոլոր տոպոլոգիական տարածություններում որպես հիմնական այլընտրանքային հասկացություն։

\begin{definition}
$(X, \tau )$ տոպոլոգիական տարածությունում $x\in X$ կետի \textbf{բաց շրջա\-կայք} կոչվում է այդ կետը պարունակող $X$-ի ցանկացած բաց ենթաբազմություն։ Այնուհետև, $x$ \textbf{կետի շրջակայք} կոչվում է ցանկացած $V\subset X$ ենթաբազմու\-թյուն, որն իր մեջ ընդգրկում է $x$-ի որևէ $U$ բաց շրջակայք։
\par Այսպիսով, կարճ՝ $V\subset X$ ենթաբազմությունը շրջակայք է $x$ կետի համար, եթե գոյություն ունի $U$ բաց ենթաբազմություն, որ $x\in U\subset V$։ Նկատենք նաև, որ ընդհանուր դեպքում կետի շրջակայքը կարող է չլինել այդ կետի բաց շրջակայք։
\end{definition}
\begin{example}
$(X,\textrm{դիսկր․})$ տարածությունում $x$ կետը պարունակող $X$-ի բոլոր ենթա\-բազմութ\-յուն\-ները (բաց) շրջակայքեր են $x$-ի համար։ Մասնավորապես, $\forall x \in X$ կետի համար $\{x\}$ ենթաբազմությունը $x$ կետի բաց շրջակայք է։ Իսկ $(X,\textrm{անտ․})$ տարածությունում ցանկացած կետ ունի միայն մի շրջակայք, որը ինքը՝ $X$-ն է։ Այս դեպքում արդեն $\{x\}$ ենթաբազմությունը $x$ կետի շրջակայք չէ, եթե $X$-ը պարունակում է մեկից ավելի կետեր։ Դիտարկենք ևս երկու օրինակ.
\end{example}

\begin{example} 
 $(\R,\textrm{սովոր․})$ տարածությունում $x=0{,}5$ կետի համար հետևյալ՝ $(0; 1)$, $[0; 1),\ (0; 1],\ [0; 1] \cup \{3\}$ ենթաբազմությունները շրջակայքեր են, ընդ որում դրանցից միայն առաջինն է բաց շրջակայք։ Իսկ $x=0, 1, 3$ կետերի համար դրանցից ոչ մեկը շրջակայք չէ (հիմնավորե՛ք)։
\end{example}
\begin{example} 
 $(\R,\textrm{վերջ․ լր․})$ տարածությունում իռացիոնալ թվերի բազմությունը շրջակայք չէ իր կետերից ոչ մեկի համար։ Իսկ $(\R, \textrm{հաշվ․ լր․})$ տարածությունում այն շրջակայք է իր ցանկացած կետի համար (հիմնավորե՛ք)։

\end{example}
	
\begin{theorem}\label{թեորեմ 1}
$(X,\tau)$ տոպոլոգիական տարածությունում որևէ $W\subset X$ ենթաբազմու\-թյուն բաց ենթաբազմություն է այն և միայն այն դեպքում, երբ այն շրջակայք է իր ցանկացած կետի համար։ %թեորեմ 1
\end{theorem}

\begin{proof}

Եթե $W$-ն բաց ենթաբազմություն է, ապա այն (բաց) շրջակայք է իր ցանկացած $w\in W$ կետի համար (ըստ կետի շրջակայքի սահմանման)։
Այժմ հակառակը՝ դիցուք $W$-ն շրջակայք է իր (ցանկացած) $w$ կետի համար։ Նշանա\-կում է՝ գոյություն ունի $U(w)\in \tau$ բաց ենթաբազմություն, որ $w\in U(w)\subset W$։ Ուստի $W$-ն կարող է ներկայացվել որպես բաց ենթաբազմությունների միավորում՝ \linebreak $W=\bigcup\limits_w U(w)$։ Հետևաբար $W$-ն բաց ենթաբազմություն է ըստ տոպոլոգիայի 2-րդ աքսիոմի։\qedhere
\end{proof}

\par Վերը մենք նշեցինք, որ կետի շրջակայք հասկացությունը հիմնային հասկացու\-թյուն է թվային ուղղի տոպոլոգիայի համար։ Պարզվում է, որ այն հիմնային հաս\-կա\-ցու\-թյուն է ցանկացած տոպոլոգիական տարածությունում։ Իր նշանա\-կու\-թյամբ այն հավասարազոր է բաց բազմություն հասկացությանը (պատմականորեն, տոպոլո\-գիայի՝ որպես մաթեմատիկայի բաժին ձևավորման առաջին փուլում տոպոլոգիա\-կան տարածությունները սահմանվել են կետերի շրջակայքերի միջոցով և կոչվել են \textbf{շրջակայքային տարածություններ}): 

\par Սա պարզաբանելու նպատակով նախ թվարկենք շրջակայքերի հիմնական հատ\-կու\-թյուն\-ները։

\begin{theorem}\label{թեորեմ 2} Դիցուք $(X, \tau )$-ն որևէ տոպոլոգիական տարածություն է։ Ապա $X$-ի կետերի շրջակայքերն օժտված են հետևյալ 4 հատկություններով.
\begin{enumerate}
\item ամեն մի $x$ կետ պատկանում է իր ցանկացած շրջակայքի,
\item եթե $V$-ն $x$ կետի շրջակայք է և $V \subset W$, ապա $W$-ն նույնպես $x$ կետի շրջակայք է,
\item $x$ կետի ցանկացած վերջավոր քանակով շրջակայքերի հատումը նորից $x$-ի շրջակայք է,
\item $x$ կետի ցանկացած $V$ շրջակայքի համար գոյություն ունի այդ կետի այնպիսի $U$ շրջակայք, որ $U \subset V$, և $U$-ն շրջակայք է իր ցանկացած կետի համար։
\end{enumerate}
\end{theorem}
\begin{proof}
Առաջին հատկությունն ակնհայտ է, ցույց տանք երկրորդը։ Եթե $V$-ն $x$ կետի որևէ շրջակայք է, ապա ըստ սահմանման գոյություն ունի $x$-ի $U$ բաց շրջակայք, որ $x \in U \subset V$։ Քանի որ $V \subset W$ և $x \in U \subset W$, ուստի $W$-ն նույնպես շրջակայք է $x$ կետի համար։

Ապացուցենք 3-ը։ Դիցուք $V_1$-ը և $V_2$-ը $x$ կետի որևէ երկու շրջակայքեր են։ \linebreak Ըստ սահմանման գոյություն ունեն $X$-ի $U_1$ և $U_2$ բաց ենթաբազմություններ, որ $x \in U_1 \subset V_1$ և $x \in U_2 \subset V_2$։ Ըստ տոպոլոգիայի երրորդ աքսիոմի՝ $U_1 \cap U_2$-ը բաց ենթաբազմություն է $X$-ում, և քանի որ $x \in U_1 \cap U_2 \subset V_1 \cap V_2 $, ուստի $V_1 \cap V_2 $-ը $x$ կետի շրջակայք է։ Այժմ հատկություն 3-ը ստացվում է պարզ ինդուկցիայով։

Ապացուցենք 4-ը։ $x$ կետի կամայական $V$ շրջակայքի համար գոյություն ունի $U$ բաց ենթաբազմություն, որ $x \in U \subset V$։ Համաձայն թեորեմ 1-ի՝ $U$-ն շրջակայք է իր ցանկացած $y$ կետի համար։ Ուստի $U \in S_y$ կամայական $y \in U $ կետի դեպքում։
\end{proof}
\par Այս 1-4 հատկությունները լիովին բնութագրում են կետերի շրջակայքերը տոպո\-լո\-գիա\-կան տարածություններում այն իմաստով, որ կարող են դիտարկվել որպես աքսիոմներ՝ բազմության վրա տոպոլոգիայի այլընտ\-րանքա\-յին սահմանման համար: Նախ $(X, \tau )$  տարածության ամեն մի $x\in X$ կետի համար նշանակենք $S_x$-ով այդ կետի բոլոր շրջակայքերի բազմությունը։ 
\begin{theorem}[(շրջակայքերի միջոցով տոպոլոգիայի տրման մասին)]\label{թեորեմ 3}
Դիցուք $X$-ը կա\-մա\-յա\-կան բազմություն է, և դիցուք ամեն մի $x\in X$ տարրի ինչ-որ եղանակով համադրված է $X$-ի ենթաբազմությունների մի ինչ-որ $\hat{S}_x$ ընտանիք այնպես, որ տեղի ունեն ստորին բերվող $1^{\ast}$-$4^{\ast}$ հատկությունները (աքսիոմները)․
\begin{enumerate}
    \item[1$^*$.]  $x$-ը պատկանում է $\hat{S}_x$-ի բոլոր տարրերին,

    \item[2$^*$.]  եթե $V\in \hat{S}_x$ և $V\subset W$, ապա $W$-ն ևս պատկանում է $\hat{S}_x$-ին,

    \item[3$^*$.]  $\hat{S}_x$-ի ցանկացած վերջավոր տարրերի հատումը նույնպես պատկանում է \linebreak $\hat{S}_x$-ին,

    \item[4$^*$.]  ցանկացած $V\in \hat{S}_x$ տարրի համար գոյություն ունի այնպիսի $U\in \hat{S}_x$ տարր, որ $U\subset V$ և $U\in \hat{S}_y$ ցանկացած $y\in U$ տարրի դեպքում։
\end{enumerate}
Ապա $X$ բազմության վրա գոյություն ունի միակ այնպիսի $\tau$ տոպոլոգիա, որ $(X, \tau )$ տոպոլոգիական տարա\-ծութ\-յունում ցանկացած կետի շրջակայքերի $S_x$ ընտանիքը համընկ\-նում ՝ ենթաբազմութ\-յուն\-ների $\hat{S}_x$ ընտանիքի հետ։
\end{theorem}
\begin{proof}
Սահմանենք $X$-ում ենթաբազմությունների $\tau$ ընտանիք կազմված $\varnothing$-ից  և այն բոլոր $U$ ենթաբազմություններից, որ $U$-ն պատկանում է $\hat{S}_x$ ընտանիքին  ամեն մի $x\in U$ տարրի դեպքում։ Այժմ 1$^{\ast}$ և 2$^{\ast}$ աքսիոմներից հետևում է, որ ինքը՝ \linebreak $X$-ը պատկանում է բոլոր $\hat{S}_x$ ընտանիքներին, ուստի $X\in \tau$։ Այսինքն $\tau$-ն բավարա\-րում է տոպոլոգիայի առաջին աքսիոմին։

Ցույց տանք, որ $\tau$-ն բավարարում է նաև տոպոլոգիայի երկրորդ աքսիոմին։ Դիցուք $U = \bigcup\limits_i U_i$, որտեղ $U_i \in \tau$, $i\in I$,  իսկ  $I$-ն ինդեքսների որևէ բազմություն է։ Կամայական $x_0 \in U$ տարրի համար գոյություն ունի $U_{i_{0}}$, որ $x_0 \in U_{i_{0}}$, $i_0\in I$։ Այժմ $\tau$ ընտանիքի սահմանումից հետևում է, որ $U_{i_{0}}\in \hat{S}_{x_{0}}$, և քանի որ $U_{i_{0}}\subset U$, ուստի $U$-ն նույնպես պատկանում է $\hat{S}_{x_{0}}$ ընտանիքին՝ ըստ 2$^\ast$ աքսիոմի։ Հետևաբար $U\in \tau$ ըստ $\tau$ ընտանիքի սահմանման։ Այսպիսով $\tau$-ն բավարարում է տոպոլոգիայի երկրորդ աքսիոմին։

Ստուգենք տոպոլոգիայի երրորդ աքսիոմը․ դիցուք ${U = U_1 \cap U_2}$, որտեղ \linebreak ${U_1, U_2 \in \tau}$, և ցույց տանք, որ $U\in\tau$։ Կամայական $x_0 \in U$ տարրի դեպքում ունենք $x_0 \in U_1$, $x_0 \in U_2$, և ունենք նաև $U_1\in \hat{S}_{x_{0}}$, $U_{2}\in \hat{S}_{x_{0}}$ (ըստ $\tau$ ընտանիքի սահմանման)։ Ուստի $U\in \hat{S}_{x_{0}}$ ըստ 3$^\ast$ աքսիոմի, հետևաբար $U\in \tau$ (ըստ $\tau$ ընտանիքի սահմանման)։ Ուրեմն $\tau$-ն բավարարում է նաև տոպոլոգիայի երրորդ աքսիոմին։

Այժմ ամեն մի ${x\in X}$ կետի համար դիտարկենք $x$-ի բոլոր շրջակայքերի $S_x$ ընտա\-նիքը $(X, \tau)$ տոպոլոգիական տարածությունում և ցույց տանք, որ տեղի ունի $S_x = \hat{S}_x$ համընկում։ 

Եթե $V\in S_x$, այսինքն $V$-ն $x$ կետի որևէ շրջակայք է $(X, \tau)$ տարածությունում, ապա գոյություն ունի $U$ բաց ենթաբազմություն, որ ${x\in U \subset V}$։ Քանի որ $U\in \hat{S}_{x_{0}}$, ուստի աքսիոմ 2$^\ast$-ից ստանում ենք՝ ${V \in \hat{S}_{x_{0}}}$, և ուրեմն ${S_x \subset \hat{S}_{x_{0}}}$։ Հակառակ՝ ${\hat{S}_{x_{0}} \subset S_x}$ ներդրումը հաստատելու համար դիտարկենք $X$-ի կամայական $V\in \hat{S}_{x_{0}}$ ենթաբազմություն։ Ըստ 4$^\ast$ աքսիոմի՝ գոյություն ունի $X$-ի $U\in \hat{S}_{x_{0}}$ ենթաբազմու\-թյուն, որ $U\subset V$, և $U$-ն պատկանում է $\hat{S}_{y}$ ընտանիքին ամեն մի $y\in U$ տարրի դեպքում։ Նշանակում է՝ $U\in \tau$ ըստ $\tau$ ընտանիքի սահմանման։ Այսպիսով ունենք՝ $x\in U \subset V,$ ուրեմն $V$-ն $x$ կետի շրջակայք է $\tau$ տոպոլոգիայում, ուստի $U \in S_x$։ Հետևաբար $\hat{S}_{x_{0}} \subset S_x$, և ունենք $S_x = \hat{S}_{x_{0}}$ համընկում։
        
Վերջապես նկատենք, որ $\tau$ տոպոլոգիայի միակությունը հետևանք է հետևյալ դիտողությունից. եթե $X$ բազմության որևէ երկու տոպոլոգիայում ցանկացած $x$ կետի շրջակայքերը համընկնում են, ապա այդ տոպոլոգիաները նույնական են։
\end{proof}


\newpage
\bigskip
\bigskip
\subsubsection*{Խնդիրներ և հարցեր թեմա 4-ի վերաբերյալ}

\begin{enumerate}[label=\thesection.\arabic*.]
    \item Ապացուցեք, որ թվային ուղղի ամեն մի ինտերվալ կարող է ներկայացվել որպես $[a, b]$ տեսքի հատվածների միավորում։
    
    \item Ապացուցեք, որ թվային ուղղի վերջավոր քանակով կամայական ինտերվալ\-ների հատումը կամ դատարկ է, կամ ինտերվալ է։
    
    \item Ապացուցեք, որ թվային ուղղի անվերջ քանակով կամայական ինտերվալների հատումը կարող է լինել $ \varnothing$, $\{a\}$,  $(a,b)$, $[a,b)$, $(a,b]$, $[a,b]$ տեսքերի ենթաբազմու\-թյուններից որևէ մեկը։ Այդ 6 տեսքերից յուրաքանչյուրի համար կառուցեք այդ\-պիսի հատումների մեկական օրինակ։
    
    \item Դիտարկենք երեք տարրից կազմված որևէ $X=\{a,b,c\}$ բազմություն և նրա ենթաբազմությունների հետևյալ ընտանիքները․
    \begin{align*}
    &\Phi_1 = \{ \varnothing, \{c\}, \{a,b\},X \}, \\
    &\Phi_2 = \{ \varnothing, \{a\}, \{b\},\{a,c\},\{b,c\},X \}, \\ 
    &\Phi_3 = \{ \varnothing, \{a\}, \{b\},\{a,b,c\}\},\\ &\Phi_4 = \{ \varnothing, \{c\}, \{a,c\},\{a,b,c\} \}:
    \end{align*}

    \begin{enumerate}
        \item[ա)] Դրանցից որո՞նք են որոշում $X$ բազմության տոպոլոգիա։
        
        \item[բ)] Գրեք $X$-ի վրա իրարից տարբեր բոլոր տոպոլոգիաները։
    \end{enumerate}


    \item Որոշու՞մ է արդյոք տոպոլոգիա բոլոր բնական թվերի $\N$ բազմության վրա նրա ենթաբազմությունների հետևյալ ընտանիքը․
        \begin{enumerate}
        \item[ա)] $\{\varnothing, \{V_n \mid V_n \subset \N,\ n \geq 1\}\}$, որտեղ $V_n=\{n,n+1,\dots\}$,
        
        \item[բ)] $\{\N, \{W_n \mid W_n \subset \N,\ n \geq 1\}\}$, որտեղ $W_n=\{x\mid x \in \N \text{ և } x<n\}$:
    \end{enumerate}
    
    \item Ճի՞շտ է արդյոք, որ թվային ուղղի սովորական տոպոլոգիայում ամեն մի ոչ դատարկ բաց ենթաբազմություն կարող է ներկայացվել որպես $[a,b]$ տեսքի հատվածների միավորում։
    
    \item Ապացուցեք, որ $\R$ թվային ուղղի ենթաբազմությունների հետևյալ համա\-խմբու\-թյուններից ոչ մեկը տոպոլոգիա չէ $\R$-ի համար․
    \begin{enumerate}
        \item[ա)] $\{\varnothing,\ \R,\ \{(-\infty,x]\text{, որտեղ $x$-ը կամայական թիվ է $\R$-ում}\}$,
        
        \item[բ)] $\{\varnothing,\ \R,\ \{(a, b)\text{, որտեղ $a, b\in\R$ կամայական են և } a<b\}\}$:
    \end{enumerate}
    
    \item $\R$ թվային ուղղի ենթաբազմությունների հետևյալ ընտանիքներից որո՞նք են որոշում տոպոլոգիա $\R$-ի վրա․
    \begin{enumerate}
        \item[ա)] $\{\varnothing,\ \R,\ \{(-\infty,x) \mid x\in\R \text{ կամայական թիվ է}\}\}$,
        
        \item[բ)] $\{\varnothing,\ \R,\ \{(-\infty,x) \mid x\in\Q \text{ կամայական ռացիոնալ թիվ է}\}\}$,

        \item[գ)] $\{\varnothing,\ \{(-\infty,-x]\cup(x,+\infty) \mid x\ge 0 \text{ կամայական թիվ է}\}\}$,

        \item[դ)] $\{\varnothing,\ \R,\ \{[-x,x) \mid x>0 \text{ կամայական իռացիոնալ թիվ է}\}$։
    \end{enumerate}

    \item Դիցուք $X$-ը որևէ ոչ հաշվելի բազմություն է։ Ապացուցեք, որ օրինակ 5-ում սահմանված ենթաբազմությունների $\tau$ ընտանիքը որոշում է տոպոլոգիա $X$-ի վրա։
    
    \item Ապացուցեք, որ $\R$ թվային ուղղի հաշվելի լրացումների տոպոլոգիան ուժեղ է $\R$-ի վերջավոր լրացումների տոպոլոգիայից։
    
    \item Ապացուցեք, որ $\R$ թվային ուղղի հաշվելի լրացումների տոպոլոգիան համեմա\-տելի չէ $\R$-ի սովորական տոպոլոգիայի հետ։
    
    \item Հանդիսանու՞մ է արդյոք շրջակայք ուղղի $\sqrt{2}$ կետի համար բոլոր իռացիոնալ թվերից կազմված ենթաբազմությունը
        \begin{enumerate}
        \item[ա)] ($\R$, սովոր․) տարածությունում,
        
        \item[բ)] ($\R$, վերջ․ լր․) տարածությունում,
        
        \item[գ)] ($\R$, հաշվ․ լր․) տարածությունում։
    \end{enumerate}
\end{enumerate}


\end{document}
