% \section{Խնդիրներ և հարցեր թեմա 1-ի վերաբերյալ}

% \subparagraph{1.1}  Կամայական $A$, $B$, $C$ բազմությունների համար ապացուցեք հետևյալ նույնությունները.
% \begin{enumerate}
%     \item[ա)] $A \cup B = A \cup (B \setminus A)$
%     \item[բ)] $(A \cup B) \setminus C = (A \setminus B)\cup(B \setminus C)$
%     \item[գ)] $A \setminus(B \cup C)  = (A \setminus B) \setminus C$
%     \item[դ)] $A \setminus (B \setminus C) = (A \setminus B) \cup (A \cap B)$
% \end{enumerate}

% \subparagraph{1.2}  Ցույց տվեք, որ \hyperref[օրինակ 6]{օրինակ 6}-ում ուղիղների $L$ ընտանիքը կարելի է ուղղակի ինդեքսավորել նաև $[0, 2\pi)$ միջակայքի թվերով։

% \subparagraph{1.3}  Դիտարկենք Էվկլիդյան կոորդինատային $\R^3$ տարածության սկզբնակետով անցնող բոլոր հարթությունների ընտանիքը: Այս ընտանիքի համար գտեք որևէ ինդեքսավորող բազմություն։

% \subparagraph{1.4}  Ապացուցել դե Մորգանի \eqref{eq:1.1} բանաձևերից առաջինը։

% \subparagraph{1.5}  \hyperref[թեորեմ 1]{Թեորեմ 1}-ի բերված ապացույցը կազմված է որպես $4$ հետևությունների հաջորդականություն։ Պարզեք՝ դրանցից ո՞րը չի հակադարձվում։

% \subparagraph{1.6}  Օգտվելով \hyperref[օրինակ 7]{օրինակ 7}-ում սահմանված $f: X \rightarrow Y$ արտապատկերումից և $X$-ում ընտրելով $X_1$ և $X_2$ հարմար ենթաբազմություններ՝ ցույց տվեք, որ ընդհանուր դեպքում \hyperref[թեորեմ 1]{թեորեմ 1}-ի $f(X_1 \cap X_2) \subset f(X_1) \cap (X_2)$ ներդրումը չի վերածվում աջ և ձախ մասերի համընկման։

% \subparagraph{1.7}  Ապացուցեք \hyperref[թեորեմ 1]{թեորեմ 1}-ի երեք նույնություններից առաջինը և երրորդը։

% \subparagraph{1.8}  Ապացուցեք \hyperref[թեորեմ 2]{թեորեմ 2}-ի երկրորդ և երրորդ նույնությունները։

% \subparagraph{1.9}  \hyperref[թեորեմ 3]{Թեորեմ 3}-ի երկու պնդումների ապացույցներում ո՞ր հետևումներն են, որ չեն հակադարձվում։

% \subparagraph{1.10}  Դիտարկենք երկու հարց կապված \hyperref[թեորեմ 3]{թեորեմ 3}-ի հետ․ ինչպիսի՞ն պետք է լինի $f:X \rightarrow Y$ արտապատկերումը, որ տեղի ունենա
% \begin{enumerate}
%     \item[ա)] $f^{-1}(f(A))=A$ համընկում ցանկացած $A \subset X$ ենթաբազմության դեպքում,
    
%     \item[բ)] $f(f^{-1}(B))=B$ համընկում ցանկացած $B \subset Y$ ենթաբազմության դեպքում։
% \end{enumerate}

% Ապացուցեք, որ ա) դեպքում անհրաժեշտ և բավարար պայման է $f$ արտապատկերման ինյեկտիվությունը, իսկ բ) դեպքում՝ $f$-ի սյուրյեկտիվությունը։

% \subparagraph{1.11}  Նշանակենք $h$-ով $f:X \rightarrow Y$, $g:Y \rightarrow Z$ արտապատկերումների $g \circ f: X \rightarrow Z$ համադրույթը՝ $h=g \circ f$։ Ապացուցեք, որ ամեն մի $T \subset Z$ ենթաբազմության դեպքում $h^{-1}(T)=f^{-1}(g^{-1}(T))$։

% \subparagraph{1.12}  Դիցուք $f:X \rightarrow Y$ և $g:Y \rightarrow Z$ արտապատկերումներն այնպիսին են, որ $g\circ f = \nuynakan_X$։ Ապացուցեք, որ $f$-ը ներդիր, $g$-ն վրադիր արտապատկերումներ են։

% \subparagraph{1.13}  Դիցուք $f:X \rightarrow Y$ և $g:Y \rightarrow X$ արտապատկերումներն այնպիսին են, որ $f\circ g = \nuynakan_Y$, $g\circ f = \nuynakan_X$։ Ապացուցեք, որ $f$-ը և $g$-ն փոխմիարժեք, մեկը մյուսին հակադարձ արտապատկերումներ են։

\bigskip
\bigskip
\subsubsection*{Խնդիրներ և հարցեր թեմա 1-ի վերաբերյալ}

\begin{enumerate}[label=\thesection.\arabic*.]
\item  Կամայական $A$, $B$, $C$ բազմությունների համար ապացուցեք հետևյալ նույ\-նու\-թյուն\-ները.
\begin{enumerate}
    \item[ա)] $A \cup B = A \cup (B \setminus A)$
    \item[բ)] $(A \cup B) \setminus C = (A \setminus B)\cup(B \setminus C)$
    \item[գ)] $A \setminus(B \cup C)  = (A \setminus B) \setminus C$
    \item[դ)] $A \setminus (B \setminus C) = (A \setminus B) \cup (A \cap B)$
\end{enumerate}

\item  Ցույց տվեք, որ \hyperref[օրինակ 1:6]{օրինակ 6}-ում ուղիղների $L$ ընտանիքը կարելի է ուղղակի ինդեքսավորել նաև $[0, 2\pi)$ միջակայքի թվերով։

\item  Դիտարկենք էվկլիդյան կոորդինատային $\R^3$ տարածության սկզբնակետով \linebreak անցնող բոլոր հարթությունների ընտանիքը: Այս ընտանիքի համար գտեք որևէ ինդեքսավորող բազմություն։

\item  Ապացուցեք դե Մորգանի \eqref{eq:1.1} բանաձևերից առաջինը։

\item  \hyperref[թեորեմ 1:1]{Թեորեմ 1}-ի բերված ապացույցը կազմված է որպես $4$ հետևությունների հա\-ջոր\-դա\-կա\-նու\-թյուն։ Պարզեք՝ դրանցից ո՞րը չի հակադարձվում։

\item  Օգտվելով \hyperref[օրինակ 1:7]{օրինակ 7}-ում սահմանված $f: X \rightarrow Y$ արտապատկերումից և $X$-ում ընտրելով $X_1$ և $X_2$ հարմար ենթաբազմություններ՝ ցույց տվեք, որ ընդհանուր դեպքում \hyperref[թեորեմ 1:1]{թեորեմ 1}-ի $f(X_1 \cap X_2) \subset f(X_1) \cap (X_2)$ ներդրումը չի վերածվում աջ և ձախ մասերի համընկման։

\item  Ապացուցեք \hyperref[թեորեմ 1:1]{թեորեմ 1}-ի երեք նույնություններից առաջինը և երրորդը։

\item  Ապացուցեք \hyperref[թեորեմ 1:2]{թեորեմ 2}-ի երկրորդ և երրորդ նույնությունները։

\item  \hyperref[թեորեմ 1:3]{Թեորեմ 3}-ի երկու պնդումների ապացույցներում ո՞ր հետևումներն են, որ չեն հակադարձվում։

\item  Ապացուցեք \eqref{eq:1.2} նույնությունները։

\item  Դիտարկենք երկու հարց կապված \hyperref[թեորեմ 1:3]{թեորեմ 3}-ի հետ․ ինչպիսի՞ն պետք է լինի $f:X \rightarrow Y$ արտապատկերումը, որ տեղի ունենա
\begin{enumerate}
    \item[ա)] $f^{-1}(f(A))=A$ համընկում ցանկացած $A \subset X$ ենթաբազմության դեպքում,
    
    \item[բ)] $f(f^{-1}(B))=B$ համընկում ցանկացած $B \subset Y$ ենթաբազմության դեպքում։
\end{enumerate}

Ապացուցեք, որ ա) դեպքում անհրաժեշտ և բավարար պայման է $f$ արտա\-պատ\-կեր\-ման ինյեկտիվությունը, իսկ բ) դեպքում՝ $f$-ի սյուրյեկտիվությունը։

\item  Նշանակենք $h$-ով $f:X \rightarrow Y$, $g:Y \rightarrow Z$ արտապատկերումների $g \circ f: X \rightarrow Z$ համադրույթը՝ $h=g \circ f$։ Ապացուցեք, որ ամեն մի $T \subset Z$ ենթաբազմության դեպքում $h^{-1}(T)=f^{-1}(g^{-1}(T))$։

\item  Դիցուք $f:X \rightarrow Y$ և $g:Y \rightarrow Z$ արտապատկերումներն այնպիսին են, որ $g\circ f = \nuynakan_X$։ Ապացուցեք, որ $f$-ը ներդիր, $g$-ն վրադիր արտապատկերումներ են։

\item  Դիցուք $f:X \rightarrow Y$ և $g:Y \rightarrow X$ արտապատկերումներն այնպիսին են, որ $f\circ g = \nuynakan_Y$, $g\circ f = \nuynakan_X$։ Ապացուցեք, որ $f$-ը և $g$-ն փոխմիարժեք, մեկը մյուսին հակադարձ արտապատկերումներ են։
\end{enumerate}
