% Զուտ թեմա երկու խնդիրներն եմ քոփի արել ու համարյա բան չեմ փոխել
\bigskip
\bigskip
\subsubsection*{Խնդիրներ և հարցեր թեմա 3-ի վերաբերյալ}

\begin{enumerate}[label=\thesection.\arabic*.]
% 3.1
\item Թեմա $1$-ի օրինակ $9$-ի նմանությամբ կառուցեք փոխմիարժեք $h$ արտապատ\-կերում $\{(x,y)| x^2 + y^2 < 1\}$ անեզր շրջանի և $\mathbb{R}^2$ հարթության կետերի միջև։ 

\begin{hint}
    Տեղադրելով $1$ շառավղով անեզր կիսասֆերան այնպես, որ շոշափի $\mathbb{R}^2$ հարթությունը $O$ կետում, նախ կառուցեք փոխմիարժեք համապատասխանություն շրջանի և կիսասֆերայի միջև, և ապա՝ կիսասֆերայի և հարթության կետերի միջև՝ ինչպես ցույց է տրված նկարում։
\end{hint}

% \textbf{Ցուցում։} Տեղադրեք $1$ շառավղով անեզր կիսասֆերան այնպես, որ շոշափի $/mathbb{R}^2$ հարթությունը $O$ կետում։
\begin{figure}
    \centering
    \includegraphics[width=0.5\linewidth]{images/Screenshot 2024-10-03 173106.png}
    \caption{Խնդիր 3.1}
    % \label{fig:enter-label}
\end{figure}

% 3.2
\item Ապացուցեք հետևյալ հավասարազորությունները․
\[
(0,1) \sim (0,1] \sim [0,1) \sim [0,1]
\]

\begin{hint}
    Նախ այդ բազմությունները ներկայացրեք որպես միայն ռացիոնալ թվերից և միայն իռացիոնալ թվերից կազմված երկու ենթաբազմությունների միավորում։ Օրինակ՝ $(0,1) = \mathbb{Q} \cup I$ և $[0,1]=(\mathbb{Q} \cup {0,1}) \cup I$, որտեղ $\mathbb{Q}$-ն և $\mathbb{I}$-ն $(0,1)$ ինտերվալի ռացիոնալ և իռացիոնալ թվերի բազմություններն են։ Այժմ կարող ենք հաստատել $(0,1) \sim [0;1]$ հավասարազորություն՝ լրացնելով $\mathbb{Q} \sim \mathbb{Q} \cup {0,1}$ հավասարազորությունը (որը տեղի ունի ըստ թեորեմ $3$-ի) նույնական $I \rightarrow I$ արտապատկերումով։ \red{vstah chem vor petq er es mathb-n hanel I-ic, mi ban bayc cher havanel hastat}
\end{hint}

% \textbf{Ցուցում:} Նախ այդ բազմությունները ներկայացրեք որպես միայն ռացիոնալ թվերից և միայն իռացիոնալ թվերից կազմված երկու ենթաբազմությունների միավորում։ Օրինակ, $(0;1) = \mathbb{Q} \cup \mathbb{I}$ և $[0;1]=(\mathbb{Q} \cup {0;1}) \cup \mathbb{I}$, որտեղ $\mathbb{Q}$-ն և $\mathbb{I}$-ն $(0;1)$ ինտերվալի ռացիոնալ և իռացիոնալ թվերի բազմություններն են։ Այժմ կարող ենք հաստատել $(0;1) \sim [0;1]$  լրացնելով $\mathbb{Q} \sim \mathbb{Q} \cup {0,1}$ հավասարազորությունը (որը տեղի ունի ըստ թեորեմ $3$-ի) նույնական $\mathbb{I} \rightarrow \mathbb{I}$ արտապատկերումով։

% 3.3
\item Ապացուցեք, որ թվային ուղղի $(-\infty, a), (-\infty,b], (c, +\infty), [d, +\infty], , (a,b), [c,d], [m,n), [p,q] $ տեսքի ցանկացած երկու ենթաբազմություն հավասարազոր են։

\begin{hint}
    Դիտարկենք հետևյալ արտապատկերումները․
\begin{enumerate}
    \item[ա)] $f_1 : (a,b) \rightarrow (0,1), f_2 : [a,b] \rightarrow [0,1], f_3 : (a,b] \rightarrow (0;1], f_4 : [a,b) \rightarrow [0;1)$, որոնք սահմանվում են $x \mapsto \dfrac{x-a}{b-a}$ համադրումով,
    \item[բ)]  $g_1 : (-\infty; 0) \rightarrow (0,1), g_2 : (-\infty; 0] \rightarrow (0,1]$, որոնք սահմանվում են $x \mapsto \frac{1}{1-x}$ համադրումով,
    \item[գ)]  $h_1 : (0; + \infty) \rightarrow  (0,1), h_2 : [0; \infty) \rightarrow  [0,1]$ որոնք սահմանվում են $x \mapsto \dfrac{x}{x+1}$ համա\-դրումով,
    \item[դ)]  $u_1 : (- \infty; a) \rightarrow (- \infty; 0 ), u_2 : (-\infty; a] \rightarrow (-\infty; 0]$, որոնք սահմանվում են $x \mapsto x-a $ համադրումով,
    \item[ե)] $v_1 : [a;+ \infty) \rightarrow ( 0;+ \infty), u_2 : ( a;+\infty] \rightarrow ( 0;+\infty]$, որոնք սահմանվում են $x \mapsto x-a $ համադրումով,
    \item[զ)] $w : ( - \infty;0) \rightarrow ( 0;+ \infty)$, սահմանվում է՝ $w(x) = -x$: 

\end{enumerate}

Հիմնավորելով սրանցից յուրաքանչյուրի փոխմիարժեքությունը՝ օգտվեք դրանց համադրույթներից և խնդիր $3.2$-ից։
\end{hint}

% 3.4
\item Ապացուցեք, որ հաշվելի $A$ բազմությունից նրա որևէ $B$ վերջավոր ենթաբազ\-մություն անջատելիս ստացված $A\setminus B$ մնացորդը դարձյալ հաշվելի բազմու\-թյուն է։

\begin{hint}
     Դիտարկեք $A$-ի տարրերի որևէ ${a_1, a_2, \dots}$ համարակալում և երկու դեպք՝ $A$-ն վերջավոր է, $A$-ն անվերջ է։ Երկրորդ դեպքում օգտվեք թեորեմ $1$-ից \red{do the ref magic here bitte}։ 
\end{hint}

% 3.5
\item Ապացուցեք. եթե $A_1, A_2, \dots, A_n$, որտեղ $n \geq 1$, բազմությունները հաշվելի են, ապա նրանց $A_1 \times A_2 \times \dots \times A_n$ ուղիղ արտադրյալը նույնպես հաշվելի բազմություն է։

\begin{hint}
    Դիտարկեք  $A_1 \times A_2 \times \dots \times A_n = \bigcup\limits_{i} A_1 \times A_2 \times \dots \times A_{n-1} \times \{ b_{i} \} $  ներկայացումը, որտեղ $A_n = \{b_1, b_2, \dots \}$, և օգտվելով թեորեմ $3$-ից՝ կիրառեք ինդուկցիա ըստ $n$-ի։
\end{hint}
 
% 3.6
\item Ապացուցեք․ հաշվելի բազմության տարրերով կազմված բոլոր վերջավոր հա\-ջոր\-դականությունների բազմությունը հաշվելի է։

\begin{hint}
    Հաշվելի $A = \{a_1, a_2, \dots \}$ բազմության տարրերով կազմված $n$ երկարության ամեն մի $a_{i_{1}}, a_{i_{2}}, \dots, a_{i_{n}}$ \red{asum a indexnery mecacnel, bayc axer ay dzer cavy tanem, vonc, petq a a-ern el hety mecacnel uremn, esim e, esim} հաջորդականություն կարելի է դիտել որպես $A \times A \times  \dots \times A$ ուղիղ արտադրյալի տարր և հակառակը։ Օգտվելով խնդիր $3.5$-ից՝ կիրառեք ինդուկցիա ըստ $n$-ի։
\end{hint}

% 3.7
\item Ունենք բազմությունների որևէ $f : A \rightarrow B$ սյուրյեկտիվ արտապատկերում։ Ապացուցեք՝ եթե $A$-ն հաշվելի է, ապա $B$-ն նույնպես հաշվելի է։

\begin{hint}
    Ամեն մի $b \in B$ տարրի համար ընտրենք որևէ $a \in f^{-1}(B)$ տարր և սահմանենք $g : B \rightarrow A$, $g(b)=a$ արտապատկերում։
Ցույց տվեք, որ $g$-ն ինյեկտիվ արտապատկերում է, և օգտվեք թեորեմ $1$-ից։
\end{hint}

% 3.8
\item Ապացուցեք, որ հաշվելի $A = \{a_1, a_2, \dots \}$ բազմության բոլոր վերջավոր ենթա\-բազմությունների բազմությունը հաշվելի է։

\begin{hint}
    Տարրերի ամեն մի $a_{i_{1}}, a_{i_{2}}, \dots , a_{i_{n}}$ վերջավոր հաջորդականություն համադրենք $A$ բազմության $\{a_{i_{1}}, a_{i_{2}}, \dots , a_{i_{n}}\}$ ենթաբազմությունը։
Արդյունքում ստանում ենք որոշակի $f$ արտապատկերում $A$ բազմության տարրերով կազմված բոլոր վեր\-ջավոր հաջորդականությունների բազմությունից դեպի $A$-ի բոլոր վերջավոր ենթաբազմություններով կազմված բազմության մեջ ցույց տվեք, որ $f$-ը սյուր\-յեկտիվ արտապատկերում է, և օգտվեք նախորդ երկու խնդիրներից։
\end{hint}

% 3.9
\item Ապացուցեք․ եթե $A$-ն անվերջ բազմություն է, իսկ $B$-ն հաշվելի բազմություն է, ապա $A \cup B$-ն հավասարազոր է $A$-ին։

\begin{hint}
    Օգտվեք հետևյալ երկու պնդումներից՝ հիմնավորելով դրանք․
\begin{enumerate}
    \item [ա)] եթե $A_1$-ը $A$ բազմության որևէ հաշվելի ենթաբազմություն է, ապա տեղի ունի $A_1 \cup B \sim A_1$ հավասարազորությունը,
    \item [բ)] $A \cup B=\left(A \setminus A_1\right) \cup\left(A_1 \cup B\right) \sim \left(A \setminus A_1\right) \cup A_1 = A_1$։
\end{enumerate}
\end{hint}


% 3.10
\item Ապացուցեք․ եթե $A$-ն անվերջ և ոչ հաշվելի բազմություն է, իսկ $B$-ն հաշվելի է, ապա $A \setminus B \sim A$։

\begin{hint}
    Օգտվեք $A=(A \setminus B) \cup B$ նույնությունից և նախորդ խնդրից։
\end{hint}

% 3.11
\item Ի՞նչ հզորություն ունի բոլոր իռացիոնալ թվերի բազմությունը։ 

\begin{hint}
    Օգտվեք խնդիր $3.9$-ից \red{screw the ref-s, ageed? pls} ՝ վերցնելով որպես $A$ բոլոր իռացիոնալ թվե\-րի, իսկ որպես $B$՝ բոլոր ռացիոնալ թվերի բազմությունները։
\end{hint}


% 3.12
\item Ճի՞շտ է արդյոք, որ բոլոր իռացիոնալ թվերի բազմությունը և $(-1, 1)$ ին\-տեր\-վալի իռացիոնալ թվերի բազմությունը ունեն նույն հզորությունը։
\end{enumerate}
