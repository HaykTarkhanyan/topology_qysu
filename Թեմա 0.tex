\documentclass[./main.tex]{subfiles}

\begin{document}
\onehalfspacing
\section{Տոպոլոգիան առաջին հայացքից․ տոպոլոգիական հատկություններ և տոպոլոգիական ինվարիանտներ, տոպոլոգիան որպես աքսիոմատիկ տեսություն։ Տարբեր հարթաչափական երկրաչափություններ․ պատկերների հավասարություն, թե՞ համարժեքություն։ Երկրաչափությունները և դրանց հիմքում ընկած ձևափոխությունների խմբերը, տոպոլոգիան որպես սահմանային երկրաչափություն։}\label{sec:0}
\hyphenation{տո-պո-լո-գիա ձևափո-խութ-յուն-ների տո-պո-լո-գիա-կան տա-րա-ծութ-յուն-ներ հա-մա-պա-տաս-խա-նութ-յուն ու-ռու-ցի-կութ-յուն երկրա-չա-փութ-յունը հաս-կա-ցութ-յուն-ները բազ-մութ-յուն-ների տա-րա-ծութ-յուն-ները նպա-տա-կա-հար-մա-րութ-յան այ-նուամե-նայ-նիվ հա-մա-պա-տաս-խա-նա-բար} 
Տոպոլոգիան երկրաչափության բաժին է, որն ուսումնասիրում է պատկերների այն հատկությունները, որոնք պահպանվում են պատկերների անընդհատ ձևափոխությունների դեպքում։ Այդպիսի հատկությունները կոչվում են պատկերների տոպոլոգիական հատկություններ։
\par  Երբ ասում են, որ $X$ երկրաչափական պատկերը ստացվում է $Y$ երկրաչափական պատկերի անընդհատ ձևափոխությունով, ապա ենթադրվում է, որ
\begin{enumerate}
\item հաստատվում է որոշակի $h$ փոխմիարժեք (մեկը մեկին) համապատասխանություն $X$ և $Y$ պատկերների միջև (մասնավորապես սա նշանակում է, որ $X$ և $Y$ պատկերների կետերի քանակությունները պիտի լինեն նույնը);
\item եթե $X$ պատկերի որևէ երկու $x_1$ և $x_2$ կետեր բավականաչափ մոտիկ են միմյանց $X$-ում, ապա $h$-ի միջոցով նրանց համապատասխանեցված $y_1$ և $y_2$ կետերը նույնպես պետք է լինեն բավականաչափ մոտիկ $Y$-ում և հակառակը։
\end{enumerate}
Տոպոլոգիայում երկրաչափական պատկերները կոչվում են տոպոլոգիական տարածություններ, իսկ 1-2 պայմաններին բավարարող տոպոլոգիական տարածությունները կոչվում են միմյանց տոպոլոգորեն համարժեք կամ հոմեոմորֆ տարածություններ։ Այս դեպքում $h$ փոխմիարժեք համապատասխանությունը կոչվում է հոմեոմորֆիզմ $X$-ի և $Y$-ի միջև։
\par Բերենք հոմեոմորֆիզմների, հոմեոմորֆ և ոչ հոմեոմորֆ տարածությունների մի քանի հասարակ օրինակներ։

\begin{example}
Պարզ է, որ մի կետից կազմված որևէ $\{x\}$ պատկեր չի կարող լինել հոմեոմորֆ որևէ $[a;b]$ հատվածի, քանի որ վերջինս կազմված է անթիվ քանակով կետերից։
\par
Հարց է առաջանում․ իսկ հոմեոմո՞րֆ են արդյոք որևէ երկու $[a;b]$ և $[c;d]$ հատվածներ։ Մասնավոր դեպքում, երբ օրինակ $[a,b]=[1;3],\ [c;d]=[1;5]$, ունենք $[1;3]\subset[1;5]$ ներդրում, որից կարելի է եզրակացնել, որ $[1;3]$ հատվածում կետերն ավելի քիչ են $[1;5]$ ինտերվալի կետերից, ուստի դրանք հոմեոմորֆ չեն։ Բայց մյուս կողմից հեշտ է տեսնել, որ $h(x)=\dfrac{d-c}{b-a}(x-a)+c$ գծային ֆունկցիան փոխմիարժեք և անընդհատ $[a;b]$ հատվածը ձևափոխում է $[c;d]$ հատվածին։ Հետևաբար $[a;b]$ և $[c;d]$ հատվածները հոմեոմորֆ են։
\par Այս օրինակը ուսանելի է նրանով, որ ցույց է տալիս․ բազմության կետերի քանակություն հասկացությունը կարիք ունի հստակեցման։
\end{example}
% or2
\begin{example}
Ցանկացած երկու ուռուցիկ բազմանկյուն հոմեոմորֆ են միմյանց։ Իրոք, նկար 1-ում պատկերված քառանկյուն և վեցանկյուն բազմանկյունների դեպքում $O$ կետից դուրս եկող ճառագայթները որոշում են $x\mapsto y $ և $y\mapsto x$ փոխմիարժեք անընդհատ համապատասխանություններ այդ բազմանկյունների կետերի միջև։ Հեշտ է նաև ցույց տալ, որ հոմեոմորֆ են նկար 2-ում պատկերված
\begin{center}
\includegraphics[scale=1]{images/id 01.PNG}
\end{center}
շրջանագիծն ու հնգանկյուն բազմանկյունը։ Այս դեպքում կարելի է օրինակ հոմեոմորֆիզմ կառուցել երկու քայլով․ նախ շրջանագծի $x$ կետին համապատասխանեցնենք օժանդակ շրջանագծի (որն ստացվում է շրջանագծի զուգահեռ տեղափոխումով) $y$ կետը և այնուհետև $y$ կետին համապատասխանեցնենք $O_1 y$ ճառագայթի հատման $z$ կետը բազմանկյան հետ։ Արդյունքում ստանում ենք պահանջվող $x\mapsto z$ փոխմիարժեք անընդհատ համապատասխանություն։
\end{example}
%or 3 
\begin{example}
Դիտարկենք ստորև նկ․ 3-ում պատկերված երկու $L_1$ և $L_2$ գծապատկերներ․
\begin{center}
\includegraphics[scale=1]{images/id 02.PNG}
\end{center}
Դրանք նույնպես հոմեոմորֆ են։ Բայց այս դեպքում հազիվ թե հնարավոր է կառուցել հոմեոմորֆիզմ նախորդ օրինակներում բերված եղանակներով։
\par Նկարագրենք այսօրինակ գծապատկերների միջև հոմեոմորֆիզմ կառուցելու մի ավելի ընդհանուր մեթոդ։ Այդ նպատակով ընտրենք մեկական $x_0, y_0$ կետեր $L_1$-ի, $L_2$-ի վրա և շրջապտույտի մեկական ուղղություններ $L_1$ և $L_2$ գծերի վրա։ 
\par Նշենք որ գիծն, իր վրա ընտրված շրջապտույտի ուղղության հետ միասին կոչվում է \textbf{կողմնորոշված գիծ}։ Յուրաքանչյուր գիծ կարելի է կողմնորոշել երկու եղանակով՝ ընտրելով երկայնքով շարժվելու երկու ուղղություններից մեկնումեկը։
\par Նշանակենք $L_1$ և $L_2$ գծերի երկարությունները $l'$ և $l''$։ Կամայական $x\in L_1$ կետի համար նշանակենք $l'_x$-ով $L_1$ գծի այն ամենակարճ «աղեղի» երկարությունը, որով պետք է շարժվել $x_0$ կետից $L_1$-ի երկայնքով ըստ ընտրված ուղղության, որպեսզի հայտնվենք $x$ կետում։ Նույնպիսի՝ $l''_y$ նշանակում կատարենք նաև $L_2$ գծի կամայական $y$ կետի համար։ Այժմ $L_1$-ի ամեն մի $x$ կետի համապատասխանեցնենք $L_2$ գծի այն միակ $y$ կետը, որի դեպքում $l''_y=\dfrac{l''}{l'}\cdot l'_x$։ Ակնհայտ է, որ $x\mapsto y$ համապատասխանությունը որոշում է հոմեոմորֆիզմ $L_1$-ի և $L_2$-ի միջև։
 \end{example}
\par Բերենք նաև տարածական պատկերների հոմեոմորֆության օրինակներ։
%or 4 
\begin{example}
Նկար 4-ում պատկերված կիսասֆերան շոշափում է հարթության մաս կազմող նույն շառավղով շրջանին նրա կենտրոնում։ Դրանք հոմեոմորֆ պատկերներ են․ կիսասֆերայի 
\begin{center}
\includegraphics[scale=1]{images/id 03.PNG}
\end{center}
յուրաքանչյուր $x$  կետի համապատասխանեցվում է նրա $y$ պրոյեկցիան հարթության վրա և հակառակը՝ շրջանի յուրաքանչյուր $y$ կետի համապատասխանեցվում է կիսասֆերայի այն միակ $x$ կետը, որի պրոյեկցիան $y$-ն է։ Այս համապատասխանությունները որոշում են հոմեոմորֆիզմ նշված պատկերների միջև (նշենք, որ երկու պատկերներն էլ դիտարկվում են կամ իրենց եզրային կետերի հետ միասին, կամ առանց եզրակետերի)։
\par Պարզվում է, որ այդ նույն կիսասֆերան (բայց առանց եզրակետերի) հոմեոմորֆ է հարթությանը։ Այս դեպքում (տես նկ․ 5-ը) կիսասֆերայի յուրաքանչյուր $x$ կետի համապատասխանեցնելով $Ox$ ուղղի հատման $y$ կետը հարթության հետ, ստանում ենք պահանջվող հոմեոմորֆիզմը։
\end{example}
\par Ավարտելով օրինակների քննարկումը, այժմ լուսաբանենք տոպոլոգիական տոպոլոգիական հատկություն, տոպոլոգիական ինվարիանտ հասկացությունները։
\par Երկրաչափական պատկերների որևէ հատկություն, կամ պատկերների հետ առնչվող որևէ թվային մեծություն կոչվում է տոպոլոգիական հատկություն կամ տոպոլոգիական ինվարիանտ մեծություն, եթե այն բանից, որ ինչ-որ պատկեր օժտված  է այդ հատկությամբ կամ այդ մեծությամբ հետևում է, որ այդ հատկությամբ օժտված են նաև նրան հոմեոմորֆ բոլոր պատկերները։
\par Վերը քննարկված օրինակները ցույց են տալիս, որ մեզ ծանոթ երկրաչափական հատկությունների և մեծություններից՝ պատկերի ուղղագծություն, պատկերի ուռուցիկություն, պատկերի սահմանափակություն կամ անսահմանափակություն, գծի երկարություն, պատկերի մակերես, և ոչ մեկը տոպոլոգիական հատկություն կամ տոպոլոգիական մեծություն չէ․ դրանք չեն պահպանվում պատկերների անընդհատ ձևափոխությունների դեպքում։ 

\par Ընթերցողի մոտ կարող է հարց առաջանալ․ իսկ կա՞ն արդյոք պատկերների տոպոլոգիական հատկություններ կամ տոպոլոգիական մեծությունններ, որո՞նք են և որքա՞ն են դրանք։ Այդպիսիք իհարկե կան, և դրանցից որոշների հետ ընթերցողը կծանոթանա դասընթացում։ Այս պահին սահմանափակվենք մի օրինակով․ ստորև նկարում
%nk7  kam nkar anhamar
\begin{center}
\includegraphics[scale=1]{images/id 04.PNG}
\end{center}
ցուցադրված են պատկերներ, որոնք ստացվում են շրջանից հեռացնելով մեկ, երկու և երեք շրջանակներ։ Դրանցից ոչ մի երկուսը միմյանց հոմեոմորֆ չեն․ պարզվում է, որ պատկերի անցքերի քանակությունը տոպոլոգիական (անփոփոխ) մեծություն է։ Այս, գրեթե ակնհայտ թվացող պնդումը իհարկե կարիք ունի խիստ ապացուցման։
\par Անցնենք հաջորդ հարցին․ ինչպիսի՞ ներքին կամ արտաքին նմանություններ ունեն տոպոլոգիան և դպրոցական դասընթացից մեզ ծանոթ էվկլիդեսյան երկրաչափությունը։
\par Ինչպես գիտենք, և՛ հարթաչափությունը, և՛ տարածաչափությունը կառուցվում են աքսիոմատիկ եղանակով։ Էվկլիդեսյան երկրաչափության հիմքում ընկած են կետ, ուղիղ, հարթություն, կետերի միջև հեռավորություն չսահմանվող հասկացությունները։ Փոխարենը սահմանվում են նրանց միջև փոխհարաբերություններ աքսիոմների տեսքով։ Հիշեցնենք հարթաչափության մի քանի այդպիսի աքսիոմներ։
\begin{enumerate}
\item Յուրաքանչյուր ուղղի պատկանում են առնվազն երկու կետեր։

\item Գոյություն ունեն առնվազն երեք կետեր, որոնք չեն գտնվում մի ուղղի վրա։

\item Ցանկացած երկու կետերով անցնում է ուղիղ և այն էլ միայն մեկը։
\end{enumerate}
Աքսիոմներով են ներմուծվում՝ «կետերիկ միջև», ճառագայթ, կիսահարթություն հասկացությունները և այլն։
 \par Այնուհետև սահմանվում են նոր երկրաչափական հասկացություններ, ձևակերպվում և դեդուկտիվ եղանակով ապացուցվում են թեորեմներ, որոնք բացահայտում են այդ հասկացությունների զանազան հատկություններ։ 
\par Տոպոլոգիան որպես երկրաչափության բաժին նույնպես կառուցվում է աքսիոմատիկ եղանակով։ Տոպոլոգիան հիմնված է բազմությունների տեսության վրա։ Այն նույնպես ունի աքսիոմատիկ կառուցվածք, որի հիմքում ընկած է \textbf{բազմության} հասկացությունը։

\par Նշենք, որ բազմություն և բազմության տարր հասկացությունները նախնական, չսահմանվող հասկացություններ են։
Բերենք այդ տեսության մի քանի աքսիոմներ։
\begin{enumerate}

\item \textbf{Ծավալման աքսիոմ․} եթե $A$ և $B$ բազմությունները կազմված են միևնույն տարրերից, ապա նրանքք համընկնում են։
\item \textbf{Գումարի աքսիոմ․} բազմությունների ամեն մի $A$ ընտանիքի համար գոյություն ունի $S$ բազմություն կազմված այն և միայն այն տարրերից, որոնք պատկանում են $A$ ընտանիքի որևէ $X$ բազմությանը։\par
Մեկնաբանելով բազմությունների ընտանիք հասկացությունը նշենք, որ բազմությունների ընտանիքը նույնպես բազմություն է, որի տարրերը բազմություններ են։
\item \textbf{Դատարկ բազմության գոյության աքսիոմ․} գոյություն ունի այնպիսի բազմություն (նշանակվում է $\varnothing$), որ նրան ոչ մի $x$ տարր չի պարունակում։

\item \textbf{Ընտրության աքսիոմ․} ոչ դատարկ, չհատվող բազմությունների ամեն մի $A$ ընտանիքի համար գոյություն ունի $B$ բազմություն, որն ունի ճիշտ մի ընդհանուր տարր $A$-ին պատկանող ամեն մի $X$ բազմության հետ։
\par Այս աքսիոմի անհրաժեշտությունը մեկնաբանելու նպատակով նկատենք, որ ընդհանուր դեպքում գոյություն չունի ժամանակի և տարածության մեջ տեղավորվող որևէ իրական գործնական եղանակ, որով հնարավոր լինի կազմավորել աքսիոմում նշված $B$ բազմությունը։
\end{enumerate}
Վերադառնալով տոպոլոգիայի աքսիոմատիկային՝ նշենք, որ նրա հիմքում ընկած է \textbf{տոպոլոգիական տարածության} հասկացությունը։ Տոպոլոգիական տարածությունը բազմություն է, որի համար, նույնպես աքսիոմներով, սահմանվում է այդ բազմության որոշակի ներքին կառուցվածք։ Այդ կառուցվածքը կոչվում է տվյալ բազմության տոպոլոգիա կամ տոպոլոգիա տվյալ բազմության վրա։
Այսպիսով տոպոլոգիա տերմինը գործածվում է երկու իմաստով․ որպես տվյալ բազմության ներքին կառուցվածք և որպես մաթեմատիկայի բաժին։
\par Ամփոփելով նշենք, որ Տոպոլոգիայի (որպես բաժին) աքսիոմների լրիվ համակարգը կազմված է բազմությունների տեսության աքսիոմներից և տոպոլոգիայի (որպես բազմության ներքին կառուցվածքի) աքսիոմներից։
\par Պատահական չէ, որ բազմությունների տեսության ստեղծման հենց սկզբից նրա հեղինակն ու հետևորդները՝ Գ․ Կանտոր, Ֆ․ Հաուսդորֆ, Մ․ Ֆրեշե, Կ․ Կուրատովսկի, Պ․ Ալեքսանդրով և այլոք, ձեռնարկեցին բազմության տոպոլոգիական կառուցվածքի մշակումը։ Եղան տարբեր մոտեցումներ։
\begin{enumerate}
\item Բազմության տոպոլոգիայի հիմքում դրվեց կետի շրջակայք հասկացությունը իր համապատասխան աքսիոմներով։ Սկզբում տոպոլոգիական տարածությունները այդպես էլ կոչվում էին՝ \textbf{շրջակայքային տարածություններ}։
\item Մյուս դեպքում տոպոլոգիայի հիմքում դրվեցին ենթաբազմության հպման կետ, ենթաբազմության փակում հասկացությունները (Կուրատովսկու աքսիոմատիկա)։
\item Երրորդ դեպքում տոպոլոգիայի հիմքում դրվեց բազմության կետերի միջև հեռավորության հասկացությունը՝ մետրիկան։ Այսպիսի տոպոլոգիական տարածությունները կոչվում են մետրական տարածություններ։

\item Բազմության վրա տոպոլոգիա կարելի է սահմանել նաև ներմուծելով բաց (փակ) ենթաբազմություն հասկացությունը (դարձյալ աքսիոմներով)

\end{enumerate}
Կան նաև այլ մոտեցումներ։ Յուրաքանչյուր դեպքում, ըստ ճաշակի և նպատակահարմարության, ընտրվել և վերացարկվել է սովորական երկրաչափական պատկերների այս կամ այլ հատկությունը։ Ընդ որում, ընդգրկված բոլոր հատկություններն ունեն անմիջական կապ երկրաչափական պատկերների արտապատկերումների անընդհատության հետ։ Նկատենք նաև, որ բոլոր մոտեցումները համարժեք են այն իմաստով, որ բերում են միևնույն արդյունքներին իրենց գործադրությունների տիրույթների ընդհանուր մասերում։
\par Այժմ բացահայտենք էվկլիդեսյան, ինչպես նաև որոշ այլ երկրաչափությունների և տոպոլոգիայի ներքին նմանությունները։ Մասնավորապես պարզաբանենք, թե ինչ տեղ է գրավում տոպոլոգիան այդ երկրաչափությունների շարքում։ Հարթության էվկլիդեսյան երկրաչափությունում երկու պատկերները (օրինակ եռանկյունները), կոչվում են հավասար, եթե դրանցից մեկը վերադրումով կարելի է համընկեցնել մյուսի հետ։ Ընդ որում համարվում է, որ վերադրման ողջ ընթացքում տեղափոխվող պատկերի ցանկացած երկու կետերի միջև հեռավորությունը մնում է անփոփոխ։ Ստորև պատկերված ուղղանկյուն եռանկյունների հավասարությունը ա) դեպքում կասկած չի հարուցում։
% nkar a and b
\begin{center}
\includegraphics[scale=1]{images/id 05.PNG}
\end{center}
Նախ կարելի է եռանկյուններից մեկը, օրինակ $ABC$-ն ինքն իրեն զուգահեռ տեղափոխելով, նրա $C$ գագաթը համընկեցնել $A'B'C'$ եռանկյան $C'$ գագաթի հետ։ Այնուհետև, կատարելով պտույտ $C'$ գագաթի շուրջ որոշ անկյունով համընկեցվում են նաև մյուս երկու գագաթները՝ $A$-ն $A'$-ի հետ, $B$-ն $B'$-ի հետ։ Քանի որ այս գործողությունների ընթացքում $ABC$ եռանկյան ցանկացած երկու կետերի միջև հեռավորությունները չեն փոխվում, արդյունքում եռանկյուն $ABC$-ն համադրվում է $A'B'C'$ եռանկյան վրա։ Մյուս՝ բ) տարբերակում, առանց հարթությունից դուրս գալու այդպիսի վերադրում իրականացնել հնարավոր չէ։ Այս դեպքում ունենք երկու հնարավորություն․ կա՛մ պատկերված $ABC$ և $ABC'$ եռանկյունները պետք է համարենք ոչ հավասար, կա՛մ էլ կարող ենք համարել հավասար, եթե մեզ թույլ տանք վերադրելիս դուրս գալ հարթությունից դեպի տարածություն, կատարելով պտույտ $AB$ ուղղի շուրջ $180^\circ$ անկյունով։
\par Միանգամից ասենք, որ երկու մոտեցումն էլ սկզբունքորեն ընդունելի են, բայց առաջին տարբերակի դեպքում ստացվում են երկու \textbf{տարբեր երկրաչափություններ}։ Դրանցից առաջինում, երբ $ABC$ և $ABC'$ եռանկյունները համարվում են ոչ հավասար, տեղի չունի եռանկյունների հավասարության հայտանիշը ըստ երեք կողմերի։ Այնուամենայնիվ նման հայտանիշ կարելի է ձևակերպել նոր տեսքով․ համապատասխանաբար հավասար կողմերով և \textbf{միատեսակ կողմնորոշված} եռանկյունները միմյանց հավասար են։ Նկատենք, որ բ) նկարում պատկերված $ABC$ և $ABC'$ եռանկյուններն ունեն տարբեր կողմնորոշումներ, այսինքն վազանցումները նրանց պարագծերով (պահելը գագաթների համապատասխանությունը) կատարվում են տարբեր ուղղություններով․ $ABC$ եռանկյան դեպքում՝ ժամ․ սլաքին հակառակ ուղղությամբ, իսկ $ABC'$ եռանկյան դեպքում ժամ․ սլաքի ուղղությամբ։
\par Ամփոփելով վերը շարադրվածը կատարենք երկու դիտողություն։\par
1. Կախված նրանից, թե ինչպիսի ձևափոխություններ են թույլատրվում պատկերների հավասարություն սահմանելիս, կարող են գոյանալ տարբեր երկրաչափություններ։ Հենց նոր տեսանք, որ գոյություն ունի հարթության առնվազն երկու երկրաչափություն՝ կողմնորոշված պատկերների հարթաչափություն և չկողմնորոշված պատկերների հարթաչափություն (սովորական էվկլիդեսյան հարթաչափություն)։
\par 2. Էվկլիդեսյան հարթաչափությունում $ABC$ և $ABC'$ եռանկյունների հավասարությունը վերադրումով հիմնավորելու համար ստիպված եղանք դուրս գալ հարթությունից։ Մինչդեռ ցանկալի է, որ հարթաչափությունը լինի ինքնաբավ, այսինքն ձևավորվի սեփական տիրույթում առանց դիմելու տարածության օգնությանը։ Դա կարելի է անել ներմուծելով պատկերների իզոմետրիկ ձևափոխություն հասկացությունը։
\par Դիցուք ունենք $X$ և $Y$ երկու պատկերներ, որոնք կետերի միջև հաստատված է փոխմիարժեք $x\mapsto y$ համապատասխանություն այնպես, որ պահպանվում են կետերի միջև հեռավորությունները․ այսինքն կամայական $x_1,x_2\in X$ կետերի դեպքում, եթե  $x_1\mapsto y_1$, $x_2\mapsto y_2$, որտեղ $y_1,y_2\in Y$, ապա $x_1$ և $x_2$ կետերի միջև $\rho(x_1,x_2)$ հեռավորությունը հավասար է $y_1$ և $y_2$ կետերի միջև $\rho(y_1,y_2)$ հեռավորությանը։ Ամեն մի այդպիսի ձևափոխություն կոչվում է հարթության իզոմետրիկ ձևափոխություն։
\par Իզոմետրիկ ձևափոխություններ են հարթության զուգահեռ տեղափոխությունները, կետերի շուրջ պտույտները, ուղիղների նկատմամբ համաչափությունները։ Տեղի է նկատել, որ ի տարբերություն զուգահեռ տեղափոխությունների և պտույտների, ուղղի նկատմամբ համաչափությունը փոխում է պատկերը կողմնորոշումը։


\includegraphics[scale=0.2]{images/top_page_9.png}
%%%%%%%%%%%%%%%%%%%%%%%%%%%%
\newcommand{\RNum}[1]{\uppercase\expandafter{\romannumeral #1\relax}}

%%%%%%%%%%%%%% 10 %%%%%%%%%%%%%%%%%%%
\par Պարզվում է, որ թվարկված ձևափոխություններով սպառվում են հարթության մեջ բոլոր իզոմետրիկ ձևափոխությունները հետևյալ իմաստով․ հարթության ամեն մի
իզոմետրիկ ձևափոխություն կամ զուգահեռ տեղափոխություն է ինչ-որ վեկտորով, կամ պտույտ է ինչ-որ կետի շուրջ ինչ-որ անկյունով, կամ համաչափություն է ինչ-որ ուղղի
նկատմամբ և կամ էլ մի քանի այդպիսի ձևափոխությունների համադրույթ է (այսինք ձևափոխությունների հաջորդական կիրառման արդյունք է)։

\par Իզոմետրիկ ձևափոխությունները օժտված են հետևյալ կարևոր հատկություններով։
\begin{enumerate}
    \item [1.] Ցանկացած պատկերի նույնական ձևափոխությունը իզոմետրիկ ձևափոխություն է։ Նույնական կոչվում է այն ձևափոխությունը, որը յուրաքանչյուր կետի համապատասխանեցնում է այդ նույն կետը $x \Longrightarrow x$:
    \item [2.] Ցանկացած պատկերի (այդ թվում հարթության) ցանկացած քանակով իզոմետրիկ ձևափոխությունների համադրույթը իզոմետրիկ ձևափոխություն է։
    \item [3.] Յուրաքանչյուր իզոմետրիկ $h$ ձևափոխության համար գոյություն ունի նրան հակադարձ $h^{-1}$ ձևափոխություն որը նույնպես իզոմետրիկ ձևափոխություն է (ըստ սահմանման, $x \shortrightarrow y$ ձևափոխության հակադարձ կոչվում է $y \shortrightarrow x$ ձևափոխությունը)։
\end{enumerate}
 
\par Իրոք, $\overrightarrow{a}$ վեկտորով զուգահեռ տեղափոխության հակադարձը $-\overrightarrow{a}$ վեկտորով զուգահեռ տեղափոխությունն է։ 
$O$ կետի շուրջ $\phi$ անկյունով պտույտի հակադարձը $O$ կետի շուրջ  $-\phi$ անկյունով (այսինքն հակառակ ուղղությամբ բայց նույն մեծությամբ անկյունով) պտույտն է։ 
Որևէ ուղղի նկատմամբ համաչափության հակադարձը այդ նույն համաչափությունն է։ Վերջապես, եթե $h_1$, $h_2$-ը իզոմետրիկ ձևափոխություններ են, ապա նախ կատարելով
$h_1$ ձևափոխությունը և ապա $h_2$ ձևափոխությունը ստանում ենք իզոմետրիկ ձևափոխություն (նշանակվում է $h1 \circ h_2$), որի հակադարձը $h1^{-1} \circ h_2^{-1}$ իզոմետրիկ ձևափոխությունն է։ 

\par Եթե ինչ-որ պատկերի (մասնավորապես հարթության կամ տարածության) որոշ ձևափոխություններից կազմված $G$ բազմության տարրերը բավարարում են 1-3 պայմաններին, ապա
ասում են որ $G$-ն տվյալ պատկերի ձևափոխությունների խումբ է։ 

\par Այսպիսով հարթության բոլոր իզոմետրիկ ձևափոխությունները կազմում են խումբ, որն ընդունված է նշանակել $O(2)$։

%%%%%%%%%%%%%% 11 %%%%%%%%%%%%%%%%%%%%
\par Համեմատության համար քննարկենք մյուս՝ հարթության կողմնորոշված պատկերների երկրաչափությունը։ Այս երկրաչափության
հիմքում ընկած են պատկերների (մասնավորապես հարթության) այդ բոլոր իզոմետրիկ ձևափոխությունները, որոնք պահպանում են դրանց կողմորոշումը։ Դրանք են՝ 
զուգահեռ տեղափոխությունները, կետերի շուրջ պտույտները և դրանց բոլոր համաչափությունները։ Սրանք նույնպես կազմում են ձևափոխությունների խումբ։ 
Այդ խումբը կոչվում է հարթության հատուկ (կամ առաջին սեռի) իզոմետրիկ ձևափոխությունների խումբ և նշանակվում է $SO(2)$։ Այս խումբը կոչվում է $O(2)$ խմբի մաս (ենթախումբ)։ Նրանում բացակայում են համաչափությունները ուղիղների նկատմամբ։ Որպես հետևանք առաջանում են
որոշ տարբերություններ $SO(2)$ և $O(2)$ խմբերի երկրաչափություների միջը։ 
\RNum{1}. $SO(2)$ խմբի երկրաչափությունում երկրաչափական պատկերների տեսականին ավելի հարուստ է։ Իրոք, այն դեպքում երբ $ABC$ և $ABC'$ եռանկյունները նույնն են (հավասար են)
$O(2)$ խմբի երկրաչափությունում, դրանք տարբեր են (հավասար չեն) $SO(2)$ խմբի երկրաչափությունում։ Բացի այդ, եթե $O(2)$ խմբի դեպքում անկյունը միարժեքորեն որոշվում է
կետով և նրանից դուրս եկով երկու $a$ և $b$ ճառագայթներով, ապա $SO(2)$ խմբի դեպքում մենք ունենք երկու տարբեր անկյուններ՝ $a$-ից մինչև $b$ ընկած անկյուն և 
$b$-ից մինչև $a$ ընկած անկյուն, որոնք որպես պատկերներ նույնը չեն։ Փոխելով շեշտադրումը նույն միտքը կարող ենք ձևակերպել նաև այսպես․ տվյալ երկրաչափության 
ձևափոխությունների խումբն ընդլայնելիս առաջանում է նոր երկրաչափություն, բայց երկրաչափական պատկերների ավելի աղքատիկ տեսականիով։

\RNum{2}. Ձևափոխությունների խումբն ընդլայնելիս որոշ չափով քչանում են (նախկինի համեմատ) նաև երկրաչափական հատկությունները։ Տվյալ դեպքում խոսքը պատկերի 
կողմնորոշման մասին է։ Պատկերը կողմնորոշելու հնարավորությունը (կամ դրա անհնարությունը) իրականում շատ կարևոր երկրաչափական հատկություններ են (օրինակ մարդիկ 
թե ներքին թե արտաքին կառուցվածքով ունեն որոշակի աջ կամ ձախ կողմնորոշում)։ Քանի որ հարթության մեջ ուղղի նկատմամը, իսկ տարածության մեջ հարթության նկատմամբ 
համաչափությունները փոխում են պատկերի կողմորոշումը, ուստի $O(2)$ խմբի երկրաչափությունում այդ դադարում է լինել երկրաչափական հատկություն։  
%%%%%%%%%%%%%%%%%%%%%%%% 12 %%%%%%%%%%%%%%%%%%%%%%%%


\par Այժմ կարող ենք սահմանել Էվկլիդեսյան հարթաչափությունում պատկերի հավասարություն, առանց այդ պատկերները 
հարթությունից դուրս բերելու․ երկու $X$ և $Y$ պատկերներ կոչվում են հավասար, եթե գոյություն ունի հարթության իզոմետրիկ ձևափոխություն, որը $X$ պատկերը 
ձևափոխում է $Y$ պատկերին։ Մի փոքր այլ ձևակերպումով դա հնչում է այսպես․ $X$ և $Y$ պատկերները հավասար են եթե գոյություն ունի $O(2)$ խմբի $h$ տարր, որ $h(X) = Y$։

\par Այս սահմանումից և իզոմետրիկ ձևափոխությունների 1-3 հատկություններից ստացվում են պատկերների հավասարության հետևյալ հատկությունները․
\begin{enumerate}
    \item [1'.] ամեն մի պատկերը հավասար է ինքն իրեն % \red{ա-ն մեծատառ գուցե}
    \item [2'.] Եթե $X$ պատկերը հավասար է $Y$ պատկերին, իսկ $Y$ պատկերը հավասար է $Z$ պատկերին, ապա $X$-ը հավասար է $Z$-ին։
    \item [3'.] Եթե $X$ պատկերը հավասար է $Y$ պատկերին, ապա $Y$-ը հավասար է $X$-ին։
\end{enumerate}
 
\par Այնուհետև, պատկերների որևէ հատկություն (պատկերների հետ առնչվող որևէ թվային մեծություն) կոչվում է երկրաչափական հատկություն (երկրաչափական մեծություն) Էվկլիդեսյան 
հարթաչափությունում, եթե այն բանից որ այդ հատկությամբ (մեծությամբ) օժտված է որևէ $X$ պատկեր, հետևում է, որ այդ հատկությամբ (մեծությամբ) օժտված են $X$-ին հավասար բոլոր պատկերները։ 

\par Այլ կերպ ասած երկրաչափական են այն հատկություններն ու մեծությունները, որոնք կայուն են, չեն փոխվում երբ պատկերը ենթարկվում է իզոմետրիկ ձևափոխությունների։ 
Այսպիսով Էվկլիդեսյան երկրաչափությունում իզոմետրիկ ձևափոխությունները որոշիչ դեր են խաղում։ Այս հանգամանքը ի նկատի ունենալով, հաճախ Էվկլիդեսյան երկրաչափությունը 
անվանում են ձևափոխությունների $O(2)$ խմբի երկրաչափություն։ Այս երկրաչափությունում երկրաչափական հատկություններ են պատկերի ուղղագծությունը (ի նկատի ունենք որ պատկերի կետերը շարված են որևէ ուղղի վրա); 
ուղիղներով կազմած անկյունը, գծերի, հատավածների երկարությունները, պատկերների մակերեսները, անկյան մեծությունը, բազմանկայն կողմերի քանակը և այլն։

%%%%%%%%%%%%%%%%%%%%%%%% 13 %%%%%%%%%%%%%%%%%%%%%%%%

\par Որպեսզի ավելի ակնառու լինի երկրաչափական պատկերների և երկրաչափական հատկություններօ դինամիկան կապված ձևափոխությունների խմբի փոփոխության հետ,
քննարկենք ձևափոխությունների ևս մի խմբի և նրանով որոշված երկրաչափության օրինակ։ 

\par Այդ երկրաչափությունը սահմանային տեսքով (առանց այդ մասին հատուկ պարզաբանումների) դիտարկվում է միջնակարգ դպրոցի երկրաչափության դասընթացում,
որպես օգտակար մեթոդ ստանալու համար, օրինակ որոշ չափական առընչություններ ուղղանկյուն եռանկյան էջերի, ներքնաձիգի, ներքնաձիգին տարված բարձրության և
ներքնաձիգի հատավածների միջը։ Դա այսպես կոչված նմանության հատկությունն է։ 

\par Այս երկրաչափության հիմքում ընկած է հարթության նմանության ձևափոխությունների խումբը։ Հարթության ձևափոխությունը կոչվում է $k>0$ գործակցով նմանության
ձևափոխություն, եթե կամայական $x_1$, $x_2$ կետերի և նրանց $y_1$, $y_2$ կերպարների դեպքում տեղի ունի $\rho(y_1,y_2) = k \cdot \rho(x_1,x_2)$ հավասարություն։
Մասնավոր դեպքում, երբ $k=1$ ստացվում է $\rho(x_1,x_2) = \rho(y_1,y_2)$  այսինքն հարթության իզոմետրիկ ձևափոխությունները նմանության ձևափոխություններ են։ Հարթության ամեն մի $O$ կետի
և $k>0$ թվի համար սահմանվում է մի հատուկ նմանության ձևափոխություն։ Այն կոչվում է հարթության հոմոտետիա $O$ կենտրոնով և $k$ գործակցով, նշանակվում է $H_k^o$ և
սահմանվում է որպես $\overrightarrow{Ox'} = R \cdot \overrightarrow{Ox}$ որտեղ $x$-ը հարթության կամայական կետ է, իսկ $x'$-ը՝ նրա կերպարը։ Հեշտությամբ ցույց է տրվում որ․
\begin{enumerate}
    \item[1.] ցանկացած $H_k^o$ հոմոտետիա նմանության ձևափոխություն է $k$ գործակցով;
    \item[2.] ցանկացած $k$ գործակցով նմանության ձևափոխություն կարող է ներկայացվել որպես որևէ իզոմետրիկ ձևափոխության և որևէ $H_k^o$ հոմոտետիայի համադրույթ;
    \item[3.] ցանկացած երկու $k_1$ և $k_2$ գործակիցներով նմանության ձևափոխությունների համադրույթը նմանության ձևափոխություն է $k_2 k_1$ գործակցով;
    %%%%%%%%%%%%%%%%%%%%%%%% 14 %%%%%%%%%%%%%%%%%%%%%%%%

    \item[4.] ցանկացած $k$ գործակցով նմանության ձևափոխության համար գոյություն ունի նրան հակադարձ ձևափոխություն․ որը նմանության ձևափոխություն է $\frac{1}{k}$ գործակցով
\end{enumerate}



\par Վերջապես, քանի որ հարթության նույնական ձևափոխությունը նույնպես ձևափոխություն է ($k=1$ գործակցով), ուստի հարթության բոլոր նմանության ձևափոխությունները
կազմում են ձևափոխությունների խումբ, որը նշանակվում է $H(2)$։ Այն իր մեջ ընդգրկում է իզոմետրիկ ձևափոխությունների $O(2)$ խումբը որպես ենթախումբ։ Այսպիսով
ունենք խմբերի և նրանց ներդրումների այսպիսի շղթա $SO(2) \in O(2) \in H(2)$

\par նմանության երկրաչափությունում երկու $M$ և $N$ պատկերներ համարվում են վերադրումով համատեղվող (կամ պարզապես նման), եթե գոյություն ունի
նմանության ձևափոխություն որը $M$-ը ձևափոխում է (արտապատկերում է) $N$-ին։ 
Այս դեպքում նպատակահարմար չէ համատեղվող պատկերները համարել հավասար, քանի որ $k \neq 1$ դեպքում նմանության ձևափոխությունը փոխում է պատկերի չափսերը, մեծացնելով 
կամ փոքրացնելով դրանք $k$ անգամ։

\par Ուստի նպատակահարմար է համատեղվող պատկերներ համար գործածել համարժեք կամ կոնգուենտ տերմինները, իսկ պատկերի հավասարության տերմինը գործածել այն դեպքում
երբ դրանք նույնն են։ 

\par Այսպիսով $H2$ խմբի երկրաչափությունում երկու պատկերներ միմյանց համարժեք են այն և միայն այն դեպքում երբ նրանք նման պատկերներ են։ Այս իմաստով
միմյանց համարժեք են բոլոր $(a,b)$ ինտերվալները, բոլոր $[a,b]$ հատվածները, բոլոր կանոնավոր եռանկյունները, բոլոր քառակուսիները, բոլոր շրջանագծերը և այլն։

\par Ստացվում է այդպես, որ բոլոր, օրինակ շրջանագծերը, միասին վերցրած հանդես են գալիս որպես մի նորագայություն, որպես մի
նոր տեսակի երկրաչափական պատկեր $H(2)$ խմբի երկրաչափությունում։ Այդ պատկերն ընդունված է անվանել շրջանագծերը համարժեքության դաս։ Նույնը տեղի է ունենում բոլոր
կանոնավոր եռանկյունների, բոլոր քառակուսիների, և ընդհանրապես՝ բոլոր միմյանց նման պատկերների դեպքում։ Սա նման է նրան, երբ մի քանի պետություններ, ելնելով որոշ
ընդհանրություններից և շահերից միավորվում են դաշինքների կամ ընդհանուր պետության մեջ,

%%%%%%%%%%%%%%%%%%%%%%%% 15 %%%%%%%%%%%%%%%%%%%%%%%%

չձուլվելով միմյանց։ % (կարողա՞ հստակեցվեցին nerqevi toxum)}
\par Այն բանից հետո, երբ հստակվեցին  հավասար և համարժեք պատկերների հատկությունները  
(հավասարները նաև համարժեք են, բայց համարժեքները կարող են հավասար չլինել), անդրադառնալով \RNum{1} և \RNum{2} դիտողություններին, տեսնում ենք որ դրանցում պետք 
է կատարել ճշգրտումներ։

\RNum{1}': $ABC$ և $ABC'$ եռանկյունները ոչ միայն տարբեր են, այլ նաև համարժեք չեն $SO(2)$ խմբի երկրաչափությունում։

\RNum{2}': Ձևափոխությունների խումբն ընդլայնելիս նախկին երկրաչափական պատկերներն ու երկրաչափական հատկությունները չեն քչանում։ Դրանք ինչպես կային, այդպես էլ մնում են։ 
Պարզապես գոյանում է նոր երկրաչափություն նոր պատկերներով, որոնք ձևավորվում են հին պատկերներից համարժեքության դասերի տեսքով։ Նոր պատկերների երկրաչափական 
հատկությունները նույնպես ձևավորվում են հինգ պատկերների երկրաչափական հատկություններից։ 
Դրանք այդ հատկություններն են որոնք չեն փոխվում նաև ավելացվող ձևափոխությունների դեպքում։ 

\par Մասնավորապես ձևափոխությունների $H(2)$ խմբի երկրաչափությունում անիմաստ է խոսել շրջանագծերի դասի շառավղի մասին, շրջնանների դասի մակերեսի մասին, 
էլիպսների որևէ դասի կիսառանցքների մասին, քանի որ դրանք չեն պահպանվում նմանության բոլոր ձևափոխությունների դեպքում։ Բայց կարելի է խոսել նման էլիպսների 
ցանկացած դասի էքսցենտրիսիտետի մասին, քանի որ միմյանց նման բոլոր էլիպսները ունեն միևնույն էքսցենտրիսիտետը։

\par Նշենք, որ երկրաչափությունների հաջորդականությունը չի ավարտվում քննարկված երեք երկրաչափություններով։ Սովորավար վերլուծական երկրաչափության 
քիչ թե շատ լիարժեք դասընթացում դիտակվում է ևս մի երկրաչափություն՝ աֆինական երկրաչափությունը։

\par Աֆինական երկրաչափության հիմքում ընկած է հարթության աֆինական ձևափոխությունների խումբը, որն իր մեջ որպես
%%%%%%%%%%%%%%%%%%%%%%%% 16 %%%%%%%%%%%%%%%%%%%%%%%%
ենթախումբ ընդգրկում է նմանության ձևափոխությունների $H(2)$ խումբը։ Ձևափոխությունների խմբի ընդլայնումը տեղի է ունենում այսպես կոչված՝ ուղիղների նկատմամբ սեղմումների։ % շարահյուսությունը սխալա ոնց-որ


\par Հարթության սեղմում $k>0$ գործակցով սևեռված $l$ ուղղի նկատմամբ կոչվում է այն ձևափոխությունը, որի դեպքում
{\includegraphics[scale=0.5]{images/top_page_16.png}};
 % չգիտեմ ոնց անենք ստեղ ա, բ-ի համարակալման պահը, բ-ն շատ երկարա

ա) $l$ ուղղի կետերը մնում են անշարժ,
բ) կամայական $M$ կետի $M'$ կերպարը որոշվում է $\overrightarrow{M_0 M'} = k \cdot \overrightarrow{M_0 M}$ պայմանով, որտեղ $M_0$-ն $M$ կետի պրոյեկցիան է $l$ ուղղի վրա (նկ․ 6-ում պատկերված է $k = \frac{1}{2}$ դեպքը)։

\par Հարթության աֆինական ձևափոխությունները գոյանում են նմանության ձևափոխություններից, ուղիղների նկատմամբ սեղմումներից և դրանց համադրույթներից։

\par Նշենք, որ աֆինական երկրաչափությունում բոլոր էլիպսները, անկախ նրանց կիսառանցքների երկարություններից, համարժեք են միմյանց
և կազմում են մի համարժեքության դաս՝ էլիպսների դաս։ Այս նոր երկրաչափական պատկերի համար,
հասկանալի պատճառով, իմաստազրկվում է էքսցենտրիսիտետ հասկացությունը։

\par Ամփոփենք․ որքան ավելի է ընդլայնվում ձևափոխությունների խումբը, այնքան ավելի են շատանում միմյանց համարժեք երկրաչափական պատկերները համարժեքության դասերում։
Դրա հետ կապված՝ նույնքան դժվարանում է համարժեքության նոր դասերի համար երկրաչափական հատկությունների որոնումը։

\par Ի վերջո առաջանում է երկու հարց․
1․ Գոյություն ունի՞ արդյոք մի վերջին, այսպես կոչված սահմանային երկրաչափություն, որով ավարտվում է երկրաչափություների այս հաջորդականությունը
2․ Պատկերների ո՞ր երկրաչափական հատկություններն են բնորոշ սահմանային երկարության պատկերներին (համարժեքության դասերին)։

\par Արդեն իսկ դիտարկված ձևափոխությունների հատկությունների համեմատումը ցույց է տալիս, որ ձևափոխությունների խումբն ընդլայնելիս
1. ավելացող (նոր) ձևափոխություններն ավելի <<լիբերալ են>> նախորդներից, այսինք ավելի թույլ պահանջներ են դնում պատկերների համարժեքության համար։
%%%%%%%%%%%%%%%%%%%%%%%% 17 %%%%%%%%%%%%%%%%%%%%%%%%

2. Բոլոր հին և նոր ձևափոխությունները ունեն մի ընդհանուր հատկություն՝ փոխմիարժեք և անընդհատ են


\par Ուստի բնական կլինի սահմանային երկրաչափական ձևափոխություններից միայն պահանջել, որ նրանք լինեն փոխմիարժեք և անընդհատ։

\par Այժմ վերադառնալով թեմայի սկզբին պարզ է դառնում, որ ստացվող սահմանային երկրաչափությունը տոպոլոգիա է։ Տեսնում ենք նաև, որ 1-3 նկարներում պատկերված գծապատկերները, ինչպես նաև 5 նկարում պատկերները, տոպոլոգորեն համարժեք են․ դրանցից ցանկացած երկուսը կարող են ստացվել մեկը մյուսից ենթարկելով անընդհատ ձևափոխությունների։

\par Երբեմն, պատկերավորության համար ասում են, որ տոպոլոգիայում երկու պատկերները համարժեք են, եթե նրանցից մեկը կարող է ստացվել մյուսից ձգելով, սեղմելով, ծռելով, բայց առանց պոկելու, ջարդելու կամ կտրելու։  

\par Երբեմն, ավելի հեռուն են գնում հայտատրերոլվ, որ տոպոլոգի համար (տոպոլոգ անվանում են տոպոլոգիայով զվաղվողներին) նույնն են տոպոլոգորեն համարժեք պատկերները (մասնավորապես նույնն են ստորև պատկերված թխվածքաբլիթն ու սուրճի գավաթը)։

{\includegraphics[scale=0.5]{images/top_page_17.png}}


\par Այս նույն տրամաբանությամբ ասում են նաև, որ տոպոլոգիան ռետինային կամ էլաստիկ երկրաչափություն է։

\par Վերջին հարց, արդյո՞ք տոպոլոգիայով վերջանում են երկրաչափությունները։ Բնավ ոչ․ ընդհանուր կամ տեսաբազմական տոպոլոգիային հաջորդեց հոմոտոպիական տոպոլոգիան։ Այս տոպոլոգիայում հոմեոմորֆիզմների դեր կատարում են այսպես կոչված հոմոտոպիական համարժեքությունները։ Ի տարբերություն հոմեոմորֆիզմների, համատոպիական համարժեքությունները որպես արտապատկերումներ կարող են չլինել փոխմիարժեք համապատասխանություններ։

\par Նշենք, որ կան նաև երկրաչափություններ, որոնք չեն սահմանվում աքսիոմատիկ եղանակով, օրինակ՝ դիֆերենցիալ, ռիմանյան, հանրահաշվական երկրաչափությունները
\end{document}