\bigskip
\bigskip
\subsubsection*{Խնդիրներ և հարցեր թեմա 2-ի վերաբերյալ}

\begin{enumerate}[label=\thesection.\arabic*.]
\item  Ո՞ր դեպքերում է հնարավոր
\begin{enumerate}
    \item[ա)] $A \times B = B \times A$ նույնություն,
    \item[բ)] $A\times(B \times C) = (A \times B) \times C$ նույնություն։
\end{enumerate}

\item  Երկրաչափորեն մեկնաբանեք $[a,b]\times[c,d]$, $[a,b]^2$, $[a,b]^3$, $[a,b]\times[c,d]^2$ ուղիղ արտադրյալները, որտեղ $[a,b]$-ն և $[c,d]$-ն թվային ուղղի հատվածներ են։

\item  Երկրաչափորեն մեկնաբանեք $A\times[0,1]$ ուղիղ արտադրյալը, որտեղ $A$-ն $\R^2$ հարթության
\begin{enumerate}
    \item[ա)] $\{(x,y)\mid x^2+y^2=1\}$ շրջանագիծն է,
    \item[բ)] $\{(x,y)\mid x^2+y^2 \le 1\}$ շրջանն է։
\end{enumerate}

\item  Ապացուցեք \hyperref[թեորեմ 2:2]{թեորեմ 2}-ի ա), գ), դ) նույնությունները։

\item  Դիցուք $X_1$, $X_2$-ը որևէ բազմություններ են։ Ապացուցեք, որ ցանկացած \linebreak $A \subset X_1$, $B \subset X_2$ ենթաբազմությունների դեպքում
\[
P_{X_1}^{-1}(A) = A\times X_2,
\quad
P_{X_2}^{-1}(B) = X_1\times B,
\quad
P_{X_1}^{-1}(A) \cap P_{X_2}^{-1}(B) = A\times B,
\]
որտեղ $P_{X_1}$-ը և $P_{X_2}$-ը $X_1\times X_2 \rightarrow X_1$, $X_1\times X_2 \rightarrow X_2$ կանոնական պրոյեկցիա\-ներն են։ 

\item  Ապացուցեք \hyperref[թեորեմ 2:3]{թեորեմ 3}-ի ա) նույնությունը։

\item  Սահմանեք $S \times S$ վերացական տորի համար զուգահեռականի և միջօրեականի հասկացություններ՝ որպես $S \times S$ ուղիղ արտադրյալի ենթաբազմություններ։

\item  Օգտվելով նախորդ խնդրից՝ ապացուցեք․

$\displaystyle \hfill h : S \times S \rightarrow S \times S, \quad h(s_1,s_2)=(s_2,s_1), \quad s_1,s_2 \in S \hfill$ արտապատկերումը

\begin{enumerate}
    \item[ա)]  $S \times S$ վերացական տորը փոխմիարժեք արտա\-պատ\-կերում է ինքն իր վրա,
    
    \item[բ)] $h$-ը վերացական տորի զուգահեռականները արտապատկերում է նրա միջօրեականների, իսկ միջօրեականները՝ զուգահեռականների։
\end{enumerate}

\item  Ապացուցեք, որ \hyperref[օրինակ 2:6]{օրինակ 6}-ի երկտեղ հարաբերությունը համարժեքության \linebreak հարա\-բերություն է։ Նկարագրեք նրանով որոշվող ֆակտոր-բազմությունը։

\item  $\R^2$ հարթության բոլոր ուղիղներով կազմված $(l_1,l_2)$ զույգերի համար սահ\-ման\-ված է ա) $R'$ երկտեղ հարաբերություն՝ «տեղի ունի $l_1R'l_2$ այն և միայն այն դեպքում, երբ $l_1$-ը և $l_2$-ը զուգահեռ են» ասույթով, և բ) $R''$ երկտեղ հարաբերություն՝ «տեղի ունի $l_1R''l_2$ այն և միայն այն դեպքում, երբ $l_1$ և $l_2$ ուղիղները կամ նույնն են, կամ զուգահեռ են» ասույթով։
\par Պարզեք, թե այս երկուսից որն է համարժեքության հարաբերություն, և որը՝ ոչ։ Առաջին դեպքում նկարագրեք համարժեքության դասերը և ֆակտոր-բազ\-մու\-թյան համար կառուցեք որևէ երկրաչափական մոդել։

\item  Թղթի թերթից կտրեք $A(-10,-1)$, $B(-10,1)$, $C(10,1)$, $D(10,-1)$ գագաթներով երկու ուղղանկյուն (մասշտաբը՝ $1$ սմ)։ Մի դեպքում առանց ոլորելու՝ ստացված $ABCD$ ուղղանկյան $AB$ կողմը սոսնձեք $DC$ կողմի հետ այնպես, որ նույնաց\-վեն $A$, $D$ գագաթները և $B$, $C$ գագաթները, իսկ մյուս դեպքում՝ կատարելով թղթի շերտի \textbf{մի} ոլորում՝ նորից սոսնձեք $AB$ կողմը $DC$ կողմի հետ այնպես, որ նույնաց\-վեն $A$, $C$ գագաթները և $B$, $D$ գագաթները։
\par Համոզվեք, որ առաջին դեպքում կստացվի \textbf{գլանային մակերևույթ}, իսկ երկ\-րորդ դեպքում կստացվի գլանաձև, բայց \textbf{ոչ գլանային մակերևույթ}, որը կոչվում է \textbf{Մյոբիուսի թերթ}։ Նաև համոզվեք, որ առաջին մակերևույթի եզրագիծը կազմված է \textbf{երկու անջատ օվալից}, իսկ երկրորդի եզրագիծը՝ \textbf{մի օվալից}։
\par Սահմանեք կոորդինատային բացահայտ տեսքով $ABCD$ ուղ\-ղանկ\-յան կետերի համար երկու (երկտեղ) համարժեքության հարաբերություն այնպես, որ արդյունքում ստացվող ֆակտոր-բազմություններից մեկը լինի գլանայինը, իսկ մյուսը՝ Մյոբիուսի թերթը։

\item Ի՞նչ կարող եք ասել մակերևույթների մասին (տե՛ս նախորդ խնդիրը), որոնք ստացվում են, երբ մինչև սոսնձումը կատարվում է թղթի շերտի $2$ ոլորում, $3$ ոլորում, և այլն։
\end{enumerate}
