\bigskip % 10 u 9 tegherov poghel a
\bigskip
\subsubsection*{Խնդիրներ և հարցեր թեմա 9-ի վերաբերյալ}

\begin{enumerate}[label=\thesection.\arabic*.]
$\underline{Ցուցում։}$ Որպես $Y$ վերցրեք $X$ բազմությունը դիսկրետ տոպոլոգիայով։9.1. Դիցուք $X$ տոպոլոգիական տարածությունը օժտված է հետևյալ հատկությամբ․ ցանկացած $Y$ տոպոլոգիական տարածության և ցանկացած $f: X \to Y$ արտապատկերման դեպքում $f$-ը անընդհատ է։ Ապացուցեք, որ $X$-ի տոպոլոգիան դիսկրետն է։9.2. Դիցուք $Y$ տոպոլոգիական տարածությունը օժտված է հետևյալ հատկությամբ․ ցանկացած $X$ տոպոլոգիական տարածության և ցանկացած $f: X \to Y$ արտապատկերման դեպքում $f$-ը անընդհատ է։ Ապացուցեք, որ $Y$-ի տոպոլոգիան անտիդիսկրետն է։$\underline{Ցուցում։}$ Որպես $X$ վերցրեք $Y$ բազմությունը անտիդիսկրետ տոպոլոգիայով։9.3. Դիցուք $f: X \to Y$ արտապատկերումն այնպիսին է, որ $X$-ում բաց ցանկացած ենթաբազմության կերպարը բաց ենթաբազմություն է $Y$-ում։ Ապացուցեք, որ այդպիսի արտապատկերումը կարող է անընդհատ չլինել։$\underline{Ցուցում։}$ Դիտարկեք որևէ $X$ բազմություն $\tau$ և $\mathcal{T}$ տոպոլոգիաներով, որտեղ $\tau < \mathcal{T}$, և $X \to X$ նույնական արտապատկերումը։9.4. Բերեք $X, Y$ տոպոլոգիական տարածությունների և $f: X \to Y$ բիյեկտիվ անընդհատ արտապատկերման օրինակ, որ $f^{-1}$ հակադարձ արտապատկերումը չլինի անընդհատ։$\underline{Ցուցում։}$ Օգտվել նախորդ խնդրի ցուցումից։Image: image_10183c.jpg9.5. Դիցուք $\mathbb{R}^2$ էվկլիդյան հարթության ինչ-որ $f: \mathbb{R}^2 \to \mathbb{R}^2$ արտապատկերում ցանկացած $x_1, x_2 \in \mathbb{R}^2$ կետերի դեպքում բավարարում է $\rho(f(x_1), f(x_2)) = r \cdot \rho(x_1, x_2)$ պայմանին, որտեղ $r$-ը ֆիքսված իրական դրական թիվ է։ Ապացուցեք, որ $f$-ը անընդհատ արտապատկերում է։9.6. Ունենք $\mathbb{R}$ թվային ուղղի $g: \mathbb{R} \to \mathbb{R}$ անընդհատ արտապատկերում։ Հայտնի է, որ $g(x)$-ը ամբողջ թիվ է ցանկացած $x \in \mathbb{R}$ կետի դեպքում։ Ապացուցեք, որ $g$-ն հաստատուն արտապատկերում է։9.7. Ապացուցեք, որ ցանկացած $f: [a, b] \to [a, b]$ անընդհատ արտապատկերում ունի անշարժ կետ (այսինքն $f(x) = x$ գոնե մի $x \in [a, b]$ կետի դեպքում)։$\underline{Ցուցում։}$ Դիտարկեք $g(x) = f(x) - x$ արտապատկերումը։9.8. Կարելի՞ է պնդել, որ ցանկացած անընդհատ ա) $\mathbb{R} \to \mathbb{R}$ տեսքի, բ) $[a, b) \to [a, b)$ տեսքի արտապատկերում ունի անշարժ կետ։ Պատասխանը հիմնավորեք դատողություններով կամ օրինակներով։Image: image_10181e.jpg9.9. Դիցուք ունենք երկու անընդհատ $f, g: X \to Y$ արտապատկերումներ, որտեղ $Y$-ը հաուսդորֆյան տարածություն է։ Դիտարկենք $F \subset X$ ենթաբազմությունը կազմված այն բոլոր $x$ կետերից, որ $f(x) = g(x)$։ Ապացուցեք, որ $F$-ը փակ ենթաբազմություն է $X$-ում։9.10. Դիցուք $f, g: X \to Y$ արտապատկերումները, $Y$ տարածությունը և $F \subset X$ ենթաբազմությունը այնպիսին են ինչպես նախորդ խնդրում։ Ապացուցեք, եթե $F$-ը նաև ամենուրեք խիտ է $X$-ում, ապա $f$ և $g$ արտապատկերումները համընկնում են $X$-ի բոլոր կետերում։$\underline{Ցուցում։}$ Ցույց տվեք, որ $F$-ը պարունակում է իր բոլոր հպման կետերը։9.11. Ունենք $f: X \to \mathbb{R}$ անընդհատ արտապատկերում, որտեղ $X$-ը կամայական տոպոլոգիական տարածություն է և $a \in \mathbb{R}$ ֆիքսված թիվ։ Ապացուցեք, որ $g: X \to \mathbb{R}$ արտապատկերումը սահմանված $g(x) = a f(x), \ x \in X$ բանաձևով նույնպես անընդհատ է։$\underline{Ցուցում։}$ Ներկայացրեք $g$-ն որպես $f$ և $h: \mathbb{R} \to \mathbb{R}, \ h(r) = ar, \ r \in \mathbb{R}$ արտապատկերումների համադրույթ:


\end{enumerate}
