\bigskip
\bigskip
\subsubsection*{Խնդիրներ և հարցեր թեմա 7-ի վերաբերյալ}

\begin{enumerate}[label=\thesection.\arabic*.]

Here is the exact text extraction from the remaining image files (`image_101b1e.jpg`, `image_10183c.jpg`, and `image_10181e.jpg`).

### **Image: `image_101b1e.jpg**`

 Որպես  վերցրեք  բազմությունը դիսկրետ տոպոլոգիայով։

**9.1.** Դիցուք  տոպոլոգիական տարածությունը օժտված է հետևյալ հատկությամբ. ցանկացած  տոպոլոգիական տարածության և ցանկացած  արտապատկերման դեպքում -ը անընդհատ է։ Ապացուցեք, որ -ի տոպոլոգիան դիսկրետն է։

**9.2.** Դիցուք  տոպոլոգիական տարածությունը օժտված է հետևյալ հատկությամբ. ցանկացած  տոպոլոգիական տարածության և ցանկացած  արտապատկերման դեպքում -ը անընդհատ է։ Ապացուցեք, որ -ի տոպոլոգիան անտիդիսկրետն է։
 Որպես  վերցրեք  բազմությունը անտիդիսկրետ տոպոլոգիայով։

**9.3.** Դիցուք  արտապատկերումն այնպիսին է, որ -ում բաց ցանկացած ենթաբազմության կերպարը բաց ենթաբազմություն է -ում։ Ապացուցեք, որ այդպիսի արտապատկերումը կարող է անընդհատ չլինել։
 Դիտարկեք որևէ  բազմություն  և  տոպոլոգիաներով, որտեղ , և  նույնական արտապատկերումը։

**9.4.** Բերեք  տոպոլոգիական տարածությունների և  բիյեկտիվ անընդհատ արտապատկերման օրինակ, որ  հակադարձ արտապատկերումը չլինի անընդհատ։
 Օգտվել նախորդ խնդրի ցուցումից։

---

### **Image: `image_10183c.jpg**`

**9.5.** Դիցուք  էվկլիդյան հարթության ինչ-որ  արտապատկերում ցանկացած  կետերի դեպքում բավարարում է  պայմանին, որտեղ -ը անունով իրական դրական թիվ է։ Ապացուցեք, որ -ը անընդհատ արտապատկերում է։

**9.6.** Ունենք  թվային ուղղի  անընդհատ արտապատկերում։ Հայտնի է, որ -ը ամբողջ թիվ է ցանկացած  կետի դեպքում։ Ապացուցեք, որ -ն հաստատուն արտապատկերում է։

**9.7.** Ապացուցեք, որ ցանկացած  անընդհատ արտապատկերում ունի անշարժ կետ (այսինքն  գոնե մի  կետի դեպքում)։
 Դիտարկեք  արտապատկերումը։

**9.8.** Կարելի՞ է պնդել, որ ցանկացած անընդհատ ա)  տեսքի, բ)  տեսքի արտապատկերում ունի անշարժ կետ։ Պատասխանը հիմնավորեք դատողություններով կամ օրինակներով։

---

### **Image: `image_10181e.jpg**`

**9.9.** Դիցուք ունենք երկու անընդհատ  արտապատկերումներ, որտեղ -ը հաուսդորֆյան տարածություն է։ Դիտարկենք  ենթաբազմությունը կազմված այն բոլոր  կետերից, որ ։ Ապացուցեք, որ -ը փակ ենթաբազմություն է -ում։

**9.10.** Դիցուք  արտապատկերումները,  տարածությունը և  ենթաբազմությունը այնպիսին են ինչպես նախորդ խնդրում։ Ապացուցեք, եթե -ը նաև ամենուրեք խիտ է -ում, ապա  և  արտապատկերումները համընկնում են -ի բոլոր կետերում։
 Ցույց տվեք, որ -ը պարունակում է իր բոլոր հպման կետերը։

**9.11.** Ունենք  անընդհատ արտապատկերում, որտեղ -ը կամայական տոպոլոգիական տարածություն է և  անունով թիվ։ Ապացուցեք, որ  արտապատկերումը սահմանված  բանաձևով նույնպես անընդհատ է։
 Ներկայացրեք -ն որպես  և  արտապատկերումների համադրույթ:

Would you like me to translate these problems or solve any of them (e.g., the fixed point theorem in 9.7)?

\end{enumerate}
