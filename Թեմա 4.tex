\documentclass[./main.tex]{subfiles}

\begin{document}
\onehalfspacing
\section{Տոպոլոգիա բազմության վրա, տոպոլոգիաների համեմատումը։ Տոպոլոգիական տարածության հասկացությունը, փոխադարձ կապը բաց ենթաբազմություն և կետի շրջակայք հասկացությունների միջև։}\label{sec:4}

\par Տոպոլոգիայի՝ որպես մաթեմատիկայի բաժին, հիմնական հասկացությունները ծագել են Ռիմանի, Պուանկարեի, Կանտորի և այլոց աշխատանքներում։ Ներկա\-յումս տոպոլոգիան ստորաբաժանվում է երեք ենթաբաժինների՝ ընդհանուր կամ տեսա\-բազմական տոպոլոգիա, հանրահաշվական տոպոլոգիա և բազմաձևությունների տոպոլոգիա։ Սույն դասընթացը դրանցից առաջինի ներածություն է։

\par Ընդհանուր տոպոլոգիան գիտություն է տոպոլոգիական տարածությունների, անընդ\-հատ արտապատկերումների և դրանց հետ կապված այլ հասկացությունների մասին։ Ամենաընդհանուր տեսքով այդ հասկացությունները երևան եկան 1914 թվականին Ֆ․ Հաուսդորֆի «Բազմությունների տեսություն» գրքում և ստացան վերջնական, ավարտուն տեսք բազմությունների տեսության զարգացման ընթացքում։

\begin{definition}
Ասում են, որ $X$ բազմության վրա տրված է \textbf{տոպոլոգիա}, եթե տրված է $X$-ի որոշ $U_i$ ենթաբազմությունների $\tau=\{U_i,\ i\in I\}$ ընտանիք, որը բավարարում է հետևյալ երեք պայմաններին (աքսիոմներին)․
\begin{enumerate}
    \item $X$-ը և դատարկ բազմությունը պետք է պատկանեն $\tau$-ին,
	\item $\tau$-ի ցանկացած քանակով տարրերի միավորումը պետք է պատկանի $\tau$-ին,
	\item $\tau$-ի ցանկացած վերջավոր քանակով տարրերի հատումը նույնպես պետք է պատկանի $\tau$-ին։
	\par Նկատենք, որ 3-րդ պայմանը կարող է փոխարինվել հետևյալ համարժեք պայմանով՝
	\item[$3'.$] $\tau$-ի ցանկացած երկու տարրերի հատումը նորից պետք է պատկանի $\tau$-ին։
\end{enumerate}


\end{definition}

\begin{example}
Մեկից ավելի տարրեր պարունակող ամեն մի $X$ բազմության վրա կարելի է սահմանել առնվազն երկու տոպոլոգիա․
\begin{enumerate}
    \item[ա)] որպես $\tau$ վերցնենք $X$-ի բոլոր ենթաբազմությունների բազմությունը։
	Այս տոպո\-լոգիան կոչվում է \textbf{դիսկրետ տոպոլոգիա} $X$-ի վրա։
	\item[բ)] որպես $\tau$ վերցնենք միայն $X$-ից և դատարկ բազմությունից կազմված ընտանիքը՝ $\tau=\{\varnothing, X\}$։ Այն կոչվում է \textbf{անտիդիսկրետ} տոպոլոգիա $X$-ի վրա։
    \par	Երկու դեպքում էլ 1-3 պայմանների ստուգումը դժվարություն չի հարուցում։
\end{enumerate}
\end{example}

\begin{example}
Դիտարկենք երկու տարր պարունակող $X=\{x_1,x_2\}$ բազմություն և սահմանենք $\tau_1=\{ \varnothing,\{x_1\},\{x_1,x_2\}\},\ \tau_2=\{\varnothing,\{x_2\},\{x_1,x_2\}\}$։ Ե՛վ $\tau_1$-ը, և՛ $\tau_2$-ը 1-3 պայմաններին բավարարում են։
\par Նկատենք, որ $\forall X=\{x_1,x_2\}$ բազմության վրա ունենք ճիշտ և ճիշտ չորս տոպոլոգիա․ դրանք են դիսկրետ, անդիսկրետ, և $\tau_{1}, \tau_{2}$ տոպոլոգիաները։
\end{example}

\begin{example}
$\R$ թվային ուղղի վրա դիտարկենք ենթաբազմությունների $\tau$ ընտանիք կազմված $\varnothing$ բազմությունից, բոլոր $(a,b)$ ինտերվալներից և դրանց բոլոր հնարավոր միավորումներից։ Առաջին երկու պայմանները բավարարվում են անմիջականորեն, ըստ $\tau$-ի սահմանման։ Ստուգենք $3'$ պայմանը։ Եթե ունենք $\tau$ ընտանիքի որևէ երկու $U_1=\bigcup\limits_j{(a_j,b_j)}, \ j\in J$ և $U_2=\bigcup\limits_k {(c_k,d_k )},\ k\in K$ տարրեր (դրանք ինտերվալների միավորումներ են ըստ ինդեքսների $J$ և $K$ բազմությունների), ապա
\[ 
U_1 \cap U_2= \bigg(\mathsmaller{\bigcup}\limits_{j}{(a_j;b_j)} \bigg) \cap \bigg(\mathsmaller{\bigcup}\limits_{k}{(c_k;d_k )}\bigg) =\mathsmaller{\bigcup}\limits_{i,\, j} \left( (a_j;b_j ) \cap (c_k;d_k)\right)
\]
\par Պարզ է, որ ինտերվալների $(a_j,b_j)\cap (c_k,d_k)$ հատումը կամ դատարկ է, կամ էլ նորից ինտերվալ է։ Ուստի $U_1\cap U_2\in \tau$ ըստ $\tau$-ի սահմանման։
\par 
Այս տոպոլոգիան կոչվում է թվային ուղղի \textbf{սովորական տոպոլոգիա}։ 
\end{example}
\begin{example}
Դիցուք $X$-ը որևէ անվերջ բազմություն է։ Դիտարկենք նրա ենթա\-բազմությունների $\tau$ ընտանիքը՝ կազմված $X$-ից, $\varnothing$-ից և այն բոլոր $U\subset X$ ենթա\-բազմություններից, որոնց $X\setminus U$ լրացումը վերջավոր բազմություն է։ Ցույց տանք, որ $\tau$-ն որոշում է տոպոլոգիա X-ի վրա։ Ստուգենք 2-րդ պայմանը։ Դիցուք $U_j\in \tau,\ j\in J$, որտեղ $J$-ն ինդեքսների ինչ-որ բազմություն է։ Ըստ Դե Մորգանի 1-ին բանաձևի՝ $X\setminus \bigcup\limits_j U_j= \bigcap\limits_j (X\setminus U_j)$։ Քանի որ $X\setminus U_j$ լրացումները վերջավոր ենթաբազմություններ են, ուստի նրանց հատումը նույնպես $X$-ի վերջավոր ենթաբազմություն է։ Հետևաբար $\bigcup\limits_j U_j \in \tau$։ Ստուգենք 3-րդ պայմանը։ Դիցուք $U_{1}, U_{2},\dots, U_{k} \in \tau$։ Ըստ Դե Մորգանի 2-րդ բանաձևի՝ $\bigcap\limits_{i=1}^k U_i$ ենթաբազմության $X\setminus \bigcap\limits_{i=1}^k U_i =\bigcup\limits_{i=1}^k (X\setminus U_i)$ լրացումը վերջավոր ենթաբազմություն է։ Ուստի $\bigcap\limits_{i=1}^k U_i\in \tau$։ Այս տոպոլոգիան կոչվում է \textbf{վերջավոր լրացումների տոպոլոգիա} $X$-ի վրա։
\end{example}

\begin{example}
Դիցուք $X$-ը ոչ հաշվելի որևէ բազմություն է։ Դիտարկենք նրա են\-թա\-բազմութ\-յուն\-նե\-րի $\tau$ ընտանիքը կազմված $X$-ից, $\varnothing$-ից և այն բոլոր $V$ են\-թա\-բազմություն\-նե\-րից, որոնց $X\setminus V$ լրացումը հաշվելի բազմություն է։ Ապա $\tau$-ն տոպոլոգիա է $X$-ի վրա (կոչվում է \textbf{հաշվելի լրացումների տոպոլոգիա})։ Ապացուցումը (թողնվում է ընթերցողին) կատարվում է նախորդ օրինակի նմանությամբ՝ հիմնվելով հաշվելի բազմությունների միավորման և հատման հատկությունների վրա։
\end{example}
Սահմանափակվելով առայժմ 1-5 օրինակներով՝ նկատենք նաև, որ եթե միևնույն $X$ բազմության վրա ունենք որևէ երկու $\tau_1$ և $\tau_2$ տոպոլոգիա, ապա նրանց $\tau_1\cap \tau_2$ հատումը նորից տոպոլոգիա է $X$-ի վրա (հիմնավորել)։ Իսկ ահա նրանց միավորումը կարող է չլինել տոպոլոգիա $X$-ի վրա (այս խնդիրը մանրամասն կքննարկվի թեմա 5-ում)։

\subsection*{Տոպոլոգիաների համեմատումը։}
\begin{definition}
Եթե $X$-ի վրա տրված են երկու՝ $\tau_1$ և $\tau_2$ տոպոլոգիա այնպես, որ $\tau_1\subset \tau_2$ ապա ասում են, որ $\tau_1$-ը \textbf{ուժեղ չէ} $\tau_2$-ից, կամ $\tau_2$-ը \textbf{թույլ չէ} $\tau_1$-ից։ Եթե $\tau_1\subset \tau_2$ և $\tau_1\neq \tau_2$, ապա ասում են, որ $\tau_1$-ը \textbf{թույլ է} $\tau_2$-ից ($\tau_2$-ը \textbf{ուժեղ է} $\tau_1$-ից)։
\end{definition}
Միևնույն $X$ բազմության վրա բոլոր տոպոլոգիաներից անտիդիսկրետ տոպոլոգիան ամենաթույլ, իսկ դիսկրետ տոպոլոգիան ամենաուժեղ տոպոլոգիաներն են։ Նկատենք, որ նույն $X$-ի վրա որևէ երկու տոպոլոգիաներ միշտ չէ, որ համեմատելի են։
\begin{example}
Եթե $X=\{x_1,x_2\}$, ապա $\tau_1=\{ \varnothing,\{x_1\},\{x_1,x_2\}\}$ և $\tau_2=\{\varnothing,\{x_2\},\{x_1,x_2\}\}$ տոպոլոգիաները համեմատելի չեն, քանի որ $\{x_1\} \in \tau_1$, բայց $\{x_1\}\not\in \tau_2$, ինչպես նաև $\{x_2\}\in \tau_2$, բայց $\{x_2\}\not\in  \tau_1$։
\end{example}

\begin{example}
Ցույց տանք, որ $\R$ թվային ուղղի սովորական տոպոլոգիան ուժեղ է $\R$-ի վերջավոր լրացումների տոպոլոգիայից։ Պարզ է, որ $(0, 1)$ ինտերվալը պատկա\-նում է դրանցից առաջինին և չի պատկանում երկրորդին։ Մյուս կողմից, եթե $U\subset \R$ ենթաբազմությունը պատկանում է երկրորդին, ապա $\R \setminus U$ լրացումը վերջավոր բազմություն է, ուստի կամ $U=\R$, կամ $\R \setminus U=\{x_1,x_2,\dots,x_n\}$, որտեղ $x_1 < x_2 < \cdots < x_n:$ Երկրորդ դեպքում, համարելով $x_1<x_2<\dots<x_n$, կունենանք $U=(-\infty,x_1) \cup (x_1,x_2)\cup \dots \cup(x_{n-1},x_n) \cup(x_n,+\infty),$ և հետևաբար $U$-ն պատկանում է $\R$-ի սովորական տոպոլոգիային (ինչի՞ց է դա հետևում)։
\par Ընթերցողին առաջարկում ենք ցույց տալ (որպես օգտակար խնդիր), որ թվային ուղղի հաշվելի լրացումների տոպոլոգիան ուժեղ է վերջավոր լրացումների տոպո\-լո\-գիայից, բայց համեմատելի չէ թվային ուղղի սովորական տոպոլոգիայի հետ։
\end{example}
\subsection*{Տոպոլոգիական տարածության հասկացությունը։}
\begin{definition}
$X$ բազմությունը նրա վրա տրված $\tau=\{U_i\}_{i\in I}$ տոպոլոգիայի հետ միասին (այսինքն $(X, \tau)$ զույգը) կոչվում է \textbf{տոպոլոգիական տարածություն}։ $X$ բազմության $x\in X$ տարրերը կոչվում են տոպոլոգիական տարածության \textbf{կետեր}, իսկ $U_i \subset X$ ենթաբազմությունները (այսինքն $\tau$ տոպոլոգիայի տարրերը) կոչվում են տվյալ տոպոլոգիական տարածության \textbf{բաց ենթաբազմություններ}։
\end{definition}

\par Վերը բերված տոպոլոգիաների օրինակներից գոյանում են համապատասխան անվանումներով տոպոլոգիական տարածություններ։ Քննարկենք դրանք։ \textbf{Դիսկրետ տոպոլոգիական տարածությունում} (կնշանակենք $(X;\textrm{դիսկր․})$) բաց ենթաբազմութ\-յուն\-ներ են համարվում $X$-ի բոլոր ենթաբազմությունները։ \textbf{Անտիդիսկրետ տոպո\-լոգիա\-կան տարածությունում} (կնշակենք $(X; \textrm{անտ․})$) բաց ենթաբազմություններ են միայն $X$-ը և $\varnothing$-ը։

\par Թվային ուղղի սովորական տոպոլոգիան շահեկանորեն առանձնանում է $\R$-ի մյուս բոլոր տոպոլոգիաներից։ Այդ տոպոլոգիայի հիմքով է կառուցվում մի փոփոխա\-կանի ֆունկցիաների մաթեմատիկական անալիզը։ Ստացված տոպոլոգիական տա\-րա\-ծությունը կնշանակենք $(\R; \textrm{սովոր․})$։ Այս տարածությունում բաց ենթաբազմու\-թյուն\-ներ են $\varnothing$-ը, բոլոր $(a,b)$ ինտերվալները և նրանց միավորումները։ 

\par Այստեղից հետևում է, որ բաց ենթաբազմություններ են նաև $\R$-ը և $(-\infty;a),\ (b; +\infty)$ տեսքերի ինտերվալները։ Իսկ ահա մի կետանոց՝ $\{a\}$ տեսքի, ինչպես նաև $[a; b],\ (a; b]$, $[a; b),\ (-\infty;a],\ [b;+\infty)$ տեսքերի ենթաբազմությունները, ինչպես նաև ռացիոնալ թվերի $\Q$, իռացիոնալ թվերի $I$ ենթաբազմությունները բաց ենթաբազմություններ չեն, քանի որ չեն կարող ներկայացվել որպես $(a,b)$ տեսքի ինտերվալների միավորումներ։

\par Կնշանակենք $(\R,\textrm{վերջ․ լր․})$ և $(\R, \textrm{հաշվ․ լր․})$ սիմվոլներով այն տոպոլոգիական տարածությունները, որոնք գոյանում են թվային ուղղի համապատասխանաբար վերջավոր լրացումների և հաշվելի լրացումների տոպոլոգիաներով։ Համեմատելով դրանք միմյանց հետ, նկատենք, որ իռացիոնալ թվերի բազմությունը բաց բազմություն չէ $(\R, \textrm{վերջ․ լր․})$ տարածությունում, բայց բաց բազմություն է $(\R, \textrm{հաշվ․ լր․})$ տոպո\-լոգիա\-կան տարածությունում (ինչո՞ւ)։ 

\par Տոպոլոգիական տարածության սահմանման մեջ գործածվեցին «կետ», «տարածու\-թյուն» տերմինները, որոնք բնորոշ են երկրաչափությանը։ Դա պատահական չէ․ տոպոլոգիան երկրաչափության բաժին է, որի նպատակն է կառուցել զանազան երկրաչափություններ կամայական բազմություններում՝ հիմք ընդունելով բազմության վրա տոպոլոգիայի հասկացությունը։ Ընդհանրապես յուրաքանչյուր երկրաչափության մեջ քննարկվում է կետից կետ կարճագույն ճանապարհով անընդհատ տեղափոխվելու հնարավորությունը։ Տարրական երկրաչափությունում այդ դերը կատարում են ուղիղ\-ները կամ նրանց մասերը՝ հատվածները։ Այս կերպ ծագում է կետերի միջև մոտիկության և կետի շրջակայք հասկացությունները։ Օրինակ, մաթեմատիկական անալիզում $a$ կետի $\varepsilon$ շրջակայք ասելով հասկանում են $(a-\varepsilon; a+\varepsilon)$ ինտերվալը։ Նկատենք, որ այն բաց բազմություն է $(\R,\textrm{սովոր․})$ տոպոլոգիական տարածությունում։ Նկատենք նաև, որ ոչ բոլոր տոպո\-լոգիա\-կան տարածություններում է հնարավոր ներմուծել կետերի միջև հեռավորության հասկացություն քիչ թե շատ ընդունելի եղանակով (այդ մասին կխոսենք հաջորդ թեմայում)։ 

Այժմ ցույց տանք, որ հնարավոր է ներմուծել \textbf{կետի շրջակայք} հասկացություն բոլոր տոպոլոգիական տարածություններում որպես հիմնական այլընտրանքային հասկացություն։

\begin{definition}
$(X, \tau )$ տոպոլոգիական տարածությունում $x\in X$ կետի \textbf{բաց շրջակայք} կոչվում է այդ կետը պարունակող $X$-ի ցանկացած բաց ենթաբազմություն։ Այնուհետև, $x$ \textbf{կետի շրջակայք} կոչվում է ցանկացած $V\subset X$ ենթաբազմու\-թյուն, որն իր մեջ ընդգրկում է $x$-ի որևէ $U$ բաց շրջակայք։
\par Այսպիսով, կարճ՝ $V\subset X$ ենթաբազմությունը շրջակայք է $x$ կետի համար, եթե գոյություն ունի $U$ բաց ենթաբազմություն, որ $x\in U\subset V$։ Նկատենք նաև, որ ընդհանուր դեպքում կետի շրջակայքը կարող է չլինել այդ կետի բաց շրջակայք։
\end{definition}
\begin{example}
$(X,\textrm{դիսկր․})$ տարածությունում $x$ կետը պարունակող $X$-ի բոլոր ենթա\-բազմութ\-յուն\-ները (բաց) շրջակայքեր են $x$-ի համար։ Մասնավորապես, $\forall x \in X$ կետի համար $\{x\}$ ենթաբազմությունը $x$ կետի բաց շրջակայք է։ Իսկ $(X,\textrm{անտ․})$ տարածությունում ցանկացած կետ ունի միայն մի շրջակայք, որը ինքը՝ $X$-ն է։ Այս դեպքում արդեն $\{x\}$ ենթաբազմությունը $x$ կետի շրջակայք չէ, եթե $X$-ը պարունակում է մեկից ավելի կետեր։ Դիտարկենք ևս երկու օրինակ.
\end{example}

\begin{example} 
 $(\R,\textrm{սովոր․})$ տարածությունում $x=0{,}5$ կետի համար հետևյալ՝ $(0; 1),\ [0; 1),\ (0; 1],\ [0; 1] \cup \{3\}$ ենթաբազմությունները շրջակայքեր են, ընդ որում դրանցից միայն առաջինն է բաց շրջակայք։ Իսկ $x=0, 1, 3$ կետերի համար դրանցից ոչ մեկը շրջակայք չէ (հիմնավորե՛ք)։
\end{example}
\begin{example} 
 $(\R,\textrm{վերջ․ լր․})$ տարածությունում իռացիոնալ թվերի բազմությունը շրջակայք չէ իր կետերից ոչ մեկի համար։ Իսկ $(\R, \textrm{հաշվ․ լր․})$ տարածությունում այն շրջակայք է իր ցանկացած կետի համար (հիմնավորե՛ք)։

\end{example}
	
\begin{theorem}\label{թեորեմ 1}
$(X;\tau)$ տոպոլոգիական տարածությունում որևէ $W\subset X$ ենթաբազմու\-թյուն բաց ենթաբազմություն է այն և միայն այն դեպքում, երբ այն շրջակայք է իր ցանկացած կետի համար։ %թեորեմ 1
\end{theorem}

\begin{proof}

    Եթե $W$-ն բաց ենթաբազմություն է, ապա այն (բաց) շրջակայք է իր ցանկացած $w\in W$ կետի համար (ըստ կետի շրջակայքի սահմանման)։
Այժմ հակառակը՝ դիցուք $W$-ն շրջակայք է իր (ցանկացած) $w$ կետի համար։ Նշանակում է՝ գոյություն ունի $U(w)\in \tau$ բաց ենթաբազմություն, որ $w\in U(w)\subset W$։ Ուստի $W$-ն կարող է ներկայացվել որպես բաց ենթաբազմությունների միավորում՝ $W=\bigcup\limits_w U(w)$։ Հետևաբար $W$-ն բաց ենթաբազմություն է ըստ տոպոլոգիայի 2-րդ աքսիոմի։\qedhere
\end{proof}

\par Վերը մենք նշեցինք, որ կետի շրջակայք հասկացությունը հիմնային հասկացություն է թվային ուղղի տոպոլոգիայի համար։ Պարզվում է, որ այն հիմնային հասկացություն է ցանկացած տոպոլոգիական տարածությունում։ Իր նշանակությամբ այն հավասարազոր է բաց բազմություն հասկացությանը (պատմականորեն, տոպոլոգիայի՝ որպես մաթեմատիկայի բաժին ձևավորման առաջին փուլում տոպոլոգիական տարածությունները սահմանվել են կետերի շրջակայքերի միջոցով և կոչվել են \textbf{շրջակայքային տարածություններ}): 

\par Սա պարզաբանելու նպատակով նախ թվարկենք շրջակայքերի հիմնական հատ\-կու\-թյուն\-ները։

\begin{theorem}\label{թեորեմ 2} Դիցուք $(X; \tau )$-ն որևէ տոպոլոգիական տարածություն է։ Ապա $X$-ի կետերի շրջակայքերն օժտված են հետևյալ 4 հատկություններով.
\begin{enumerate}
\item ամեն մի $x$ կետ պատկանում է իր ցանկացած շրջակայքի,
\item եթե $V$-ն $x$ կետի շրջակայք է և $V \subset W$, ապա $W$-ն նույնպես $x$ կետի շրջակայք է,
\item $x$ կետի ցանկացած վերջավոր քանակով շրջակայքերի հատումը նորից $x$-ի շրջակայք է
\item $x$ կետի ցանկացած $V$ շրջակայքի համար գոյություն ունի այդ կետի այնպիսի $U$ շրջակայք, որ $U \subset V$ և $U$-ն շրջակայք է իր ցանկացած կետի համար։
\end{enumerate}
\end{theorem}
\begin{proof}
Առաջին հատկությունն ակնհայտ է, ցույ տանք երկրորդը։ Եթե $V$-ն $x$ կետի որևէ շրջակայք է, ապա ըստ սահմանման գոյություն ունի $x$-ի $U$ բաց շրջակայք, որ $x \in U \subset V$։ Քանի որ $V \subset W$ և $x \in U \subset W$, ուստի $W$-ն նույնպես շրջակայք է $x$ կետի համար։
Ապացուցենք 3-ը։ Դիցուք $V_1$-ը և $V_2$-ը $x$ կետի որևէ երկու շրջակայքեր են։ Ըստ սահմանման գոյություն ունեն $X$-ի $U_1$ և $U_2$ բաց ենթաբազմություններ, որ $x \in U_1 \subset V_1$ և $x \in U_2 \subset V_2$։ Ըստ տոպոլոգիայի երրորդ աքսիոմի $U_1 \cap U_2$-ը բաց ենթաբազմություն է $X$-ում, և քանի որ $x \in U_1 \cap U_2 \subset V_1 \cap V_2 $, ուստի $V_1 \cap V_2 $-ը $X$ կետիի շրջակայք է։ Այժմ հատկություն 3-ը ստացվում է պարզ ինդուկցիայով։
Ապացուցենք 4-ը։ $X$ կետի կամայական $V$ շրջակայք համար գոյություն ունի $U$ բաց ենթաբազմություն, որ $x \in U \subset V$։ Համաձայն թեորեմ 1-ի $U$-ն շրջակայք է իր ցանկացած $y$ կետի համար։ Ուստի $U \in S_y$ կամայական $y \in U $ կետի դեպքում։
\end{proof}
\par Այս 1-4 հատկությունները լիովին բնութագրում են կետերի շրջակայքերը տոպո\-լո\-գիա\-կան տարածություններում այն իմաստով, որ կարող են դիտարկվել որպես աքսիոմներ՝ բազմության վրա տոպոլոգիայի այլընտ\-րանքա\-յին սահմանման համար: ?? $(X; \tau )$  տարածության ամեն մի $x\in X$ կետի համար նշանակենք $S_x$-ով այդ կետի բոլոր շրջակայքերի բազմությունը։ 
\begin{theorem}\label{թեորեմ 3}[(շրջակայքերի միջոցով տոպոլոգիայի տրման մասին)]
Դիցուք $X$-ը կա\-մա\-յա\-կան բազմություն է, և դիցուք ամեն մի $x\in X$ տարրի ինչ-որ եղանակով համադրված է $X$-ի ենթաբազմությունների մի ինչ-որ $\hat{S}_x$ ընտանիք այնպես, որ տեղի ունեն $1^{\ast} - 4^{\ast}$ հատկությունները (աքսիոմները)։ 
$1^{\ast}.$ $x$-ը պատկանում է $\hat{S_x}$-ի բոլոր տարրերին,
$2^{\ast}.$ եթե $V\in \hat{S_x}$ և $V\subset W$, ապա $W$-ն ևս պատկանում է $\hat{S_x}$-ին,
$3^{\ast}.$ $\hat{S_x}$-ի ցանկացած վերջավոր տարրերի հատումը նույնպես պատկանում է $\hat{S_x}$-ին,
$4^{\ast}.$ ցանկացած $V\in \hat{S_x}$ տարրի համար, գոյություն ունի այնպիսի $U\in \hat{S_x}$ տարր, որ $U\subset V$ և $V\in \hat{S_y}$ ցանկացած $y\in U$ տարրի դեպքում։

Ապա $X$ բազմության վրա գոյություն ունի միակ այնպիսի $\tau$ տոպոլոգիա, որ $(X, \tau )$ տոպոլոգիական տարա\-ծութ\-յունում կետերի շրջակայքերի $S_x$ ընտանիքները համընկնում են ենթաբազմութ\-յուն\-ների $\hat{S}_x$ ընտանիքների հետ։
\end{theorem}
\begin{proof}
Սահմանենք $X$-ում ենթաբազմությունների $\tau$ ընտանիք կազմված $\emptyset$-ից  և այն բոլոր $U$ ենթաբազմություններից, որ $U$-ն պատկանում է $\hat{S_x}$ ընտանիքին  ամեն մի $x\in U$ տարրի դպեքում։ Այժմ $1^{\ast}$ և $2^{\ast}$ աքսիոմներից հետևում է, որ ինքը՝ $X$-ը պատկանում է բոլոր $\hat{S_x}$ ընտանիքներին, ուստի $X\in \tau$։ Այսինքն $\tau$-ն բավարարում է տոպոլոգիայի առաջին աքսիոմին։
    Ցույց տանք, որ $\tau$-ն բավարարում է նաև տոպոլոգիայի երկրորդ աքսիոմին։ Դիցուք $U = \bigcup\limits_i U_i$, որտեղ $U_i \in \tau$, $i\in I,$  իսկ,  $I$-ն ինդեքսների որևէ բազմություն է։ Կամայակն $x_0 \in U$ տարրի համար գոյություն ունի $U_{i_{0}}$, որ $x_0 \in U_{i_{0}}, i_0\in I$։ Այժմ $\tau$ ընտանիքի սահմանումից հետևում է, որ $U_{i_{0}}\in \hat{S}_{x_{0}}$, և քանի որ $U_{i_{0}}\in U$, ուստի $U$-ն նույնպես պատկանում է $\hat{S}_{x_{0}}$ ընտանիքին ըստ $2^\ast$ աքսիոմի։ Հետևաբար $U\in \tau$ ըստ $\tau$ ընտանիքի սահմանման։ Այսպիսով $\tau$-ն բավարարում է տոպոլոգիայի երկրորդ աքսիոմին։
        Ստուգենք տոպոլոգիայի երրորդ աքսիոմը․ դիցուք $U = U_1 \cap U_2$, որտեղ $U_1, U_2 \in \tau$ և ցույց տանք, որ $U\in\tau:$Կամայական $x_0 \in U$ տարրի դեպքում ունենք$x_0 \in U_1, x_0 \in U_2$ և ունենք նաև $U_1\in \hat{S}_{x_{0}}, U_{2}\in \hat{S}_{x_{0}}$ (ըստ $\tau$ ընտանիքի սահմանման)։ Ուստի $U\in \hat{S}_{x_{0}}$ ըստ $3^\ast$ աքսիոմի, հետևաբար $U\in \tau$ (ըստ $\tau$ ընտանիքի սահմանման)։ Ուրեմն $\tau$-ն բավարարում է նաև տոպոլոգիայի երրորդ աքսիոմին։
        Այժմ ամեն մի $x\in X$ կետի համար դիտարկենք $x$-ի բոլոր շրջակայքերի $S_x$ ընտանիքը $(X, \tau)$ տոպոլոգիական տարածությունում և ցույց տանք, որ տեղի ունի $S_x = \hat{S_x}$ նույնացումը։
        Եթե $V\in S_x$, այսինքն $V$-ն $x$ կետի որևէ շրջակայք է $(X, \tau)$ տարածությունում, ապա գոյություն ունի $U$ բաց ենթաբազմություն, որ $x\in U \in V$։ Քանի որ $U\in \hat{S}_{x_{0}}$, ուստի աքսիոմ $2^\ast$-ից ստանում ենք՝ $V \in \hat{S}_{x_{0}}$, և ուրեմն $S_x \subset \hat{S}_{x_{0}}$։ Հակառակ` $\hat{S}_{x_{0}} \subset S_x$ ներդրումը հատատելու համար դիտարկենք $X$-ի կամայական $V\in \hat{S}_{x_{0}}$ ենթաբազմություն։ Ըստ $4^\ast$ աքսիոմի գոյություն ունի $X$-ի $U\in \hat{S}_{x_{0}}$ ենթաբազմություն, որ $U\subset V$և $U$-ն պատկանում է $\hat{S}_{y}$ ընտանիքին ամեն մի $y\in U$ տարրի դեպքում։ Նշանակում է $U\in \tau$ ըստ $\tau$ ընտանիքի սահանման։ Այսպիսով ունենք՝ $x\in U \in V,$ ուրեմն $V$-ն , $x$ կետի շրջակայքմ է $\tau$ տոպոլոգիայում, ուստի $U \in S_x$։ Հետևաբար $\hat{S}_{x_{0}} \subset S_x$ և ունենք $S_x = \hat{S}_{x_{0}}$ նույնացում։
        Վերջապես նկատենք, որ $\tau$ տոպոլոգիայի միակությունը հետևանք է հետևյալ դիտողությունից. եթե $X$ բազմության որևէ երկու տոպոլոգիայում ցանկացած $x$ կետի շրջակայքերը համընկնում են, ապա այդ տոպոլոգիաները նույնական են։
\end{proof}


% \newpage
\bigskip
\bigskip
\subsubsection*{Խնդիրներ և հարցեր թեմա 4-ի վերաբերյալ}

\begin{enumerate}[label=\thesection.\arabic*.]
    \item Ապացուցեք, որ թվային ուղղի ամեն մի ինտերվալ կարող է ներկայացվել որպես $[a, b]$ տեսքի հատվածների միավորում։
    
    \item Ապացուցեք, որ թվային ուղղի վերջավոր քանակով կամայական ինտերվալ\-ների հատումը կամ դատարկ է, կամ ինտերվալ է։
    
    \item Ապացուցեք, որ թվային ուղղի անվերջ քանակով կամայական ինտերվալների հատումը կարող է լինել $ \varnothing$, $\{a\}$,  $(a,b)$, $[a,b)$, $(a,b]$, $[a,b]$ տեսքերի ենթաբազմու\-թյուններից որևէ մեկը։ Այդ 6 տեսքերից յուրաքանչյուրի համար կառուցեք այդ\-պիսի հատումների մեկական օրինակ։
    
    \item Դիտարկենք երեք տարրից կազմված որևէ $X=\{a,b,c\}$ բազմություն և նրա ենթաբազմությունների հետևյալ ընտանիքները․
    \begin{align*}
    &\Phi_1 = \{ \varnothing, \{c\}, \{a,b\},X \}, \\
    &\Phi_2 = \{ \varnothing, \{a\}, \{b\},\{a,c\},\{b,c\},X \}, \\ 
    &\Phi_3 = \{ \varnothing, \{a\}, \{b\},\{a,b,c\}\},\\ &\Phi_4 = \{ \varnothing, \{c\}, \{a,c\},\{a,b,c\} \}:
    \end{align*}

    \begin{enumerate}
        \item[ա)] Դրանցից որո՞նք են որոշում $X$ բազմության տոպոլոգիա։
        
        \item[բ)] Գրեք $X$-ի վրա իրարից տարբեր բոլոր տոպոլոգիաները։
    \end{enumerate}


    \item Որոշու՞մ է արդյոք տոպոլոգիա բոլոր բնական թվերի $\N$ բազմության վրա նրա ենթաբազմությունների հետևյալ ընտանիքը․
        \begin{enumerate}
        \item[ա)] $\{\varnothing, \{V_n \mid V_n \subset \N,\ n \geq 1\}\}$, որտեղ $V_n=\{n,n+1,\dots\}$,
        
        \item[բ)] $\{\N, \{W_n \mid W_n \subset \N,\ n \geq 1\}\}$, որտեղ $W_n=\{x\mid x \in \N \text{ և } x<n\}$:
    \end{enumerate}
    
    \item Ճի՞շտ է արդյոք, որ թվային ուղղի սովորական տոպոլոգիայում ամեն մի ոչ դատարկ բաց ենթաբազմություն կարող է ներկայացվել որպես $[a,b]$ տեսքի հատվածների միավորում։
    
    \item Ապացուցեք, որ $\R$ թվային ուղղի ենթաբազմությունների հետևյալ համա\-խմբու\-թյուններից ոչ մեկը տոպոլոգիա չէ $\R$-ի համար․
    \begin{enumerate}
        \item[ա)] $\{\varnothing,\ \R,\ \{(-\infty,x]\text{, որտեղ $x$-ը կամայական թիվ է $\R$-ում}\}$,
        
        \item[բ)] $\{\varnothing,\ \R,\ \{(a, b)\text{, որտեղ $a, b\in\R$ կամայական են և } a<b\}\}$:
    \end{enumerate}
    
    \item $\R$ թվային ուղղի ենթաբազմությունների հետևյալ ընտանիքներից որո՞նք են որոշում տոպոլոգիա $\R$-ի վրա․
    \begin{enumerate}
        \item[ա)] $\{\varnothing,\ \R,\ \{(-\infty,x) \mid x\in\R \text{ կամայական թիվ է}\}\}$,
        
        \item[բ)] $\{\varnothing,\ \R,\ \{(-\infty,x) \mid x\in\Q \text{ կամայական ռացիոնալ թիվ է}\}\}$,

        \item[գ)] $\{\varnothing,\ \{(-\infty,-x]\cup(x,+\infty) \mid x\ge 0 \text{ կամայական թիվ է}\}\}$,

        \item[դ)] $\{\varnothing,\ \R,\ \{[-x,x) \mid x>0 \text{ կամայական իռացիոնալ թիվ է}\}$։
    \end{enumerate}

    \item Դիցուք $X$-ը որևէ ոչ հաշվելի բազմություն է։ Ապացուցեք, որ օրինակ 5-ում սահմանված ենթաբազմությունների $\tau$ ընտանիքը որոշում է տոպոլոգիա $X$-ի վրա։
    
    \item Ապացուցեք, որ $\R$ թվային ուղղի հաշվելի լրացումների տոպոլոգիան ուժեղ է $\R$-ի վերջավոր լրացումների տոպոլոգիայից։
    
    \item Ապացուցեք, որ $\R$ թվային ուղղի հաշվելի լրացումների տոպոլոգիան համեմա\-տելի չէ $\R$-ի սովորական տոպոլոգիայի հետ։
    
    \item Հանդիսանու՞մ է արդյոք շրջակայք ուղղի $\sqrt{2}$ կետի համար բոլոր իռացիոնալ թվերից կազմված ենթաբազմությունը
        \begin{enumerate}
        \item[ա)] ($\R$, սովոր․) տարածությունում,
        
        \item[բ)] ($\R$, վերջ․ լր․) տարածությունում,
        
        \item[գ)] ($\R$, հաշվ․ լր․) տարածությունում։
    \end{enumerate}
\end{enumerate}


\end{document}