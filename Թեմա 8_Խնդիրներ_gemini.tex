\bigskip
\bigskip
\subsubsection*{Խնդիրներ և հարցեր թեմա 7-ի վերաբերյալ}

\begin{enumerate}[label=\thesection.\arabic*.]
Here is the exact text extraction from the provided images, preserving the original language (Armenian) and mathematical notation.

### **Image 1 (Continuation of Axioms)**

...աքսիոմների  հաջորդականության դեպքում։ Պատճառը կայանում է նրանում, որ  կամ  աքսիոմից չի հետևում ցանկացած մի կետանոց  ենթաբազմության փակությունը -ում։ Բայց եթե լրացուցիչ պահանջենք, որ -ն բավարարի նաև  աքսիոմին, ապա շնորհիվ թեորեմ 4-ի, նշված օրինաչափությունը կպահպանվի։

 -ն կոչվում է  (կամ՝ -ը ռեգուլյար է), եթե բավարարում է  և  աքսիոմներին և կոչվում է , եթե բավարարում է  և  աքսիոմներին։

Այսպիսով ռեգուլյար տարածությունները հաուսդորֆյան են, իսկ նորմալ տարածությունները ռեգուլյար են (կհիմնավորենք)։

Հայտնի են հաուսդորֆյան, բայց ոչ ռեգուլյար, և ռեգուլյար, բայց ոչ նորմալ տարածությունների օրինակներ։ Դրանց կառուցումը բավական բարդ է և այստեղ դրանք չենք քննարկի։

Հետագայում մենք կտեսնենք, որ նորմալ տարածությունների դասը բավական լայն է և իր մեջ ընդգրկում է բոլոր «լավ» տարածությունները (տես թեորեմ 6-ը թեմա 7-ում)։

---

### **Image 2 (Problems 8.1 - 8.5)**

**8.1.** Ապացուցեք.  տարածությունը բավարարում է հաշվելիության առաջին աքսիոմին։
 Դիտարկելով կամայական  կետ, ցույց տվեք որ աջից կիսափակ ինտերվալների  ընտանիքը  կետի շրջակայքների հաշվելի բազիս է։

**8.2.** Ճի՞շտ է արդյոք, որ ցանկացած  տարածություն, որտեղ -ը ոչ հաշվելի բազմություն է, չի բավարարում հաշվելիության առաջին աքսիոմին։
 Օգտվել օրինակ 2-ում բերված ապացուցման ընթացքից։

**8.3.** Ապացուցեք.  տարածությունը, որտեղ -ը ամբողջ թվերի բազմությունն է, բավարարում է հաշվելիության առաջին աքսիոմին։
 Օգտվելով թեմա 3-ի 3.8 խնդրից նախ ցույց տվեք, որ  տարածությունը բավարարում է հաշվելիության երկրորդ աքսիոմին։

**8.4.** Ճի՞շտ է արդյոք, որ  տարածությունը բավարարում է հաշվելիության առաջին աքսիոմին։
 Տես օրինակ 2-ը։

**8.5.** Գտեք մետրիկական տարածություն, որը չի բավարարում հաշվելիության երկրորդ աքսիոմին։

---

### **Image 3 (Problems 8.6 - 8.9)**

**8.6.** Ճի՞շտ է արդյոք որ գոյություն ունի ցանկացած հզորության  տոպոլոգիական տարածություն, որի յուրաքանչյուր միկետանոց ենթաբազմություն ամենուրեք խիտ է -ում։

**8.7.** Ապացուցեք. որևէ  տարածության տոպոլոգիան դիսկրետ տոպոլոգիա է այն և միայն այն դեպքում, երբ -ում գոյություն ունի միայն մի ամենուրեք խիտ ենթաբազմություն և դա հենց -ն է։

**8.8.** Ճի՞շտ է արդյոք, որ կամայական տոպոլոգիական տարածությունում ցանկացած երկու ամենուրեք խիտ ենթաբազմությունների ա) միավորումը, բ) հատումը ամենուրեք խիտ է։

**8.9.** Դիցուք  ենթաբազմությունը ամենուրեք խիտ է -ում։
Ապացուցեք. ցանկացած  բաց ենթաբազմության դեպքում  ենթաբազմության փակումը -ում համընկնում է -ի փակման հետ՝ ։
 Դիտարկենք կամայական  կետ և այդ կետի...

---

### **Image 4 (Problem 8.9 cont. & 8.10)**

...ցանկացած  բաց շրջակայք։ Հիմնվելով թեորեմ 4-ի վրա ցույց տվեք, որ  (դրանից կհետևի, որ )։

**8.10.** Ապացուցեք. Կամայական տոպոլոգիական տարածության վերջավոր քանակով ամենուրեք խիտ բաց ենթաբազմությունների հատումը նույնպես ամենուրեք խիտ բաց ենթաբազմություն է։
 Դիտարկեք  ենթաբազմություններն ամենուրեք խիտ են -ում։ Կիրառեք խնդիր 8.9-ը, վերցնելով ։

Would you like me to translate any of these topology problems or provide the mathematical definitions used (e.g., Sorgenfrey line, Cofinite topology)?

\end{enumerate}
