\bigskip
\bigskip
\subsubsection*{Խնդիրներ և հարցեր թեմա 8-ի վերաբերյալ}

\begin{enumerate}[label=\thesection.\arabic*.]

% 8.1
\item Ապացուցեք. $(\R, \to)$ տարածությունը բավարարում է հաշ-վելիության առաջին աքսիոմին:


\begin{hint}
Դիտարկելով կամայական $a \in R$ կետ, ցույց տվեք, որ աջից կիսափակ ինտերվալների $\{[a, x); x > a, x \in Q\}$ ընտանիքը $a$ կետի շրջակայքերի հաշվելի բազա է:
\end{hint}

% 8.2
\item Ճի՞շտ է արդյոք, որ ցանկացած $(X, \text{վերջ. տ.})$ տարածություն,որտեղ $X$-ը ոչ հաշվելի բազմություն է, չի բավարարում հաշվելիության առաջին աքսիոմին:

\begin{hint}
Օգտվեք օրինակ 2-ում բերված ապացուցման ընթացքից:
\end{hint}

% 8.3
\item Ապացուցեք. $(Z, \text{վերջ. տ.})$ տարածությունը, որտեղ $Z$-ը ամբողջ թվերի բազմությունն է, բավարարում է հաշվելիության առաջին աքսիոմին:

\begin{hint}
Օգտվելով թեմա 3-ի 3.8 խնդրից նախ ցույց տվեք,որ $(Z, \text{վերջ. տ.})$ տարածությունը բավարարում է հաշվելիության երկրորդ աքսիոմին:
\end{hint}

% 8.4
\item Ճի՞շտ է արդյոք, որ $(\mathbb{R}, \text{հաշվ. տ.})$ տարածությունը բավարարում է հաշվելիության առաջին աքսիոմին:

\begin{hint}
Տես օրինակ 2-ը:
\end{hint}

% 8.5
\item Գտեք մետրիկային տարածություն, որը չի բավարարում
հաշվելիության երկրորդ աքսիոմին:

% 8.6
\item Ճի՞շտ է արդյոք որ գոյություն ունի ցանկացած հզորության $X$ տոպոլոգիական տարածություն, որի յուրաքանչյուր միկետանոց ենթաբազմություն ամենուրեք խիտ է $X$-ում:

% 8.7
\item Ապացուցեք. որևէ $X$ տարածության տոպոլոգիան դիսկրետ տոպոլոգիա է այն և միայն այն դեպքում, երբ $X$-ում գոյություն ունի միայն մի ամենուրեք խիտ ենթաբազմություն, և դա ինքը՝ $X$-ն է:

% 8.8
\item Ճի՞շտ է արդյոք, որ կամայական տոպոլոգիական տարածությունում ցանկացած երկու ամենուրեք խիտ ենթաբազմությունների ա) միավորումը, բ) հատումը ամենուրեք խիտ է:

% 8.9
\item Դիցուք $A$ ենթաբազմությունը ամենուրեք խիտ է $X$-ում: Ապացուցեք. ցանկացած $U \subset X$ բաց ենթաբազմության դեպքում$A \cap U$ ենթաբազմության փակումը $X$-ում համընկնում է $U$-ի փակման հետ՝ $\overline{A \cap U} = \overline{U}$:

\begin{hint}
Դիտարկենք կամայական $x \in \overline{U}$ կետ և այդ կետի ցանկացած $V$ բաց շրջակայք: Հիմնվելով թեորեմ 4-ի վրա ցույց տվեք, որ $V \cap (A \cap U) \neq \varnothing$ (դրանից կհետևի, որ $\overline{U} \subset \overline{A \cap U}$):
\end{hint}

% 8.10
\item Ապացուցեք. կամայական տոպոլոգիական տարածության վերջավոր քանակով ամենուրեք խիտ բաց ենթաբազմությունների հատումը նույնպես ամենուրեք խիտ բաց ենթաբազմություն է:

\begin{hint}
    Դիցուք $U_1, U_2 \subset X$ ենթաբազմություններն ամենուրեք խիտ են $X$-ում: Կիրառեք խնդիր 8.9-ը, վերցնելով $A=U_1$,$U=U_2$:
\end{hint}

\end{enumerate}
