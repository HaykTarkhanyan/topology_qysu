\bigskip
\bigskip
\subsubsection*{Խնդիրներ և հարցեր թեմա 7-ի վերաբերյալ}
% 9, 10 tegherob poxel a

\begin{enumerate}[label=\thesection.\arabic*.]

10.1. Նկարագրեք բոլոր զուգամետ հաջորդականությունները կապակցված կետային տոպոլոգիայով տարածությունում՝ $X = \{x_1, x_2\}, \ \mathcal{T} = \{\emptyset, \{x_1\}, \{x_1, x_2\}\}$։10.2. Դիցուք $X$-ը անվերջ բազմություն է վերջավոր լրացումների տոպոլոգիայով։ Ապացուցեք, որ ցանկացած $\{x_n\} \subset X$ հաջորդականություն կազմված $X$-ի տարբեր կետերից զուգամիտում է $X$-ի ցանկացած կետի։10.3. Մաթեմատիկական անալիզի դասընթացում ապացուցվում է $(\mathbb{R}, \text{սովոր.})$-ում մոնոտոն և սահմանափակ ամեն մի հաջորդականություն ունի միակ սահման։ Ապացուցեք, որ այսպիսի հաջորդականությունները զուգամետ չեն $(\mathbb{R}, \text{վերջ.})$ տարածությունում։10.4. Դիտարկենք որևէ հաշվելի $X = \{x_1, x_2, \dots\}$ բազմություն։ Սահմանեք $X$-ի համար այնպիսի $\mathcal{T}$ տոպոլոգիա, որ $(X, \mathcal{T})$ տարածությունում բոլոր զուգամետ հաջորդականություններն ունենան միևնույն սահմանը։$\underline{Ցուցում։}$ Քննարկեք $\{\emptyset, X, \{x_1\}, \{x_1, x_2\}, \{x_1, x_2, x_3\}, \dots\}$ ընտանիքը։Image: image_06399d.jpg10.5. Դիտարկենք որևէ $(X, \text{հաշվ. լր.})$ տարածություն և դիցուք $A$-ն $X$-ի որևէ ոչ հաշվելի ենթաբազմություն է։ Ապացուցեք՝ա) ցանկացած $b \in X \setminus A$ կետ հպման կետ է $A$-ի համար;բ) գոյություն չունի $\{a_n\} \subset A$ հաջորդականություն, որ $\lim a_n = b$։10.6. Ապացուցեք․ եթե $\{x_n\} \subset X$ հաջորդականությունն ունի $\lim x_n = a$ սահման, ապա $a$-ն պատկանում է $\{x_n\}$ փակմանը՝ $a \in \overline{\{x_n\}}$։10.7. Ապացուցեք․ ցանկացած $\{x_n\} \subset X$ զուգամետ հաջորդականության ամեն մի $\{y_n\}$ ենթահաջորդականություն ունի նույն սահմանը (սահմանները), ինչը որ ունի $\{x_n\}$-ը։10.8. Հաջորդականության ենթահաջորդականություն սահմանելիս պահանջեցինք $h: \mathbb{N} \to \mathbb{N}$ արտապատկերման գոյությունը, որը պետք է բավարարի $h(i) > h(j)$ պայմանին, ամեն անգամ երբ $i > j$։$\underline{Հարց։}$ Ինչպիսի՞ հետևանքներ կարող են առաջանալ, եթե սահմանման մեջ վերոհիշյալ սահմանը փոխարինենք $h(i) \ge h(j)$, երբ $i > j$ ավելի թույլ պայմանով։Image: image_063995.jpgՀաջորդ երեք խնդիրներում հստակեցվում է հարցադրումը։10.9. Ճի՞շտ է արդյոք, որ ցանկացած ստացիոնար հաջորդականության ամեն մի ենթահաջորդականություն նույնպես ստացիոնար հաջորդականություն է։10.10. Ճի՞շտ է արդյոք պնդումը․ $\{x_n\} \subset X$ զուգամետ հաջորդականության ցանկացած ենթահաջորդականություն ա) նույնպես զուգամետ է, բ) ունի նույն սահմանները, ինչը որ ունի $\{x_n\}$ հաջորդականությունը։10.11. Նշանակենք $\{\lim x_n\}$ սիմվոլով $\{x_n\} \subset X$ հաջորդականության բոլոր սահմանների բազմությունը (այն կարող է լինել նաև $\emptyset$ բազմությունը)։ Ճի՞շտ է արդյոք պնդումը․ $\{x_n\}$ հաջորդականության ցանկացած $\{y_n\}$ ենթահաջորդականության դեպքում տեղի ունի $\{\lim y_n\} \subset \{\lim x_n\}$ ներդրումը:


\end{enumerate}
