\documentclass[./main.tex]{subfiles}

\begin{document}
% էս հռոմեական մեկ-ի պահը կդզենք հետո
\begin{section}
{Տոպոլոգիա(\rom{1} միջանկյալ քննություն)}
    Քննական տարբերակը պարունակում է $4$ հարց․ մեկական հարց հետևյալ $4$ բաժիններից՝ 
    \begin{enumerate}
        \item[1.]բազմությունների տեսություն,
        \item[2.]տոպոլոգիան բազմության վրա,
        \item[3.]կետի շրջակայք, տոպոլոգիայի բազա
        \item[4.]ենթաբազմության ներքնամաս, ենփաբազմության փակում
    \end{enumerate}
    $1$-ին հարցում կարող է լինել
    \begin{enumerate}
        \item [ա)]ֆորմուլների ապացուցում կապված բազմությունների միավորման, հատման, տար\-բե\-րու\-թյան հետ։ Օրինակ։ Ապացուցել, որ $(\bigcup\limits_{i=1}^nAi)\bigcap B=\bigcup\limits^n_{i=1}(Ai\bigcap B)$ և այլն։
        \item [բ)] Կարող է լինել ֆորմուլների ապացուցում կապված բազմությունների ար\-տա\-պատ\-կե\-րում\-նե\-րի հետ։ Օրինակ ունենք $f:x\rightarrow y,Ai\subset X$:Ապացուցել, որ $f(\bigcup\limits _iAi)=\bigcup\limits _if(Ai)$: Կամ, ունենք $f:X\rightarrow Y,Bi\subset Y,i \in I$: Ապացուցել, որ $f^{-1}(\bigcup\limits _iBi)=U_if^{-1}(Bi)$:
        \item [գ)]Կարող է լինել ոչ բարդ խնդիր համարժեքության հարաբերություն, ֆակտոր բազմություն թեմայից։
        \item [դ)]Խնդիր բազմությունների հավասարազորության վերաբերյալ։  
\end{enumerate}
Պարզեցնենք դ)-ն:\\
Հիշեցնենք, որ երկու $X$ և $Y$ բազմություններ կոչվում են հավասարազոր, եթե գոյություն ունի փոխմիարժեք (բիյեկտիվ) համապատասխանություն նրանց տար\-րե\-րի միջև։ Ստորև $X$ և $Y$ բազմությունների հավասարազորությունը կգրառենք $X\sim Y$ տեսքով։ 
Դիտարկենք $\mathds{R}$ թվային ուղղի հետևյալ ենթաբազմությունները՝ 
$(a,b),$ $ [a,b),(a,b], [a,b], (-\infty,a),(-\infty,a],(a,+\infty),[a,+\infty) (\ast)$:
Դրանք բոլորը միմյանց հա\-վա\-սա\-րա\-զոր բազմություններ են։
\begin{example}
 Ցույց տանք, որ $(2,3)$ ինտերվալը և $[2,3]$ հատվածը հավասարազոր բազմություններ են։\\
Ներկայացնենք $(2,3)$-ը $(2,3)=\mathds{Q}'\bigcup \mathds{I}'$ տեսքով, որտեղ $\mathds{Q}'$-ն $(2,3)$ ինտերվալի բոլոր ռացիոնալ թվերի բազ\-մու\-թյունն է, իսկ $\mathds{I}'$-ը՝ $(2,3)$-ի բոլոր իռացիոնալ թվերի բազ\-մու\-թյունն է ։ Նման ձևով դիտարկենք $[2,3]$ հատվածի $(2,3)=\mathds{Q}''\bigcup \mathds{I}''$ ներկայացումը։ Պարզ է, որ $\mathds{Q}''=\mathds{Q}'\bigcup {2,3}, \mathds{I}''=\mathds{I}'$:Քանի որ $\mathds{R}$ թվային ուղղի բոլոր ռացիոնալ թվերի $\mathds{Q}$ բազմությունը հաշվելի անվերջ բազմություն է(հետևանք թեորեմ $3$-ից թեմա $3$-ում), ուստի նրա $\MATHDS{Q}'$ անվերջ ենթաբազմությունը նույնպես հաշվելի անվերջ բազմություն է։Մյուս կողմից $\mathds{Q}''=\mathds{Q}'\bigcup {2;3}$ բազմությունը նույնպես հաշվելի անվերջ բազմություն է որպես հաշվելի անվերջ $\mathds{Q}'$ բազմության և ${2;3}$ հաշվելի (վերջավոր) բազմության միավորում (տես թեորեմ $3$-ը թեմա $3$-ում)։Այսպիսով \mathds{Q}'-ը և $\mathds{Q}''$-ը լի\-նե\-լով հավասարազոր բնական թվերի $\mathds{N}$ բազմությանը, հավասարազոր են միմյանց` $\mathds{Q}'\sim \mathds{Q}''$: Նշանակում է գոյություն ունի $\varphi:\mathds{Q}'\rightarrow \mathds{Q}''$  փոխմիարժեք հա\-մա\-պա\-տաս\-խա\-նու\-թյուն (արտապատկերում)։ Ունենք նաև $\Uppsi:\mathds{I}'\rightarrow \mathds{I}''$ (նույնական) փոխ\-մի\-ար\-ժեք ար\-տա\-պատ\-կե\-րում։ Ուստի $\varphi$-ն և $\Uppsi$-ն միասին որոշում են $(2,3)\rightarrow[2,3]$ փոխմիարժեք արտապատկերում։ Հետևաբար $(2,3)\sim [2,3]$ ։
\end{example}
\begin{example}
Տեղի ունի $(\sqrt{2},5]\sim [\sqrt{2},5)$ հավասարազորություն։Ապացուցվում է օրինակ $1$-ի նմանությամբ։Դիտարկենք $(\sqrt{2},5)=\mathds{Q}'\bigcup \mathds{I}'$ ներկայացումը։ Ունենք $(\sqrt{2},5]=\mathds{Q}'\bigcup{5}\bigcup\mathds{I}', [\sqrt{2},5)=\mathds{Q}'\bigcup{\sqrt{2}}\bigcup\mathds{I}'$։ Ունենք ակնհայտ $f:\mathds{Q}'\bigcup{5}\rightarrow\mathds{Q}'\bigcup{\sqrt{2}}$ փոխմիարժեք համապատասխանություն (նույնական $\mathds{Q}$'-ի վրա և $f(5)=\sqrt{2}$) , ինչպես նաև $g:\mathds{I}'\rightarrow\mathds{I}'$ նույնական համապատասխանություն։ Դրանց միջոցով ստանում ենք $(\sqrt{2},5]\rightarrow[\sqrt{2},5)$ փոխմիարժեք համապատասխանություն։ Ուրեմն $(\sqrt{2},5]\sim[\sqrt{2},5)$։
Վերը բերված $\ast$ շարքի ցանկացած երկու բազմությունների հա\-վա\-սա\-րա\-զո\-րու\-թյունը կարելի է հաստատել, օգտվելով հետևյալ փոխմիարժեք ար\-տա\-պատ\-կե\-րում\-նե\-րից։
\begin{enumerate}
    \item [1.]$U:(-\infty;a)\rightarrow (-\infty;0)$, որտեղ $U(x)=x-a$;
    \item [2.]$\mho:(a;+\infty)\rightarrow(0;+\infty)$,որտեղ $\mho(x)=x-a$;
    \item [3.]$\omega:(-\infty;0)\rightarrow(0;+\infty)$, որտեղ $\omega(x)=-x$;
    \item [4.]$f:(a,b)\rightarrow(0;1)$, որտեղ $f(x)=\dfrac{x-a}{b-a}$;
    \item [5.]$g:(-\infty;0)\rightarrow(0;1)$, որտեղ $g(x)=\dfrac{1}{1-x}$;
    \item [6.]$h:(0;+\infty)\rightarrow(0;1)$, որտեղ $h(x)=\dfrac{x}{x+1}$:
\end{enumerate}
\end{example}
\begin{example}
    Ապացուցենք $(-5;1)\sim [3;4]$ հավասարազորությունը։ Երկու անգամ օգտվենք $4$ ձևափոխությունից։
    \begin{enumerate}
        \item[ա)] $f_1:(-5;1)\rightarrow(0;1),f_1(x)=\dfrac{x+5}{6}$
        \item[բ)] $f_2:[3;4]\rightarrow[0;1],f_2=x-3$
    \end{enumerate}
    Այսպիսով ունենք $(-5;1)\sim (0;1)$ և $[3;4]\sim[0;1]$ հավասարազորություններ։ Մնում է ցույց տալ $(0;1)\sim[0;1]$ հավասարազորությունը, որը կատարվում է ինչպես օրինակ $1$-ում։
\end{example}
\begin{example}
    Ապացուցենք $(2;5]\sim(-\infty;4)$ հավասարազորությունը։ Ըստ $4$ ձևա\-փո\-խու\-թյան ունենք $f:(2;5]\rightarrow(0;1],f(x)=\dfrac{x-2}{3}$, ըստ $1$ ձևափոխության ունենք $U:(-\infty;4)\rightarrow(-\infty;0)$, $U(x)=x-4$: Ըստ $5$ ձևափոխության ունենք $g:(-\infty;0)\rightarrow(0;1)$, $g(x)=\dfrac{1}{1-x}$։Հետևաբար $g\ocircU$ համադրույթը ապահովում է $(-(-\infty;4)\sim(0;1)$ հավասարազորություն, իսկ $f$`$(2;5]\sim(0;1]$ հա\-վա\-սա\-րա\-զո\-րու\-թյուն։ Մնում է ցույց տալ հավասարազորութւյունը, որ կատարվումէ ինչպես օրինակ $1$-ում։ 
\end{example}
\end{section}
\begin{sectiom}{Բազմություններ և արտապատկերումներ}
Ապացուցել, որ
\begin{enumerate}
    \item [1.] $A-\bigcup\limits^n_{i=1}B_i=\bigcap\limits^n_{i=1}(A\setminus B_i)$,$A\setminus\bigcap\limits^n_{i=1}B_i=\bigcup\limits^n_{i=1}(A\setminus B_i)$
    \item [2.]$(\bigcup\limits^n_{i=1}A_i)\bigcap B=\bigcup\limits^n_{i=1}(A_i\bigcap B)$, $(\bigcup\limits^n_{i=1}A_i)\bigcup B=\bigcap\limits^n_{i=1}(A_i\bigcup B)$
    \item [3.]$A\setminus(B\setminus C)=(A\setminus B)\bigcup(A\bigcap C)$
    \item [4.]$(A\setminus B)\setminus C=A\setminus(B\bigcup C)$
    \item [5.]$(A\setminus B)\setminus C=(A\setminusC)\setminus(B\setminus C)$
\end{enumerate}
Ունենք ։ Ապացուցել 
\end{document}
