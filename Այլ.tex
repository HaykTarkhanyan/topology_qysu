\documentclass[./main.tex]{subfiles}

\usepackage[document]{ragged2e}
\setlength{\RaggedRightParindent}{\parindent}

\begin{document}
\RaggedRight
%\iraggedright

\section{Ներածություն}

Օրինակի համար փորձենք գրենք Թեմա 10-ը, 
գրքի հղումը սա \textcolor{blue}{\url{https://drive.google.com/file/d/1t7GouY9XHaCEfxCtURgO6ECMvh5DvAfr/view}} (Էջ 75)

Առաջին պարբերությունը տեսնում ենք որ մի մատ խորքիցա, դրա համար պետքա գրենք \par , այսինքն՝

\par Մինչև այժմ մենք դիտարկում էինք տոպոլոգիական տարածություններն առանձին-առանձին, միմյանցից անկախ։ Այժմ զբաղվելու ենք դրանց համեմատումով։ Այդ նպատակով ներմուծվում է տոպոլոգիական տարածությունների անընդհատ արտապատկերման հասկացությունը, որը երկրորդ կարևորագույն հասկացությունն է տոպոլոգիական տարածություն հասկացությունից հետո։

\textbf{Հաջորդ պարբերության սկզբնական տեսքը}՝ 

Դիցուք ունենք X,Yտոպոլոգիական տարածություններ ևf:X→Y արտապատկերում:Ապա f-ը կոչվում է  անընդհատ x0∈X կետում, եթեy0=f(x0)∈Y կետի ամեն միV շրջակայքի համար գոյություն ունիx0 կետիU շրջակայք, որf(U)⊂V:
\\

\textbf{վերջնարդյունքը}


\begin{definition}
Դիցուք ունենք $X$, $Y$ տոպոլոգիական տարածություններ ևf:X→Y արտապատկերում:Ապա $f$-ը կոչվում է  անընդհատ $x_0 \in X$ կետում, եթե $y_0=f(x_0) \in Y$ կետի ամեն մի $V$ շրջակայքի համար գոյություն ունի $x_0$ կետի $U$ շրջակայք, որ $f(U) \subset V$ :
\end{definition}

տեսնում ենք որ Սահմանում է տրված, դրա համար ունենք արդեն սահմանված հրաման, սկզբում պետք է գրենք բեգին դեֆինիշն ու պարբերության վերջում՝ էնդ դեֆինիշն (հիշում եք չէ՞ որ հայերենով գրելիս լատիներեն էլ չէր ստացվում գրել :) )

նկատեք որ սահմանում բառը չենք, գրում, ինքը արդեն նախապատրաստածա որ հասկանա ավտոմատ գրել 

տեսնում ենք որ պռաբելներ կան բաց թողած, ու որոշ հավասարումներ ունենք, հիմա պետքա դրանց դզելուն անցնենք՝

նախ "դիցուք ունենքից" հետո իքս և իգրեկը դնենք դոլարի նշանների մեջ, ապա "տարածությունների և-ից հետո ֆ իքս սլաք իգրեկը, դա ուղղելու համար գրում ենք $f: X \Rightarrow Y $, Ապա-ից հետո ֆ-ն էլ դնենք դոլարի նշանների մեջ, հետո գալիսա իքս 0 ն մեծատառ իքս մասը, դա կգրենք՝ $x_i \in X$, ապա հաջորդ հավասարումը կլինի $y_0 = f(x_0) \in Y$, հետո մնաց Վ-ն իքս 0ն ՈՒ-ն ու վերջին արջտահայտությունը, դրանք էլ կլինեն՝ $V$, $x_0$, $f(U) \subset V$, վերջնական տեսքը կլինի՝ 



Հաղորդ պարբերությունը արդեն գիտենք ոնց անենք որ մեկ մատ խորքից լինի՝ 

\par Հետևյալպնդումը երբեմն հեշտացնում է անընդհատության պայմանի ստուգումը։

\textbf{Հաջորդ պարբերությունումը}՝ 

Թեորեմ 1։ Ունենքf:X→Yարտապատկերում,f(x_0 )=y_0, իսկβ_(x_0 )={U_i (x_0 ),i∈I} և β_(y_0 )={V_j (y_0 ),j∈J}ընտանիքները համապատասխանաբարx_0 ևy_0 կետերի շրջակայքերի որևէ բազաներ ենX ևY տարածություններում: Ապա f-ը անընդհատ էx_0 կետում այն և միայն այն դեպքում, երբ ∀V_j (y_0)∈β_(y_0 ) համար∃U_i (x_0 )∈β_(x_0 ),, որ f(U_i (x_0 ))⊂V_j (y_0):

իսկ հիմա այն ինչ պետքա դարձնենք՝
\begin{theorem}
Ունենք $f: X \Rightarrow Y$ արտապատկերում, $f(x_0) = y_0$, իսկ $\beta_{x_0} = {U_i(x_0), i \in I}$ և  $\beta_{(y_0)}=\{V_j (y_0), j \in J\}$ ընտանիքները համապատասխանաբար $x_0$ և $y_0$ կետերի շրջակայքերի որևէ բազաներ են $X$ և $Y$ տարածություններում: Ապա $f$-ը անընդհատ է $x_0$ կետում այն և միայն այն դեպքում, երբ $\forall\ V_j (y_0) \in \beta_{(y_0)}$ համար $\exists\ U_i (x_0 ) \in \beta_{(x_0)}$, որ $f(U_i (x_0)) \subset V_j (y_0)$:
\end{theorem}

Ապացույցի մնացած մասը անում ենք նույն ձև ուղղակի գրելով theorem բառի փոխարեն proof

հաջորդ մի քանի պարբերությունը նախորդներին նման են, հետո երևի գալիսա հետևանք թեորեմ մեկից ը այդտեղ գրում ենք՝ 
\begin{hetevanq4}
պարբերության տեքստը
\end{hetevanq4}

էջ 81ում(թեմա 11) կարող ենք տեսնել որ համարակալում կա՝ 
դա կարող ենք անել հետևյալ կերպ 
\begin{enumerate}
    \item թիվը կարող ենք չգրել, ավտոմատա դա արվում
    \item[բ] եթե այսենք փակ փակագծում ենք դնում, ինքը հասկանումա որ պետքա թվի փոխարեն գրի բ
\end{enumerate}

Հաստ գրել ու նման բաները կարող եք նայել Արամի ֆայլ անունով ֆայլում, ինչպես նաև ուղղակի գուգլելով արագ կգտնեք, իսկ նշանների հարցով տելեգրամում ֆայլ ենք տեղադրել, ու ընդահանուր ինչ-որ բաներ հաստատ էս ֆայլում չենք ասել, ինչ անհասկանալի բան լինի միանգամից գրեք, ու ապրեք շատ որ միացաք պռոեկտին ու էս ֆայլը լրիվ կարդացիք :)
\end{document}
