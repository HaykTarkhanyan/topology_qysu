\documentclass[./main.tex]{subfiles}
\DeclareMathOperator{\exter}{ext}

\begin{document}
\onehalfspacing


% ISSUES:
% - Տերմիններ․  մետրիկ -> մետրիկա, մետրիկ -> մետրիկական/մետրական/մետրիկային/․․․


\section{Մետրիկ բազմության վրա և մետրիկ տարածություն հասկացությունները, օրինակներ։ Մետրիկ տարածության տոպոլոգիան, մետրիկային տարածություններ, օրինակներ։ Ենթաբազմության ներքինը, արտաքինը և փակույթը մետրիկային տարածություններում։}\label{sec:7}


%\par \textbf{Մետրական տարածություններ։} % էխ...
\par Մետրիկ տարածությունները սահմանվել են ֆրանսիացի մաթեմատիկոս Մ․ Ֆրե\-շեի կողմից 1906 թվին։ Նպատակն է՝ կամայական բազմություններում ներ\-մու\-ծե\-լով կետերի միջև հեռավորության, հետևաբար և՝ կետերի մոտիկության հասկացու\-թյուն, զարգացնել անալիզին և երկրաչափությանը բնորոշ ամենաընդհանուր հատ\-կություններով տեսություն։
\par Այնուհետև, մետրիկ տարածության հիմքով, արդեն տոպոլոգիայում ներմուծ\-վե\-ցին մետրիկային տարածությունները որպես տոպոլոգիական տարածություններ։ Այժմ հաջորդաբար քննարկենք այդ երկու հասկացությունները։
\par Մետրիկ տարածության հիմքում ընկած է բազմության վրա մետրիկ հասկացու\-թյունը։ Դիցուք $X$-ը որևէ բազմություն է, $\R$-ը թվային ուղիղն է։

\begin{definition}Որևէ $\rho:X \times X \rightarrow \R$ որևէ արտապատկերում կոչվում է \textbf{մետրիկ} $X$ \textbf{բազմու\-թյան վրա}, եթե կամայական $x, y, z \in X$ տարրերի դեպքում բավարար\-վում են հետևյալ երեք պայմանները․
\begin{enumerate}
    \item $\rho(x,y)=0 \Leftrightarrow x=y$ (նույնականացման աքսիոմ),
    \item $\rho(x,y)=\rho(y,x)$ (համաչափության աքսիոմ),
    \item $\rho(x,z) \leq \rho(x,y)+\rho(y,z)$ (եռանկյան աքսիոմ)։
\end{enumerate}
\end{definition}

\begin{hetevanqax}
Վերցնելով 3-ում $z=x$ ստանում ենք՝ $\rho(x,y) \geq 0$, $\forall x,y \in X$ տարրերի դեպքում։ Այժմ տեսնում ենք, որ բազմության վրա մետրիկը տարրական երկրաչափությունից հայտնի կետերի միջև հեռավորության ընդհանրա\-ցումն է կա\-մա\-յա\-կան բազմությունների դեպքում։
\end{hetevanqax}

\begin{definition} $(X,\rho)$ զույգը (այսինքն $X$ բազմությունը՝ իր վրա տրված $\rho$ մետրիկի հետ միասին) կոչվում է \textbf{մետրիկ տարածություն}, իսկ $\rho(x,y)$ թիվը՝ $x$ և $y$ \textbf{կետերի միջև հե\-ռա\-վո\-րութ\-յուն} տվյալ մետրիկ տարածությունում։
\end{definition}

\begin{example} \label{օրինակ 1}
Ցանկացած $X$ բազմություն կարելի է վերածել մետ\-րիկ տա\-րա\-ծութ\-յան առնվազն մի եղանակով՝ $\rho(x,y)=1$, երբ $x \neq y$ և $\rho(x,y)=0$, երբ $x=y$։ Այս մետրիկ տարածությունը կոչվում է \textbf{դիսկրետ մետրիկ տարածություն} (անվանումը կպարզաբանվի մի փոքր ուշ)։
\end{example}


\begin{example} \label{օրինակ 2}
$\R$ թվային ուղղի վրա սահմանենք մետրիկ՝ $\rho(x,y)= \abs{x-y}$ բանա\-ձևով։ Այն կոչվում է թվային ուղղի \textbf{սովորական} կամ \textbf{էվկլիդյան մետրիկ}, իսկ $   (\R, \rho)$ զույգը կոչվում է էվկլիդյան թվային ուղիղ։
\end{example}


\par 1-3 աքսիոմների ստուգումը օրինակներ \hyperref[օրինակ 1]{1}, \hyperref[օրինակ 2]{2}-ում թողնվում է ընթերցողին՝ որպես հեշտ, բայց օգտակար խնդիր։ Հաջորդ կարևոր օրինակը ընդհանրացնում է \hyperref[օրինակ 2]{օրինակ 2}-ի էվկլիդ\-յան մետրիկը։

\begin{example} \label{օրինակ 3}
Դիտարկենք $n$ չափականության $\R^n$ կոորդինատային տարածությու\-նը և սահմանենք $x=(x_1,\dots,x_n )$, $y=(y_1,\dots,y_n)$ տարրերի միջև հեռավորություն $\rho(x,y)=\sqrt{(x_1-y_1)^2+ \dots +(x_n-y_n)^2}$ բանաձևով։ Մետրիկի 1-2 աքսիոմները ակն\-հայ\-տո\-րեն բավարարվում են, իսկ 3-րդ աքսիոմը (սովորաբար դժվարություն է այս աքսիոմի ստուգումը) հետևում է Կոշի-Բունյակովսկու ան\-հա\-վա\-սա\-րու\-թյու\-նից՝
\[ \sum_{k=1}^n a_k b_k \leq \sqrt{\sum_{k=1}^n a_k^2}\cdot \sqrt{\sum_{k=1}^n b_k^2},
\]
որն արդարացի է թվերի կամայական $a_1, a_2, \dots, a_n$ և $b_1, b_2, \dots, b_n$ հաջորդա\-կա\-նու\-թյունների դեպքում։
\par Իրոք, $\forall x,y,z \in \R^n$ տարրերի համար 3-րդ աքսիոմն ընդունում է 
\[\sqrt{\sum_{k=1}^n (x_k-z_k)^2}\leq \sqrt{\sum_{k=1}^n (x_k-y_k)^2} +\sqrt{\sum_{k=1}^n (y_k-z_k)^2}
\]
տեսքը։ Նշանակելով $x_k-y_k=a_k,\ y_k-z_k=b_k$ ստանում ենք՝ $x_k-z_k=a_k+b_k$, և 3-րդ աքսիոմն ընդունում է նոր՝ $\sqrt{\sum\limits_k (a_k+b_k)^2} \leq \sqrt{\sum\limits_k a_k^2} + \sqrt{\sum\limits_k b_k^2}$ տեսք։ Այն համարժեք է ${\sum\limits_k a_k \cdot b_k \leq \sqrt{\sum\limits_k a_k^2}} \cdot {\sqrt{\sum\limits_k b_k^2}}$ անհավասարությանը։ Այսպիսով մնում է ապացուցել Կոշի-Բունյակովսկու անհավասարությունը։

\par Նկատենք, որ կամայական $t \in \R$ թվի դեպքում ունենք $\sum\limits_k \left(a_k-t\cdot b_k\right)^2 \geq 0$ հավաստի անհավասարություն, որը կարելի է գրել $\left(\sum\limits_k b_k^2\right) t^2-2\left(\sum\limits_k a_k b_k\right)t+\left(\sum\limits_k a_k^2 \right) \geq 0$ տեսքով։ Իսկ դա հնարավոր է միայն այն դեպքում, երբ տեղի ունի $\left(\sum\limits_k a_k b_k\right)^2 \leq \left(\sum\limits_k a_k^2\right)\cdot\left(\sum\limits_k b_k^2\right)$ անհավասարությունը, որից էլ ստանում ենք \linebreak $\sum\limits_k a_k b_k \leq \sqrt{\sum\limits_k a_k^2} \cdot \sqrt{\sum\limits_k b_k^2}$։ \qed
\end{example}

\begin{example} Եթե $\rho$-ն մետրիկ է $X$ բազմության վրա, ապա ցանկացած $c>0$ թվի դեպքում $c\rho: X \times X \rightarrow \R $ արտապատկերումը (սահմանվում է $(c\rho)(x,y)=c\cdot\rho(x,y)$ բանաձևով) ակնհայտորեն նույնպես մետրիկ է $X$-ի վրա։ Մետրիկ է նաև $\widebar{\rho} = \dfrac{\rho}{1+\rho}$ արտապատկերումը՝ սահմանված $\widebar{\rho}(x,y) = \dfrac{\rho(x,y)}{1+\rho(x,y)}$ բանաձևով։ Ստուգենք 3-րդ աքսիոմը $\widebar{\rho}$-ի համար։ Նախ նկատենք, որ
\begin{center}
$\widebar{\rho}(x,z) = \dfrac{\rho(x,z)}{1+\rho(x,z)} \leq \dfrac{\rho(x,y) + \rho(y,z)}{1+\rho(x,y) + \rho(y,z)}$
\end{center}
(ստուգվում է անմիջականորեն՝ ազատվելով հայտարարներից)։ Այնուհետև
\begin{equation*}
    \begin{aligned}
    \widebar{\rho}(x,z) &\leq \dfrac{\rho(x,y)}{1+\rho(x,y) + \rho(y,z)} +  \dfrac{\rho(y,z)}{1 + \rho(x,y) + \rho(y,z)} \leq  \dfrac{\rho(x,y)}{1+\rho(x,y)} + \dfrac{\rho(y,z)}{1+\rho(y,z)} =\\ &= \widebar{\rho}(x,y) + \widebar{\rho}(y,z)
    \end{aligned}
\end{equation*}
\qed
\end{example}
\par Նշենք, որ ստացված $(X, \widebar{\rho})$ մետրիկ տարածությունում  $\widebar{\rho}(x,y) < 1,\ \forall x,y \in X$ կետերի դեպքում։

% +

\begin{example}
Դիտարկենք $\R^{n+1}$ էվկլիդյան կոորդինատային տարածության $O$\linebreak  սկզբնակետով անցնող բոլոր ուղիղների բազմությունը։ Այն վերածվում է մետրիկ տարածության՝ սահմանելով նրա վրա այսպես կոչված \textbf{անկյունային մետրիկ։}
\par Պարզության նկատառումով բոլոր անհրաժեշտ դատողությունները կկատարենք $n=2$ մասնավոր դեպքում։
\par Նախ $\R^3$-ի $O$ կետով անցնող յուրաքանչյուր $l$ ուղղի վրա ընտրենք մի որոշակի $L$ կետ այնպես, որ $\overrightarrow{OL}$ վեկտորն ունենա միավոր երկարություն՝ $|\overrightarrow{OL}|=1$:
\par Այժմ սահմանենք կամայական երկու $l_1$ և $l_2$ ուղիղների միջև հեռավորություն $d(l_1,l_2)=\arccos|(\overrightarrow{OL_1}, \overrightarrow{OL_2})|$ բանաձևով (այստեղ $(\overrightarrow{OL_1},\overrightarrow{OL_2})$-ը $\overrightarrow{OL_1}$ և $\overrightarrow{OL_2}$ վեկ\-տորնե\-րի սկալյար արտադրյալն է, իսկ $d(l_1,l_2)$-ը անկյուն է $[0^{\circ},90^{\circ}]$ միջակայքից)։
\par Պարզ է, որ $d(l_1,l_2)$-ը բավարարում է մետրիկի 1-2 աքսիոմներին։ Ցույց տանք, որ այն բավարարում է նաև երրորդ աքսիոմին՝
\begin{equation}
d(l_1,l_2)+d(l_2,l_3)\geq d(l_1,l_3) \tag{$\ast$}
\end{equation}
ցանկացած երեք $l_1,l_2,l_3$ ուղիղների դեպքում։ Նկատենք, որ եթե այդ ուղիղներից որևէ երկուսը նույնն են, ապա ($\ast$)-ը բավարարվում է։
\par Այժմ դիտարկենք որևէ երեք ոչ համահարթ $l_1$, $l_2$, $l_3$ ուղիղներ։ Այս դեպքում $OL_1$, $OL_2$, $OL_3$ ճառագայթները կազմում են $O$ գագաթով եռանիստ անկյուն։ Հետա\-գա հիմ\-նա\-վո\-րում\-ներում օգտվելու ենք տարրական տա\-րա\-ծա\-չա\-փու\-թյու\-նից հայտ\-նի հետև\-յալ թեո\-րե\-մից․ կա\-մա\-յա\-կան $(O; OL_1, OL_2, OL_3)$ եռանիստ անկյան երեք՝ $\alpha = \angle L_1OL_2$, $\beta = \angle L_2OL_3$, $\gamma = \angle L_1OL_3$ հարթ անկյուններից ցանկացած երկուսի գումարը մեծ է երրորդից (նշենք, որ դրանց մեծությունները գտնվում են $(0^{\circ},180^{\circ})$ միջակայքում)։
\par Կախված $l_i$ ուղիղների վրա $L_i$, $i=1,2,3$ կետերի դիրքերից՝ հնարավոր են հե\-տև\-յալ չորս դեպքերը․
\begin{enumerate}
\item եթե $\alpha,\beta,\gamma$ անկյունները սուր են, ապա $d(l_1,l_2)=\alpha,\;d(l_2,l_3)=\beta,\;d(l_1,l_3)=\gamma$,
    
\item եթե $\alpha,\beta,\gamma$ անկյունները բութ են, ապա $d(l_1,l_2)=180^{\circ}-\alpha,\;d(l_2,l_3)=180^{\circ}-\beta,\;d(l_1,l_3)=180^{\circ}-\gamma$,

\item եթե $\alpha,\beta,\gamma$ անկյուններից որևէ երկուսը, օրինակ՝ $\alpha$-ն և $\beta$-ն բութ են, իսկ երրորդը՝ $\gamma$-ն սուր է, ապա $d(l_1,l_2)=180^{\circ}-\alpha,\;d(l_2,l_3)=180^{\circ}-\beta,\;d(l_1,l_3)=\gamma$,

\item եթե $\alpha,\beta,\gamma$ անկյուններից որևէ երկուսը, օրինակ՝ $\alpha$-ն և $\beta$-ն սուր են, իսկ երրորդը՝ $\gamma$-ն բութ է, ապա $d(l_1,l_2)=\alpha,\;d(l_2,l_3)=\beta,\;d(l_1,l_3)=180^{\circ}-\gamma$:
\end{enumerate}

\par Հասարակ գծագրի միջոցով կարելի է տեսնել, որ 1-3 դեպքերից յուրաքանչյու\-րում $d(l_1,l_2),\ d(l_2,l_3),\ d(l_1,l_3)$ անկյունները $O$ գագաթով որոշակի եռանիստ անկյան հարթ անկյուններ են, ուստի ($\ast$)-ը տեղի ունի ըստ վերը բերված թեորեմի։ Իսկ 4-րդ դեպքում այդ երեք անկյունները չեն հանդիսանում որևէ եռանիստ անկյան հարթ անկյուններ։ Ուստի այս դեպքում ($\ast$)-ը կապացուցենք իր բոլոր ա), բ), գ) տարբե\-րակ\-նե\-րով՝ օգտվելով ստորև բերվող սխեմատիկ գծագրից, որտեղ մի աղեղով նշված է $\gamma$ բութ անկյունը, իսկ երկուական աղեղներով՝ սուր անկյունները։

\begin{center}
\begin{tikzpicture}

% Define coordinates for the diagram
\coordinate (O) at (0, 0);
\coordinate (A1) at (-3,-1.5);  % endpoint for ℓ₁
\coordinate (A2) at (3,1.5);    % endpoint for ℓ₃
\coordinate (B1) at (-3,1.5);   % endpoint for ℓ₂
\coordinate (B2) at (3,-1.5);   % endpoint for ℓ₄
\coordinate (C1) at (0, 2.5);
\coordinate (C2) at (0, -2.5);
\coordinate (alpha_angle) at (-0.65, -0.75);
\coordinate (beta_angle) at (0.3, -0.9);
\coordinate (gamma_angle) at (0.6, -1.4);

% node O
\filldraw (O) circle (1pt) node[above right, yshift=3, xshift=-2] {$O$};

% angles alpha, beta, gamma
\draw pic[draw=black, double, angle radius=6.5mm] {angle=A1--O--C2};
\draw (alpha_angle) node[right, scale=0.7] {$\alpha$};
\draw pic[draw=black, double, angle radius=8mm] {angle=C2--O--B2};
\draw (beta_angle) node[right, scale=0.7] {$\beta$};
\draw pic[draw=black, angle radius=14mm] {angle=A1--O--B2};
\draw (gamma_angle) node[right, scale=0.7] {$\gamma$};
% angle 180 - alpha (left)
\draw pic[draw=black, double, angle radius=10mm, node font=.5] {angle=B1--O--A1};
\node[scale=0.7] at ($(O)!16mm!(180:1)$) {$180^\circ-\alpha\;\;$};
% angle 180 - alpha (right)
\draw pic[draw=black, double, angle radius=10mm, node font=.5] {angle=B2--O--A2};
\node[scale=0.7] at ($(O)!16mm!(0:1)$) {$\;\;180^\circ-\alpha$};


% l1
\draw (A1) -- (A2);
\draw (A1) node[left] {$l_1$};
\draw (A2) node[right] {$l_1$};
% l2
\draw (C1) -- (C2);
\draw (C2) node[right] {$l_2$};
\draw (C1) node[right] {$l_2$};
% l3
\draw (B1) node[below left] {$l_3$} -- (B2) node[right] {$l_3$};

\end{tikzpicture}
\end{center}

Քանի որ
\begin{enumerate}
\item[ա)]  $\alpha + \beta > \gamma > 180^{\circ} - \gamma$, ուստի $d(l_1,l_2)+d(l_2,l_3) > d(l_1,l_3)$,

\item[բ)] $\beta + (180^{\circ}-\gamma)>180^{\circ} - \alpha>\alpha$, ուստի $d(l_2,l_3)+d(l_1,l_3) > d(l_1,l_2)$,

\item[գ)] $\alpha + (180^{\circ}-\gamma)>180^{\circ} - \beta>\beta$, ուստի $d(l_1,l_2)+d(l_1,l_3) > d(l_2,l_3)$:
\end{enumerate}

\par Մյուս դեպքում, երբ $l_1,l_2,l_3$ ուղիղները համահարթ են, ($\ast$)-ը բավարարվում է ակնհայտորեն։ \qed
\end{example}

\par \hyperref[Օրինակ 5]{Օրինակ 5-ի} մետրիկ տարածությունը ընդհանուր դեպքում կոչվում է \textbf{$n$-չա\-փա\-կա\-նու\-թյան իրական պրոյեկտիվ տարածություն} և նշանակվում է $\mathbb{R} \mathbb{P}^n$, կամ պարզա\-պես $\mathbb{P}^n$: Մասնավոր, $n=2$ դեպքում $\mathbb{R} \mathbb{P}^2$ տարածությունը կոչվում է նաև \textbf{իրական պրոյեկտիվ հարթություն}։ Այդ տարածությունները կարևոր դեր են կատարում դասական երկրաչափությունում և տոպոլոգիայում։


\begin{definition}
Երկու $(X_1, \rho_1)$ և $(X_2, \rho_2)$ մետրիկ տարածություններ կոչվում են \textbf{իզոմետրիկ տարածություններ}, եթե գոյություն ունի $\varphi : X_1 \rightarrow X_2$ բիյեկտիվ ար\-տա\-պատ\-կե\-րում, որ $\rho_1(x,y) = \rho_2(\varphi(x), \varphi(y)),\ \forall x, y \in X$ կետերի դեպքում։
\end{definition}

Հեշտ է տեսնել, որ $X = \mathbb{R}$ թվային ուղղի վրա \hyperref[օրինակ 1]{օրինակներ 1}-ում և \hyperref[օրինակ 2]{2}-ում սահման\-ված մետրիկներից ստացվող մետրիկ տա\-րա\-ծու\-թյուն\-ները իզոմետրիկ տա\-րա\-ծութ\-յուն\-ներ չեն (ինչո՞ւ)։
\par Իզոմետրիկ չեն նաև $(X, \rho)$ և $(X, \widebar{\rho})$ մետրիկ տարածությունները, եթե թեկուզ մի զույգ $x,y \in X$ կետերի դեպքում $\rho(x,y)\geq 1$։
\par Իսկ \hyperref[օրինակ 1]{օրինակ 1}-ում վերցնելով մի դեպքում $X = \mathbb{Q}$, իսկ մյուս դեպքում՝ $X = \mathbb{Z}$, ստանում ենք իզոմետրիկ տարածություններ (հիմնավորե՛ք)։

\begin{definition}
Դիցուք $(X, \rho)$-ն որևէ մետրիկ տարածություն է, $a\in X$, $r>0$։
Հետևյալ ենթաբազմությունները՝
\begin{align*}
&\mathcal{D}(a, r) = \{x\in X\mid \rho(a, x) < r\}, \\
&\mathcal{B}(a, r) = \{x\in X\mid \rho(a, x) \leq r\}, \\
&\mathcal{S}(a, r) = \{x\in X\mid \rho(a, x) = r\},
\end{align*}
կոչվում են $(X,\rho)$ մետրիկ տարածության հա\-մա\-պա\-տաս\-խա\-նա\-բար $a$ կենտրոնով և $r$ շառավղով \textbf{անեզր գունդ, եզրով գունդ} և \textbf{սֆերա}։
    
\par Այս անվանումները մեկնաբանելու համար նշենք, որ $\mathcal{B}(a, r)$ և $\mathcal{D}(a, r)$ գնդերի համար եզր է համարվում $\mathcal{S}(a, r)$ սֆերան։ Այսպիսով $\mathcal{B}(a, r)$ գունդը պարունակում է իր եզրը, իսկ $\mathcal{D}(a, r)$ գունդը՝ ոչ։
\end{definition}

\begin{example}
\label{օրինակ 6}
Դիսկրետ $(X,\rho)$ մետրիկ տարածությունում ունենք․
\par եթե $r < 1$, ապա $\mathcal{D}(a, r) = \mathcal{B}(a, r) = \{a\}$ և $\mathcal{S}(a, r) = \varnothing$;
\par եթե $r > 1$, ապա $\mathcal{D}(a, r) = \mathcal{B}(a, r) = X$ և $\mathcal{S}(a, r) = \varnothing$;
\par իսկ $r = 1$ դեպքում՝ $\mathcal{D}(a, 1) = \{a\},\ \mathcal{B}(a, 1) = X,\ \mathcal{S}(a, 1) = X \setminus \{a\}$։
\end{example}

\begin{example} \label{օրինակ 7}Պարզաբանենք՝ ինչ են $\mathbb{R}^n$ էվկլիդյան մետրիկ տարածության անեզր և եզրով գնդերն ու սֆերաները $n = 1$, $2$, $3$ դեպքերում։
\par $n=1$ դեպքում ունենք՝ 
\begin{align*}
&\mathcal{D}(a, r) = \{ x \in \mathbb{R}: |x-a| < r\} = (a-r, a+r), \\
&\mathcal{B}(a, r) = \{ x \in \mathbb{R}: |x-a|\leq r \} = [ a-r, a+r ], \\ 
&\mathcal{S}(a, r) = \{ x \in \mathbb{R}: |x-a| = r \} = \{a-r, a+r\}։
\end{align*}
Դրանք էվկլիդյան $\mathbb{R}$ թվային ուղղի բաց ինտերվալները, փակ հատվածներն ու կետազույգերն են։
\par $n=2$ դեպքում ունենք՝ 
\begin{align*}
&\mathcal{D}(a,r) = \{ (x_1, x_2) \in \mathbb{R}^2 : (x_1 - a_1)^2 + (x_2 - a_2)^2 < r^2 \}, \\
&\mathcal{B}(a, r) = \{ (x_1, x_2) \in \mathbb{R}^2 : (x_1 - a_1)^2 + (x_2 - a_2)^2 \leq r^2 \}, \\ 
&\mathcal{S}(a, r) = \{ (x_1, x_2) \in \mathbb{R}^2 : (x_1 - a_1)^2 + (x_2 - a_2)^2 = r^2 \}։
\end{align*}
Դրանք $\mathbb{R}^2$ էվկլիդյան հարթության բաց և փակ շրջաններն են ու շրջանագծերը։
\par $n=3$ դեպքում՝ դրանք $\mathbb{R}^3$ էվկլիդյան տարածության անեզր և եզրով գնդերն են ու գնդային մակերևույթները (այդ տերմինների սովորական իմաստով)։
\end{example}

% this used to be theorem 2, now it's theorem 1 because WHERE ARE THE T_0, T_1, T_2 AXIOMS?!
\begin{theorem}
$(X,\rho)$ մետրիկ տարածության բոլոր $\mathcal{D}(x,r)$ անեզր գնդերի բազ\-մու\-թյունը հանդիսանում է բազա $X$-ի ինչ-որ տոպոլոգիայի համար (այդ տոպոլոգիան կոչվում է $X$-ի \textbf{մետրիկային տոպոլոգիա}, իսկ $B=\{\mathcal{D}(x, r)\}$ ընտանիքը՝ նրա կանո\-նա\-կան բազա։
%որի մասին ասում են, որ այն ծնված է $X$-ի $\rho$ մետրիկայով)։
\end{theorem}
\label{թեորեմ 2}
% \renewcommand*{\proofname}{\hspace{18pt}\textbf{Ապացուցման}\nopunct}
% \begin{proof} 

\textbf{Ապացուցման} հիմքում ընկած է \textcolor{red}{թեմա 5-ի թեորեմ 2}-ը, ըստ որի՝ պետք է ստու\-գենք երկու պայման։



\begin{enumerate}
    \item $\bigcup \mathcal{D}(x,r)=X$ (հետևում է նրանից, որ միշտ $x \in \mathcal{D}(x, r))$։
    \item Ամեն մի ոչ դատարկ $\mathcal{D}(a,r_1) \cap \mathcal{D}(b,r_2)$ հատում ներկայացվում է որպես անեզր գնդերի միավորում։ Բավական է ցույց տալ, որ ցանկացած $x \in \mathcal{D}(a,r_1) \cap \mathcal{D}(b,r_2)$ կետի համար գոյություն ունի $\mathcal{D}(x,r)$ գունդ, որ $\mathcal{D}(x,r) \subset \mathcal{D}(a,r_1) \cap \mathcal{D}(b,r_2)$։ Այդ գնդի $r$ շառավղի մեծությունը որոշելու համար դիտարկենք մաս\-նա\-վոր դեպք․ որպես $(X, \rho)$ մետրիկ տարածություն վերցնենք $\mathbb{R}^2$ հար\-թու\-թյունը սովորական էվկլիդյան մետրիկով (տե՛ս գծագիրը)։
\end{enumerate}

\begin{center}
    \includegraphics[scale=0.7]{images/id12.png}
\end{center}

Ունենք՝ $\rho(a,x)<r_1,\rho(b,x)<r_2$։ Գծագրից երևում է․ եթե $r$-ը այնպիսին է, որ
\[
\begin{cases}
    \mathcal{D}(x,r) \subset D(a,r_1)\\
    \mathcal{D}(x,r) \subset D(b,r_2)
\end{cases}\hspace{-1em},
\text{ ապա }
\begin{cases}
       \rho(a,x) + r \leq r_1\\
       \rho(b,x)+r \leq r_2
\end{cases} \hspace{-1em}\ \Rightarrow\
\begin{cases}
    r \leq r_1 - \rho(a,x)\\
    r \leq r_2 - \rho(b,x)
\end{cases}
\]

Ուստի, ընդհանուր դեպքում որպես $\mathcal{D}(x,r)$ գնդի որոնելի շառավիղ վերցնենք $r=\min(r_1-\rho(a,x),{r_2-\rho(b,x))}$ թիվը։ Այժմ դիտարկենք $\forall y\in \mathcal{D}(x,r)$ կետ և ցույց տանք, որ $y\in \mathcal{D}(a,r_1)\cap \mathcal{D}(b,r_2)$։ Դրա\-նից կհետևի, որ $x\in \mathcal{D}(x,r)\subset \mathcal{D}(a,r_1)\cap \mathcal{D}(b,r_2)$։

Կարող ենք գնահատել՝ 
\[ 
\rho(a,y)\leq \rho(a,x)+\rho(x,y)<\rho(a,x)+r\leq \rho(a,x)+r_1-\rho(a,x)=r_1,
\]
ուստի $y\in \mathcal{D}(a,r_1)$։ Նման ձևով ստանում ենք $y\in \mathcal{D}(b,r_2)$, ուստի $y\in \mathcal{D}(a,r_1)\cap \mathcal{D}(b,r_2)$։\qed 

\begin{note}
Ապացույցի ընթացքում $\mathcal{D}(x,r)$ գնդի $r$ շառավղի մեծությունը որոշվեց տեսողաբար՝ $\mathbb{R}^2$ հարթության մասնավոր օրինակի միջոցով։ Անդրա\-դառ\-նալով ապացույցի մանրամասներին՝ նշենք, որ ամենասկզբում օգտվեցինք հետևյալ գծագրային հուշումից․ եթե $\mathcal{D}(x,r)\subset \mathcal{D}(a,r_1)$, ապա $\rho(a,x)+r\leq r_1$։ Այլ, ավելի ընդհանուր ձևակերպումով դա հնչում է այսպես․ եթե կամայական մետրիկ տարա\-ծու\-թյունում մի գունդ ընկած է մեկ այլ գնդի մեջ, ապա առաջին գնդի շառավիղը փոքր կամ հավասար է երկրորդ գնդի շառավղից։ Բոլոր փորձերը այս պնդումը դարձնել թեորեմ (դուրս բերել մետրիկի 1-3 աքսիոմներից) դատա\-պարտ\-ված են անհաջողության, քանի որ իրողությունն այլ է․ գոյություն ունեն մետրիկ տարա\-ծու\-թյուն\-ներ, որոնցում փոքր շառավղով գունդն իր մեջ ներառում է ավելի մեծ շառավղով գունդ՝ չհամընկնելով նրա հետ (ընթերցողը կարող է ծանոթանալ այդ\-պիսի օրինակի հետ Б. Гелбаум, Дж. Олмстед, "Контрпримеры в анализе", 1967 գրքում)։ Արժեզրկվո՞ւմ է արդյոք սրանով \hyperref[թեորեմ 2]{թեորեմ 1}-ի վերը բերված ապացույցը։ Սկզբունքորեն՝ ոչ, քանի որ ձևականորեն բերված ապացույցն անթերի է։ Բայց հոգեբանորեն ինչ-որ տեղ՝ գուցե։ Թերևս սա է պատճառը, որ А.Н. Колмо\-го\-ров, С.В. Фомин, "Элементы теории функции и функционального анализа" դա\-սա\-գրքի երկրորդ գլխում հեղինակները գերադասել են մետրիկ տարածություն\-ներում մետրիկային տոպոլոգիա սահմանել համարժեք, բայց այլ եղանակով՝ դրանով իսկ խուսափելով $\mathcal{D}(x,r)$ գնդի շառավղի մեծության վերոհիշյալ ընտրությունից։ Այդ եղանակը հիմնված է Կուրատովսկու փակման գործողության վրա։ Կարծում ենք՝ ընթերցողին հետաքրքիր և օգտակար կլինի ծանոթանալ նաև այդ եղանակի հետ։
\end{note}

\renewcommand*{\proofname}{\hspace{18pt}\textbf{Ապացուցում։}\nopunct}
Որպես հետևանք \hyperref[թեորեմ 2]{թեորեմ 1}-ից ստանում ենք, որ անեզր գնդերը բաց բազ\-մու\-թյուն\-ներ են տվյալ մետրիկային տոպոլոգիայում (այդ պատճառով կոչվում են նաև \textbf{բաց գնդեր})։ Ցույց տանք նաև, որ $\mathcal{B}(a,r)$ եզրով գնդերը փակ բազմություններ են մետրիկային տոպոլոգիայում (այդ պատճառով կոչվում են նաև \textbf{փակ գնդեր})։ 

\begin{center}
\begin{tikzpicture}
    % Draw circles
    \draw (0,0) circle (2cm);
    \draw (3,0) circle (1cm);

    % Draw radii
    \draw (0,0) -- (2,0) node[midway,above] {$r$};
    \draw (3,0) -- (2,0) node[midway,above] {$r_1$};

    % Add dots for the centers of the circles
    \filldraw[black] (0,0) circle (1.5pt);
    \filldraw[black] (3,0) circle (1.5pt);

    % Label centers
    \node at (0,-0.2) {$a$};
    \node at (3,-0.2) {$x$};

    % Adjust point y position higher
    \filldraw[black] (2.7,-0.5) circle (1pt) node[below right] {$y$};
\end{tikzpicture}
\end{center}

Դրա համար բավական է ապացուցել, որ $X\setminus \mathcal{B}(a,r)$ լրացումը շրջակայք է իր կամայական $x$ կետի համար, և հետևաբար բաց բազմություն է։ Սա հետևում է նրանից, որ $\mathcal{D}(x,r_1)\subset X\setminus \mathcal{B}(a,r)$, որտեղ $r_1=\rho(a,x)-r$։ Իրոք, $\forall y\in \mathcal{D}(x,r_1)$ \linebreak կետի համար ունենք՝
\[
\rho(a,y)\geq \rho(a,x)-\rho(y,x)>\rho(a,x)-r_1=r, \text{ ուստի } \mathcal{D}(x,r_1) \subset X\setminus \mathcal{B}(a,r)։
\]

Նման ձևով ապացուցվում է, որ $\mathcal{S}(a,r)$ սֆերաները փակ բազմություններ են մետրիկային տոպոլոգիայում (հիմնավորե՛ք)։

% Used to be theorem 3
\begin{theorem}
\label{թեորեմ 3}
$(X,\rho)$ մետրիկային տարածության $A\subset X$ ենթաբազմությունը բաց ենթաբազմություն է տվյալ մետրիկային տոպոլոգիայում այն և միայն այն դեպքում, երբ իր կամայական $a$ կետի հետ միասին պարունակում է $a$ կենտրոնով որևէ բաց գունդ։
\end{theorem}

\begin{center}
    \includegraphics[scale=0.75]{images/id14.png}
\end{center}

\begin{proof}
Ինչպես արդեն գիտենք, եթե $A$-ն $(X, \rho)$-ի որևէ բաց բազմություն է, ապա այն բաց գնդերի միավորում է (ըստ \hyperref[թեորեմ 2]{թեորեմ 1}-ի)։ Մնում է ցույց տալ, որ եթե $a \in A$ կետը պատկանում է որևէ $\mathcal{D}(b,r)$ բաց գնդի, ապա $\mathcal{D}(b,r)$-ն իր մեջ պարունակում է $a$ կենտրոնով որևէ բաց գունդ։

Վերցնենք $r'=r-\rho(a,b)$ և ցույց տանք, որ $\mathcal{D}(a,r')\subset \mathcal{D}(b,r)$։
Եթե $y\in \mathcal{D}(a,r')$, ապա $\rho(y,b)\leq \rho(y,a)+\rho(a,b)<r'+\rho(a,b)=r$ գնահատումից հետևում է, որ $y\in \mathcal{D}(b,r)$, ուստի $\mathcal{D}(a,r')\subset \mathcal{D}(b,r)$։ Թեորեմի հակառակ պնդումն ակնհայտ է։
\end{proof}

\begin{example}
\hyperref[թեորեմ 2]{Թեորեմ 1}-ից, \hyperref[օրինակ 2]{օրինակներ 2}-ից և \hyperref[օրինակ 7]{7}-ից հետևում է, որ $\mathbb{R}$ թվային ուղղի էվկլիդյան մետրիկից ստացվող մետրիկային տոպոլոգիան համընկնում է $\mathbb{R}$-ի սովորական տոպոլոգիայի հետ։
\end{example}
\begin{example}
    \hyperref[թեորեմ 2]{Թեորեմ 1}-ից, \hyperref[օրինակ 3]{օրինակներ 3}-ից և \hyperref[օրինակ 7]{7}-ից հետևում է, որ $\mathbb{R}^2$ էվկլիդյան հարթության մետրիկային տոպոլոգիայի համար կանոնական բազա են կազմում բոլոր անեզր (բաց) շրջանները, իսկ $\mathbb{R}^3$ էվկլիդյան տարածության համար՝ բոլոր սովորական անեզր (բաց) գնդերը (կամայական կենտրոններով և շառավիղներով)։
\end{example}

Մետրիկային տարածություններն օժտված են կարևոր հատկությամբ՝ բավարա\-րում են անջատելիության $\textrm{T}_2$ աքսիոմին։ Իրոք, եթե $x_1$, $x_2 \in X$ և $x_1\neq x_2$, ապա $\rho(x_1,x_2) \neq 0$, ուստի $D(x_1, \frac{1}{2} \rho(x_1,x_2))$ և $D(x_2, \frac{1}{2} \rho(x_1,x_2))$ բաց գնդերը $x_1$ և $x_2$ կետերի չհատվող շրջակայքեր են։ Դա ցույց տալու համար ենթադրենք հակառակը՝ գո\-յու\-թյուն ունի նրանց պատկանող որևէ $y$ կետ։ Ըստ եռանկյան աքսիոմի՝ $\rho(x_1, x_2) \leq \rho(x_1,y) + \rho(x_2,y) < \frac{1}{2}\rho(x_1,x_2)+\frac{1}{2}\rho(x_1,x_2) = \rho(x_1,x_2)$, հակասություն։
% \end{proof}

\begin{definition} 
Միևնույն $X$ բազմության վրա տրված երկու մետրիկ կոչվում են \textbf{համարժեք}, եթե նրանցով ծնված մետրիկային տոպոլոգիաները նույնն են։
\end{definition}

\begin{example}
Ցույց տանք, որ նույն $X$-ի վրա $\rho$ և $\widebar{\rho} = \dfrac{\rho}{1+\rho}$ մետրիկները համար\-ժեք են (չնայած, որ ընդհանուր դեպքում $(X,\rho)$ և $(X,\widebar{\rho})$ մետրիկ տարա\-ծու\-թյուն\-ները իզոմետրիկ չեն)։ Իրոք, $(X,\rho)$ տարածության ամեն մի $\mathcal{D}(a,r)\subset X$ բաց գունդ կարող է դիտարկվել որպես $(X,\widebar{\rho})$ տարածության $\widebar{\mathcal{D}}(a,R)\subset X$ բաց գունդ, որտեղ $R=\dfrac{r}{1+r}$, և հակառակը՝ $(X,\widebar{\rho})$ տարածության ամեն մի $\widebar{\mathcal{D}}(a, R)$ գունդ, եթե $R<1$, կարող է դիտարկվել որպես $(X,\rho)$ տարածության $\mathcal{D}(a,r)$ գունդ, որտեղ $r=\dfrac{R}{1-R}$։ Պարզունակ ստուգումը թողնվում է ընթերցողին։
\end{example}

\begin{definition}
$(X,\tau)$ տոպոլոգիական տարածությունը կոչվում է \textbf{մետրիկացվող տա\-րա\-ծու\-թյուն}, եթե $X$ բազմության վրա գոյություն ունի որևէ $\rho$ մետրիկ այնպես, որ $X$ մետրիկային տոպոլոգիան համընկնում է $X$-ի $\tau$ տոպոլոգիայի հետ։
\end{definition}

\begin{example}
Ցանկացած $(X, \textrm{դիսկր․})$ տարածություն մետրիկացվող տարա\-ծութ\-յուն է, քանի որ նրա տոպոլոգիան համընկնում է $X$-ի վրա $\rho(x,y)= \begin{cases} 0, \text{ երբ } x=y \\ 1, \text{ երբ } x \neq y \end{cases}$ դիսկրետ մետրիկով ծնված տոպոլոգիայի հետ։
\par Իրոք, ըստ \hyperref[օրինակ 6]{օրինակ 6-ի} և \hyperref[թեորեմ 2]{թեորեմ 1}-ի՝ $X$-ում ամեն մի $\mathcal{D}(a,r)$ գունդ $r<1$ դեպքերում $X$-ի մի կետանոց $\{a\}$ բաց ենթաբազմությունն է։ Հետևաբար $X$-ի բոլոր ենթաբազմությունները բաց բազ\-մութ\-յուն\-ներ են տվյալ մետրիկային տոպոլոգիա\-յում։
\end{example}

\par Բերենք նաև չմետրիկացվող տարածության օրինակ․ մեկից ավելի կետեր պա\-րու\-նակող ամեն մի ($X$, անտիդ․) տարածություն չի կարող մետրիկացվել, քանի որ այն չի բավարարում անջատելիության $\mathrm{T}_2$ աքսիոմին։

\par Վերջում, անդրադառնալով մետրիկ տարածություններին՝ նշենք, որ բազմության վրա տրված մետրիկը թույլ է տալիս սահմանել հեռավորության հասկացություն ոչ միայն երկու կետերի միջև, այլև՝ կետի և ենթաբազմության, ինչպես նաև երկու ենթաբազմությունների միջև։

\begin{definition}
$(X, \rho)$ մետրիկ տարածության որևէ $b$ կետի հեռավորություն $X$-ի որևէ $A$ ենթաբազմությունից կոչվում է $\inf\{\rho(b,a);\ a \in A\}$ թիվը (նշանակվում է՝ $\rho(b,A)$):
\end{definition}

\begin{theorem}
$x \in X$ կետը պատկանում է $(X, \rho)$ մետրիկային տարածության $A \subset X$ ենթաբազմության $\widebar{A}$ փակույթին այն և միայն այն դեպքում, երբ $\rho(x, A)= 0$:
\end{theorem}

\begin{proof}
Եթե $x \in \widebar{A}$, ապա ըստ ենթաբազմության փակույթի սահմանման՝ ամեն մի $\mathcal{D}(x, r_n)$, $r_n = \frac{1}{n}$ բաց գնդում գոյություն ունի $A$ ենթաբազմության որևէ $a_n$ կետ, ուստի $\rho(x, A) = 0$:
\par Այժմ հակառակը․ դիցուք ինչ-որ $x \in X$ կետի դեպքում $\rho(x, A)= 0$: Դիտարկենք այդ կետի կամայական $U$ շրջակայք և ցույց տանք, որ $U\cap A \neq \varnothing$ (դրանից կհետևի, որ $x \in \widebar{A}$): Ըստ մետրիկային տոպոլոգիայի սահմանման և \hyperref[թեորեմ 3]{թեորեմ 2}-ի՝ գոյություն ունի $x$ կետի $\mathcal{D}(x,r)$ բաց գնդային շրջակայք, որ $\mathcal{D}(x,r) \subset U$: Այժմ $\rho(x,A)$ հեռա\-վո\-րու\-թյան սահմանումից և $\rho(x,A)=0$ պայմանից հետևում է․ գոյություն ունի որևէ $a\in A$ կետ, որ $\rho (x,a) < \varepsilon$: Քանի որ $a \in \mathcal{D}(x,r) \subset U$, ուստի $a\in U \cap A$: Ուրեմն $U \cap A \neq \varnothing$:
\end{proof}

\begin{theorem}
$(X, \rho)$ մետրիկային տարածության $x$ կետը $A \subset X$ ենթաբազմության արտաքին կետ է այն և միայն այն դեպքում, երբ $\rho (x, A) > 0$:
\end{theorem}

\begin{proof}
Դիցուք $x \in \exter A$: Ըստ ենթաբազմության արտաքին կետի սահ\-ման\-ման՝ գոյություն ունի $x$-ի որևէ $U$ շրջակայք, որ $U \subset X \setminus A$: Հետևաբար գոյու\-թյուն ունի նաև $x$-ի բաց գնդային $\mathcal{D}(x,r)$ շրջակայք, որ $\mathcal{D}(x,r) \subset U$:
Ուստի $\mathcal{D}(x,r) \subset X \setminus A$, և ուրեմն $\mathcal{D}(x,r) \cap A = \varnothing$: Այժմ, ենթադրելով, որ $\rho(x,A)$ թիվը դրական չէ, ստանում ենք $\rho(x,A) = 0$, որից (ըստ նախորդ թեորեմի) հետևում է $x \in \widebar{A}$: Իսկ դրանից հետևում է՝ $\mathcal{D}(x,r) \cap A \neq \varnothing$ (հակասություն)։
\par Ապացուցենք նաև հակառակ պնդումը․ դիցուք $\rho(x,A)=r>0$: Ցույց տանք, որ $\mathcal{D}(x,r) \subset X \setminus A$ (դրանից կհետևի, որ $x \in \exter A$): Ցանկացած $y \in \mathcal{D}(x,r)$ կետի դեպքում ըստ մետրիկի եռանկյան աքսիոմի ունենք՝ $\rho(x,y) + \rho(y,a) \geq \rho(x,a)$, որտեղ $a$-ն կամայական կետ է $A$-ից։ Քանի որ $\rho(x,a)>r$ և $\rho (x,y) = r$, ուստի $\rho(y,a) > 0$: Իսկ դրանից հետևում է՝ $\mathcal{D}(x,r) \subset X \setminus A$։
\end{proof}

\begin{theorem}
    $(X, \rho)$ մետրիկային տարածության $x$ կետը $A \subset X$ ենթաբազմության եզրային կետ է այն և միայն այն դեպքում, երբ $\rho (x,A) = \rho (x, X \setminus A) = 0$:
\end{theorem}
\par Ապացույցը որպես հեշտ խնդիր թողնվում է ընթերցողին։

\newpage 

\bigskip
\bigskip
\subsubsection*{Խնդիրներ և հարցեր թեմա 7-ի վերաբերյալ}

\begin{enumerate}[label=\thesection.\arabic*.]
% 7.1 
\item Ապացուցեք, որ $$\rho : \R \times \R \to \R, \qquad \rho(x,y)=(x-y)^2$$ արտապատկերումը չի որոշում մետրիկ $\R$ թվային ուղղի վրա։ 

% 7.2
\item Ցույց տվեք, որ $n>1$ դեպքերում 
\[
\rho: \R^n \times \R^n \to \R, \qquad \rho(x, y ) = \min_{1\le i \le n} |x_i-y_i|,
\]
\[
x=(x_1, x_2, \dots, x_n),\qquad y=(y_1, y_2, \dots, y_n)
\] 
արտապատկերումը չի որոշում մետրիկ $\R^n$-ում։

% 7.3
\item Ապացուցեք․ $X$ բազմության վրա մետրիկի 1-3 աքսիոմները համարժեք են հետևյալ 
երկու աքսիոմներին՝
\begin{enumerate}
\item[1.] $\rho(x,y)=0$ այն և միայն այն դեպքում, երբ $x=y$,
\item[2.] $\rho(x,y) \le \rho(z,x)+\rho(z,y)$ բոլոր $x, y, z \in X$ տարրերի դեպքում։
\end{enumerate}


% 7.4
\item Ապացուցեք․ եթե $\rho$-ն որևէ մետրիկ է $X$ բազմության վրա, ապա 
\[
\tilde{\rho} : X \times X \to \R,
\qquad 
\tilde{\rho}(x,y) = \min\left(1, \rho(x,y)\right)
\]
արտապատկերումը նույնպես մետրիկ է $X$-ի վրա։ 

% 7.5
\item Ցույց տվեք, որ $X$ բազմության վրա նախորդ խնդրում դիտարկված $\rho$ և $\tilde{\rho}$ մետրիկները համարժեք են (որոշում են $X$-ի միևնույն տոպոլոգիան):

% 7.6
\item Գտեք այնպիսի մետրիկային տարածություն և նրանում երկու այնպիսի փակ գնդեր, որ դրանցից ավելի մեծ շառավղով գունդը պարունակվի ավելի փոքր շառավղով գնդի մեջ՝ չհամընկնելով նրա հետ։

% 7.7
\item Դիտարկենք $\R^2$ կոորդինատային հարթությունը և
\[
\sigma: \R^2 \times \R^2 \to \R, \qquad \sigma(x, y ) = |x_1-y_1|+|x_2-y_2|, \qquad
x=(x_1, x_2),\ y=(y_1, y_2)
\]
արտապատկերումը:
\begin{enumerate}
\item[ա)] Ապացուցեք, որ $\sigma$-ն որոշում է մետրիկ $\R^2$-ի վրա։

\item[բ)] Նկարագրեք ստացվող մետրիկային տոպոլոգիայի կանոնական բազան։
\end{enumerate}

\par \textbf{Ցուցում} բ)-ի վերաբերյալ․ Սևեռված $x^0 = (x_1^0, x_2^0)$ կետի դեպքում՝ $\mathcal{D}(x^0, r) = \{y \in \R^2 : |x_1^0-y_1|+|x_2^0-y_2|<r\}$։ Մասնավորապես $x^0=O=(0,0)$ կետի դեպքում $\mathcal{D}(0, r) = \{y\in\R^2 : |y_1|+|y_2|<r\}$ բաց շրջանը անեզր քառակուսի է, որի կենտրոնը կոորդինատների սկզբնակետն է, իսկ $2r$ երկարությամբ անկյունագծերը գտնվում են կոորդինատային առանցքների վրա։ Ցույց տվեք, որ ընդհանուր դեպքում $\mathcal{D}(x^0, r)$ շրջանը ստացվում է $\mathcal{D}(O, r)$ քառակուսուց՝ նրա $O$ կենտրոնի զուգահեռ տեղափոխությունով $x^0$ կետ։

\begin{center}
\begin{tikzpicture}[scale=1.2]

% Left diamond
\draw[->] (-3,0) -- (1.5,0); % x-axis
\draw[->] (-1.25,-1.5) -- (-1.25,2.2); % y-axis

\begin{scope}[shift={(-1.25,-0.704*1.5)}, rotate=45]
    \draw[fill=white] (0,0) rectangle (1.5,1.5);
    \foreach \x in {0.15,0.3,...,1.5}
        \draw[opacity=0.5] (\x,0) -- (\x,1.5);
\end{scope}
% Mark center and theta
\node at (-1.25,0) [circle,fill=black,inner sep=1pt]{};
\node at (-1.0,0.2) {$O$};

\begin{scope}[shift={(1.5,0)}]


% Right diamond
\draw[->] (1.5,0) -- (6,0); % x-axis
\draw[->] (3.25,-1.5) -- (3.25,2.2); % y-axis
\begin{scope}[shift={(5,-0.3)}, rotate=45]
    \draw[fill=white] (0,0) rectangle (1.5,1.5);
    \foreach \x in {0.15,0.3,...,1.5}
        \draw[opacity=0.5] (\x,0) -- (\x,1.5);
\end{scope}
% Mark center and points
\node at (5.1,0.75) [circle,fill=black,inner sep=1pt]{};
% \node at (3.55,0.3) {$0$};
\node at (5.35,0.75) {$x^0$};
\end{scope}
\end{tikzpicture}
\end{center}


% 7.8
\item Լուծեք նախորդ խնդիրը
\[
\mu : \R^2 \times \R^2 \to \R, 
\qquad 
\mu(x, y) = \max\left\{ |x_1-y_1| + |x_2-y_2| \right\}
= \max_{i=1, 2} |x_i-y_i| 
\]
արտապատկերման դեպքում, որտեղ $x=(x_1,x_2)$, $y=(y_1, y_2)$:


\begin{center}
\begin{tikzpicture}[scale=1.2]

% Left diamond
\draw[->] (-3,0) -- (1.5,0); % x-axis
\draw[->] (-1.25,-1.5) -- (-1.25,2.2); % y-axis

\begin{scope}[shift={(-2,-1.5/2)}, ]
    \draw[fill=white] (0,0) rectangle (1.5,1.5);
    \foreach \x in {0.15,0.3,...,1.5}
        \draw[opacity=0.5] (\x,0) -- (\x,1.5);
\end{scope}
% Mark center and theta
\node at (-1.25,0) [circle,fill=black,inner sep=1pt]{};
\node at (-1.0,0.2) {$O$};

\begin{scope}[shift={(1.5,0)}]


% Right diamond
\draw[->] (1.5,0) -- (6,0); % x-axis
\draw[->] (3.25,-1.5) -- (3.25,2.2); % y-axis
\begin{scope}[shift={(5.1-1.5/2-0.2,0.75-1.5/2-0.2)}]
    \draw[fill=white] (0,0) rectangle (1.5,1.5);
    \foreach \x in {0.15,0.3,...,1.5}
        \draw[opacity=0.5] (\x,0) -- (\x,1.5);
\end{scope}
% Mark center and points
\node at (5.1-0.2,0.75-0.2) [circle,fill=black,inner sep=1pt]{};
% \node at (3.55,0.3) {$0$};
\node at (5.35-0.2,0.75-0.2) {$x^0$};
\end{scope}
\end{tikzpicture}
\end{center}


\par \textbf{Ցուցում} բ)-ի վերաբերյալ․ Ցույց տվեք, որ սևեռված $x^0 = (x_1^0, x_2^0)$ կետի դեպ\-քում $x^0$ կենտրոնով, $r$ շառավղով $\mathcal{D}(x^0, r) = \{y \in \R^2 : \mu(x^0 , y)<r\}$ բաց շրջանը ստացվում է $O=(0, 0)$ կենտրոնով և $2r$ կողմով $\mathcal{D}(O, r) = \{y \in \R^2 : |y_1|<r,\ |y_2|<r\}$ անեզր քառակուսուց՝ նրա $O$ կենտրոնի զուգահեռ տեղափոխությունով $x^0$ կետ։

% 7.9
\item Ապացուցեք, որ 7.7 և 7.8 խնդիրներում սահմանված $\sigma$ և $\mu$ մետրիկները համ\-ար\-ժեք են (որոշում են $\R^2$-ի նույն տոպոլոգիան)։

% 7.10
\item Երկրաչափորեն նկարագրեք՝ ինչ են պրոյեկտիվ հարթության մետրիկային տոպոլոգիայի կանոնական բազայի տարրերը։ 

\begin{hint} Սևեռելով $\R^3$-ում $O$ սկզբնակետով անցնող որևէ $l^0$ ուղիղ և որևէ $\alpha$ սուր անկյուն՝ դիտարկեք $O$ կետով անցնող բոլոր $l$ ուղիղների բազմությունը, որոնք $l^0$ ուղղի հետ կազմում են $\alpha$-ն չգերազանցող անկյուն։ Պարզեք, թե այդ անեզր գնդի համար երկրորդ կարգի ո՞ր մակերևույթն է հանդիսանում եզրային սֆերա։ 
\end{hint}


% 7.11
\item Դիտարկենք $\R^n$, $n \ge 1$ կոորդինատային տարածությունը \hyperref[օրինակ 3]{օրինակ 3}-ում \linebreak սահ\-մանված $\rho(x,y) = \sqrt{(x_1-y_1)^2+(x_2-y_2)^2+\dots+(x_n-y_n)^2}$ մետրիկով։ Ապացուցեք, որ այդ մետրիկային տարածությունում ամեն մի $\mathcal{S}(a, r)$ սֆերա հանդիսանում է տվյալ $\mathcal{D}(a, r)$ գնդի եզր՝ ըստ թեմա 6-ում սահմանված ենթա\-բազմության եզր հասկացության։

\end{enumerate}



\end{document}