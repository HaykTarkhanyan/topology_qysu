\documentclass[./main.tex]{subfiles}


\begin{document}
\onehalfspacing
\section{Հաշվելիության առաջին և երկրորդ աքսիոմները, կապը նրանց միջև, Լինդելյոֆի թեորեմը։ Սեպարաբել տարածություններ, կապը հաշվելիության երկրորդ աքսիոմի և սեպարաբելության միջև։
}\label{sec:8}
 
\par Մետրիկային տարածություններն օժտված են ևս մի կարևոր հատկությամբ՝ բավա\-րա\-րում են այսպես կոչված հաշվելիության առաջին աքսիոմին։

% 2
\begin{definition}
Դիցուք $x$-ը $X$ տոպոլոգիական տարածության որևէ սևեռված կետ է, իսկ $\beta_x$-ը այդ կետի որոշ շրջակայքերի համախմբություն է։ Ասում են, որ $\beta_x$-ը $x$ կետի \textbf{շրջակայքերի բազա} է, եթե $x$-ի ցանկացած $U$ շրջակայքի համար գոյություն ունի $V \in \beta_x$ շրջակայք, որ $V \subset U$։
\end{definition}

% 3
\begin{definition}
Ասում են, որ $X$ տոպոլոգիական տարածությունը բավարարում է \textbf{հաշվելիության առաջին աքսիոմին}, եթե նրա ցանկացած կետի համար գոյություն ունի շրջակայքերի հաշվելի բազա։
\end{definition}
% 4
% Երբեմն $x$ կետի շրջակայքերի հաշվելի բազան կնշանակենք $\{V_i;\, i\in I \subset \N \}$ սիմվոլով, որտեղ $V_i$-ն $x$ կետի շրջակայք է, իսկ $I$-ն կա՛մ բոլոր բնական թվերի $\N$ բազմությունն է, կա՛մ այդ բազմության որևէ $\{1,2,\dots, n\}$ հատված է։
\begin{example}
    
Հեշտ է տեսնել, որ ցանկացած $(X, \textrm{դիսկր․})$ և $(X, \textrm{անտիդ․})$ տարածությունները բավարարում են հաշ\-վելիութ\-յան առաջին աքսիոմին (\textbf{ինչո՞ւ})։
\end{example}

\begin{theorem}
\label{thm:8.1}
Ցանկացած $X, \rho$ մետրական տարածություն բավարարում է հաշվելիութ\-յան առաջին աքսիոմին \textbf{ինչո՞ւ}։
\end{theorem}

\begin{proof}
Ցույց տանք, որ կամայական $x \in X$ կետի համար բոլոր $\mathcal{D}(x,r),\ r \in \Q$ բաց գնդերը կազմում են $x$ կետի շրջակայքերի հաշվելի բազա։ Եթե $V$-ն $x$-ի որևէ շրջակայք է, ապա ըստ կետի շրջակայքի սահմանման՝ գոյություն ունի $x$-ի $U$ բաց շրջակայք, որ $x \in U \subset V$։ Համաձայն թեմա 6-ում թեորեմ 3-ի \red{fyi, iskzbane chkain ref-er}՝ գոյություն ունի $x$ կենտրոնով $\mathcal{D}(x,R)$ բաց գունդ, որ $x \in \mathcal{D}(x,R) \subset U$։
% 8
\par Վերցնենք որևէ ռացիոնալ $r$ թիվ այնպես, որ  $R>r>0$։ Ունենք $\mathcal{D}(x,r) \subset \mathcal{D}(x, R) \subset U \subset V$, որից հետևում է, որ $x \in \mathcal{D}(x,r) \subset V$, ուստի ռացիոնալ շառավիղներով $\mathcal{D}(x,r)$ բաց գնդերը կազմում են $x$ կետի շրջակայքերի հաշվելի բազա։
\end{proof}

\par Այժմ բերենք տոպոլոգիական տարածության օրինակ, որը չի բավարա\-րում հաշվելիության առաջին աքսիոմին։

% 5
\begin{example}
Դիտարկենք $(\R, \textrm{վերջ․ լր․})$ տարածությունը։ Ըստ սահմանման՝ նրա\-նում բաց բազմություններ են համարվում վերջավոր ենթաբազմությունների լրա\-ցում\-ները։ Ցույց տանք, որ $0 \in \R$ կետի համար գոյություն չունի շրջակայքերի հաշվելի բազա։ Ենթադրենք հակառակը։ Դիցուք $0$ կետի համար գոյություն ունի շրջա\-կայ\-քերի $\beta=\{ U_i, i \in I \subset \N\}$ հաշվելի բազա։ Նախ ցույց տանք, որ $\bigcap\limits_i U_i=\{0\}$։ Նկատենք, որ ամեն մի $r \neq 0$ թվի դեպքում $V(r)=\R \setminus \{r\}$ ենթաբազմությունը $0$ կետի բաց շրջակայք է, հետևաբար գոյություն ունի $0$-ի $U(r) \in \beta$ շրջակայք, որ $0 \in U(r) \subset V(r)$։ Քանի որ $r \notin V(r)$, ուստի $\bigcap\limits_r V(r)=\{0\}$։ Հետևաբար $\bigcap\limits_r U(r)=\{0\}$, և ուրեմն նաև $\bigcap\limits_i U_i=\{0\}$, որտեղ $i \in I$։ Ստացվում է, որ է $\R \setminus \bigcap\limits_i U_i= \R \setminus \{0\}$ բազմությունը ոչ հաշվելի բազմություն է։ Բայց մյուս կողմից, ըստ Դե Մորգանի երկրորդ բանաձևի \red{but?} $\R \setminus \bigcap\limits_i U_i= \bigcup\limits_i(\R \setminus U_i)$ բազմությունը հաշվելի բազմություն է՝ որպես հաշվելի քանակով վերջավոր բազմությունների միավորում։ Ստացանք հակասություն։ 

\par Ուստի $(\R, \textrm{վերջ․ լր․})$ տարածությունը չի բավարարում հաշվելիության առաջին աքսիոմին։

\par Այստեղից, որպես կողմնակի հետևանք ստանում ենք․ $(\R, \textrm{վերջ․ լր․})$ տարածությունը չի կարող մետրիկացվել \qed
\qed
\end{example}
% 6

% 7

% 9
\par Երկրաչափական որոշ բարդ պատկերներ (օրինակ՝ ողորկ բազմաձևությունները) սահմանվում են ավելի պարզ պատկերների սոսնձումներով։ Այդպիսի կառուցումները էապես հեշտանում են, եթե նախապես պահանջում են, որ կառուցվող տարածությունը (բազմաձևությունը) բավարարի այսպես կոչված հաշվելիության երկրորդ աքսիոմին։ \red{file-um moracel a nshi voncvor te "erkrord"}
% 10
\begin{definition}
Ասում են, որ $X$ տոպոլոգիական տարածությունը բավարարում է \textbf{հաշվելիության երկրորդ աքսիոմին}, եթե նրա տոպոլոգիայի համար գոյություն ունի որևէ հաշվելի բազա։
\end{definition}

\begin{example}
Ինչպես գիտենք, ռացիոնալ ծայրակետերով բոլոր $(r_1, r_2)$ ինտերվալների ընտանիքը հաշվելի բազա է $\R$ թվային ուղղի սովորական տոպոլոգիայի համար: Ուստի $(\R, \text{սովոր})$ տարածությունը բավարարում է հաշվելիության երկրորդ աքսիոմին:
\end{example}


 % 11
\begin{theorem}
\label{thm:8.2} 
Հաշվելիության երկրորդ աքսիոմին բավարարող ամեն մի տարածություն բավարարում է նաև հաշվելիության առաջին աքսիոմին։
\end{theorem}
% 12
\begin{proof}
Դիցուք $(X, \tau)$ տարածության համար $B=\{U_i\}$ %$B=\{ U_i;\ i \in I \subset \N \}$
համախմբությունը $\tau$ տոպոլոգիայի հաշվելի բազա է։ Կամայական $x \in X$ կետի համար դիտարկենք $B\textrm{-ի}$ %այն բոլոր տարրերի $\beta_x$ բազմությունը,
$B_x$ ենթաբազմությունը՝ կազմված $B$-ի այն բոլոր տարրերից, որոնք պարունակում են $x$ կետը։ 
%Ցույց տանք, որ $\beta_x$ բազմությունը $x$ կետի շրջակայքերի հաշվելի բազա է։
Պարզ է, որ $B_x$-ը հաշվելի բազմություն է։ Եթե $V$-ն $x$-ի որևէ շրջակայք է, ապա գոյություն ունի $U \in \tau$ բաց բազմություն, որ $x \in U \subset V$։ Ըստ պայմանի՝ $U$-ն ներկայացվում է $U=\bigcup U_j$ տեսքով, որտեղ $U_j \in B$։ Նշանակում է՝ $x$-ը պատկանում է դրանցից որևէ մեկին՝ $x \in U_{j_0},\ j_0 \in J$։ Ուստի $U_{j_0} \in B_x$, և $x \in U_{j_0} \subset V$։
% Այսպիսով $x \in U_{j_0} \subset V$, որտեղ $U_{j_0} \in \beta_x$ և $j_0 \in \N$։
\end{proof}
% 13
\par Հաշվելիության առաջին աքսիոմին բավարարող տարածությունը կարող է չբավարարել հաշվելիության երկրորդ աքսիոմին։
% 14
\par Որպես պարզ օրինակ վերցնենք որևէ ոչ հաշվելի բազմություն դիսկրետ տոպո\-լոգիա\-յով։ Ինչպես գիտենք, նրա ցանկացած $B$  բազա իր մեջ պարունակում է բոլոր մի կետանոց ենթաբազմությունները։ Ուստի $B$-ն ոչ հաշվելի բազա է։
% 15
\par Հաշվելիության երկրորդ աքսիոմին բավարարող տարածությունների մյուս կարևոր հատկությունը կապված է բազմության ծածկույթ, ենթածածկույթ հասկացությունների հետ։
% 16
\begin{definition}
Դիցուք ունենք $X$ բազմության $A$ ենթաբազմություն և $X$-ի ենթաբազմությունների ինչ-որ $U_i \subset X,\ i \in I$ ընտանիք։ Ասում են, որ $\{ U_i;\ i \in I \}$ ընտանիքը \textbf{$\boldsymbol{A}$ ենթաբազմության ծածկույթ է}, եթե $A \subset \bigcup\limits_iU_i$։ Մասնավորապես, $A=X$ դեպքում $\{U_i,\ i \in I\}$ ընտանիքը $X$-ի ծածկույթ է, եթե $X=\bigcup\limits_i U_i$։ Ծածկույթը կոչվում է \textbf{բաց ծածկույթ}, եթե նրա տարրերը $X$ տարածության բաց ենթաբազմություններ են։ Ծածկույթը կոչվում է \textbf{հաշվելի} ծածկույթ, եթե ինդեքսների $I$ բազմությունը հաշվելի բազմություն է։ 
%17
\par Դիցուք ունենք $A \subset X$ ենթաբազմության երկու՝ $\{U_i; i \in I\}$ և $\{V_j; j \in J \}$ ծածկույթներ։ Եթե ամեն մի $i \in I$ տարրի համար գոյություն ունի $j \in J$ տարր այնպես, որ $U_i=V_j$, ապա ասում են, որ $\{U_i; i \in I\}$ ծածկույթը $\{V_j;  j \in J\}$ \textbf{ծածկույթի ենթածածկույթ է}։ \red{vstah chem vor ;-i formatting-y lav a} \red{vonc haskaca uzum a es cackueti entacackuty bold chlini el}
\end{definition}


% 18
%\begin{example}
Օրինակ $(\R, \textrm{սովոր․})$ տարածության համար $U_i=(i-1;i+1),\ i \in I= \R$ ինտերվալները կազմում են նրա բաց ծածկույթ, իսկ $V_j=(j-1;j+1),\ j \in J= \Z$ ինտերվալները կազմում են այդ ծածկույթի հաշվելի ենթածածկույթ։
%\end{example}
\begin{definition}
    Տոպոլոգիական տարածությունը կոչվում է \textbf{լինդելյոֆյան տարա\-ծու\-թյուն}, եթե նրա ցանկացած բաց ծածկույթի համար գոյություն ունի հաշվելի ենթածածկույթ։
\end{definition}
% 19
\begin{theorem} Հաշվելիության երկրորդ աքսիոմին բավարարող ամեն մի տարածություն լինդելյոֆյան տարածություն է։
\end{theorem}
% 20
\begin{proof}
Դիցուք $\{U_i,\ i \in I\}$-ն $(X, \tau)$ տարածության որևէ բաց ծածկույթ է, իսկ $B$-ն $\tau$ տոպոլոգիայի որևէ հաշվելի բազա է։ Ունենք $X=\bigcup\limits_i U_i$։ Ամեն մի $x \in X$ կետի համար ընտրենք որևէ $U_i$, որ $x \in U_i$։ Այդ $U_i$-ն վերանշանակենք $U_x$ (նկատենք, որ տարբեր $x_1,x_2\in X$ կետերի դեպքում հնարավոր է $U_{x_1}=U_{x_2}$ համընկում)։ Գոյություն ունի $B$-ին պատկանող $V_x$ տարր, որ $x \in V_x \subset U_x$։ Պարզ է, որ $\{V_x;\ x \in X\}$ համախմբությունը հաշվելի է որպես հաշվելի $B$ բազմության ենթաբազմություն։ Այժմ յուրաքանչյուր $V_x$-ի համար ընտրենք մի $U_x$, որ $V_x \subset U_x$։ Քանի որ $X=\bigcup\limits_x V_x$, ուստի և $X=\bigcup\limits_x U_x$։ Ուրեմն $\{U_x\}$-ը $\{ U_i,\ i \in I \}$ բաց ծածկույթի հաշվելի ենթածածկույթ է։
\end{proof}


% \begin{hetevanqlind}
% $\R$ թվային ուղղի (ընդհանուր դեպքում $\R^n\textrm{-ի}$) ամեն մի ծածկույթից կարելի է անջատել հաշվելի ենթածածկույթ։
% \end{hetevanqlind}
% 21
% \subsection*{Սեպարաբել տարածություններ։}
% 22

Հաշվելիության երկրորդ աքսիոմին բավարարող տարածությունները կարևոր են ոչ միայն թեորեմ 3-ի իմաստով: Դրա հետ կապված առաջանում է խնդիր. ինչպե՞ս պարզել՝ բավարարում է արդյոք տվյալ տարածությունը հաշվելիության երկրորդ աքսիոմին: Որոշ դեպքերում այդ խնդրի դրական կամ բացասական պատասխաններ ստացվում են ամենուրեք խիտ ենթաբազմություն, սեպարաբել տարածություն հասկացությունների տերմիններով:

\begin{definition}
Ասում են, որ $A \subset X$ \textbf{ենթաբազմությունը ամենուրեք խիտ է} $X, \tau$ տոպոլոգիական տարածությունում, եթե $A$-ի փակումը $X$-ն է՝ $\widebar{A} = X$։ 

\par Այդպիսիք են, օրինակ ա) $X$-ը կամայական $X, \tau$ տարածությունում, բ) ռացիոնալ թվերի և իռացիոնալ թվերի ենթաբազմություն\-ները $(\R, \textrm{սովոր․})$-ում, գ) ցանկացած անվերջ ենթաբազմություն $(X, \textrm{վերջ․ լր․})$-ում։ Սրանք հեշտ հիմնավորվում են շնորհիվ հետևյալի։
\end{definition}
% 23
\begin{theorem}
\label{thm:8.4}
$X$ տարածության $A$ ենթաբազմությունը ամենուրեք խիտ է $X$-ում այն և միայն այն դեպքում, երբ $A$-ն ունի ոչ դատարկ հատում $X$-ի ցանկացած բաց ոչ դատարկ ենթաբազմության հետ։ 
\end{theorem}
%4
\textbf{Անհրաժեշտությունը։} Դիցուք $\widebar{A} =X,\ U \in \tau,\ U \neq \varnothing$։ Եթե $U \cap A=\varnothing$, ապա $U$-ի ոչ մի կետ հպման կետ չէ $A$-ի համար, ուստի $\widebar{A} \neq X$ (հակասություն)։
% 25
\par \textbf{Բավարարությունը։} Վերցնենք $\forall x_0 \in X$ կետ և ցույց տանք, որ $x_0 \in \widebar{A}$։ Դիցուք $V$-ն $x_0$ կետի որևէ շրջակայք է։ Ըստ կետի շրջակայքի սահմանման՝ գոյություն ունի $x_0$-ի $U$ բաց շրջակայք, որ $x_0 \in U \subset V$։ Համաձայն պայմանի՝\\ $U \cap A \neq \varnothing \Rightarrow V \cap A \neq \varnothing \Rightarrow x_0 \in \widebar{A} \Rightarrow \widebar{A} =X$։\qed
% 26
\begin{definition}
Տոպոլոգիական տարածությունը կոչվում է \textbf{սեպարաբել տարածու\-թյուն}, եթե այն ունի որևէ ամենուրեք խիտ հաշվելի ենթաբազմություն։
\end{definition}
% 27
\begin{separabelex}
ա) ցանկացած $(X, \tau)$ տարածու\-թյուն, որտեղ $X$-ը հաշվելի բազմություն է, բ) $(\R, \textrm{սովոր․})$-ը, գ) $(\R, \rightarrow)$ և $(\R, \leftarrow)$ տարածությունները, դ) $\R^n\textrm{-ը}$ սովորական մետրիկային տոպոլոգիայով, ե) ցանկացած $(X, \textrm{անտիդ․})$ տարածություն։ Իսկ $(X, \textrm{դիսկր․})$-ը սեպարաբել չէ, եթե $X$-ը ոչ հաշվելի բազմություն է։
\end{separabelex}

\begin{theorem}
Բոլոր $\R^n, n \ge 1$ էվկլիդեսյան տարածությունները
սեպարաբել տարածություններ են:

\begin{proof}

Ինչպես գիտենք ռացիոնալ կոորդինատներով բոլոր $(q_1, q_2, ..., q_n), q_i \in \Q$ կետերի $\Q^n = \Q \times \Q \times ... \times \Q \subset \R^n$ ենթաբազմությունը հաշվելի է (տես խնդիր 3.5-ը թեմա 3-ում): Ցույց տանք, որ այն ունի ոչ դատարկ հատում ամեն մի $D(x, r)$ գնդի հետ, որտեղ $x=(x_1, x_2, ..., x_n), r>0$: Յուրաքանչյուր $i=1, 2, ..., n$ ինդեքսի համար ընտրենք որևէ $q_i$ ռացիոնալ թիվ $(x_i - \frac{r}{\sqrt{n}}, x_i + \frac{r}{\sqrt{n}})$ ինտերվալից: Ունենք՝

$x_i - \frac{r}{\sqrt{n}} < q_i < x_i + \frac{r}{\sqrt{n}} \implies -\frac{r}{\sqrt{n}} < q_i - x_i < \frac{r}{\sqrt{n}} \implies |q_i - x_i| < \frac{r}{\sqrt{n}} \implies$
$\sum_{i=1}^n |q_i - x_i|^2 < n \cdot \frac{r^2}{n} \implies \sum_{i=1}^n (q_i - x_i)^2 < r^2 \implies (q_1, q_2, ..., q_n) \in D(x, r) \implies$
$\Q^n \cap D(x, r) \neq \varnothing$:

\par Քանի որ $D(x, r)$ գնդերը կազմում են բազա $\R^n$-ի մետրիկային տոպոլոգիայի համար, ուստի $\Q^n$-ի հատումը դատարկ չէ $\R^n$-ի ցանկացած բաց ենթաբազմության հետ: Հետևաբար $\Q^n$-ը ամենուրեք խիտ է $\R^n$-ում ըստ թեորեմ 4-ի: Այսպիսով $\Q^n$-ը հաշվելի, ամենուրեք խիտ ենթաբազմություն է $\R^n$-ում, ուստի $\R^n$-ը սեպարաբել տարածություն է: \qed

\end{proof}
\end{theorem}


%  maybe mapsto and mapsfrom

% 29
\begin{theorem}
\label{thm:8.5}
Հաշվելիության երկրորդ աքսիոմին բավարարող ցանկացած $(X, \tau)$ տարա\-ծու\-թյուն սեպարաբել տարածություն է։
\end{theorem}
% 30
\begin{proof}
Դիցուք $B$-ն $\tau$ տոպոլոգիայի որևէ հաշվելի բազա է։ Յուրաքանչյուր $U_n \in B$ ենթաբազմությունում ընտրենք որևէ $a_n$ կետ։ Ստացված հաշվելի $A=\{a_n\}$ բազմությունը բավարարում է \hyperref[thm:8.2]{թեորեմ 4}-ի \red{voncvor compile-i problem a?} պայմանին (ինչո՞ւ)։ Հետևաբար $\widebar{A} =X$, ուստի $X$-ը սեպարաբել տարածություն է։
\end{proof}
% 31
\par Հակառակը ճիշտ չէ․ սեպարաբել տարածությունը կարող է չունենալ հաշվելի բազա։ Բերենք երկու օրինակ։
% 32
\begin{example}
Դիտարկենք $(\R, \textrm{վերջ․ լր․})$ տարածությունը։ Այն սեպարաբել տարա\-ծու\-թյուն է։ Իրոք, ռացիոնալ թվերի $\Q \subset \R$ ենթաբազմությունն ունի ոչ դատարկ հատում ցանկացած ոչ դատարկ բաց ենթաբազմության հետ, ուստի այն հաշվելի ամենուրեք խիտ ենթաբազմություն է $\R$-ում։ Բայց $(\R, \textrm{վերջ․ լր․})$ տարածությունը չունի հաշվելի բազա, քանի որ ունենալու դեպքում կբավարարվեր հաշվելիության առաջին աքսիոմը համա\-ձայն \hyperref[thm:8.2]{թեորեմ 2}-ի \red{petq a vor ok liner, bayc yndhanur label-nery hin en}։ Մինչդեռ թեմայի սկզբում ցույց ենք տվել, որ այդ տարածությունը չի բավարարում հաշվելիության I աքսիոմին։
\end{example}
% 33
\begin{example}
Դիտարկենք $(\R, \textrm{աջից կիս․ ինտ․})$ տարածությունը։ Այն սեպարաբել տարածություն է, քանի որ $\widebar{\Q} = \R$ (ինչո՞ւ)։ Այժմ ենթադրենք, որ նրա համար գոյություն ունի հաշվելի $B$ բազա։ Նշանակում է ցանկացած սևեռված $a \in \R$ կետի $[a,b)$ բաց շրջակայքի համար պետք է գոյություն ունենա $U \in B$ տարր, որ $a \in U \subset[a,b)$։ Այսինքն $U \subset \R$ ենթաբազմությունն ունի փոքրագույն տարր ի դեմս $a$-ի։ Բայց բոլոր $a \in \R$ կետերի բազմությունը (այսինքն  $\R$-ը) ոչ հաշվելի բազմություն է, ուստի $\Rightarrow B$-ն հաշվելի բազմություն չէ (հակասություն)։
\end{example}
% 34
\par Մետրիկային տարածությունների դեպքում սեպարաբելությունը համարժեք է հաշ\-վե\-լիութ\-յան երկրորդ աքսիոմին։
% 35
\begin{theorem}
\label{thm:8.6}
Որևէ մետրիկային տարածությունը սեպարաբել է այն և միայն այն դեպքում, երբ այն բավարարում է հաշվելիության երկրորդ աքսիոմին։
\end{theorem}
% 36
\begin{proof}
Պայմանի բավարարությունը հետրում է \hyperref[thm:8.5]{Թեորեմ 5}-ից, ուստի մնում է ապացուցել պայմանի անհրաժեշտու\-թյունը։ Դիցուք $(X,\rho)$ մետրիկական տարածությունը սեպարաբել տարածություն է, և $A=\{a_n,\ n \in \N \} \subset X$ ենթաբազմությունը ամենուրեք խիտ է $X$-ում։ Ցույց տանք, որ բաց գնդերի $B=\{\mathcal{D}(a_n,r) \mid a_n \in A,\ r \in \Q\}$ ընտանիքը հաշվելի բազա է $(X,\rho)\textrm{-ի}$ համար։ Նախ պարզ է, որ $B$-ն հաշվելի է, քանի որ հաշվելի է $(a_n,r)$ զույգերի բազմությունը համաձայն թեմա 3-ում թեորեմ 3-ի \red{ref?}։
%որպես հաշվելի թվով հաշվելի բազմությունների միավորում։
% 37
\par Այժմ վերցնենք որևէ $x_0 \in X$ կետ և նրա որևէ $U$ շրջակայք։ Բավական է ցույց տալ, որ գոյություն ունի այնպիսի $\mathcal{D}(a_n,r) \in B$ գունդ, որ $x_0 \in \mathcal{D}(a_n,r) \subset U$։ Դրանից կհետևի, որ $(X,\rho)$-ն բավարարում է հաշվելիության երկրորդ աքսիոմին (\textbf{ինչո՞ւ})։
% 38
\par Ըստ կետի շրջակայքի և մետրիկային տոպոլոգիայի սահմանումների գոյություն ունի $\mathcal{D}(x_0,r_1)$ $r_1 \in \R$, բաց գունդ, որ $x_0 \in \mathcal{D}(x_0,r_1) \subset U$։ Ըստ \hyperref[thm:8.1]{թեորեմ 1}-ի ապացույցում բերված դատողության կարող ենք համարել, որ $r_1 \in \Q$։ Դիտարկենք $\mathcal{D}\left(x_0, \frac{1}{3} r_1\right)$ գունդը, պարզ է, որ $\mathcal{D}(x_0, \frac{1}{3} r_1) \subset \mathcal{D}(x_0,r_1)$։ Քանի որ ${\widebar{A}=X}$, ուստի (համաձայն \hyperref[thm:8.4]{թեորեմ 4}-ի) գոյություն ունի $a_n \in A$ կետ, որ $a_n \in \mathcal{D}(x_0, \frac{1}{3} r_1)$։\\
% 39
\import{tikz/}{8nkar1.tex}
\par Դիտարկենք $\mathcal{D}\left(a_n, \frac{1}{3} r_1 \right)$ գունդը։ Ցույց տանք, որ $\mathcal{D}(a_n,\frac{1}{3} r_1 ) \subset \mathcal{D}(x_0,r_1 )$։ Իրոք, եթե $x \in \mathcal{D}(a_n,\frac{1}{3} r_1 )$, ապա $\rho(x,a_n ) < \frac{1}{3} r_1 \Rightarrow \rho(x,x_0) \leq \rho(x,a_n )+\rho(a_n, x_0) < \frac{1}{3} r_1+ \frac{1}{3} r_1 <r_1$։ Հետևաբար $\mathcal{D}(a_n,\frac{1}{3} r_1) \subset U$։ Մյուս կողմից, քանի որ $a_n \in \mathcal{D}(x_0,\frac{1}{3} r_1)$,  ուստի $\rho(x_0,a_n) < \frac{1}{3} r_1$։ Այսպիսով $x_0 \in \mathcal{D}(a_n,r) \subset U$, որտեղ $r=\frac{1}{3} r_1 \in \Q$։ Նշանակում է $\{\mathcal{D}(a_n,r) \mid a_n \in A,\, r \in \Q\}$ ընտանիքը $(X,\rho)$ տարածության տոպոլոգիայի հաշվելի բազա է։
\end{proof}

\begin{theorem}
    Բոլոր $\R^n,\, n \ge 1$ էվկլիդյան տարածությունները
    \begin{enumerate}
    \item[ա)] բավարարում են հաշվելիության երկրորդ աքսիոմին,
    \item[բ)] լինդելյոֆյան տարածություններ են։
    \end{enumerate}
\end{theorem}

\begin{proof}
Ունենք, որ $\R^n$-ը սեպարաբել տարածություն է համաձայն
թեորեմ 5-ի: Այժմ թեորեմ 7-ից հետևում է, որ $\R^n$-ը բավարարում է
հաշվելիության երկրորդ աքսիոմին, ուստի այն լինդելյոֆյան տարածու-
թյուն է ըստ թեորեմ 3-ի: \qed 
\end{proof}

\begin{hetevanq}
Ցանկացած $\R^n, n \ge 1$ էվկլիդեսյան մետրիկային տարածության ամեն մի բաց ծածկույթից կարելի է անջատել հաշվելի ենթածածկույթ:
\end{hetevanq}


% Իրոք, ռացիոնալ կոորդինատներով բոլոր $(q_1,q_2,\dots,q_n)$ կետերի ենթաբազմութ\-յունը հաշվելի անվերջ ենթաբազմություն է $\R^n$-ում և ունի ոչ դատարկ հատում ամեն մի $\mathcal{D}(x,r)$ բաց գնդի հետ։ Ուստի $\R^n$-ը սեպարաբել տարածություն է համաձայն \hyperref[thm:8.4]{թեորեմ 4}-ի և լինդելյոֆյան տարածություն է համաձայն \hyperref[thm:8.6]{թեորեմ 6}-ի։ \qed
\end{document}