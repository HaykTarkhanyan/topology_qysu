\documentclass[./main.tex]{subfiles}

\begin{document}
\onehalfspacing
\section{\hspace{12pt} Կոմպակտ տարածություն, կոմպակտ ենթաբազմություն, օրինակներ։ Թեորեմ տոպոլոգիական տարածությունների ուղիղ արտադրյալի կոմպակտության մասին։ Թեորեմ $\R^n$ էվկլիդյան տարածության կոմպակտ ենթաբազմությունների մասին։}\label{sec:17}

Վերհիշենք ծածկույթ և ենթածածկույթ հասկացությունները։ Եթե $A\subset X$, ապա $A$ ենթաբազմության ծածկույթ կոչվում է $X$-ի որոշ ենթաբազմություններից կազմ\-ված այնպիսի $\{ U_i,\, i\in I\} $ ընտանիքը, որ $A\subset \bigcup\limits_i U_i$։ Ծածկույթը կոչվում է վերջավոր ծածկույթ, եթե $I$-ն վերջավոր բազմություն է։
\par Եթե $\{V_j,\, j\in J\}$ նույնպես $A$-ի ծածկույթ է և $\forall j\in J$ ինդեքսի համար գոյություն ունի $i\in I$ ինդեքս այնպես, որ $V_j=U_i$, ապա $\{V_j,\, j\in J\}$ ծածկույթը կոչվում է $\{U_i,\, i\in I\}$ ծածկույթի ենթածածկույթ։
\par $A$ ենթաբազմության ծածկույթը կոչվում է բաց ծածկույթ, եթե բոլոր $U_i$-ները բաց ենթաբազմություններ են $X$-ում։
\begin{definition} $X$-ի $A$ ենթաբազմությունը կոչվում է \textbf{կոմպակտ ենթաբազմություն}, եթե նրա կամայական բաց ծածկույթի համար գոյություն ունի նրա վերջավոր են\-թա\-ծած\-կույթ (հաճախ ասում են նաև, $A$-ի ցանկացած բաց ծածկույթից կարելի է անջատել նրա վերջավոր ենթածածկույթ)։
\end{definition}
\par Մասնավորապես $X$-ը կոչվում է \textbf{կոմպակտ տարածություն}, եթե նրա կա\-մա\-յա\-կան $\{U_i,\, i\in I\},\ X=\bigcup\limits_i U_i$ բաց ծածկույթի համար գոյություն ունի նրա վերջավոր $\{V_j,\, j=1,2,\dots,n\}$ ենթածածկույթ։
\begin{example}
$X$-ը անտիդիսկրետ տոպոլոգիայով կոմպակտ տարածություն է, իսկ $X$-ը դիսկրետ տոպոլոգիայով կոմպակտ տարածություն է այն և միայն այն դեպքում, երբ $X$-ը վերջավոր բազմություն է (ինչո՞ւ)։
\end{example}
\begin{example}
$\mathbb{R}$ թվային ուղիղը սովորական մետրական տոպոլոգիայով կոմպակտ չէ, քանի որ $\{(n,n+2),\, n\in \mathbb{Z}\}$ բաց ծածկույթից հնարավոր չէ անջատել վերջավոր ենթածածկույթ։
\end{example}
\begin{theorem}
$A\subset X$ ենթաբազմությունը կոմպակտ ենթաբազմություն է այն և միայն այն դեպքում, երբ $A$-ն կոմպակտ տարածություն է $X$-ից $A$-ի վրա մակածված տոպոլոգիայով։
\end{theorem} %թեորեմ 1

\par Ապացուցումը հեշտությամբ հետևում է այն բանից, որ եթե ${\{ U_i,\, i\in I\} }$-ն բաց ծածկույթ է $A$ ենթաբազմության համար, ապա $\{ U_i\cap A,\, i\in I\} $ ընտանիքը բաց ծածկույթ է $A$ ենթատարածության համար։

\begin{theorem} \label{թեորեմ 2}
$(\mathbb{R}, \textrm{սովոր․})$ տարածությունում թվային ուղղի $[a,b]$ փակ հատվածը կոմպակտ տարածություն է։
\end{theorem} %թեորեմ 2

\begin{proof}
Դիցուք $S=\{U_i\mid U_i\subset \mathbb{R},\, i\in I\}$ ընտանիքը $[a,b]\subset \mathbb{R}$ են\-թա\-բազ\-մու\-թյան բաց ծածկույթ է՝ $[a,b]\subset \bigcup\limits_i U_i$։ Ենթադրենք $[a,b]$-ն կոմպակտ չէ։ Նշանակում է $\left[ a,\dfrac{a+b}{2}\right],\ \left[\dfrac{a+b}{2},b\right] $ հատվածներից որևէ մեկը հնարավոր չէ ծածկել $S$-ի որևէ վերջավոր ենթածածկույթով։ Նշանակենք այդ հատվածը $[a_1,b_1]$։ Նույն դա\-տո\-ղու\-թյամբ $\left[ a_1,\dfrac{a_1+b_1}{2}\right],\ \left[ \dfrac{a_1+b_1}{2},b_1\right] $ հատվածներից գոնե մեկը հնարավոր չէ ծածկել $S$-ի վերջավոր ենթածածկույթով, նշանակենք այդ հատվածը $[a_2,b_2]$ և այդպես շարունակ։ Ստանում ենք $[a,b]\supset [a_1,b_1]\supset[a_2,b_2]\supset \dots \supset [a_n,b_n]\supset \dots$ անվերջ ներդրումներ և ${b_n-a_n=\dfrac{b-a}{2^n}}$։
\par Քանի որ $a\leq a_1\leq a_2\leq \dots \leq a_n<b_m $ ցանկացած $n$-ի և տվյալ $m$-ի դեպքում, ուստի գոյություն ունի $\sup \{ a_i\} =\widebar{a}$, և քանի որ $\widebar{a}\leq b_m$ ցանկացած $m$-ի դեպքում, ուստի գոյություն ունի $\inf\{ b_i\}=\widebar{b}$ և $\widebar{a}\leq \widebar{b}$։

Նրանից, որ $a_n\leq \widebar{a}\leq \widebar{b}\leq b_n$ և $b_n-a_n=\dfrac{b-a}{2^n}$ ցանկացած $n$-ի դեպքում, հետևում է, որ $\widebar{a}=\widebar{b}$։

Քանի որ $[a,b]\subset \bigcup\limits_i U_i$, գոյություն ունի այնպիսի $U_i\in S$, որ $\widebar{a}\in U_i$ և այնպիսի $\varepsilon >0$, որ $(\widebar{a}-\varepsilon,\widebar{a}+\varepsilon )\subset U_i$։ Ընտրենք $N$ բնական թիվ այնպես, որ $\dfrac{b-a}{2^N}<\varepsilon$, և ուրեմն $b_N-a_N<\varepsilon $։ Քանի որ $\widebar{a} \in [a_N,b_N ]$, ուստի $\widebar{a}-a_N<\dfrac{b-a}{2^N} < \varepsilon $ և $b_N-\widebar{a}<\dfrac{b-a}{2^N}<\varepsilon$։ Հետևաբար $[a_N,b_N]\subset (\widebar{a}-\varepsilon,\widebar{a}+\varepsilon )\subset U_i$, որը հակասում է նրան, որ $[a_N,b_N]$-ը հնարավոր չէ ծածկել $S$-ի որևէ վերջավոր ենթածածկույթով։
\end{proof}
\begin{theorem} \label{թեորեմ 3}
Կոմպակտ տարածության կերպարը անընդհատ արտապակերման դեպքում կոմպակտ տարածություն է։
\end{theorem} %թեորեմ 3

\begin{proof}
Ունենք $f:X\rightarrow Y$ անընդհատ արտապատկերում և ${f(X)=Y}$։ Եթե $S=\{U_i,\, i\in I\}$ ընտանիքը $Y$-ի որևէ բաց ծածկույթ է, ապա ${T=\{f^{-1}(U_i),\, i\in I\} }$  ընտանիքը $X$-ի բաց ծածկույթ է։ Ըստ պայմանի գոյություն ունի $T$-ի ${f^{-1}(U_{i_1})}$, ${f^{-1}(U_{i_2}), \dots , f^{-1}(U_{i_n})}$ վերջավոր ենթածածկույթ՝ $X=\bigcup\limits_{k=1}^n U_{i_k}$։
\par Ունենք՝ $Y= f(X)=f\left(\bigcup\limits_{k=1}^n f^{-1}(U_{i_k})\right)=\bigcup\limits_{k=1}^n f(f^{-1}(U_{i_k}))=\bigcup\limits_{k=1}^nU_{i_k}$, ուստի $Y$-ը կոմպակտ է։
\end{proof}
\begin{hetevanq_counter}
Եթե $A$-ն $X$ տարածության կոմպակտ ենթաբազմություն է, ${f:X\rightarrow Y}$ անընդհատ արտապատկերում է, ապա $f(A)$-ն $Y$ տարածության կոմպակտ են\-թա\-բազ\-մու\-թյուն է։
\end{hetevanq_counter} %հետևանք 1
\begin{hetevanq_counter}
Կոմպակտությունը տոպոլոգիական հատկություն է (ինչո՞ւ)։
\end{hetevanq_counter} %հետևանք 2
\begin{hetevanq_counter}
Եթե $X$-ը կոմպակտ տարածություն է, ապա նրա ցանկացած $\faktor{X}{\sim}$ ֆակտոր-տարածությունը նույնպես կոմպակտ տարածություն է (ինչո՞ւ)։
\end{hetevanq_counter} %հետևանք 3
Սրանից հետևում է, որ շրջանագիծը, տորը, Կլեյնի շիշը, պրոյեկտիվ հար\-թու\-թյունը կոմպակտ տարածություններ են։
\begin{theorem}\label{թեորեմ 4}
Կոմպակտ տարածության $\forall$ փակ ենթաբազմություն կոմպակտ է։
\end{theorem} %թեորեմ 4

\begin{proof}
Դիցուք $X$-ը կոմպակտ է, իսկ $A$-ն $X$-ի փակ ենթաբազմություն է։ Դիցուք $S=\{ U_i,\, i\in I\} $-ն $A$-ի բաց ծածկույթ է, $A\subset \bigcup\limits_i U_i$։ Ապա $U_i,\, i\in I$ և $X\setminus A$ ենթաբազմություններով կազմված ընտանիքը $X$-ի բաց ծածկույթ է։ Քանի որ $X$-ը կոմպակտ է, գոյություն ունի այդ ծածկույթի վերջավոր ենթածածկույթ այնպես, որ կամ $X=U_{i_1}\cup U_{i_2}\cup \dots U_{i_n}$ կամ $X=U_{i_1}\cup U_{i_2}\cup \dots U_{i_n}\cup(X\setminus A)$։ Ակնհայտ է, որ երկու դեպքում էլ $A\subset U_{i_1}\cup U_{i_2}\cup \dots U_{i_n}$, ուստի $A$-ն կոմպակտ է։
\end{proof}
\begin{theorem} \label{թեորեմ 5}
Հաուսդորֆյան տարածության ցանկացած կոմպակտ են\-թա\-բազ\-մու\-թյուն փակ ենթաբազմություն է։
\end{theorem} %թեորեմ 5
\begin{proof}
Դիցուք $X$-ը հաուսդորֆյան տարածություն է և $A\subset X$ կոմպակտ ենթաբազմություն է։ Սևեռենք $x_0 \in X\setminus A$ կետ։ Ըստ պայմանի $x_0$-ի և ցանկացած $a\in A$ կետի համար (նկատենք, որ $x_0\neq a$) գոյություն ունեն չհատվող $U_a$ և $V_a$ բաց շրջակայքեր այնպես, որ $x_0 \in U_a,\, a \in V_a$։ Քանի որ $A\subset \bigcup\limits_a V_a$, գոյություն ունի $A$-ի $\{V_a,\, a\in A\}$ ծածկույթի վերջավոր $V_{a_1},V_{a_2},\dots,V_{a_n}$ ենթածածկույթ։ Դիտարկենք $x_0\textrm{-ի}$ ${U(x_0)=\bigcap\limits_{i=1}^nU_{a_i}}$  բաց շրջակայքը։ Քանի որ $U(x_0)\cap V_{a_i}=\varnothing, \ i=1,2,\dots,n$, ուստի՝ $U(x_0)\subset (X\setminus A)$։ Այսպիսով $(X\setminus A)$-ն շրջակայք է իր կամայական $x_0$ կետի համար $\Rightarrow (X\setminus A)$-ն բաց ենթաբազմություն է, ուստի $A$-ն փակ ենթաբազմություն է։
\end{proof}
\begin{theorem}\label{թեորեմ 6}
Մետրական տարածության ցանկացած կոմպակտ ենթաբազմություն սահմանափակ ենթաբազմություն է։
\end{theorem} %թեորեմ 6
\begin{proof}
Դիցուք $A$-ն $(X,\rho)$ մետրական տարածության կոմպակտ են\-թա\-բազ\-մու\-թյուն է։ Ցույց տանք, որ $A$-ն կարելի է ներառնել որևէ գնդի մեջ։ Ամեն մի $a\in A$ կետի համար ընտրենք $a$ կենտրոնով որևէ $\mathcal{D}(a,r)$ բաց գունդ։ Պարզ է, որ ${A\subset \bigcup\limits_a \mathcal{D}(a,r)}$, և քանի որ $A$-ն կոմպակտ է, գոյություն ունի ${\{\mathcal{D}(a,r),\, a\in A\} }$ ծածկույթի վերջավոր $\mathcal{D} (a_1,r_1), \mathcal{D}(a_2,r_2 ),\dots,\mathcal{D}(a_n,r_n)$ ենթածածկույթ, որ \\ $A\subset {\bigcup\limits_{i=1}^n \mathcal{D}(a_i,r_i)}$։ Նշանակենք $r'=\max \rho(a_i,a_j),\, r''=\max(r_i),\, r=\max(r',r'')$ և ցույց տանք, որ $A\subset \mathcal{D}(a_1,2r)$։ Իրոք, եթե $a\in A$ կամայական կետ է, ապա գոյություն ունի $\mathcal{D}(a_k,r_k )$ գունդ, որ $a\in \mathcal{D}(a_k,r_k)$։ Պարզ է, որ $\rho(a,a_1)\leq \rho(a_k,a)+\rho(a_k,a_1)<r_k+r' \leq r''+r'\leq 2r $, ուստի $a\in \mathcal{D}(a_1,2r)$։
\end{proof}
\begin{theorem}
Տոպոլոգիական տարածությունների $X\times Y$ արտադրյալը կոմպակտ է այն և միայն դեպքում, երբ կոմպակտ են $X$-ը և $Y$-ը։
\end{theorem}%թեորեմ 7
\label{թեորեմ 7}
\textbf{Պայմանի անհրաժեշտությունը։} Դիցուք ${X\times Y}$-ը կոմպակտ է։ Դիտարկենք ${P_X:X\times Y\rightarrow X}$ կանոնական պրոյեկցիան։ Քանի որ $P_X$-ը անընդհատ է,\\ ${P_X (X\times Y)=X}$, ուստի $X$-ը կոմպակտ է ըստ \hyperref[թեորեմ 3]{թեորեմ 3}-ի։ Նույն ձևով $Y$-ը ևս կոմպակտ է։
\par \textbf{Պայմանի բավարարության} ապացուցումը կատարենք երկու քայլով։
\par \textbf{1.} Դիցուք $X$-ը և $Y$-ը կոմպակտ են և $W=\{ U_i\times V_i,\, i\in I\}$ ընտանիքը $X\times Y$-ի որևէ բաց ծածկույթ է (այստեղ $U_i$-ները բաց են $X$-ում, իսկ $V_i$-ները բաց են $Y$-ում)։ Ցույց տանք, որ $W$ ծածկույթից կարելի է անջատել $X\times Y$-ի վերջավոր ենթածածկույթ։ Սևեռենք որևէ $x_0\in X$ կետ և դիտարկենք $W$-ի այն բոլոր $U_j (x_0) \times V_j (x_0 ),\, j\in J$ տարրերը, որ $x_0\in U_j (x_0)$ (այստեղ $J$-ն ինդեքսների $I$ բազմության ենթաբազմություն է և յուրաքանչյուր $U_j$ $(x_0)\times V_j(x_0)$ ենթաբազմություն $W$-ի որևէ տարր է)։ 
\par Դիտարկենք $W$ ընտանիքի $W(x_0)=\{ U_j (x_0 )\times V_j (x_0 ),\, j\in J\}$ ենթաընտանիքը։ Պարզ է, որ $X\times Y$-ի յուրաքանչյուր $(x_0,y)$ կետ պատկանում է $W(x_0)$-ի որևէ տարրի (հակառակ դեպքում կունենանք $(x_0,y)\notin W)$։ Ուստի $W(x_0)$-ն $\{ x_0\} \times Y$ ենթաբազ\-մութ\-յան բաց ծածկույթ է։ Քանի որ $\{x_0\}\times Y$-ը $X\times Y$-ի կոմպակտ ենթաբազմություն է (հետևում է $Y$-ի կոմպակտությունից և \hyperlink{sec:14}{թեմա $14$}-ի թեորեմ $6$-ից), ուստի նրա $W(x_0)$ ծածկույթից կարելի է անջատել վերջավոր են\-թա\-ծած\-կույթ։ Վերահամարակալենք այդ ենթածածկույթը՝
\[ U_1 (x_0)\times V_1 (x_0), U_2 (x_0)\times V_2 (x_0),\dots,U_{n(x_0)} (x_0)\times V_{n(x_0)}(x_0)\]
տեսքով, որտեղ $n(x_0)$ թիվը որոշվում է $x_0$ կետով։
\par Այժմ դիտարկենք $X$-ի $U(x_0)=\bigcap\limits_{i=1}^{n(x_0)}U_i(x_0)$ բաց ենթաբազմությունը։ Պարզ է, որ $x_0\in U(x_0 )$։ Ստացանք $X$-ի $\{ U(x),\, x\in X\}$ բաց ծածկույթ։ Քանի որ $X$-ը կոմպակտ է, նրա $\{U(x),\, x\in X\}$ բաց ծածկույթից կարելի է անջատել $X$-ի վերջավոր $U(x_1),U(x_2),\dots,U(x_m)$ ենթածածկույթ։ Արդյունքում ունենք հետևյալ բաց ենթա\-ծած\-կույթ\-ները, որտեղ \par
1) $U_1 (x_1)\times V_1(x_1),\ U_2 (x_1)\times V_2(x_1 ),\ \dots,\ U_{n(x_1)}(x_1)\times V_{n(x_1)} (x_1)$ ընտանիքը ծածկույթ է $U(x_1)\times Y$-ի համար,\par 2)
$U_1(x_2)\times V_1 (x_2),\ U_2(x_2 )\times V_2(x_2 ),\ \dots,\ U_{n(x_2)} (x_2)\times V_{n(x_2)} (x_2)$ ընտանիքը ծածկույթ է $U(x_2)\times Y$-ի համար,\par
$\dots\ \dots\ \dots\ \dots\ \dots\ \dots\ $ \par $m)$
$U_1 (x_m)\times V_1(x_m),\ U_2 (x_m )\times V_2(x_m),\ \dots,\ U_{n(x_m)} (x_m)\times V_{n(x_m)}(x_m)$ ընտանիքը ծածկույթ է $U(x_m)\times Y$-ի համար (հիշեցնենք, որ յուրաքանչյուր $n(x_i)$-ն բնական թիվ է, որոշված տվյալ $x_i$ կետի համար)։
\par Հետևաբար թվարկվածները կազմում են $W$-ի վերջավոր ենթածածկույթ  $X\times Y$-ի համար։
\par \textbf{2.} Դիցուք այժմ ունենք $X\times Y$-ի կամայական $W=\{ W_i,\, i\in I\}$ բաց ծածկույթ։ Ըստ $X\times Y$ ուղիղ արտադրյալի տոպոլոգիայի սահմանման $W_i=\bigcup\limits_{k\in K}(U_{i,k}\times V_{i,k}),$ որտեղ $U_{i,k}$-ները բաց են $X$-ում, իսկ $V_{i,k}$-ները բաց են $Y$-ում (այստեղ $K$-ն ինդեքսների բազմություն է որոշված տվյալ $i$ ինդեքսի համար՝ $K=K(i)$)։
\par Պարզ է, որ $\{U_{i,k}\times V_{i,k},\, i\in I,\, k\in K(i)\}$ ընտանիքը նույնպես $X\times Y$-ի բաց ծածկույթ է։ Ըստ նախորդ $1.$ դեպքի՝ այդ ծածկույթից կարելի է անջատել $X\times Y$-ի վերջավոր ենթածածկույթ։ 
\par Դիցուք այն (վերահամարակալումից հետո) կազմված է $U_1\times V_1, U_2\times V_2,\dots,\\ {U_p\times V_p}$ տարրերից։ Այժմ յուրաքանչյուր $U_s\times V_s,\, s=1,2,\dots,p$ տարրի համար ընտ\-րե\-լով $W$ ծածկույթի մի այնպիսի $W_i$ տարր, որ $U_s\times V_s$-ը $W_i=\bigcup\limits_{k\in K}(U_{i,k}\times V_{i,k})$ միա\-վորման բաղադրիչ է, կստանանք $X\times Y$-ի $W$ ծածկույթի վերջավոր ենթածածկույթ։
\par \hyperref[թեորեմ 7]{Թեորեմ 7}-ն ընդհանրացվում է ցանկացած վերջավոր (անգամ անվերջ) թվով $X_1,X_2,\dots,X_n$ տարածությունների և նրանց $X_1\times X_2\times \dots\times X_n $ արտադրյալի դեպքում՝ $X_1\times X_2\times \dots\times X_n$ արտադրյալը կոմպակտ է այն և միայն այն դեպքում, երբ կոմպակտ են բոլոր $X_i$ արտադրիչները։ Այստեղից և \hyperref[թեորեմ 2]{թեորեմ $2$}-ից ստանում ենք։
\begin{hetevanq}
$[a_1,b_1]\times [a_2,b_2]\times \dots \times [a_n,b_n]$ արտադրյալը, որտեղ $[a_i,b_i]\subset \mathbb{R}$, կոմպակտ է որպես $\mathbb{R}^n=\underbrace{\mathbb{R}\times \mathbb{R} \times \dots \times \mathbb{R}}_\text{$n$}$ էվկլիդյան տարածության են\-թա\-բազ\-մու\-թյուն։ Կոմպակտ է նաև  $\underbrace{S^K\times S^K\times\dots\times S^K}_\text{$n$}$ տարածությունը ($S^K$-ն $\mathbb{R}^{k+1}$-ում միավոր սֆե\-րան է) որպես $\mathbb{R}^{nk+n}=\underbrace{\mathbb{R}^{k+1}\times \mathbb{R}^{k+1}\times \dots \times \mathbb{R}^{k+1}}_\text{$n$}$ էվկլիդյան տա\-րա\-ծու\-թյան ենթաբազմություն։
\end{hetevanq}

\begin{theorem}\label{թեորեմ 8}
Էվկլիդյան $\mathbb{R}^n$ տարածության $M$ ենթաբազմությունը կոմպակտ է այն և միայն այն դեպքում, երբ այն փակ է և սահմանափակ $\mathbb{R}^n$-ում։
\end{theorem} %թեորեմ 8
\begin{proof}
Եթե $M$-ը կոմպակտ է, ապա այն սահմանափակ է $\mathbb{R}^n$-ում ըստ \hyperref[թեորեմ 6]{թեորեմ $6$}-ի, և $\mathbb{R}^n$-ի փակ ենթաբազմություն է ըստ \hyperref[թեորեմ 5]{թեորեմ $5$}-ի։ Այժմ հակառակը․ դիցուք $M$-ը փակ ենթաբազմություն է $\mathbb{R}^n$-ում։ Սահմանափակությունից հետևում է, որ գոյություն ունի $n$-չափականության $ N=[ a_1,b_1 ]\times [ a_2,b_2 ] \times \dots \times [ a_n,b_n ]$  ուղ\-ղանկ\-յու\-նա\-նիստ, որ $M\subset N$։  Քանի որ $M$-ը և $N$-ը փակ են $R^n$-ում և $M\subset N$, ուստի $M$-ը փակ է նաև $N$-ում (ըստ \red{թեորեմ $1$-ի նմանակի թեմա $12$}-ում)։ Վերջապես, քանի որ $N$-ը կոմպակտ է ըստ \hyperref[թեորեմ 7]{թեորեմ 7}-ից հետևանքի, ուստի $M$-ը նույնպես կոմպակտ է ըստ \hyperref[թեորեմ 4]{թեորեմ $4$}-ի։
\end{proof}
\begin{theorem}[(Վեյերշտրաս)] \label{թեորեմ 9} Կոմպակտ տարածության վրա սահմանված ան\-ընդ\-հատ թվային ֆունկցիան սահմանափակ է և ընդունում է մեծագույն և փոքրագույն արժեքներ։
\end{theorem} %թեորեմ 9

\begin{proof}
Դիցուք $X$-ը կոմպակտ տոպ․ տարածություն է և $f:X\rightarrow \mathbb{R}$ արտապատկերումն անընդհատ է։ Քանի որ $f(X)$-ը, ըստ \hyperref[թեորեմ 3]{թեորեմ $3$}-ի, $\mathbb{R}$ թվային ուղղի կոմպակտ ենթաբազմություն է, ուստի այն փակ և սահմանափակ է համաձայն \hyperref[թեորեմ 8]{թեորեմ $8$}-ի։ Դիտարկենք $\inf(f(X))$ և $\sup⁡(f(X))$ կետերը $\mathbb{R}$-ում։ Որպես $f(X)$ փակ ենթաբազմության հպման կետեր՝ դրանք պատկանում են $f(X)$-ին։ Հետևաբար գոյություն ունեն $x_1,x_2\in X$ կետեր, որ $f(x_1)=\inf(f(X))$ և $f(x_2)=\sup(f(X))$։ 
\end{proof}
\par Վերջում, անդրադառնալով \hyperref[թեորեմ 9]{թեորեմ 9}-ին, նկատենք, որ գոյություն ունեն $\mathbb{R}^n$ էվկլիդյան տարածության ենթաբազմությունների կոմպակտության նաև այլ հայ\-տա\-նիշ\-ներ։
\par Նախ ձևակերպենք մի փոքր ավելի ընդհանուր հայտանիշ։
\begin{theorem}
Դիցուք $X$ տոպոլոգիական տարածությունը բավարարում է հաշ\-վե\-լիու\-թյան երկրորդ աքսիոմին։ Ապա $x$-ը կոմպակտ է այն և միայն այն դեպքում, երբ ամեն մի $\{x_n\}\subset X$ հաջորդականություն ունի ենթահաջորդականություն, որը զուգամիտում է $X$-ի որևէ կետի։ 
\end{theorem} %թեորեմ 10
Ապացույցի հետ կարելի է ծանոթանալ \cite{Calley} գրքում։ Այստեղից որ\-պես հետևանք ստանում ենք։
\begin{theorem}
$\mathbb{R}^n$ տարածության $X$ ենթաբազմությունը կոմպակտ է այն և միայն այն դեպքում, երբ $X$-ում ամեն մի հաջորդականություն ունի են\-թա\-հա\-ջոր\-դա\-կա\-նու\-թյուն, որը զուգամիտում է $X$-ի ինչ-որ կետի։
\end{theorem} %թեորեմ 11
\end{document}