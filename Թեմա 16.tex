\documentclass[./main.tex]{subfiles}


\begin{document}
\onehalfspacing
\section{Գծորեն կապակցված տարածություններ, գծային կապակցվածությունը որպես տոպոլոգիական հատկություն: Կապը կապակցվածություն և գծային կապակցվածություն հասկացությունների միջև: Տոպոլոգիական տարածության գծային կապակցվածության բաղադրիչները:}\label{sec:16}

\par Նախորդ թեմայում պարզաբանեցինք կապակցված տոպոլոգիական տա\-րա\-ծութ\-յուն և տարածության կապակցվածության բաղադրիչ հասկացությունները։ Մաս\-նա\-վո\-րա\-պես տեսանք, որ թվային ուղիղը կապակցված է և ունի կապակցվածության մի բա\-ղա\-դրիչ՝ ինքը $\mathbb{R}$-ը: Իսկ $\mathbb{R}\setminus 0=(-\infty,0) \cup (0,+\infty)$ տարածությունը (որպես $\mathbb{R}\textrm{-ի}$ են\-թա\-բազ\-մութ\-յուն) կապակցված չէ և ունի կապակցվածության երկու բաղադրիչ՝ $(-\infty,0)$ և $(0,+\infty)$:
	
\par Այս օրինակներում ենթագիտակցաբար (կամ բառացի) կապակցվածությունը կարող է ընկալվել նաև որպես տվյալ տարածության կամայական մի կետից մի այլ կետ ինչ-որ ճանապարհով (ուղիով) անընդհատ տեղափոխվելու հնարավորություն: Բնական է համարել, որ $\mathbb{R}$-ում դա հնարավոր է, իսկ ահա $\mathbb{R}\setminus 0$ տարածությունում $-1$ կետից $+1$ կետ որևէ ուղիով անընդհատ տեղափոխվել հնարավոր չէ ($0$ կետի հեռացումը խախտում է $\mathbb{R}$-ի ամբողջականությունը): Այս ամենի հստակեցումը մեզ բերում է գծորեն կապակցված տարածություն հասկացությանը:

\begin{definition}
$X$ տոպոլոգիական տարածությունում \textbf{ուղի} կոչվում է $I=[0;1]$ հատվածի ամեն մի $f:I\rightarrow X$ անընդհատ արտապատկերում: Եթե $f(0)=x_0,\ f(1)=x_1$, ապա $x_0$-ն և $x_1$-ը կոչվում են $f$ ուղու \textbf{սկիզբ} և \textbf{վերջ}: Եթե $f: I\rightarrow X$ ուղի է, որը միացնում է $x_0$ կետը $x_1$ կետին, ապա $\widebar{f}:I \rightarrow X,\ \widebar{f}(t)=f(1-t),\ t\in I$ արտապատկերումը նույնպես ուղի է, որը միացնում է $x_1$-ը $x_0$-ին: Եթե $f$ ուղին այնպիսին է, որ $f(t)=x_0,\, \forall t\in I$, ապա $f$-ը կոչվում է \textbf{հաստատուն ուղի $\boldsymbol{x_0}$ կետում} և նշանակվում է $\varepsilon_{x_0}$:
\end{definition}

\par Դիցուք ունենք $f,g:I\rightarrow X$ ուղիներ $X$-ում, ընդ որում $f(0) =x_0$, $f(1) =x_1$ և $g(0)=x_1$, $g(1) =x_2$: Սահմանենք նոր՝ $h :I\rightarrow X$ արտապատկերում
\[
h(t) =\begin{cases}
    f(2t),\quad  0 \leq t \leq \dfrac{1}{2}\\
    g(2t-1),\quad \dfrac{1}{2} \leq t \leq 1
\end{cases}
\]
բանաձևով: Նկատենք, որ $h(0)=f(0)=x_0$, $h(1)=g(1)=x_2$։

\par Այս արտապատկերումը կոչվում է $f$ և $g$ ուղիների արտադրյալ և նշանակվում է $f*g$:

\begin{theorem} % թեորեմ 1
Ուղիղների $f*g$ արտադրյալը ուղի է, որն սկսվում է $x_0$ կետից և ավարտվում է $x_2$ կետում:
\end{theorem}
	
\par Ապացուցելու համար մնում է ցույց տալ $f*g$ արտապատկերման ան\-ընդ\-հա\-տութ\-յու\-նը: Այն անմիջապես հետևում է \red{թեմա $12$-ի թեորեմ $3$-ից}:

\begin{definition}
$X$ տոպոլոգիական տարածությունը կոչվում է \textbf{գծորեն 
կապակցված տարածություն}, եթե նրա ցանկացած $x_1$ և $x_2$ կետեր կարող են միացվել որևէ ուղիով $X$-ում:
\end{definition}

\begin{theorem}\label{թեորեմ 2} % թեորեմ 2
Դիցուք $x_0 \in X$ որևէ սևեռված կետ է: Ապա $X$-ը գծորեն կապակցված է այն և միայն այն դեպքում, երբ $X$-ի ցանկացած կետ միացվում է $x_0$-ին որևէ ուղիով:
\end{theorem}

\begin{proof}
Պայմանի անհրաժեշտությունն ակնհայտ է: Ցույց տանք բա\-վա\-րա\-րութ\-յու\-նը: Ըստ պայմանի՝ $\forall x_1$, $x_2 \in X$ կետերի համար գոյություն ունեն $f,g :I \rightarrow X$ ուղիներ, որ $f(0) =g(0) =x_0$ և $f(1)=x_1$, $g(1)=x_2$: Ուստի $\widebar{f}*g$ ուղին $x_1$-ը միացնում է $x_2$-ին:
\end{proof}

\begin{definition}
$X$ տարածության $Y$ ենթաբազմությունը կոչվում է $X$-ի \textbf{գծորեն կա\-պակ\-ցված ենթաբազմություն}, եթե $Y$-ը գծորեն կապակցված է որպես $X$-ի են\-թա\-տարա\-ծութ\-յուն: Սա համարժեք է նրան, որ $Y$-ի կամայական $y_1$, $y_2$ կետեր կարող են միաց\-վել որևէ $f :I \rightarrow Y$ ուղիով:
\end{definition}

Նկատենք, որ շնորհիվ $I \rightarrow Y \subset X$ համադրույթի անընդհատության՝ $f$-ը նաև ուղի է $X$-ում:

\begin{theorem} % թեորեմ 3
\label{թեորեմ 3}
Դիցուք $X$ տարածությունը ներկայացված է որպես իր որոշ $X_j$, $j \in J$ գծորեն կապակցված ենթաբազմությունների միավորում: Եթե $\bigcap\limits_j X_j\neq \varnothing$, ապա $X$-ը ևս գծորեն կապակցված է:
\end{theorem}

\begin{proof}
Դիցուք $x_0\in \bigcap\limits_j X_j$ որևէ սևեռված կետ է, դիտարկենք կամայական $x_1\in X$ կետ: Գոյություն ունի այնպիսի $X_j$, որ $x_1\in X_j$ և այնպիսի $f :I \rightarrow X_j$ ուղի, որ $f(0)=x_0$, $f(x)=x_1$: Եթե $h_j :X_j \rightarrow X$ արտապատկերումը $X_j$ ենթաբազմության ներդումն է $X$-ի մեջ (այսինքն $h_j (x)=x$, $\forall x\in X_j$ կետի դեպքում), ապա $h_j\circ f$ ուղին միացնում է $x_0$ կետը $x_1$ կետին։ Ուստի $X$-ը գծորեն կապակցված է ըստ \hyperref[թեորեմ 2]{թեորեմ 2}-ի:
\end{proof}

\begin{theorem} % թեորեմ 4   
$\mathbb{R}^n$ էվկլիդյան տարածության ցանկացած ուռուցիկ ենթաբազմություն գծորեն կապակցված է:
\end{theorem}

\begin{proof}
Եթե $x, y \in \R^n$, ապա $(1-t)x+ty$, $t \in I$ տեսքի բոլոր կետերի բազմությունը $x,y$ ծայրակետերով $[x,y]$ հատվածն է $\R^n$-ում (տե՛ս \red{թեմա 14}-ի վերջում): Հետևաբար, եթե $W$-ն ուռուցիկ ենթաբազմություն է $\R^n$-ում և $x,y \in W$, ապա $[x,y]$ հատվածը պարունակվում է $W$-ում և $f :I \rightarrow W$, $f(t)=(1-t)x+ty$, $t \in I$ արտապատկերումը ուղի է $W$-ում, որը միացնում է $x$-ը $y$-ին:
\end{proof}

\begin{theorem} % թեորեմ 5
Տարածության գծային կապակցվածությունը տոպոլոգիական հատ\-կութ\-յուն է:
\end{theorem}

\begin{proof}
Բավական է ցույց տալ, որ գծորեն կապակցված տարածության կերպարը անընդհատ արտապատկերման դեպքում նույնպես գծորեն կապակցված է: Դիցուք $h :X\rightarrow Y$ անընդհատ է և $h(X)=Y$: Դիտարկենք $\forall y_1, y_2 \in Y$ կետեր: Պայմանից հետևում է. գոյություն ունեն $x_1$, $x_2 \in X$ կետեր, որ $h(x_1)=y_1$, $h(x_2)=y_2$, և գոյություն ունի այնպիսի $f :I\rightarrow X$ ուղի, որ $f(0)=x_1$, $f(1)=x_2$: Ուստի $g=h\circ f$ համադրույթը ուղի է $Y$-ում և $g(0)=y_1$, $g(1)=y_2$:
\end{proof}

\subsection*{Կապը կապակցվածության և գծային կապակցվածության միջև}

\begin{theorem}\label{թեորեմ 6} % թեորեմ 6
Ամեն մի գծորեն կապակցված տարածություն նաև կապակցված տարածություն է:
\end{theorem}

\begin{proof}
Դիցուք $X$-ը գծորեն կապակցված է, բայց կապակցված չէ: Նշա\-նա\-կում է գոյություն ունեն $X$-ի ոչ դատարկ, չհատվող $U$ և $V$ բաց են\-թա\-բազ\-մութ\-յուն\-ներ, որ $X=U\cup V$: Վերցնենք որևէ $x_1 \in U,\ x_2 \in V$ կետեր: Ապա գոյություն ունի $f :I\rightarrow X$ ուղի, որ $f(0)=x_1$, $f(1)=x_2$: Ունենք $I=f^{-1}(U) \cup f^{-1}(V)$, ընդ որում $f^{-1}(U)$-ն և $f^{-1}(V)$-ն ոչ դատարկ, չհատվող բաց ենթաբազմություններ են $[0;1]$ հատվածում, ինչը հակասում է $[0;1]$-ի կապակցվածությանը:
\end{proof}

\par Կապակցված տարածությունը կարող է գծորեն կապակցված չլինել: Բերենք համապատասխան օրինակ: Ներկայացնելով $\R^2=\R\times \R$ հարթության $(x,y)$ կե\-տե\-րը նաև որպես $x+iy$ կոմպլեքս թվեր՝ դիտարկենք $\R^2$-ի $A$ և $B$ երկու են\-թա\-բազ\-մութ\-յուն\-ներ՝ $A=\{i\}$, $B=[0;1]\cup \left\{\dfrac{1}{n}+yi,\ n \in \N,\ 0\leq y\leq 1\right\}$, որտեղ $[0;1]$-ը $OX$ առանցքի $\{(x,0);  0\leq x\leq 1\}$ են\-թա\-բազ\-մութ\-յունն է: Դիտարկենք $\R^2$ հարթության $C=A\cup B$ ենթաբազմությունը $\R^2$-ի սո\-վո\-րա\-կան մետրական տոպոլոգիայից մակածված տո\-պո\-լո\-գիա\-յով: 

\begin{tikzpicture}\label{նկար ա}
\node at (0.5, 2.5){ա)};
\node at (1,5)[circle,fill,inner sep=1.5pt]{};
\node[anchor=west] at (1,5){$i$};
\draw (1,0) -- (6, 0) node[anchor=north] at (6,-0.1){1} node[anchor=north] at (1,-0.1) {$0$};
\node[anchor=north] at (1.5,-0.25){$\dots$};
\draw[->] (6,0) -- (7, 0)  node[anchor=north] at (7,-0.1){$OX$}; 
\draw (2,0) -- (2, 5) node[anchor=north] at (2,0) {$\frac{1}{n}$};
\node[anchor=west] at (1.9,2.5) {$B_n$};
\draw (1+5/3,0) -- (1+5/3, 5) node[anchor=north] at (1+5/3,0) {$\frac{1}{3}$} node[anchor=west] at (1+5/3,2.5) {$B_3$};
\draw (3.5,0) -- (3.5, 5) node[anchor=north] at (3.5,0) {$\frac{1}{2}$} node[anchor=west] at (3.5,2.5) {$B_2$};
\draw (6,0) -- (6, 5) node[anchor=west]{$(1,1)$};

\node at (8.5, 2.5){բ)};
\node at (9,5)[circle,fill,inner sep=1.5pt]{};
\node[anchor=west] at (9,5){$i$};
\draw (9,0) -- (14, 0) node[anchor=north] at (14,-0.1){$1$};
\node[anchor=north] at (9,-0.1) {$0$};
\node[anchor=north] at (9.5,-0.25){$\dots$};
\draw[->] (14,0) -- (15, 0) node[anchor=north] at (15,-0.1){$OX$}; 
\draw (10,0) -- (10, 5) node[anchor=north] at (10,0) {$\frac{1}{n}$} node[anchor=west] at (10,3.5) {$V_n$};
\draw (9+5/3,0) -- (9+5/3, 5) node[anchor=north] at (9+5/3,0) {$\frac{1}{3}$};
\node[anchor=west] at (9+5/3,4.5) {$V_3$};
\draw (11.5,0) -- (11.5, 5) node[anchor=north] at (11.5,0) {$\frac{1}{2}$};
\draw[scale=1, domain=9:11.5, smooth, variable=\x, blue] plot ({\x}, {((\x-9)*(\x-9)*(\x-9)*(\x-9) +39.0625)/15.625});
\draw (14,0) -- (14, 5) node[anchor=west]{$(1,1)$};
\end{tikzpicture}

\begin{theorem}\label{թեորեմ 7} % թեորեմ 7
$C$-ն կապակցված, բայց ոչ գծորեն կապակցված տարածություն է:
\end{theorem}

\par Ապացույցը ստացվում է հետևյալ երեք պնդումներից:

\begin{statement}
$C$-ն կապակցված տարածություն է:
\end{statement}

\par Իրոք, քանի որ $B$-ի յուրաքանչյուր $B_n=[0;1]\cup \left\{\dfrac{1}{n}+yi,\ 0\leq y\leq 1\right\}$ են\-թա\-բազ\-մութ\-յու\-նը կա\-պակ\-ցված է որպես երկու հատվող հատվածների միավորում (տե՛ս \hyperref[նկար ա]{նկար ա}-ն) և $\bigcap\limits_{n \in \mathbb{N}} B_n\neq \varnothing$, ուստի $C$-ի $B=\bigcup\limits_{n\in \mathbb{N}}B_n$ ենթաբազմությունը կա\-պակ\-ցված է ըստ \red{թեմա 15-ում թեորեմ 3-ի}:

\par Մյուս պնդումը ձևակերպելու համար կանգ առնենք $i$ կետի շրջակայքերի մի առանձնահատկության վրա։ Դիտարկենք $i$ կետի $U=C\cap \left\{z \in \R^2;\ |z-i|<\dfrac{1}{2}\right\}$ բաց շրջակայքը $C$ տարածությունում (տես \hyperref[նկար ա]{նկար բ}-ն): Այստեղ $\left\{z \in \R^2;\ |z-i|<\dfrac{1}{2}\right\}$ ենթաբազմությունը $i$ կենտրոնով $r=\dfrac{1}{2}$ շառավղով բաց շրջան է $\R^2$-ում: Նկատենք, որ $U\cap B$ ենթաբազմությունը զույգ առ զույգ իրար հետ չհատվող $V_n=\\ U\cap \left\{\dfrac{1}{n}+yi;\ 0\leq y\leq 1\right\}$, $n>2$ ենթաբազմությունների միավորումն է:

\begin{statement}\label{պնդում:2}
Յուրաքանչյուր $V_n$ բաց և փակ ենթաբազմություն է $U$ տա\-րա\-ծութ\-յու\-նում:
\end{statement}

\par Իրոք, քանի որ ամեն մի $\left\{\dfrac{1}{n}+yi;\ 0\leq y\leq 1\right\}$ ենթաբազմություն փակ է $\R^2$-ում (ինչո՞ւ), ուստի $V_n$-ը փակ է $U$ ենթաբազմությունում ըստ մակածված տոպոլոգիայի սահմանման: Մյուս կողմից $V_n$-ը կարող է ներկայացվել \[V_n=U\cap \left\{(x,y); \dfrac{1}{n-1} <x <\dfrac{1}{n+1},\ y\in \R\right\}\] տեսքով, որտեղ $\left\{(x,y);\ \dfrac{1}{n-1 }<x <\dfrac{1}{n+1},\ y\in \R\right\}$ են\-թա\-բազ\-մութ\-յու\-նը բաց է $\R^2\textrm{-ում}$ որպես երկու՝ $\left\{(x,0);\ \dfrac{1}{n-1} <x <\dfrac{1}{n+1}\right\}$ և $\left\{(0;y),\ y\in \R \right\}$ բաց են\-թա\-բազ\-մութ\-յուն\-նե\-րի ուղիղ արտադրյալ: Հետևաբար $V_n$-ը նաև բաց են\-թա\-բազ\-մութ\-յուն է $U$ տա\-րա\-ծութ\-յու\-նում:

\par Վերջապես ապացուցելու համար, որ $C$-ն գծորեն կապակցված չէ, բավական է ցույց տալ, որ $i$ կետը հնարավոր չէ $C$-ում ուղիով միացնել $B$-ի որևէ կետի հետ: Դրա համար ցույց կտանք, որ եթե $f :I\rightarrow C$ ուղին այնպիսին է, որ $f(0)=i$, ապա $f$-ը հաստատուն ուղի է, այսինքն՝ $f(t)=i,\ \forall t\in I$ դեպքում: Նախ ապացուցենք:

\begin{statement}\label{պնդում:3}
Ցանկացած $t_0 \in f^{-1}(i)$ կետ ներքին կետ է $f^{-1}(i)\subset I$ են\-թա\-բազ\-մութ\-յան համար:
\end{statement}

\begin{proof}
Դիտարկենք $f(t_0)=i$ կետի վերը բերված $U$ բաց շրջակայքը $C\textrm{-ում}$: Քանի որ $f$-ը անընդհատ է $t_0$ կետում, ուստի գոյություն ունի $\varepsilon>0$ թիվ, որ $f(t)\in U$, երբ $|t-t_0|<\varepsilon$, այսինքն՝ երբ $t\in [0,1]\cap (t_0-\varepsilon,t_0+\varepsilon)$: Ցույց տանք, որ $t_0$-ի այդ շրջակայքն ընկած է $f^{-1}(i)$ ենթաբազմությունում: Ենթադրենք հակառակը. դիցուք գոյություն ունի $t_1 \in [0,1]\cap (t_0-\varepsilon,t_0+\varepsilon)$ կետ և $f(t_1)\neq i$: Նշանակում է $f(t_1) \in U\cap B=\bigcup\limits_{n\in \N} V_n$, ուստի գոյություն ունի $n_1>2$ բնական թիվ, որ $f(t_1) \in V_{n_1}$: Քանի որ $[0,1]\cap (t_0-\varepsilon, t_0+\varepsilon)$ ենթաբազմությունը հետևյալ երեք՝ $(t_0-\varepsilon, t_0+\varepsilon)$, $[0; t_0+\varepsilon)$ և $(t_0-\varepsilon,1]$ ենթաբազմություններից որևէ մեկն է, ուստի այն կապակցված ենթաբազմություն է $[0,1]$-ում: Հետևաբար $W=f\left([0,1]\cap (t_0-\varepsilon, t_0+\varepsilon)\right)\subset U$ են\-թա\-բազ\-մութ\-յու\-նը կապակցված է շնորհիվ $f$-ի անընդհատության: Քանի որ $W\subset U$, իսկ $V_{n_1}$-ը համաձայն \hyperref[պնդում:2]{պնդում 2}-ի բաց և փակ ենթաբազմություն է $U$-ում, ուստի $V_{n_1}\cap W$-ն բաց և փակ է $W$-ում (ըստ $U$-ից $W$-ի վրա մակածված տոպոլոգիայի սահմանման)։ Բացի այդ $f(t_1)\in V_{n_1}\cap W$ և ուրեմն $V_{n_1}\cap W\neq \varnothing$: Այժմ $W$-ի կապակցվածությունից հետևում է, որ $W=V_{n_1}\cap W$, ուստի $W\subset V_{n_1}$: Բայց դա անհնարին է, քանի որ մի կողմից $i=f(t_0)\in W \subset V_{n_1}$, իսկ մյուս կողմից $i\not\in V_{n_1}$: Հետևաբար $[0,1]\cap (t_0-\varepsilon, t_0+\varepsilon) \subset f^{-1}(i)$, և ուրեմն ցանկացած $t_0\in f^{-1}(i)$ կետ ներքին կետ է $f^{-1}(i)$ ենթաբազմության համար:
\end{proof}

Այժմ պատրաստ ենք ավարտելու \hyperref[թեորեմ 7]{թեորեմ 7}-ի ապացուցումը:

Նախ, որպես հետևանք \hyperref[պնդում:3]{պնդում 3}-ից ստանում ենք, որ $f^{-1}(i)$-ն ոչ դատարկ բաց ենթաբազմություն է $[0,1]$-ում: Մյուս կողմից, $\{i\}$-ն փակ ենթաբազմություն է $C$-ում, ուստի $f^{-1}(i)$-ն փակ ենթաբազմություն է $[0,1]$-ում շնորհիվ $f$-ի անընդհատության: Հետևաբար $f^{-1}(i)=[0,1]$ շնորհիվ $[0,1]$-ի կապակցվածության: \qed

\subsection*{Տոպոլոգիական տարածության գծային կապակցվածության բաղադրիչները} \begin{center}(սահմանվում են կապակցվածության բաղադրիչների նմանությամբ)
\end{center}

\begin{definition}
Տոպոլոգիական տարածության գծային կապակցվածության բա\-ղա\-դրի\-չը կոչվում է նրա ամեն մի գծորեն կապակցված ենթաբազմություն, որը չի պա\-րու\-նակ\-վում իրենից տարբեր որևէ գծորեն կապակցված ենթաբազմությունում:
\end{definition}

\par Ինչպես կապակցվածության բաղադրիչների դեպքում տեղի ունեն հետևյալ հատ\-կութ\-յուն\-նե\-րը.

\begin{enumerate}
    \item[Հ1․] \label{Հ1 հատկություն} 
Տարածության ամեն մի կետ պատկանում է գծային կապակցվածության որևէ բաղադրիչի,


\item[Հ2․]\label{Հ2 հատկություն}
Գծային կապակցվածության ցանկացած երկու բաղադրիչ կամ չեն հատվում, կամ համընկնում են,

\item[Հ3․]\label{Հ3 հատկություն}
Տարածության գծային կապակցվածության բաղադրիչների քանակը (հզո\-րութ\-յու\-նը) տոպոլոգիական ինվարիանտ է:
\end{enumerate}
\par Այդուհանդերձ կապակցված ենթաբազմության և կապակցվածության բա\-ղա\-դրիչ\-նե\-րի ոչ բոլոր հատկություններն են տարածվում գծորեն կապակցված են\-թա\-բազ\-մութ\-յան և գծային կապակցվածության բաղադրիչների վրա:

\par Մասնավորպես գծորեն կապակցված ենթաբազմության փակումը կարող է չլի\-նել գծորեն կապակցված ենթաբազմություն:

\par Որպես օրինակ դիտարկենք վերը քննարկված $\R^2$-ի $C=A\cup B$ ենթաբազմության $B$ գծորեն կա\-պակ\-ցված ենթաբազմությունը: Նրա $\widebar{B}=C$ փակումը գծորեն կա\-պակ\-ցված չէ:

\par Այժմ բերենք մի բավարար պայման, որի դեպքում տարածության կա\-պակ\-ցվա\-ծութ\-յունն ու գծային կապակցվածությունը համարժեք են:

\begin{theorem}\label{թեորեմ 8} % թեորեմ 8
Դիցուք $X$ տարածության յուրաքանչյուր կետ ունի գծորեն կա\-պակ\-ցված շրջակայք: Ապա $X$-ը գծորեն կապակցված է այն և միայն այն դեպքում, երբ $X$-ը կա\-պակ\-ցված է:
\end{theorem}

\begin{proof}
Դիցուք $X$-ը կապակցված տարածություն է, դիտարկենք նրա գծա\-յին կապակցվածության որևէ $U$ բաղադրիչ (այդպիսին գոյություն ունի շնորհիվ \hyperref[Հ1 հատկություն]{Հ1 հատկության}): Ըստ թեորեմի պայմանի՝ $\forall x\in U$ կետի համար գոյություն ունի $x$-ի $V$ գծորեն կապակցված շրջակայք: \hyperref[թեորեմ 3]{Թեորեմ 3}-ից հետևում է, որ $V\cup U$-ն $X$-ի գծորեն կապակցված ենթաբազմություն է, ուստի $V\subset U$: Այսպիսով $U$-ն շրջակայք է իր կամայական կետի համար $\Rightarrow U$-ն $X$-ի բաց ենթաբազմություն է: Ըստ \hyperref[Հ2 հատկություն]{Հ2}-ի՝ $X\setminus U$-ն $X$-ի մյուս բոլոր գծային կապակցվածության բաղադրիչների միավորումն է: Ուստի $X\setminus U$-ն բաց ենթաբազմություն է որպես բաց ենթաբազմությունների միավորում: Քանի որ $U$-ն և $(X\setminus U)$-ն բաց են, նրանք նաև փակ ենթաբազմություններ են: Այսպիսով $U$-ն $X$ կապակցված տարածության ոչ դատարկ, բաց և փակ են\-թա\-բազ\-մութ\-յուն է $\Rightarrow X=U\Rightarrow X$-ը գծորեն կապակցված է:
\end{proof}


\par Հակառակ պնդումը հետևում է \hyperref[թեորեմ 6]{թեորեմ 6}-ից։

\par \hyperref[թեորեմ 8]{Թեորեմ 8}-ից ստանում ենք:

\begin{hetevanq_counter}
Եթե $X$ տարածության ամեն մի կետ ունի գծորեն կապակցված շրջակայք, ապա նրա ամեն մի գծային կապակցվածության բաղադրիչ $X$-ի բաց ենթաբազմություն է:
\end{hetevanq_counter}

\begin{hetevanq_counter}
Էվկլիդյան $\R^2$ տարածության բաց ենթաբազմությունների համար կապակցվածությունն ու գծային կապակցվածությունը համարժեք են (հիմնավորել): 
\end{hetevanq_counter}

\end{document}