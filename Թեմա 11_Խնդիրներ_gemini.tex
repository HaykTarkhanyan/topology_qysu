\bigskip
\bigskip
\subsubsection*{Խնդիրներ և հարցեր թեմա 11-ի վերաբերյալ}

\begin{enumerate}[label=\thesection.\arabic*.]

% 11.1
\item Դիցուք ունենք կամայական $f: X \to Y$ անընդհատ արտապատկերում։ Ճի՞շտ է արդյոք պնդումը․ ամեն մի $x_0 \in X$ կետի ցանկացած $U$ շրջակայքի $f(U)$ կերպարը շրջակայք է $f(x_0) \in Y$ կետի համար։

\begin{hint}
Դիտարկեք որևէ $f: X \to Y$ հաստատուն արտապատկերում, որտեղ $Y$-ը օժտված է անտիդիսկրետ տոպոլոգիայով և պարունակում է մեկից ավելի թվով տարրեր։
\end{hint}

% 11.2
\item Ունենք $X$ բազմության որևէ երկու $\tau$ և $\sigma$ տոպոլոգիաներ։ Ապացուցեք․ $X \to X$ նույնական արտապատկերումը $(X, \tau)$ և $(X, \sigma)$ տարածությունների հոմեոմորֆիզմ է այն և միայն այն դեպքում, երբ $\tau$-ն և $\sigma$-ն նույնն են։

% 11.3
\item Դիցուք $f: X \to Y$ արտապատկերումը $(X, \tau)$ և $(Y, \mathcal{T})$ տոպոլոգիական տարածությունների հոմեոմորֆիզմ է։ Ապացուցեք․ եթե որևէ $U \subset X$ ենթաբազմություն շրջակայք է որևէ $x_0 \in X$ կետի համար, ապա $f(U) \subset Y$ ենթաբազմությունը շրջակայք է $f(x_0)$ կետի համար։

% 11.4
\item Ճի՞շտ է արդյոք պնդումը․ վերջավոր թվով տարրերից կազմված տոպոլոգիական տարածությունների դեպքում տարրերի քանակությունը տոպոլոգիական ինվարիանտ է։

% 11.5
\item Ընտրեք անջատելիության $T_0, T_1, T_2$ աքսիոմներից որևէ մեկը և ապացուցեք որ այն տոպոլոգիական հատկություն է։

% 11.6
\item Բերեք տոպոլոգիական տարածությունների որևէ $f: X \to Y$ փակ, բայց ոչ բաց արտապատկերման օրինակ։

\begin{hint}
Դիտարկեք $\mathbb{R}$ թվային ուղիղը և $\mathbb{R}^2$ կոորդինատային հարթությունը սովորական (էվկլիդյան) մետրիկայով տոպոլոգիաներով։ Ապացուցեք, որ $f: \mathbb{R} \to \mathbb{R}^2, \ f(x)=(x, 0), \ x \in \mathbb{R}$ արտապատկերումը փակ, բայց ոչ բաց արտապատկերում է։
\end{hint}

% 11.7
\item Ունենք $f: X \to Y$ և $g: Y \to Z$ արտապատկերումներ, ընդ որում $g \circ f: X \to Z$ համադրույթը անընդհատ է։ Ապացուցեք․ եթե $f$-ը սյուրյեկտիվ բաց (կամ փակ) արտապատկերում է, ապա $g$-ն անընդհատ արտապատկերում է։

% 11.8
\item Ճի՞շտ է արդյոք, որ թվային ուղղի $\mathbb{R} \to \mathbb{R}$ նույնական արտապատկերումը որոշում է տոպոլոգիական տարածությունների $(\mathbb{R}, \leftarrow) \to (\mathbb{R}, \to)$ հոմեոմորֆիզմ, որտեղ $\leftarrow$ և $\to$ սիմվոլները թվային ուղղի ձախից կիսաբաց և աջից կիսաբաց ինտերվալների տոպոլոգիաներն են։

% 11.9
\item Ապացուցեք՝ $(\mathbb{R}, \to)$ և $(\mathbb{R}, \leftarrow)$ տոպոլոգիական տարածությունները հոմեոմորֆ են։

\begin{hint}
Դիտարկեք $f: \mathbb{R} \to \mathbb{R}, \ f(x) = -x$ արտապատկերումը և օգտվեք թեորեմ 6-ից։
\end{hint}

% 11.10
\item Ճի՞շտ է արդյոք, որ $(\mathbb{R}, \to)$ և $(\mathbb{R}, \text{սովոր.})$ տարածությունները հոմեոմորֆ են։

\begin{hint}
Ապացուցեք, որ $[a, b)$ տեսքի ենթաբազմությունները միաժամանակ բաց և փակ ենթաբազմություններ են $(\mathbb{R}, \to)$ տարածությունում, և օգտվեք թեորեմ 6-ից։
\end{hint}

