\bigskip
\bigskip
\subsubsection*{Խնդիրներ և հարցեր թեմա 7-ի վերաբերյալ}

\begin{enumerate}[label=\thesection.\arabic*.]

Here is the exact text extraction from the remaining images provided.

### **Image: `image_06391d.jpg**`

**11.1.** Դիցուք ունենք կամայական  անընդհատ արտապատկերում։ Ճի՞շտ է արդյոք պնդումը․ ամեն մի  կետի ցանկացած  շրջակայքի  կերպարը շրջակայք է  կետի համար։
 Դիտարկեք որևէ  հաստատուն արտապատկերում, որտեղ -ը օժտված է անտիդիսկրետ տոպոլոգիայով և պարունակում է մեկից ավելի թվով տարրեր։

**11.2.** Ունենք  բազմության որևէ երկու  և  տոպոլոգիաներ։ Ապացուցեք․  նույնական արտապատկերումը  և  տարածությունների հոմեոմորֆիզմ է այն և միայն այն դեպքում, երբ -ն և -ն նույնն են։

**11.3.** Դիցուք  արտապատկերումը  և  տոպոլոգիական տարածությունների հոմեոմորֆիզմ է։ Ապացուցեք․ եթե որևէ  ենթաբազմություն շրջակայք է որևէ  կետի համար, ապա  ենթաբազմությունը շրջակայք է  կետի համար։

---

### **Image: `image_0638fb.jpg**`

**11.4.** Ճի՞շտ է արդյոք պնդումը․ վերջավոր թվով տարրերից կազմված տոպոլոգիական տարածությունների դեպքում տարրերի քանակությունը տոպոլոգիական ինվարիանտ է։

**11.5.** Ընտրեք անջատելիության  աքսիոմներից որևէ մեկը և ապացուցեք որ այն տոպոլոգիական հատկություն է։

**11.6.** Բերեք տոպոլոգիական տարածությունների որևէ  փակ, բայց ոչ բաց արտապատկերման օրինակ։
 Դիտարկեք  թվային ուղիղը և  կոորդինատային հարթությունը սովորական (էվկլիդյան) մետրիկայով տոպոլոգիաներով։ Ապացուցեք, որ  արտապատկերումը փակ, բայց ոչ բաց արտապատկերում է։

**11.7.** Ունենք  և  արտապատկերումներ, ընդ որում  համադրույթը անընդհատ է։ Ապացուցեք․ եթե -ը սյուրյեկտիվ բաց (կամ փակ) արտապատկերում է, ապա -ն անընդհատ արտապատկերում է։

**11.8.** Ճի՞շտ է արդյոք, որ թվային ուղղի  նույնական արտապատկերումը որոշում է տոպոլոգիական տարածությունների...

---

### **Image: `image_0638db.jpg**`

...  հոմեոմորֆիզմ, որտեղ  և  սիմվոլները թվային ուղղի ձախից կիսաբաց և աջից կիսաբաց ինտերվալների տոպոլոգիաներն են։

**11.9.** Ապացուցեք՝  և  տոպոլոգիական տարածությունները հոմեոմորֆ են։
 Դիտարկեք  արտապատկերումը և օգտվեք թեորեմ 6-ից։

**11.10.** Ճի՞շտ է արդյոք, որ  և  տարածությունները հոմեոմորֆ են։
 Ապացուցեք, որ  տեսքի ենթաբազմությունները միաժամանակ բաց և փակ ենթաբազմություններ են  տարածությունում, և օգտվեք թեորեմ 6-ից։
    
\end{enumerate}
