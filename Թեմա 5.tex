\documentclass[./main.tex]{subfiles}


\begin{document}
\onehalfspacing
\section{Տոպո\-լոգիայի բազա, բազայի հայտանիշ, տոպո\-լոգիայի տրում բազայի միջոցով։ Տոպո\-լոգիական տարածության փակ ենթաբազմությունները, տոպո\-լոգիայի այլընտրանքային սահմանում։ Անջատելիության $\boldsymbol{\textrm{T}}_0$, $\boldsymbol{\textrm{T}}_1$, $\boldsymbol{\textrm{T}}_2$ աքսիոմները, ռեգուլյար և նորմալ տարածություններ}\label{sec:5}

Որևէ $X$ բազմության վրա տոպո\-լոգիա տալու համար պարտադիր չէ հատիկ-հատիկ թվարկել $\tau$-ի բոլոր տարրերը ($X$-ի բաց բազմությունները)։ Բավական է տալ կամ նկարագրել դրանց մի մասը՝ պայմանով, որ մյուսները ստացվեն դրանց միավորումներով։ Այս կերպ գալիս ենք տոպո\-լոգիայի բազա հասկացությանը։
\begin{definition}
$(X, \tau)$ տոպո\-լոգիական տարածության բաց ենթաբազմությունների $B \subset \tau$ համախմբությունը կոչվում է $\tau$ \textbf{տոպո\-լոգիայի բազա}, եթե ցանկացած ոչ դատարկ ${U}$ բաց ենթաբազմություն ներկայացվում է որպես $B$-ի որոշ քանակով տարրերի միա\-վո\-րում։
\end{definition}
Պարզ է, որ ցանկացած $\tau$ տոպոլոգիայի համար ինքը՝ $\tau$-ն բազա է։
\begin{example} ա) Ակնհայտ է, որ $(X, \textrm{անտիդ․})$ տարածության համար կա տոպո\-լոգիայի միայն մի բազա՝ $B=\{X\}$, իսկ $(X, \textrm{դիսկր․})$ տարածության դեպքում տոպո\-լոգիայի ցանկացած բազա իր մեջ պետք է պարունակի բոլոր մի կե\-տա\-նոց ${\{x\} \subset X}$  ենթա\-բազմությունները։ Նկատենք նաև, որ $(X, \textrm{դիսկր․})$-ում իրենք՝ բոլոր մի կե\-տա\-նոց ենթաբազմությունները ևս կազմում են տոպո\-լոգիայի բազա, և այն որոշակի իմաս\-տով «նվազագույն» բազա է (պարունակվում է ցանկացած այլ բազայում)։
    
\par բ) $\R$ թվային ուղղի սովորական տոպո\-լոգիայի համար (ըստ սահմանման) բազա են կազմում բոլոր $(a,b)$ ինտերվալները, նաև դրա մասը կազմող ռացիոնալ \linebreak ծայրա\-կետերով բոլոր $(r_1, r_2),\ {r_1, r_2 \in \Q}$ ինտերվալները։ Սա հիմնավորելու հա\-մար նկա\-տենք, որ ցանկացած իրական թիվ կարելի է ցանկացած ճշտությամբ և՛ հա\-վե\-լոր\-դով, և՛ պակասորդով մոտարկել ռացիոնալ թվերով։ Դրա շնորհիվ ունենք  ${(a,b) =\bigcup{(r_1,r_2 )}}$ ներկայացում, որտեղ միավորումը տարածվում է ռացիոնալ \linebreak ծայրա\-կետերով այն բոլոր $(r_1, r_2)$ ին\-տեր\-վալ\-ների վրա, որոնք բավարարում են \linebreak ${a<r_1<r_2<b}$ պայմանին։
\end{example}
Բերված օրինակներից հետևում է, որ ընդհանուր դեպքում տվյալ տոպոլոգիայի համար բազան կարող է միակը չլինել։
\par Այժմ բերենք բազայի հայտանիշ կետերի շրջակայքերի տերմիններով (հիշենք, որ կետի շրջակայք հասկացությունը բաց բազմություն հասկացությանը հա\-վա\-սա\-րար\-ժեք հասկացություն է)։
\begin{theorem}
Դիցուք ունենք $(X, \tau)$ տոպոլոգիական տարածություն և $X$-ի բաց են\-թա\-բազ\-մու\-թյուն\-նե\-րի մի $B \subset \tau$ համախմբություն՝ $B=\{W_j\mid W_j \in \tau,\, j\in J\}$։ Ապա $B$-ն $\tau$ տոպո\-լոգիայի բազա է այն և միայն այն դեպքում, երբ $\forall{x} \in X$ կետի ցանկացած $V$ շրջակայքի համար գոյություն ունի որևէ $W \in B$ տարր, որ ${x \in W \subset V}$։ 
\end{theorem}
\begin{proof}
ա) Դիցուք $B$-ն $\tau$ տոպո\-լոգիայի որևէ բազա է, իսկ $V$-ն ${x \in X}$ կետի որևէ շրջակայք է։ Ըստ կետի շրջակայքի սահմանման՝ գոյություն ունի $U \in \tau$ բաց ենթաբազմություն, որ $x \in U \subset V$։ Ըստ տոպո\-լոգիայի բազայի սահմանման ունենք՝ $U=\bigcup\limits_{k \in K} W_k$, որտեղ $K$-ն ինդեքսների $J$ բազմության ենթաբազմություն է։ Ուստի գոյություն ունի $k \in K$ տարր, որ $x \in W_k$։ Այսպիսով $x \in W_k \subset V,\ W_k \in B$։ 
\par բ) Ապացուցենք հակառակ պնդումը․ դիտարկենք կամայական $U \in \tau$ բաց ենթա\-բազ\-մութ\-յուն և ցույց տանք, որ $U$-ն կարելի է ներկայացնել որպես $B$-ի որոշ տար\-րե\-րի միավորում։ Որպես բաց բազմություն՝ $U$-ն շրջակայք է իր ամեն մի կետի համար (տե՛ս \hyperref[թեորեմ 1]{թեորեմ 1}-ը թեմա 4-ում)։ Ըստ պայմանի՝ տվյալ $x \in U$ կետի համար գոյություն ունի $W(x)\in B$ տարր, որ $x\in W(x)\subset U$։ Հետևաբար, $U=\bigcup\limits_{x\in U}W(x)$, որից էլ հետևում է, որ $B$-ն $\tau$ տոպո\-լոգիայի բազա է։
\end{proof}

\par Տոպո\-լոգիայի բազա հասկացությունը հաճախ հեշտացնում կամ պարզեցնում է բազմության վրա տոպո\-լոգիա սահմանելու ընթացքը։ Այն նաև թույլ է տալիս պարզեցնել շատ թեորեմների ապացույցները, ինչում կհամոզվենք հետագայում։
\begin{theorem}[(բազայի միջոցով տոպո\-լոգիայի տրման մասին)] % թեորեմ 2
\label{թեորեմ 2}
Դիցուք տրված է $X$ բազմության որոշ ենթաբազմությունների $B=\{W_i\mid W_i\subset X,\ i\in I\}$ ընտանիք այնպես, որ 
\begin{enumerate}
    \item $B$-ի բոլոր տարրերի միավորումը $X$-ն է՝ $ \bigcup\limits_{i \in I} W_i=X $,
    \item $B$-ի ցանկացած երկու տարրերի (ոչ դատարկ) հատումը կարող է ներկայացվել որպես $B$-ի որոշ քանակով տարրերի միավորում։
\end{enumerate}
\par Ապա $X$-ի վրա գոյություն ունի, ընդ որում միակ, այնպիսի $\tau$ տոպո\-լոգիա, որի համար $B$-ն տոպո\-լոգիայի բազա է։
\end{theorem}	
\begin{proof}
Որպես $\tau $ վերցնենք $B$-ի բոլոր տարրերը, նրանց բոլոր հնարավոր միավորումները և $\varnothing$-ը։ Տոպո\-լոգիայի առաջին երկու աքսիոմները $\tau$-ի համար ստուգ\-վում են հեշտությամբ (հետևում են $1$-ից և $\tau$-ի սահմանումից)։ Մնում է ստուգել $3$, կամ նրան համարժեք $3'$ աքսիոմը։ Դիցուք $U_1,U_2 \in \tau$, և $U_1=\bigcup\limits_{k\in K} W_k,\ U_2=\bigcup\limits_{l \in L} W_{l}$, որտեղ $K$-ն և $L$-ը ինդեքսների $I$ բազմության են\-թա\-բազ\-մու\-թյուն\-ներ են։ Ունենք՝ 
\[ U_1\cap U_2=\left(\mathsmaller{\bigcup}\limits_{k} W_k\right) \cap \left(\mathsmaller{\bigcup}\limits_{l} W_l\right)=\mathsmaller{\bigcup}\limits_{k,\, l} (W_k \cap W_l)։\]
\par Ըստ \hyperref[թեորեմ 2]{թեորեմի} $2$-րդ պայմանի՝ $W_k\cap W_l\neq \varnothing$ հատումը $B$-ի որոշ տարրերի մի-\linebreak ա\-վո\-րում է։ Ուստի $U_1\cap U_2$-ը ևս $B$-ի որոշ տարրերի միավորում է, հետևաբար \linebreak $U_1\cap U_2\in \tau $։ Այսպիսով $\tau$-ն տոպո\-լոգիա է, որի համար $B$-ն տոպո\-լոգիայի բազա է։ Ստացված տոպո\-լոգիայի միակությունը հետևում է նրանից, որ տվյալ բազայով տոպո\-լոգիա որոշվում է միարժեքորեն (ինչո՞ւ)։ 

Այժմ \hyperref[թեորեմ 2]{թեորեմ 2}-ի կիրառումով սահմանենք ևս մի հետաքրքիր տոպո\-լոգիա $\mathbb{R}$ թվային ուղղի վրա։ Նրա համար որպես բազա վերցնենք բոլոր $[a, b)$ տեսքի կիսաբաց ին\-տեր\-վալ\-նե\-րը։ Դրանք բավարարում են թեորեմի $1$-$2$ պայմաններին, քանի որ մի կողմից $\bigcup\limits_{n\in \mathbb{Z}} [n, n+1)=\mathbb{R}$, և մյուս կողմից էլ $[a, b)$ տեսքի որևէ երկու միջակայքերի հատումը կամ $\varnothing$ է, կամ էլ նույն տեսքի կիսաբաց ինտերվալ է։
\par Ստացված տոպո\-լոգիան կոչվում է \textbf{աջից կիսաբաց ինտերվալների տոպո\-լոգիա}։ Ընթերցողին, որպես օգտակար խնդիր, առաջարկում ենք ապացուցել, որ այս տո\-պո\-լո\-գիան ավելի ուժեղ է թվային ուղղի սովորական տոպո\-լոգիայից։
\par Նման ձևով սահմանվում է \textbf{ձախից կիսաբաց} $(a, b]$ \textbf{ինտերվալների տոպո\-լո\-գիան}։ $\mathbb{R}$ թվային ուղիղը, վերցված աջից կամ ձախից կիսաբաց ինտերվալների տոպոլո\-գիա\-յով, նշանակվում են համապատասխանաբար $(\R, \mapsto)$ և $(\R, \mapsfrom)$։ Այդ \linebreak տոպոլոգիական տարածությունները կոչվում են \textbf{Զորգենֆրեյի ուղիղներ}։
\par Այժմ անդրադառնանք թեմա $4$-ում առաջարկված խնդրին․ տոպոլո՞գիա է արդյոք երկու տոպոլոգիաների միավորումը։ Նախ ճշտենք հարցադրումը․ $X$ բազմության վրա տրված երկու $\tau_1$ և $\tau_2$ տոպոլոգիաների միավորում ասելով հասկանալու ենք $X\textrm{-ի}$ ենթաբազմությունների $\tau_1 \cup \tau_2$ ընտանիքը։ Հարցը հետևյալն է․ ճի՞շտ է արդյոք, որ կամայական բազմության ցանկացած երկու տոպոլոգիաների միավորումը դարձյալ տոպոլոգիա է այդ բազմության համար։
\par Ցույց տանք, որ թվային ուղղի աջից կիսաբաց $\tau_1$ և ձախից կիսաբաց $\tau_2$ տոպոլո\-գիա\-ների $\tau_1 \cup \tau_2$ միավորումը տոպոլոգիա չէ այդ ուղղի համար (չի բավարարվում տոպոլոգիայի 2-րդ՝ միավորման աքսիոմը)։ Իրոք, ունենք՝ $[a, b)\in \tau_1,\ (a, b]\in \tau_2$, բայց նրանց $[a, b)\cup (a, b]=[a, b]$ միավորումը չի պատկանում ո՛չ $\tau_1$-ին, և ո՛չ էլ $\tau_2$-ին (հիմնավորե՛ք)։
\end{proof}

\par Վերը թվային ուղղի սովորական և կիսաբաց ինտերվալների տոպոլոգիաները սահմանվեցին համապատասխանաբար $\{(a, b)\}, \{[a, b)\}, \{(a, b]\}$ բազաների միջոցով։ Դրա հետ կապված առաջանում է հարց․ հանդիսանո՞ւմ է արդյոք  թվային ուղղի որևէ տոպոլոգիայի բազա բոլոր $[a, b]$ փակ հատվածների բազմությունը։ Հեշտ է ցույց տալ, որ $\{[a, b]\}$ ընտանիքը, ի տարբերություն նախորդ երեք օրինակների, չի բավարարում թեորեմ $2$-ի երկրորդ պայմանին, և ուրեմն չի հանդիսանում բազա թվային ուղղի որևէ տոպոլոգիայի համար (հիմնավորե՛ք):

\par Մյուս կողմից, թվային ուղղի \textbf{բոլոր} փակ ենթաբազմությունները (այսինքն այն բոլոր ենթաբազմությունները, որոնք պարունակում են իրենց բոլոր հպման կե\-տե\-րը), օժտված են հետևյալ կարևոր հատկությամբ․ ցանկացած $F$ փակ ենթա\-բազ\-մու\-թյան $\R \setminus F$ լրացումը բաց ենթաբազմություն է։ Իրոք, եթե $x \in \R \setminus  F$, ապա $x$-ը հպման կետ չէ $F$-ի համար, ուստի գոյություն ունի $x$-ի որևէ $U(x, \varepsilon) = (x-\varepsilon, x+\varepsilon)$ շրջակայք, որ $U(x, \varepsilon) \subset \R \setminus  F$։ Սրանից հետևում է, որ $R \setminus  F = \bigcup\limits_{x} U(x,\varepsilon)$ լրացումը բաց ենթաբազմություն է $\R$-ում՝ որպես ինտերվալների միավորում։ Բնականաբար հակառակը նույնպես ճիշտ է՝ թվային ուղղի բաց ենթաբազմությունների լրա\-ցում\-ները փակ ենթա\-բազ\-մու\-թյուններ են։

\par Թվային ուղղի բաց և փակ ենթաբազմությունների այս հատկության հիմքով ներմուծվում է փակ ենթաբազմության հասկացություն կամայական տոպոլոգիական տարածությունում։ 

 % \par Ընթերցողին առաջարկում ենք, որպես փոքրիկ խնդիր, պատասխանել հետևյալ հարցին․ կարելի՞ է արդյոք սահմանել տոպո\-լոգիա թվային ուղղի վրա այնպես, որ նրա համար տոպո\-լոգիայի բազա կազմեն բոլոր ա) $[a,b],\ a<b $ հատվածների բազմությունը, բ) $[a,b],\ a\leq b$ տեսքի ենթաբազմությունները։

% \subsection*{Տոպո\-լոգիական տարածության փակ ենթաբազմությունները։}

\begin{definition}
$(X, \tau)$ տոպո\-լոգիական տարածության \textbf{փակ ենթաբազմություն} կոչվում է $X$-ի ամեն մի $F$ ենթաբազմություն, որի $X\setminus F$ լրացումը բաց են\-թա\-բազ\-մու\-թյուն է $X$-ում (այսինքն՝ $(X \setminus F)\in \tau $)։
\end{definition}
\begin{example} ($X$, անտիդ․)-ում փակ են միայն $\varnothing$-ը և $X$-ը, իսկ ($X$, դիսկր․)-ում փակ են $X$-ի բոլոր ենթաբազմությունները։ Այնուհետև, $(\R, \textrm{սովոր․})$-ում փակ են բոլոր $[a,b]$ հատվածները, $(-\infty ;a]$ և $[a; +\infty)$ հատվածները, բոլոր մի կետանոց $\{a\}$ ենթաբազմությունները, ինչպես նաև դրանց վերջավոր միավորումները։
\par $(\mathbb{R}, \textrm{վերջ․ լր․})$ տարածությունում փակ են\-թա\-բազմութ\-յուն\-ները $\mathbb{R}$-ի վերջավոր են\-թա\-բազմութ\-յուն\-ներն են։ 
\par $(\mathbb{R}, \mapsto)$ տարածությունում $[a,b)$ ենթաբազմությունները միա\-ժա\-մա\-նակ և՛ բաց, և՛ փակ են\-թա\-բազմութ\-յուն\-ներ են (ինչո՞ւ)։ Իսկ $(a,b),\ (a,b],\ (a,+\infty )$ են\-թա\-բազ\-մու\-թյուն\-նե\-րը փակ չեն, քանի որ դրանց լրացումները չեն հանդիսանում շրջակայք $a$ կետի համար (հիմնավորե՛ք)։
\end{example}
\paragraph{Փակ բազմությունների հիմնական հատ\-կու\-թյուն\-նե\-րը։} Ցանկացած $(X,\tau)$ տոպ․ տա\-րա\-ծութ\-յունում
\begin{enumerate}
    \item[F1.] $\varnothing$-ը և $X$-ը փակ ենթաբազմություններ են (ակնհայտ է),
    \item[F2.] $X$-ի ցանկացած վերջավոր քանակով փակ ենթաբազմությունների միա\-վո\-րու\-մը փակ ենթաբազմություն է,
    \par (Իրոք, եթե $F_1, F_2, \dots,  F_n$-ը փակ են $X$-ում, ապա $X\setminus F_1, X\setminus F_2, \dots, X\setminus F_n$-ը բաց են $X\textrm{-ում}$։ Ուստի բաց է նաև նրանց $\bigcap\limits_{i=1}^n (X\setminus F_i)$  հատումը։ Այժմ դե Մորգանի առաջին բանաձևից ստանում ենք, որ $X\setminus \bigcup\limits_{i=1}^n F_i = \bigcap\limits_{i=1}^n (X\setminus F_i)$ ենթաբազմությունը բաց է $X$-ում, ուստի $\bigcup\limits_{i=1}^n F_i$ միավորումը փակ ենթա\-բազ\-մու\-թյուն է $X$-ում)։
    \item[F3.] $X$-ի ցանկացած քանակով փակ ենթաբազմությունների հատումը $X$-ի փակ ենթաբազմություն է (ապացուցվում է նախորդ հատկության նմանությամբ, դե Մոր\-գա\-նի մյուս բանաձևի միջոցով)։
\end{enumerate}
\par Փակ բազմությունների F1-F3 հատկությունները լիովին բնութագրում են $(X,\tau )$ տարածության տոպո\-լոգիան հետևյալ իմաստով։
\begin{theorem}
Դիցուք տրված է $X$ բազմության ենթաբազմությունների մի որոշ $\sigma$ համախմբություն այնպես, որ տեղի ունեն հետևյալ պայմանները․ %թեորեմ3
\label{թեորեմ 3} 
\begin{enumerate}
    \item[1*.] $\varnothing$-ը և $X$-ը պատկանում են $\sigma$-ին,
    \item[2*.] $\sigma$-ի ցանկացած վերջավոր քանակով տարրերի միավորումը պատկանում է $\sigma\textrm{-ին}$,
    \item[3*.] $\sigma$-ի ցանկացած քանակով տարրերի հատումը պատկանում է $\sigma$-ին։
\end{enumerate}
\par Ապա $X$-ի վրա գոյություն ունի միակ $\tau$ տոպո\-լոգիա, որի նկատմամբ փակ են\-թա\-բազ\-մու\-թյուն\-նե\-րը ճիշտ և ճիշտ $\sigma$-ի տարրերն են։
\end{theorem} 
\begin{proof}
Սահմանենք $\tau $ տոպո\-լոգիա $X$-ի վրա՝ վերցնելով որպես $\tau$-ի տար\-րեր $\sigma$-ի տարրերի լրացումները։ Այսինքն $X$-ում բաց ենթաբազմություններ ենք հա\-մա\-րում $X$-ի այն և միայն այն ենթաբազմությունները, որոնց լրացումները պատ\-կա\-նում են $\sigma$-ին։ Տոպո\-լոգիայի 1-3 պայմանների ստուգումը կատարվում է 1*-3* աք\-սիոմ\-նե\-րի և դե Մորգանի բանաձևերի միջոցով։ 
\end{proof}

\par Այսպիսով \hyperref[թեորեմ 3]{թեորեմ 3}-ը  տալիս է բազմության վրա տոպո\-լոգիա սահմանելու ևս մի եղանակ՝ որպես հիմք ընդունելով բազմության փակ ենթաբազմություն չսահմանվող հասկացությունը, իսկ որպես աքսիոմներ՝ 1*, 2*, 3*-ը։

\par Առայժմ բավարարվելով վերը շարադրվածով՝ մենք արդեն կարող ենք ձեռնարկել տոպոլոգիական տարածությունների որոշ ուսումնասիրություն՝ հիմնվելով դրանց այս կամ այլ կարևոր հատկության վրա։

\par Որպես առաջին այդպիսի հատկություն կդիտարկենք այսպես կոչված անջա\-տե\-լի\-ության աքսիոմները։

\par Անջատելիության աքսիոմները լուծում են հետևյալ խնդիրը․ եթե տվյալ տո\-պո\-լո\-գիա\-կան տարածությունում ունենք երկու կետեր կամ (ընդհանուր դեպքում) երկու ենթաբազմություններ, հնարավո՞ր է արդյոք դրանք մասնակիորեն կամ լիովին տար\-ան\-ջատել միմյանցից չհատվող շրջակայքերով։ Ստորև կքննարկենք երկու կե\-տե\-րի տարանջատման խնդիրը երեք տարբերակով ($\textrm{T}_0$, $\textrm{T}_1$, $\textrm{T}_2$ աքսիոմներ)։


\begin{definition} Ասում են, որ $X$ տոպոլոգիական տարածությունը բավարարում է \textbf{անջատելիության $\boldsymbol{\textrm{T}}_0$ աքսիոմին} (կարճ՝ $X$-ը $\textrm{T}_0$ տարածություն է), եթե $X$-ի կամայական երկու տարբեր կետերից գոնե մեկն ունի շրջակայք, որը չի պա\-րու\-նա\-կում մյուս կետը։
\end{definition}

\par Նկատենք, որ ոչ բոլոր տարածություններն են բավարարում այդ պայմանին․ այդպիսի օրինակ է երկուսից ոչ պակաս կետերով $(X, \textrm{անտիդ․})$-ը (հիմնավորե՛ք)։

\begin{definition} Ասում են, որ $X$ տոպոլոգիական տարածությունը բավարարում է \textbf{անջատելիության $\boldsymbol{\textrm{T}}_1$ աքսիոմին}, եթե նրա $\forall x_1 \neq x_2$ կետերից յուրաքանչյուրն ունի շրջակայք, որը չի պարունակում մյուս կետը։
\end{definition}

\par Պարզ է, որ $\textrm{T}_1$ աքսիոմին բավարարող տարածությունը բավարարում է նաև $\textrm{T}_0$ աքսիոմին։ Հակառակը ճիշտ չէ․ $X=\{ x_1,x_2 \}$ բազմությունը $\tau_1= \{\varnothing, \{x_1\}, \{x_1,x_2\}\}$ տոպոլոգիայով $\textrm{T}_0$-տարածություն է, բայց $\textrm{T}_1$-տարածություն չէ (ինչո՞ւ)։

\begin{definition} Ասում են, որ $X$ տոպ․ տարածությունը \textbf{հաուսդորֆյան}, կամ $T_2$-\textbf{տարածություն} է, եթե նրա $\forall x_1 \neq x_2$ կետեր ունեն չհատվող շրջակայքեր։
\end{definition} 
$\textrm{T}_2$-տարածության պարզագույն օրինակ է որևէ երկկետանոց բազմություն դիսկ\-րետ տոպոլոգիայով։
\par Պարզ է, որ $\textrm{T}_2$ աքսիոմին բավարարող ամեն մի տարածություն բավարարում է նաև $\textrm{T}_1$-ին։ Հակառակը ճիշտ չէ․ օրինակ կբերենք քիչ հետո։ Իսկ հիմա․

\begin{theorem} $X$ տոպոլոգիական տարածությունը $\textrm{T}_1$-տարածություն է այն և միայն այն դեպքում, երբ նրա յուրաքանչյուր մի կետանոց $\{x\}$ ենթաբազմություն փակ ենթաբազմություն է $X$-ում։
\end{theorem} 

\begin{proof} \textbf{Պայմանի անհրաժեշտությունը։} Դիցուք $X$-ը $\textrm{T}_1$-տարածություն է, ցույց տանք, որ $\forall x \in X$ կետի դեպքում $X \setminus \{x\}$ ենթաբազմությունը բաց են\-թա\-բազ\-մութ\-յուն է, որից կհետևի, որ $\{x\}$-ը փակ է։ Դիտարկենք $\forall y \in X \setminus \{x\}$ կետ։ Ըստ պայմանի՝ գոյություն ունի $y$ կետի $U_y$ շրջակայք, որ $x \notin U_y$։  Նշանակում է $U_y \subset X \setminus \{x\}$։ Այսպիսով $X \setminus \{x\}$-ը շրջակայք է իր ամեն մի կետի համար $\Rightarrow X \setminus \{x\}\textrm{-ը}$ բաց ենթաբազմություն է $\Rightarrow \{x\}$-ը փակ ենթաբազմություն է։

\textbf{Պայմանի բավարարությունը։} Ունենք, որ $\forall \{x\} \subset X$ ենթաբազմություն փակ ենթաբազմություն է։ Կամայական $x_1 \neq x_2$ կետերի համար $X \setminus \{x_2\}$-ը և $X \setminus \{x_1\}$-ը համապատասխանաբար $x_1$-ի և $x_2$-ի բաց շրջակայքեր են, ընդ որում՝ $x_1 \notin X \setminus \{x_1\}$ և $x_2 \notin X \setminus \{x_2\}$։
\end{proof}

\par Այժմ բերենք $\textrm{T}_1$-տարածության օրինակ, որը $\textrm{T}_2$-տարածություն չէ։ Դիտարկենք որևէ $(X, \textrm{վերջ․ լր․})$ տարածություն, որտեղ $X$-ը անվերջ բազմություն է։ Սա $\textrm{T}_1$ \linebreak տարա\-ծու\-թյուն է (քանի որ ցանկացած մի կետանոց ենթաբազմություն փակ ենթա\-բազ\-մու\-թյուն է), բայց $\textrm{T}_2$-տարածություն չէ, քանի որ այս տարածությունում ցանկա\-ցած երկու բաց ենթաբազմություններ ունեն ոչ դատարկ հատում (հիմ\-նա\-վո\-րե՛ք)։

% Apr xndrum em karciqd kases es ktori voraki pahov (inkati unem havaqely eli, imastain che)
Բացի $T_0, T_1, T_2$ աքսիոմներից տոպոլոգիայում դիտարկվում են անջատելիության նաև այլ աքսիոմներ։ 

\begin{definition} Ասում են, որ $(X, \tau)$ տարածությունը բավարարում է $T_3$ աքսիոմին, եթե նրա կամայական $x$ կետի և այդ կետը չպարունակող կամայական $F$ փակ ենթաբազմության համար գոյություն ունեն $U_x$ և $U_F$ չհատվող բաց ենթաբազմություններ, որ $x \in U_x, F \subset U_F$։
\end{definition}

\par Նման ձևով սահմանվում է $T_4$ աքսիոմը․ պահանջվում է, որ $X$-ի ցանկացած երկու չհատվող $F_1$ և $F_2$ փակ ենթաբազմությունների համար գոյություն ունենան $X$-ի չհատվող $U_{F_1}$ և $U_{F_2}$ բաց ենթաբազմություններ, որ $F_1 \subset U_{F_1}, F_2 \subset U_{F_2}$։

\par Նշենք․ եթե $T_0, T_1, T_2$ աքսիոմներից յուրաքանչյուր հաջորդը ավելի ուժեղ էր նախորդից, ապա այդ օրինաչափությունը խախտվում է աքսիոմների $T_2, T_3, T_4$ հաջորդականության դեպքում։ Պատճառը կայանում է նրանում, որ $T_3$ կամ $T_4$ աքսիոմից չի հետևում ցանկացած մի կետանոց $\{x\}$ ենթաբազմության փակությունը $X$-ում։ Բայց եթե լրացուցիչ պահանջենք, որ $(X, \tau)$-ն բավարարի նաև $T_1$ աքսիոմին, ապա շնորհիվ թեորեմ 4-ի \red{ref pls}, նշված օրինաչափությունը կպահպանվի։ 

\begin{definition} $(X, \tau)$-ն կոչվում է $\textbf{ռեգուլյար \ տարածություն}$ (կամ՝ $X$-ը ռեգուլյար է), եթե բավարարում է $T_1$ և $T_3$ աքսիոմներին և կոչվում է $\textbf{նորմալ տարածություն}$, եթե բավարարում է $T_1$ և $T_4$ աքսիոմներին։
    
\par Այսպիսով ռեգուլյար տարածությունները հաուսդորֆյան են, իսկ նորմալ տարածությունները ռեգուլյար են (կհիմնավորենք)։ 

\par Հայտնի են հաուսդորֆյան, բայց ոչ ռեգուլյար, և ռեգուլյար, բայց ոչ նորմալ տարածությունների օրինակներ։ Դրանց կառուցումը բավական բարդ է և այստեղ դրանք չենք քննարկի։

\par Հետագայում մենք կտեսնենք, որ նորմալ տարածությունների դասը բավական լայն է և իր մեջ ընդգրկում է բոլոր «լավ» տարածությունները (տես թեորեմ 6-ը թեմա 7-ում) \red{ref pls}։
\end{definition}



% \newpage
\bigskip
\bigskip
\subsubsection*{Խնդիրներ և հարցեր թեմա 5-ի վերաբերյալ}

\begin{enumerate}[label=\thesection.\arabic*.]
% 5.1
\item Հանդիսանո՞ւմ է արդյոք ենթաբազմությունների $\{[a, b];\ a\leq b]\}$ ընտանիքը թվային ուղղի որևէ տոպոլոգիայի բազա։

% 5.2
\item Ապացուցեք, որ թեորեմ 2-ի երկրորդ պայմանը կարելի է փոխարինել հետևյալ համարժեք պայմանով․ ցանկացած  $W_i,\  W_j \in B$ տարրերի և ամեն մի \linebreak $x \in W_i \cap W_j$ տարրի համար գոյություն ունի $W_k \in B$ տարր, որ $x \in W_k$ և  \linebreak $W_k \subset W_i \cap W_j$: 

% 5.3
\item Դիտարկենք $\R^2$ կոորդինատային հարթության ենթաբազմությունների $\Phi_1$  և $\Phi_2$ ընտանիքներ կազմված բոլոր այնպիսի անեզր քառակուսիներից (կող\-մերն ու գագաթները հեռացված են), որ առաջին ընտանիքում քառակուսիների կող\-մերը զուգահեռ են կոորդինատային առանցքներին, իսկ երկրորդ ընտա\-նի-  \linebreak քում քառակուսիների անկյունագծերն են զուգահեռ կոորդինատային առանցք\-նե\-րին:

    %Here can be your graph
    
Ապացուցեք․ $\Phi_1$-ը և $\Phi_2$-ը ծառայում են որպես բազա $\R^2$-ի ինչ-որ $\tau_1$ և $\tau_2$ տոպոլոգիաների համար։

\begin{hint}
Դիտարկելով որևէ երկու հատվող քառակուսի առաջին ընտանի\-քից և օգտվելով խնդիր 5․2-ից՝ ստուգեք թեորեմ 2-ի երկրորդ պայմանը (առա\-ջին պայմանը բավարարվում է ակնհայտորեն)։ Երկրորդ ընտանիքի դեպքը բեր\-վում է առաջին ընտանիքի դեպքին՝ կատարելով պտույտ կոորդինատների սկզբնա\-կետի շուրջը $45^\circ$-ով։
\end{hint}

% 5.4
\item Դիտարկենք $\R^2$ հարթության բոլոր անեզր շրջանների (եզրային շրջանագծերը հեռացված են) $\Phi_3$ ընտանիքը։ Ապացուցեք, որ $\Phi_3$-ը բազա է $\R^2$-ի ինչ-որ $\tau_3$ տոպոլոգիայի համար։

% 5.5
\item Ճի՞շտ է արդյոք, որ $\R^2$ հարթության բոլոր եզրով (փակ) շրջանների ընտանիքը կազմում է բազա $\R^2$-ի ինչ-որ տոպոլոգիայի համար։
 
% 5.6
\item Ապացուցեք, որ 5․3 և 5․4 խնդիրներում նկարագրված $\Phi_1$, $\Phi_2$, $\Phi_3$ բազաները որոշում են թվային ուղղի նույն տոպոլոգիան։

\begin{hint}
  Ցույց տվեք, որ յուրաքանչյուր $\Phi_i$ բազայի ($i=1, 2, 3$) ցանկացած տարր կարող է ներկայացվել որպես $\Phi_j,\ j \not= i$ բազայի անվերջ քանակությամբ որոշ տարրերի միավորում:
\end{hint}

% 5.7
\item Ապացուցեք, որ բնական թվերից կազմված բոլոր անվերջ թվաբանական պրոգ\-րե\-սիաների համախմբությունը բոլոր բնական թվերի բազմության ինչ-որ  \linebreak տոպո\-լո\-գիայի բազա է։

\begin{hint}
Դիցուք ունենք բնական թվերից կազմված $A=\{a_1, a_2, \dots\}$ և $B=\{b_1, b_2, \dots\}$ երկու անվերջ թվաբանական պրոգրեսիա համապատաս\-խա\-նաբար $d_1$ և $d_2$ տարբերություններով։ Դիցուք $c_1$-ը $A\cap B$ բազմության փոք\-րա\-գույն տարրն է։ Ցույց տվեք, որ $C=A \cap B=\{c_1, c_2, \dots \}$ բազմությունը ան\-վերջ թվաբանական պրոգրեսիա է $d_3=[d_1, d_2]$ տարբերությունով, որտեղ $[d_1, d_2]$-ը $d_1$ և $d_2$ թվերի ամենափոքր ընդհանուր բազմապատիկն է։
\end{hint}

%5.8
\item Ապացուցեք․ ցանկացած $\textrm{T}_1$ տարածության ամեն մի վերջավոր ենթա\-բազ\-մու\-թյուն փակ բազմություն է։

%5.9
\item Ապացուցեք, որ աջից կիսաբաց ինտերվալների $(\R, \mapsto)$ տոպոլոգիական  \linebreak տարա\-ծու\-թյունում
\begin{enumerate}
    \item[ա)] ամեն մի $[a, +\infty)$ ենթաբազմություն և բաց է, և փակ է,
    
    \item[բ)] ամեն մի $(a, b]$ ենթաբազմություն ոչ բաց է, ոչ էլ փակ է:
\end{enumerate}


%5.10
\item Ապացուցեք․ խնդիր 5․4-ում դիտարկված $(\R^2, \tau_3)$ տարածությունը հաուս\-դորֆ\-յան տարածություն է։

%5.11
\item Պարզեք՝ ստորև բերված տարածություններից որո՞նք են հաուսդորֆյան  \linebreak տարա\-ծու\-թյուն․
\begin{enumerate}
    \item[ա)] դիսկրետ տարածություն,
    \item[բ)] անտիդիսկրետ տարածություն,
    \item[գ)] աջից կիսաբաց ինտերվալների տարածություն։
\end{enumerate}

\end{enumerate}

\end{document}