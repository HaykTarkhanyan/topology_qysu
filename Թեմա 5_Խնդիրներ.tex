\bigskip
\bigskip
\subsubsection*{Խնդիրներ և հարցեր թեմա 5-ի վերաբերյալ}

\begin{enumerate}[label=\thesection.\arabic*.]
% 5.1
\item Հանդիսանո՞ւմ է արդյոք ենթաբազմությունների $\{[a, b];\ a\leq b]\}$ ընտանիքը թվային ուղղի որևէ տոպոլոգիայի բազա։

% 5.2
\item Ապացուցեք, որ թեորեմ 2-ի երկրորդ պայմանը կարելի է փոխարինել հետևյալ համարժեք պայմանով․ ցանկացած  $W_i,\  W_j \in B$ տարրերի և ամեն մի \linebreak $x \in W_i \cap W_j$ տարրի համար գոյություն ունի $W_k \in B$ տարր, որ $x \in W_k$ և  \linebreak $W_k \subset W_i \cap W_j$: 

% 5.3
\item Դիտարկենք $\R^2$ կոորդինատային հարթության ենթաբազմությունների $\Phi_1$  և $\Phi_2$ ընտանիքներ կազմված բոլոր այնպիսի անեզր քառակուսիներից (կող\-մերն ու գագաթները հեռացված են), որ առաջին ընտանիքում քառակուսիների կող\-մերը զուգահեռ են կոորդինատային առանցքներին, իսկ երկրորդ ընտա\-նի-  \linebreak քում քառակուսիների անկյունագծերն են զուգահեռ կոորդինատային առանցք\-նե\-րին:

    %Here can be your graph
    
Ապացուցեք․ $\Phi_1$-ը և $\Phi_2$-ը ծառայում են որպես բազա $\R^2$-ի ինչ-որ $\tau_1$ և $\tau_2$ տոպոլոգիաների համար։

\begin{hint}
Դիտարկելով որևէ երկու հատվող քառակուսի առաջին ընտանի\-քից և օգտվելով խնդիր 5․2-ից՝ ստուգեք թեորեմ 2-ի երկրորդ պայմանը (առա\-ջին պայմանը բավարարվում է ակնհայտորեն)։ Երկրորդ ընտանիքի դեպքը բեր\-վում է առաջին ընտանիքի դեպքին՝ կատարելով պտույտ կոորդինատների սկզբնա\-կետի շուրջը $45^\circ$-ով։
\end{hint}

% 5.4
\item Դիտարկենք $\R^2$ հարթության բոլոր անեզր շրջանների (եզրային շրջանագծերը հեռացված են) $\Phi_3$ ընտանիքը։ Ապացուցեք, որ $\Phi_3$-ը բազա է $\R^2$-ի ինչ-որ $\tau_3$ տոպոլոգիայի համար։

% 5.5
\item Ճի՞շտ է արդյոք, որ $\R^2$ հարթության բոլոր եզրով (փակ) շրջանների ընտանիքը կազմում է բազա $\R^2$-ի ինչ-որ տոպոլոգիայի համար։
 
% 5.6
\item Ապացուցեք, որ 5․3 և 5․4 խնդիրներում նկարագրված $\Phi_1$, $\Phi_2$, $\Phi_3$ բազաները որոշում են թվային ուղղի նույն տոպոլոգիան։

\begin{hint}
  Ցույց տվեք, որ յուրաքանչյուր $\Phi_i$ բազայի ($i=1, 2, 3$) ցանկացած տարր կարող է ներկայացվել որպես $\Phi_j,\ j \not= i$ բազայի անվերջ քանակությամբ որոշ տարրերի միավորում:
\end{hint}

% 5.7
\item Ապացուցեք, որ բնական թվերից կազմված բոլոր անվերջ թվաբանական պրոգ\-րե\-սիաների համախմբությունը բոլոր բնական թվերի բազմության ինչ-որ  \linebreak տոպո\-լո\-գիայի բազա է։

\begin{hint}
Դիցուք ունենք բնական թվերից կազմված $A=\{a_1, a_2, \dots\}$ և $B=\{b_1, b_2, \dots\}$ երկու անվերջ թվաբանական պրոգրեսիա համապատաս\-խա\-նաբար $d_1$ և $d_2$ տարբերություններով։ Դիցուք $c_1$-ը $A\cap B$ բազմության փոք\-րա\-գույն տարրն է։ Ցույց տվեք, որ $C=A \cap B=\{c_1, c_2, \dots \}$ բազմությունը ան\-վերջ թվաբանական պրոգրեսիա է $d_3=[d_1, d_2]$ տարբերությունով, որտեղ $[d_1, d_2]$-ը $d_1$ և $d_2$ թվերի ամենափոքր ընդհանուր բազմապատիկն է։
\end{hint}

%5.8
\item Ապացուցեք․ ցանկացած $\textrm{T}_1$ տարածության ամեն մի վերջավոր ենթա\-բազ\-մու\-թյուն փակ բազմություն է։

%5.9
\item Ապացուցեք, որ աջից կիսաբաց ինտերվալների $(\R, \mapsto)$ տոպոլոգիական  \linebreak տարա\-ծու\-թյունում
\begin{enumerate}
    \item[ա)] ամեն մի $[a, +\infty)$ ենթաբազմություն և բաց է, և փակ է,
    
    \item[բ)] ամեն մի $(a, b]$ ենթաբազմություն ոչ բաց է, ոչ էլ փակ է:
\end{enumerate}


%5.10
\item Ապացուցեք․ խնդիր 5․4-ում դիտարկված $(\R^2, \tau_3)$ տարածությունը հաուս\-դորֆ\-յան տարածություն է։

%5.11
\item Պարզեք՝ ստորև բերված տարածություններից որո՞նք են հաուսդորֆյան  \linebreak տարա\-ծու\-թյուն․
\begin{enumerate}
    \item[ա)] դիսկրետ տարածություն,
    \item[բ)] անտիդիսկրետ տարածություն,
    \item[գ)] աջից կիսաբաց ինտերվալների տարածություն։
\end{enumerate}

\end{enumerate}