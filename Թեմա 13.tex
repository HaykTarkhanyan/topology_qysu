\documentclass[./main.tex]{subfiles}
\usepackage{wrapfig}
\usepackage{floatflt}
\begin{document}
\onehalfspacing
\section{Ֆակտոր-տոպոլոգիա, ֆակտոր-տարածություն, օրինակներ։ Տոպոլոգիական տարածություններում «սոսնձման» (նույնացման) գործողությունը, օրինակներ։ Միակողմանի և երկկողմանի մակերևույթներ։}\label{sec:13}

% tau-ի փոխարեն uptau անելու գաղափար կա

\par Նախորդ թեմայում $(X,\tau)$ տոպ․ տարածության $A \subset X$ ենթաբազմության վրա սահմանվեց մակածված տոպոլոգիա՝ որպես $A$ բազմության ամենաթույլ տո\-պո\-լո\-գիա, որի դեպքում $i:A \to X$ ներդրումը անընդհատ է։

\par Այժմ դիտարկենք այդ նույն խնդիրը մի փոքր այլ, ավելի ընդհանրացված տեսքով։ Դիցուք ունենք $(X,\tau)$ տոպ․ տարածություն, $A$ բազմություն (որը կարող է և չլինել $X\textrm{-ի}$ ենթաբազմություն) և $f:A\to X$ արտապատկերում։ Ինչպե՞ս սահմանել տո\-պո\-լո\-գիա $A$ բազմության վրա, որպեսզի $f$ արտապատկերումը լինի անընդհատ։ Կարելի է օրինակ, $A$-ն օժտել ամենաուժեղ՝ դիսկրետ տոպոլոգիայով, և իհարկե $f$-ը կլինի անընդհատ (ինչո՞ւ)։ Բայց մենք ցանկանում ենք գտնել $A$-ի այն ամենաթույլ տո\-պո\-լո\-գիան, որի դեպքում $f$-ը կլինի անընդհատ։ Հասկանալի է, որ այդ տոպոլոգիան իր մեջ պարունակելու է բոլոր $f^{-1} (V)$ ենթաբազմությունները, որտեղ $V\in\tau$։ Հեշտ է տեսնել, որ այդ պայմանը, լինելով անհրաժեշտ, նաև բավարար պայման է, քանի որ $A$ բազմության ենթաբազմությունների $\{f^{-1}(V)\mid V\in\tau\}$ ընտանիքը, շնորհիվ
\[
f^{-1}\left(\mathsmaller{\bigcup}\limits_{i} V_i\right) = \mathsmaller{\bigcup}\limits_{i} f^{-1}(V_i), \quad f^{-1}\left(\mathsmaller{\bigcap}\limits_{i} V_i\right) = \mathsmaller{\bigcap}\limits_{i} f^{-1}(V)
\]
նույնացումների, բավարարում է տոպոլոգիայի 1-3 աքսիոմներին։

\par Մասնավոր դեպքում, երբ $A\subset X$ իսկ $f$-ը $A\to X$ ներդրումն է, այս տոպոլոգիան համընկնում է $\tau_A$ մակածված տոպոլոգիայի հետ։ 

\par Հիմա քննարկենք մեկ այլ, այս խնդրին նման խնդիր։ Դիցուք ունենք $(X,\tau)$ տոպ․ տարածություն, ինչ-որ $B$ բազմություն և $f:X\to B$ արտապատկերում։ Ի՞նչ տոպոլոգիա վերցնենք $B$-ի վրա, որպեսզի $f$-ը լինի անընդհատ արտապատկերում։ Կարելի է օրինակ $B$-ն օժտել անտիդիսկրետ տոպոլոգիայով և իհարկե $f$-ը կլինի անընդհատ արտապատկերում։ Մենք ցանկանում ենք գտնել $B$-ի համար այն ա\-մե\-նա\-ու\-ժեղ տոպոլոգիան, որի դեպքում $f$-ը կլինի անընդհատ։ Պարզ է, որ այդ տոպոլոգիան իր մեջ պարունակելու է $B$-ի այն բոլոր $V$ ենթաբազմությունները, որոնց $f^{-1}(V)$ նախակերպարները բաց ենթաբազմություններ են $X$-ում։ Դարձյալ վերը բեր\-ված երկու նույնացումներից հետևում է, որ այդ պայմանը ոչ միայն ահրա\-ժեշտ, այլ նաև բավարար է․ այսինքն $B$ բազմության ենթաբազմությունների $\{V\in B\mid f^{-1} (V)\in \tau\}$ ընտանիքը բավարարում է տոպոլոգիայի բոլոր 1-3 աքսիոմներին։

\par Ամփոփենք․ եթե ունենք $f:X\to Y$ արտապատկերում, որտեղ $X$-ը տո\-պո\-լո\-գիա\-կան տարածություն է, իսկ $Y$-ը բազմություն է, ապա $Y$-ի այն բոլոր $V$ են\-թա\-բազ\-մութ\-յուն\-ները, որոնց նախակերպարները բաց են $X$-ում, կազմում են $Y$-ի տոպոլոգիա։ Ընդ որում, դա այն ամենաուժեղ տոպոլոգիան է $Y$-ի վրա, որի դեպքում $f$-ը ա\-նընդ\-հատ է։ Մասնավոր դեպքում, երբ $f$-ը սյուրյեկտիվ արտապատկերում է (այսինքն $f(X)=Y$), այդ տոպոլոգիան կոչվում է \textbf{$\boldsymbol f$ արտապատկերումով ծնված ֆակտոր-տոպոլոգիա} $X$ բազմության վրա, իսկ $Y$ բազմությունը, օժտված այդ ֆակտոր-տոպոլոգիայով, կոչվում է \textbf{$\boldsymbol X$ տարածության ֆակտոր-տարածություն որոշված $\boldsymbol f$ արտապատկերումով}։

\par Պարզաբանենք այս անվանումները։ Եթե $X$ բազմության վրա ունենք $R$ հա\-մար\-ժե\-քութ\-յան հարաբերություն (տես \hyperlink{sec:2}{թեմա 2}-ում), ապա $\faktor{X}{R}$ ֆակտոր-բազմության համար ունենք $P:X\to \faktor{X}{R}$ սյուրյեկտիվ արտապատկերում (կանոնական պրո\-յեկ\-ցիա), որը $\forall x \in X$ կետի համապատասխանեցնում է նրա $P(x)=[x]$ համարժեքութ\-յան դասը։ Եթե $X$-ը նաև տոպոլոգիական տարածություն է, ապա վերը նկարագրված եղանակով $\faktor{X}{R}$ ֆակտոր-բազ\-մութ\-յունը վերածվում է տոպոլոգիական տարածութ\-յան (ֆակտոր-տարածութ\-յան)՝ օժտված $P$ արտապատկերումով որոշված ֆակտոր-տո\-պո\-լոգիայով։

\par Մյուս կողմից՝ ամեն մի $f:X\to Y$ սյուրյեկտիվ արտապատկերում որոշում է $R$ համարժեքության հարաբերություն $X$-ի վրա, ըստ որի $x_1,x_2\in X$ տարրերը համարվում են համարժեք այն և միայն այն դեպքում, երբ $f(x_1)=f(x_2)$։ Դիտարկենք $h:\faktor{X}{R}\to Y$ արտապատկերում՝ սահմանելով $h([x])=f(x)$։ Հեշտությամբ ստուգ\-վում է, որ $h$-ը փոխմիարժեք արտապատկերում է և
\begin{center}
\begin{tikzcd}[row sep=large, column sep = huge,every label/.append style={font=\normalsize}]
& X \arrow{dl}{P} \arrow{dr}{f} & \\
\faktor{X}{R} \arrow{rr}{h}
& & Y
\end{tikzcd}
\end{center}
դիագրամը կոմուտատիվ է, այսինքն $f=h\circ P$։ Համեմատելով $\faktor{X}{R}$ և $Y$ բազմութ\-յուն\-ների վրա $P$ և $f$ արտապատկերումներով որոշված ֆակտոր-տոպոլոգիաները՝ տեսնում ենք, որ $h$-ը, ինչպես նաև $h^{-1}$-ը, բաց ենթաբազմությունները արտա\-պատ\-կերում են բաց ենթաբազմությունների։ Ուստի $h$-ը հոմեոմորֆիզմ է։ Այն տոպո\-լո\-գորեն նույնացնում է $\faktor{X}{R}$ և $Y$ բազմությունները, ինչպես նաև $P$ և $f$ արտապատ\-կերում\-ները։

\par \textbf{Ամփոփենք․} ֆակտոր-տարածություն կարելի է սահմանել երկու համարժեք եղա\-նա\-կով՝ որպես ելակետ վերցնելով կամ \par
ա) $X$ տոպ․ տարածություն, $Y$ բազմություն և որևէ $f:X\to Y$ սյուրյեկտիվ արտապատ\-կերում, կամ էլ 
\par
բ) $X$ տոպոլոգիական տարածություն և $X$ բազմության վրա տրված որևէ $R$ հա\-մար\-ժե\-քութ\-յան հարաբերություն (այսինքն $X$ բազմության որևէ տրոհում)։ \par
Ասվածը ցուցադրենք օրինակներով։
%orinak 1
\begin{example}
\par Վերցնենք $X=[0,1]$ հատվածը $\R$-ի սովորական տոպոլագիայից մակածված տոպոլոգիայով, վերցնենք նաև որևէ երեք՝ $\alpha,\ \beta,\ \gamma$ տարրերից կազմված $ Y=\{\alpha,\beta,\gamma\}$ բազմություն և $f:X\to Y$ սյուրյեկտիվ արտապատկերում, սահմանե\-լով $f(0)=\alpha,\ f(1)=\beta$ և $f(x)=\gamma,\ \forall x \in (0,1)$։ Ըստ ֆակտոր-տոպոլո\-գիայի սահ\-ման\-ման, $Y$-ում բաց են համարվում նրա այն և միայն այն ենթաբազմությունները, որոնց նախակերպարները բաց են $[0,1]$-ում։ Դիտարկելով $Y$-ի բոլոր $2^3=8$ են\-թա\-բազ\-մութ\-յուն\-նե\-րը՝ ստանում ենք․
\par $f^{-1}\varnothing= \varnothing$, $f^{-1}\{\alpha\} = \{0\}$, $f^{-1}\{\beta\} = \{1\}$, $f^{-1}\{\gamma\} = (0,1)$, $f^{-1}\{\alpha,\beta\} = \{0;1\}$, $f^{-1}\{\alpha,\gamma\} = [0,1)$, $f^{-1}\{\gamma,\beta\} = (0,1]$, $f^{-1}\{\alpha,\beta,\gamma\} = [0,1]$։

\par Այս ութ ենթաբազմություններից միայն հինգն են բավարարում վե\-րո\-հիշ\-յալ պայ\-մանին։ Ուստի $f$ արտապատկերումով ծնված՝ $X$-ի ֆակտոր-տարածությունը երեք տարրից կազմված $\{\alpha,\beta,\gamma\}$ բազ\-մությունն է` օժտված $\tau = \{\varnothing,\{\gamma\},\{\alpha,\gamma\},\{\beta,\gamma\},\{\alpha,\beta,\gamma\}\}$ ֆակտոր-տո\-պո\-լո\-գիա\-յով։

\par Այժմ այս նույն օրինակը քննարկենք մյուս՝ երկրորդ ելակետային մոտեցումով․ սահմանենք $\sim$ երկտեղ հարաբերություն $X=[0,1]$ հատվածի կետերի համար, ըստ որի համարելու ենք $0\sim 0,\ 1\sim 1$, և $x_1\sim x_2,\ \forall x_1,x_2\in(0,1)$ կետերի դեպքում (այս և ստորև օրինակներում $xRy$ հարաբերությունը գրառելու ենք $x\sim y$ տեսքով)։

\par Սա համարժեքության հարաբերություն է, ուստի այն որոշում է $[0,1]$ հատվածի ֆակտոր-բազմություն, որի տարրերը համարժեքության երեք դասերն են՝ $[0],\ [1],\ \big[\frac{1}{2}\big]$ (այս վերջին հա\-մար\-ժե\-քութ\-յան դասը կարող է ներկայացվել իր ցանկացած այլ $t\in(0,1)$ ներկայացուցչով՝ $[t]$)։ Նշանակելով $\alpha =[0],\ \beta=[1],\ \gamma=\big[\frac{1}{2}\big]$, կունենանք $\faktor{[0,1]}{\sim}=\{\alpha,\beta,\gamma\}$ նույնացում և կանոնական $P:[0,1]\to\{\alpha,\beta,\gamma\}$ պրոյեկցիա՝ $P(0)=[0],\ P(1)=[1]$ և $P(x)=[\frac{1}{2}]$, երբ $x\in (0,1)$։

\par Ինչ վերաբերում է $\{\alpha,\beta,\gamma\}$ ֆակտոր-բազմության ֆակտոր-տոպոլոգիային, ապա այն որոշվում է ինչպես նախորդ դեպքում և բերում է նույն արդյունքին։ Նկատենք, որ այս դեպքում $f$ արտապատկերման դեր կատարում է $P$ պրոյեկցիան։

\begin{note} $\faktor{X}{R}$ ֆակտոր-տարածությունը կարող է չժառանգել $X$ տա\-րա\-ծութ\-յան որոշ տոպոլոգիական հատկություններ։ Քիչ առաջ դիտարկված օրինակում $[0,1]$-ը հաուս\-դոր\-ֆյան տարածություն է, մինչդեռ $\faktor{[0,1]}{\sim}$ տարածությունը հաուս\-դորֆ\-յան չէ (ինչո՞ւ)։
\end{note}
\end{example}

%orinak 2
\begin{example}
Նորից դիտարկենք $X=[0,1]$ հատվածը $\R$-ի սովորական տո\-պո\-լո\-գիա\-յից մակածված տոպոլոգիայով և սահմանենք համարժեքության $\sim$ հա\-րա\-բե\-րու\-թյուն $[0, 1]$-ի վրա համարելով $x\sim x,\ \forall x\in[0,1]$ կետի դեպքում, և բացի այդ $0\sim 1$։ Այս օրինակում համարժեքության դասերն անթիվ են․ դրանք $[0,1]$ հատվածի մի կետանոց $\{x\},\ x\in(0,1)$ ենթաբազմություններն են և երկկետանոց $\{0,1\}$ ենթաբազմությունը։ Ստացված $\faktor{[0,1]}{\sim}$ ֆակտոր-բազմությունը կարելի է նույ\-նացնել 1 միավոր երկարությամբ շրջանագծի կետերի բազմության հետ (տես \red{օրինակ 9-ը թեմա 2}-ում)։

\par Անցումը $[0, 1]$-ից շրջանագծի կարելի է պատկերացնել հետևյալ գծագրի օգ\-նութ\-յամբ․ մենք պարզապես նույնացնում կամ «սոսնձում» ենք $[0, 1]$ հատվածի 0 և 1 ծայրակետերը, հատվածից անցնելով շրջանագծի։ 
%====================
% NKAR 1
%====================
\import{tikz/}{id20.tex}
\par Այժմ նկարագրենք ստացված շրջանագծի ֆակտոր-տոպոլոգիան։ Նկատենք, որ շրջանագծի ամեն մի անեզր $\alpha\beta$ աղեղ, որը չի պարունակում ֆակտոր-բազմության $\{0; 1\}$ կետը (դասը), պատկանում է ֆակտոր-տոպոլոգիային, քանի որ նրա $P^{-1}(\alpha\beta)$ նա\-խա\-կեր\-պարը $P$ պրո\-յեկ\-ցիա\-յի դեպքում $(\alpha,\beta),\ 0<\alpha<\beta<1$ տեսքի բաց ինտերվալ է $[0, 1]$ հատվածում։ Իսկ եթե շրջանագծի որևէ $\delta\gamma$ անեզր աղեղ պարունա\-կում է $\{0; 1\}$ կետը, ապա նրա $P^{-1} (\delta\gamma)$ նախակերպարը $[0,\gamma)\cup(\delta,1]$ միավորումն է, որը բաց ենթաբազմություն է $[0, 1]\textrm{-ում}$ (թվային ուղղից մակածված տոպոլոգիայում)։
%====================
% NKAR 2
%====================
\import{tikz/}{id21.tex}

\par Այսպիսով շրջանագծի (որպես ֆակտոր-բազմության) ֆակտոր-տոպոլոգիան\\ կազմված է նրա բոլոր անեզր աղեղներից և դրանց միավորումներից։ Նկատենք, որ այն ակնհայտորեն համընկնում է $\R^2$ հարթությունից շրջանագծի վրա մակածված տոպոլոգիայի հետ։

\par \textbf{Դիտողություն։} Ընթերցողի մոտ կարող է հարց առաջանալ․ արդյո՞ք պարտադիր էր \hyperref[օրինակ 2]{օրինակ 2}-ում որպես ֆակտոր-բազմություն վերցնել 1 միավոր երկարությամբ շրջանագիծը։ Այս մասին խոսվել է \hyperlink{sec:2}{թեմա 2}-ում․ ֆակտոր-բազմության մոդելի ընտրու\-թյան հարցում մենք ունենք ազատություն։ Կարելի է շրջանագծի փոխարեն վերցնել ցանկացած էլիպս կամ բազմանկյուն, կամ օվալ (ինքն իրեն չհատող փակ գիծ)։ 

\par Հեշտ է տեսնել, որ այս բոլոր մոդելները, հարթությունից մակածված տոպոլո\-գիայով, միմյանց հո\-մեո\-մորֆ տոպոլոգիական տարածություններ են և կարող են դիտարկվել որպես $\faktor{[0,1]}{\sim}$ ֆակտոր-տարածության դրսևորումներ (կրկնօրինակներ)։ Ստորև գծագրում\\
% WARNING
% եթե նկարը կարողացանք տեղավորել ապա երևի կից գծագրում
%=====================
% NKAR 3
%=====================
%\begin{minipage}[b]{
% \begin{wrapfigure}{l}{0.25\textwidth}
%   \begin{center}
\import{tikz/}{id22.tex}
% \includegraphics[scale=0.05]{images/id 9.png}
% \end{center}
% \end{wrapfigure}
%\end{minipage}
% \begin{floatingfigure}[l]{30mm}
%    \includegraphics[scale=0.05]{images/id 9.png}
% \end{floatingfigure}
պատկերված օվալն ու հնգանկյուն բազմանկյունը հոմեոմորֆ են միմյանց․ օրինակ, որպես հոմեոմորֆիզմ կարող է դիտարկվել որևէ սևեռված $O$ կետից հնգանկյան կենտ\-րո\-նա\-կան $f$ պրոյեկտումը օվալի վրա։

\par \hyperref[Օրինակ 2]{Օրինակ 2}-ի իմաստով $\faktor{X}{\sim}$ ֆակտոր-տարածության մասին հաճախ ասում են, որ այն ստացվում է $X$-ից նրա որոշ տարրերի \textbf{սոսնձումներով} կամ \textbf{նույնացումներով}։

\par Ստորև բերվող ֆակտոր-տարածությունների օրինակներում մենք կբավարար\-վենք նշելով նույնացվող (սոսնձվող) տարրերը, պատկերելով ստացվող տարա\-ծութ\-յուն\-ները մակերևույթների տեսքով $\R^3$-ում։
\end{example}

%orinak 3
\begin{example}
Վերցնենք թղթի շերտ $ABCD$ ուղղանկյան տեսքով և նույնացնենք $AB$ և $DC$ կողմերի կետերը զույգ առ զույգ, պահպանելով նշված ուղղությունները։
%=================================
% NKAR 4
%=================================

\par Արդյունքում ստանում ենք գլան։ Նրա եզրը կազմված է երկու շրջանագծերից։ Որպես վարժանք ընթերցողին առաջարկում ենք․ $ABCD$ ուղղանկյան կետերի տվյալ նույնացումները ներկայացնել որպես համարժեքության հարաբերության դասեր։ Ըն\-թեր\-ցո\-ղին առաջարկում ենք ցույց տալ, որ գլանի ֆակտոր-տոպոլոգիան համընկ\-նում է $\R^3$-ից նրա վրա մակածված տոպոլոգիայի հետ։
%================================
% NKAR 5
%================================
\begin{center}
\begin{tikzpicture}[lablum/.style={name=img-#1},
marr/.style={line width=1mm,-latex}]
    \node[lablum=1] at (-3,0) {\includegraphics[width=7cm]{images/id24.jpg}} ;
    \node at (-6.75,-1.5) {$A$};
    \node at (-6.75,1.5) {$B$};
    \node at (0.75,-1.5) {$D$};
    \node at (0.75,1.5) {$C$};
    \node[lablum=2] at (5,0) {\includegraphics[width=4cm]{images/id24_2.png}} ;
    \draw[thick, ->] (img-1) -- (img-2);

\end{tikzpicture}
\end{center}

%orinak 4
\begin{example}
Նորից վերցնենք $ABCD$ ուղղանկյունը, բայց այս անգամ նույնաց\-նենք $BA$ և $DC$ ուղղորդված հատվածների կետերը՝ պահպանելով դրանց ուղղութ\-յուն\-ները։

\par Ստացված ֆակտոր-տարածությունը կոչվում է \textbf{Մյոբիուսի թերթ}։ Այն միակող\-մա\-նի մակերևույթ է $\R^3$-ում (նկատենք, որ նրա եզրը շրջանագիծ է)։
\begin{center}
\import{tikz/}{id23.tex}
\end{center}
\end{example}
\par Պարզաբանենք՝ ինչ է նշանակում միակողմանի մակերևույթ $\R^3$-ում։ Սովորա\-բար դա ներկայացնում են այսպես․ գլանը կամ սֆերան կարելի է ներկել երկու գույնով՝ դրսից մի գույնով (օրինակ՝ կարմիր), իսկ ներսից այլ գույնով (օրինակ՝ կապույտ)։ Մինչդեռ Մյոբիուսի թերթի դեպքում սկսելով այն ներկել ինչ-որ տեղից, ասենք կանաչ գույնով, անընդհատ շարժվելով մակերևույթով, կպարզվի, որ մակե\-րևույթը թե՛ «ներսից» և թե՛ «դրսից» ներկվել է մի՝ կանաչ գույնով։ Այսինքն՝ այս մակերևույթի համար իմաստ չունեն ներս, դուրս հասկացությունները, ուստի այն ունի մի երես (չնայած, որ յուրաքանչյուր կետի բավականաչափ փոքր շրջակայք ունի երկու երես, որոնք կարելի է ներկել տարբեր գույներով)։

\par Սկզբունքորեն սա ընդունելի է, եթե մենք համարում ենք, որ մակերևույթն ունի \textbf{հաստություն}։ Մինչդեռ մաթեմատիկայում մակերևույթները չունեն հաստություն և անիմաստ է երկու երես հասկացությունը։

\par Մյուս կողմից՝ եթե պատկերացնենք, որ դիտորդը կանգնած է մակերևույթի վրա, ապա նա անընդհատ տեղաշարժվելով կարող է հայտնվել մակերևույթի նկատմամբ կամ մի, կամ մյուս կողմում։ Այս հանգամանքը թույլ է տալիս ներմուծել միակողմանի կամ երկկողմանի մակերևույթ հասկացությունները։

\par Դիցուք մակերևույթի որևէ $a$ կետում ընտրված է մակերևույթի երկու նորմալ միավոր վեկ\-տոր\-նե\-րից մեկը՝ $\vec n_{a}$։ Դիտարկենք մակերևույթի վրա $a$ կետով անցնող որևէ օվալ և շարժվելով $a$ կետից օվալի երկայնքով, անընդհատ տեղափոխենք $\vec n_{a}$ վեկտորն այնպես, որ այն շարունակ մնա ուղղահայաց մակերևույթին։ Ապա $a$ կետ վերադառ\-նալիս հնարավոր է երկու դեպք՝ ա) նորմալ վեկտորը հայտնվում է նախկին դիրքում, բ) նորմալ վեկտորը փոխել է իր ուղղությունը (այսինքն համընկել է $-\vec n_{a}$ վեկտորի հետ)։ Եթե տեղի ունի ա)-ն բոլոր հնարավոր օվալների դեպքում, ապա ասում են, որ մակերևույթը \textbf{երկկողմանի մանկերևույթ} է, իսկ եթե մակերևույթի վրա գոյություն ունի օվալ, որ տեղի ունի բ)-ն, ապա մակերևույթը կոչվում է \textbf{միա\-կող\-մա\-նի մակերևույթ}։

\par Նկատենք, որ մակերևույթի միակողմանի կամ երկկողմանի լինելը մակերևույթի ներքին հատկություն չէ։ Այն հետևանք է մակերևույթի ներդրումից $\R^3$-ում։

\par Մյոբիուսի թերթի դեպքում հետևյալ գծագիրը ցույց է տալիս, որ միջին գծի եր\-կայն\-քով մի պտույտ կատարելիս մակերևույթի նորմալը փոխում է իր ուղղությունը, ուստի Մյո\-բի\-ուսի թերթը միակողմանի մակերևույթ է։
%===========================
%  NKAR 6
% ==========================
\begin{center}
\includegraphics[scale=0.5]{images/id25.png}
\end{center}
\par Նկատենք նաև, որ եթե որևէ մակերևույթ իր մեջ պարունակում է Մյոբիուսի թերթ, ապա այն միակողմանի մակերևույթ է։

\par Մի քանի կարևոր, առանց եզր մակերևույթներ ստացվում են ուղղակյան կամ քառակուսու բոլոր կողմերի զույգ առ զույգ նույնացումներով։
\end{example}
%orinak 5
\begin{example}
Նախ նույնացնելով $ABCD$ քառակուսու $AB$ կողմը $DC$ կողմի հետ և այնուհետև նույնացնելով ստացված գլանի վերին և ստորին հիմքերը պահպանելով նշված ուղղությունները, ստացվում է տոր։ 
%============================
%NKAR 7
%============================
\begin{center}
\import{tikz/}{id26.tex}
\end{center}
\par Փորձեք ցույց տալ, որ այս դեպքում ևս տորի ֆակտոր-տոպոլոգիան համընկնում է նրա վրա $\R^3$-ից մակածված տոպոլոգիայի հետ։
\par Նկատենք, որ տորը որպես ֆակտոր-տարածություն կարող է ստացվել նաև $\R^2$ հարթությունից (տես օրինակ 9-ը \hyperlink{sec:2}{թեմա 2}-ում)։
\begin{note} Տոր կարելի է ստանալ $ABCD$ ուղղանկյունուց փոխելով կողմերի նույնացումների հաջորդականությունը․ նախ սոսնձելով $DA$ և $CB$ կողմերը և ապա սոսնձելով $AB$ և $DC$ կողմերը։ Արդյունքում կստանանք տորի մեկ ուրիշ իրացում $\R^3$-ում։
%===================
% NKAR 8
%===================
% \begin{center}
% \import{tikz/}{id27.tex}
% \end{center}
\begin{center}
\begin{tikzpicture}[lablum/.style={name=img-#1},
marr/.style={line width=1mm,-latex}]
    \node[lablum=1] at (-3,0) {
    \begin{tikzpicture}
        \filldraw[thick, pattern=vertical lines, pattern color=gray] (0,0) rectangle (3,2);
        \node[above left] (b) at (0,2) {$A$};
        \node[below left] (a) at (0,0) {$D$};
        \node[above right] (c) at (3,2) {$B$};
        \node[below right] (d) at (3,0) {$C$};
        \draw[thick] (0,0) -- (0,2);
        \draw[thick] (0,0) -- (3,0);
        \draw[thick] (0,2) -- (3,2);
        \draw[thick] (3,0) -- (3,2);
        \draw[-{Stealth[length=2.5mm, width=2.5mm]}] (0,0) -- (0,1);
        \draw[-{Stealth[length=2.5mm, width=2.5mm]}] (0,0) -- (1.5,0);
        \draw[-{Stealth[length=2.5mm, width=2.5mm]}] (0,2) -- (1.5,2);
        \draw[-{Stealth[length=2.5mm, width=2.5mm]}] (3,0) -- (3,1);
    \end{tikzpicture}
    } ;
    
    \node[lablum=2] at (5,2) {\includegraphics[width=4cm]{images/id28_1.jpg}} ;
    \node[lablum=3] at (5,-2) {\includegraphics[width=4cm]{images/id28_2.jpg}} ;
    \draw[thick, ->] (img-1) -- (img-2);
    \draw[thick, ->] (img-1) -- (img-3);
\end{tikzpicture}
\end{center}
\par Ընթերցողին առաջարկում ենք կառուցել կանոնական հոմեոմորֆիզմ դրանց միջև այնպես, որ մի իրացման զուգահեռականներն ու միջօրեականները փո\-խա\-կերպ\-վեն համապատասխանաբար մյուս իրացման միջօրեականների ու զու\-գա\-հե\-ռա\-կան\-նե\-րի։
\end{note}
\end{example}
%orinak 6
\begin{example}
Նորից դիտարկենք $ABCD$ ուղղանկյուն, բայց կողմերի այլ կողմ\-նո\-րո\-շում\-նե\-րով։ 
%=======================
%NKAR 9
%=======================
\begin{center}
\includegraphics[scale=1.25]{images/id27.png}
\end{center}
\par Նույնացնելով $AB$-ն $DC$-ի հետ, այնուհետև նույնացնելով ստացված գլանի վերին և ստորին հիմքերը ըստ նշված ուղղությունների, ստացվում է անեզր մակերևույթ, որը կոչվում է \textbf{Կլեյնի շիշ}։ Այն իր մեջ պարունակում է Մյոբիուսի թերթ (հիմնավորել) և այդ պատճառով նույնպես միակողմանի մակերևույթ է։ Կլեյնի շիշը հնարավոր չէ առանց ինքնահատումների պատկերել եռաչափ տարածության մեջ։ Առանց ինք\-նա\-հա\-տում\-նե\-րի այն իրացվում է քառաչափ տարածությունում։
\end{example}
%orinak 7
\begin{example}
Գոյություն ունի ուղղանկյան կողմերի զույգ առ զույգ նույնացման ևս մի եղանակ (տես գծագիրը)․ այժմ $AB$-ն նույնացվում է $CD$-ի հետ, իսկ $BC\textrm{-ն}$՝ $DA$-ի հետ։
%==============================
%NKAR 10
%==============================
\import{tikz/uniform/}{29.tex}
%\begin{center}
%\includegraphics[scale=1]{images/id29.png}
%\end{center}
Ստացվող մակերևույթը կոչվում է \textbf{պրոյեկտիվ հարթություն}։ Այն հա\-մար\-ժեք ձևով կարող է ստացվել շրջանից, եթե նույնացնենք շրջանի եզրագծի (շրջա\-նագծի) յուրա\-քանչյուր կետ տրամագծորեն իրեն հակադիր կետի հետ (գծա\-գրում նույնացվող կետերը նշված են նույն տառերով)։ Այս մակերևույթը նույնպես իր մեջ պարունակում է Մյոբիուսի թերթ (հիմնավորել) և նույնպես միակողմանի մակերևույթ է։ Այն հնա\-րա\-վոր չէ առանց ինքնահատումների իրացնել $\R^3$-ում, բայց հնարավոր է դա անել $\R^4$-ում (այդ մասին տես Д. Гильберт, С. Кон – Фоссен «Наглядная геометрия» գրքում)։ Ցույց տանք, թե ինչպես է հնարավոր պրոյեկտիվ հարթությունը պատկերել $\R^3$-ում (ինքնահատումով)։
%============================
% NKAR 11
%============================
\begin{center}
\begin{tikzpicture}[lablum/.style={name=img-#1},
marr/.style={line width=1mm,-latex}]
    \node[lablum=1] at (0,0) {
    \begin{tikzpicture}[scale=0.75]
        \filldraw[black, thick, pattern=vertical lines, pattern color=gray] (0,0) rectangle (3,2);
        \node[above left] (b) at (0,2) {$B$};
        \node[below left] (a) at (0,0) {$A$};
        \node[above right] (c) at (3,2) {$C$};
        \node[below right] (d) at (3,0) {$D$};
        \draw[-{Stealth[length=2mm, width=2mm]}] (0,0) -- (0,1);
        \draw[-{Stealth[length=2mm, width=2mm]}] (0,2) -- (1.5,2);
        \draw[-{Stealth[length=2mm, width=2mm]}] (3,0) -- (1.5,0);
        \draw[-{Stealth[length=2mm, width=2mm]}] (3,2) -- (3,1);
        \draw[thick] (0,0) -- (0,2);
        \draw[thick] (0,2) -- (3,2);
        \draw[thick] (3,0) -- (0,0);
        \draw[thick] (3,2) -- (3,0);
    \end{tikzpicture}
    };
    
    \node[lablum=3] at (8,0) {
    \tikzset{every picture/.style={line width=0.75pt}}      
    %\resizebox{width=0.5\textwidth}{!}{
    \begin{tikzpicture}[x=0.75pt,y=0.75pt,yscale=-1,xscale=1, scale=0.8]
\draw [black ]   (263.6,165.7) .. controls (226.8,165.5) and (178,127.3) .. (231.4,51.9) ;

\draw [black ]   (263.6,165.7) .. controls (301.2,167.3) and (354.6,105.5) .. (276,53.1) ;

\draw [black, line join=round]   (263.6,165.7) .. controls (254.8,158.5) and (250.6,130.3) .. (253,74.7) .. controls (262.71,64.79) and (264.43,63.36) .. (276,53.1) .. controls (269.86,54.21) and (261.14,60.07) .. (254.71,66.07) .. controls (249.14,60.79) and (235.57,52.21) .. (231.4,51.9) .. controls (238.71,58.07) and (248.29,68.64) .. (253,74.7) ;

\draw [black ]   (297.86,151.79) .. controls (311.71,139.36) and (320.29,101.64) .. (276,53.1) ;

\begin{scope}[on behind layer]
\draw [gray,line width=.6pt] [ dotted, line cap=round]  (297.86,151.79) .. controls (272.14,174.36) and (172.29,152.64) .. (231.4,51.9) ;
\draw [gray,line width=.6pt] [ dotted, line cap=round]  (263.6,165.7) .. controls (278.29,163.07) and (284.57,145.64) .. (254.71,66.07) ; 
\end{scope}

\draw (225.53,45.6) node  [font=\scriptsize]  {$A$};

\draw (254.96,55.74) node  [font=\scriptsize]  {$B$};

\draw (279.53,45.6) node  [font=\scriptsize]  {$C$};

\draw (242.96,78.17) node  [font=\scriptsize]  {$D$};
\end{tikzpicture}
};

    \node[lablum=2] at (4,0) {
    \tikzset{every picture/.style={line width=0.75pt}}      
    %\resizebox{width=0.5\textwidth}{!}{
    \begin{tikzpicture}[x=0.75pt,y=0.75pt,yscale=-1,xscale=1, scale=0.8]

\draw  [black ] (44,104.75) .. controls (44,74.93) and (68.18,50.75) .. (98,50.75) .. controls (127.82,50.75) and (152,74.93) .. (152,104.75) .. controls (152,134.57) and (127.82,158.75) .. (98,158.75) .. controls (68.18,158.75) and (44,134.57) .. (44,104.75) -- cycle ;

\draw [black ]   (81.22,63.93) .. controls (68.78,63.89) and (62.56,66.78) .. (55,72.11) ;

\draw [black ]   (142.56,135) .. controls (153.22,103.67) and (126.33,79.22) .. (115.22,72.56) ;

\draw [black ]   (81,156.04) .. controls (67,148.04) and (74.56,105.44) .. (90.78,75.44) ;

\draw [black ]   (105.22,58.56) .. controls (106.92,55.88) and (110.92,53.29) .. (114.33,53.78) ;

\draw [black ]   (90.78,75.44) .. controls (93.42,74.13) and (113.58,71.79) .. (115.22,72.56) .. controls (116.86,73.32) and (106.28,59.49) .. (105.22,58.56) .. controls (104.17,57.63) and (80.5,62.46) .. (81.22,63.93) .. controls (81.94,65.41) and (89.5,74.71) .. (90.78,75.44) -- cycle ;

\begin{scope}[on behind layer]
\draw [gray,line width=.6pt] [ dotted, line cap=round]  (114.33,53.78) .. controls (134.11,61.56) and (105.44,158.78) .. (81,156) ;
\draw [gray,line width=.6pt] [ dotted, line cap=round]  (142.56,135) .. controls (111.44,171) and (28.11,111.89) .. (55,72.11) ;
\end{scope}

\draw (80.86,57.57) node  [font=\scriptsize]  {$A$};

\draw (102.43,54.57) node  [font=\scriptsize,fill=white, fill opacity=.7, text opacity=1, inner sep=0pt]  {$B$};

\draw (112.43,79.14) node  [font=\scriptsize]  {$C$};

\draw (82.43,76.29) node  [font=\scriptsize]  {$D$};

\end{tikzpicture}};

\node[lablum=4] at (12,0) {\includegraphics[scale=0.125]{images/id30.png}};
\draw[thick, ->] (1.6,0) -- (2.6,0) node[midway, above] {$f_1$};
\draw[thick, ->] (5.6,0) -- (6.6,0) node[midway, above] {$f_2$};
\draw[thick, ->] (9.5,0) -- (10.5,0) node[midway, above] {$P$};
\end{tikzpicture}
\end{center}
\par Այստեղ $f_1$-ը և $f_2$-ը հո\-մեո\-մոր\-ֆիզմ\-ներ են, իսկ $P$-ն նույնացնում է («սոսնձում» է) $DA$-ն $BC$-ի հետ և $AB$-ն $CD$-ի հետ։
\end{example}
\par Ավարտելով օրինակները՝ կանգ առնենք $P:X\to \faktor{X}{R}$ կանոնական պրոյեկցիայի այսպես կոչված ունիվերսալության հատկության վրա։

\begin{theorem}
Դիցուք $Y$-ը $X$ տարածության ֆակտոր-տարածություն է՝ $Y=\faktor{X}{R}$ ըստ համարժեքության ինչ-որ $R$ հարաբերության, իսկ $P:X\to Y$ արտապատկե\-րու\-մը կանոնական պրոյեկցիան է։ Ապա որևէ $g:Y\to Z$ արտապատկերում անընդ\-հատ է այն և միայն այն դեպքում, երբ անընդհատ է $g\circ P:X\to Z$ համադրույթը։
\end{theorem}
\begin{proof} Եթե $g$-ն անընդհատ է, ապա $g\circ P$-ն անընդ\-հատ է որպես անընդ\-հատ արտապատկերումների համադրույթ։ Այժմ հակառակը․ դիցուք $g\circ P$-ն անընդ\-հատ է։ Ցույց տանք, որ $g$-ն անընդհատ է (նկատենք, որ $g$-ի ա\-նընդ\-հա\-տութ\-յունը չի կարող դուրս բերվել \red{թեմա 11-ի թեորեմ 5-ից}, քանի որ ընդհանուր դեպքում $P:X\to \faktor{X}{R}$ կանոնական պրոյեկցիան չի հանդիսանում բաց կամ փակ ար\-տա\-պատ\-կե\-րում)։ Վերցնենք որևէ $U\in Z$ բաց ենթաբազմություն և ցույց տանք, որ $g^{-1} (U)$-ն բաց է $Y$-ում։ Ունենք՝ $(g\circ P)^{-1} (U)=P^{-1} (g^{-1} (U))$, որը բաց է $X$-ում ըստ պայմանի։ Ուստի $g^{-1} (U)$-ն բաց է $Y$-ում ըստ ֆակտոր-տոպոլոգիայի սահմանման։
\end{proof}

\end{document}