\documentclass[./main.tex]{subfiles}

% 9, 10 tegherob poxel a
\begin{document}
\onehalfspacing
\section{Տոպոլոգիական տարածությունների անընդհատ արտապատկերումներ, թեորեմ մետրիկային տարածությունների արտապատկերման անընդհատության մասին։ Անընդհատության հայտանիշներ՝ բաց (փակ) ենթաբազմությունների տերմիններով, ենթաբազմությունների փակման և ներքնամասի տերմիներով։}
\label{sec:10}

\par Մինչև այժմ մենք դիտարկում էինք տոպոլոգիական տարածություններն առանձին-առանձին, միմյանցից անկախ։ Այժմ զբաղվելու ենք դրանց համեմատմամբ։ Այդ նպատակով ներմուծվում է տոպոլոգիական տարածությունների անընդհատ արտա\-պակերման հասկացությունը, որը երկրորդ կարևորագույն հասկացությունն է տո\-պոլոգիական տարածություն հասկացությունից հետո։

\begin{definition}
Դիցուք ունենք $X, \tau$, $Y, \sigma$ տոպոլոգիական տարածություններ և ${f:X \rightarrow Y}$ արտապատկերում։ Ապա $f$-ը կոչվում է \textbf{անընդհատ $\boldsymbol{x_0\in X}$ կետում}, եթե $Y$ տարածությունում $f(x_0)=y_0$ կետի ամեն մի $V$ շրջակայքի համար գոյություն ունի $X$ տարածությունում $x_0$ կետի $U$ շրջակայք, որ $f(U)\subset V$։

\end{definition}
\par Հետևյալ պնդումը երբեմն հեշտացնում է անընդհատության պայմանի ստուգումը։

\begin{theorem} Ունենք տոպոլոգիական տարածությունների $f:X \rightarrow Y$ արտապատկերում, $f(x_0)=y_{0}$, իսկ $\beta_{x_0}=\{U_i(x_0) \mid i\in I\}$ և $\beta_{y_0}=\{V_j(y_0), j\in J\}$ ընտանիքները համապատասխանաբար $x_0$ և $y_0$ կետերի շրջակայքերի որևէ բազաներ են $X$ և $Y$ տարածություններում։ Ապա $f\textrm{-ը}$ անընդհատ է $x_0$ կետում այն և միայն այն դեպքում, երբ $\forall\, V_j(y_0)\in \beta_{y_0}$ շրջակայքի համար գոյություն ունի $U_i(x_0)\in \beta_{x_0}$ շրջակայք, որ $f(U_i (x_0 ))\subset V_j(y_0)$։
\label{թեորեմ $1$}
\end{theorem}

\begin{proof} Դիցուք $f$-ը անընդհատ է $x_0$ կետում, և ունենք որևէ $V_j(y_0)\in \beta_{y_0}$։ Ըստ կետում անընդհատության սահմանման, գոյություն ունի $x_0$ կետի $U$ շրջակայք, որ $f(U)\subset V_j(y_0)$։ Համաձայն կետի շրջակայքերի բազայի սահմանման, գոյություն ունի $U_i(x_0)\in \beta_{x_0}$, որ $U_i(x_0)\subset U$։ Այժմ ստանում ենք՝ $f(U_i(x_0))\subset f(U)\subset V_j(y_0 )$։
\par Հիմա հակառակը․ դիցուք $V$-ն $y_0$ կետի որևէ շրջակայք է։ Գոյություն ունի $V_j(y_0)\in \beta_{y_0}$, որ $V_j(y_0)\subset V$։ Ըստ պայմանի, գոյություն ունի $U_i(x_0)\in \beta_{x_0}$, որ $f(U_i(x_0))\subset V_j(y_0)\subset V$։ Տվյալ $V$ շրջակայքի համար որպես $U$ շրջակայք վերցնելով $U_i(x_0)$-ն ստանում ենք $f(U)\subset V$։ Ուստի $f$-ը անընդհատ է $x_0$ կետում։ 
\end{proof}
\par Կիրառենք այս թեորեմը մասնավոր դեպքում, երբ տոպոլոգիական տարածութ\-յունները մետրիկային են՝ $(X,\rho)$ և $(Y,\rho')$։
\par Ինչպես գիտենք, այդ տարածություններում $\{\mathcal{D}(x_0,r)\mid r>0\}$ և $\{\mathcal{D}{(y_0,{r})\mid r>0}\}$ անեզր գնդերի ընտանիքները կազմում են բազա համապատասխանաբար $x_0$ և $y_0$ կետերի շրջակայքերի համար (տե՛ս \hyperlink{sec:8}{թեորեմ $2$}-ը թեմա $7$-ում)։ Այստեղից և \hyperref[թեորեմ $1$]{թեորեմ $1$}-ից ստանում ենք․ \red{ref teorems}
\begin{hetevanq} Մետրիկային տարածությունների $f:X \rightarrow Y$ արտապատկերումը անընդհատ է $x_0\in X$ կետում այն և միայն դեպքում, երբ $Y$-ում ամեն մի $\mathcal{D}(y_0,{r'})$ գնդի համար, որտեղ $y_0=f(x_0)$, գոյություն ունի $X$-ում $\mathcal{D}(x_0,r)$ գունդ, որ $f(\mathcal{D}(x_0,r))\subset \mathcal{D}(y_0,{r'})$։ 
\end{hetevanq}

\begin{definition}
Ունենք $X$ և $Y$ տոպոլոգիական տարածություններ և $f:X  \rightarrow  Y$ արտա\-պատկերում, ապա $f$-ը կոչվում է \textbf{անընդհատ արտապատկերում}, եթե այն անընդ\-հատ է $X$-ի բոլոր կետերում։
\end{definition}
Սկզբի համար բերենք անընդհատ արտապատկերումների երկու պարզ օրինակ։
\par ա) Ցանկացած տոպոլոգիական տարածության նույնական արտապատկերումը ինքն իր վրա անընդհատ արտապատկերում է։ Իրոք, $f=\nuynakan_X:X  \rightarrow  X,\ f(x)=x$ արտա\-պատ\-կեր\-ման դեպքում $y_0=f(x_0)=x_0$ կետի $V$ շրջակայքի համար որպես $x_0$ կետի $U$ շրջակայք վերցնելով $V$-ն, կստանանք $f(U)=V\subset V$։ \par բ) Դիցուք $(X,\tau)$-ն և $(Y,\sigma )$-ն կամայական տոպոլոգիական տարածություններ են, $y_0\in Y$ սևեռված կետ է։ Ընդունված է $c:X  \rightarrow Y,\ c(x)=y_0,\ \forall x\in X$ արտապատկե\-րումն անվանել $X$-ի \textbf{հաստատուն արտապատկերում} $Y$-ի $y_0$ կետի վրա։ Այն անընդհատ արտապատ\-կերում է։ Իրոք, քանի որ $c^{-1}(y_0)=X$, ուստի $y_0$-ի $\forall\, V$ շրջակայքի համար որպես $ \forall x_0\in X$ կետի $U$ շրջակայք վերցնելով $X$-ը, կունենանք $f(U)\subset V$։

\begin{theorem} 
\label{թեորեմ $2$}
Ունենք $(X, \rho)$ և $(Y, \rho')$ մետրիկային տոպոլոգիական տարածություններ։ Ապա $f:X  \rightarrow  Y$ արտապատկերումն անընդհատ է այն և միայն այն դեպքում, երբ ցանկացած $x_0\in X$ կետի և $\varepsilon >0$ թվի համար գոյություն ունի $\delta=\delta(x_0, \varepsilon)>0$ թիվ, որ եթե $\rho(x,x_0)<\delta$, ապա ${\rho'} (f(x),f(x_0 ))< \varepsilon$։
\end{theorem}
Ապացուցումը անմիջականորեն հետևում է անընդհատության սահմանումից և \hyperref[թեորեմ $1$]{թեորեմ $1$}-ի հետևանքից \red{ref theorem}։
\begin{theorem}\label{թեորեմ $3$}
Ունենք $X$ և $Y$ տոպոլոգիական տարածություններ և ${f:X  \rightarrow  Y}$ արտապատկերում։ Ապա $f\textrm{-ը}$ անընդհատ է այն և միայն այն դեպքում, երբ $Y\textrm{-ում}$ բաց ցանկացած ենթաբազմության նախակերպարը բաց ենթաբազմություն է $X\textrm{-ում}$։
\end{theorem}
\begin{proof}
    ա)
    Ցույց տանք պայմանի անհրաժեշտությունը։ Դիտարկենք որևէ $x_0\in f^{-1} (V)$ կետ։ Քանի որ $f(x_0 )\in V$, ուստի $x_0$ կետում $f$-ի անընդ\-հա\-տութ\-յունից հետևում է, որ գոյություն ունի $x_0$-ի $U$ շրջակայք, որ $f(U)\subset V$։
    \par Ունենք՝ $U\subset f^{-1}(f(U))\subset f^{-1} (V)$, որից հետևում է, որ $f^{-1} (V)$-ի ցանկացած $x_0$ կետ ներքին կետ է նրա համար, ուստի $f^{-1} (V)$-ն բաց ենթաբազմություն է $X$-ում։
    \par բ) Այժմ ապացուցենք պայմանի բավարարությունը։ Դրա համար ցույց տանք, որ $f$-ը անընդհատ է ցանկացած $x_0\in X$ կետում։ Դիցուք $V$-ն $y_0=f(x_0)$ կետի որևէ շրջակայք է։ Գոյություն ունի $y_0$-ի $W$ բաց շրջակայք, որ $W\subset V$։ Ըստ պայմանի $f^{-1}(W)$-ն բաց ենթաբազմություն է $X$-ում։ Ուրեմն $U=f^{-1}(W)$-ն (բաց) շրջակայք է $x_0$-ի համար։ Ունենք՝ $f(U)=f(f^{-1}(W))\subset W\subset V$, ուստի $f$-ը անընդհատ է $x_0$ կետում։\qedhere
\end{proof}

\par Նշենք, որ թեորեմն ապացուցելիս մենք օգտվեցինք հետևյալ ներդրումներից (տե՛ս թեորեմ $3$-ը թեմա $1$-ում)․ եթե ունենք $f:X  \rightarrow  Y$ արտապատկերում, $A\subset X$, $B\subset Y$, ենթաբազմություններ ապա \[f(f^{-1}(B))\subset B,\quad A\subset f^{-1} (f(A))։\]


\begin{hetevanq}
Եթե հայտնի է $Y$ տարածության տոպոլոգիայի որևէ բազա, ապա $f: X \to Y$ արտապատկերման անընդհատության համար բավական է պահանջել, որ $X$-ում բաց լինեն տվյալ բազայի բոլոր տարրերի նախապատկերները (հիմնավորե՛ք):
\end{hetevanq}

\begin{hetevanq} Եթե $f:X  \rightarrow  Y$, $g:Y  \rightarrow  Z$ արտապատկերումները անընդհատ են, ապա նրանց $g\circ f$ համադրույթը նույնպես անընդհատ է (հիմնավորե՛ք)։
\end{hetevanq}

\begin{theorem}[(անընդհատության հայտանիշ փակ ենթաբազմությունների տերմին-\newline ներով)] \label{թեորեմ $4$}
Ունենք $X$ և $Y$ տոպոլոգիական տարածություններ։ Ապա որևէ $f:X  \rightarrow  Y$ արտապատկե\-րու\-մը անընդհատ է այն և միայն այն դեպքում, երբ $Y$-ում փակ ցանկացած ենթա\-բազ\-մութ\-յան նախակերպարը փակ ենթաբազմություն է $X$-ում։
\end{theorem}

\par Ապացուցելիս կօգտվենք ${f^{-1} (Y \setminus Z)=X \setminus f^{-1}} (Z)$ նույնությունից, որտեղ $Z\subset Y$։
\par ա) Ենթադրելով, որ $f$-ը անընդհատ է, կունենանք․ եթե $F$-ը փակ է $Y$-ում, ապա $(Y\setminus F)$-ը բաց է $Y$-ում, և ուրեմն $f^{-1} (Y  \setminus F)$-ը բաց է $X$-ում, ըստ թեորեմ 3-ի \red{ref}։ Ուստի $f^{-1}(F)=X  \setminus f^{-1}(Y  \setminus F)$ ենթաբազմությունը փակ է $X$-ում։
\par բ) Հակառակը․ եթե $V$-ն բաց է $Y$-ում $ \Rightarrow  (Y  \setminus V)$-ն փակ է $Y$-ում $  \Rightarrow  f^{-1} (Y  \setminus V)$-ն փակ է $X$-ում $  \Rightarrow  f^{-1} (V)=X  \setminus f^{-1} (Y  \setminus V)$ ենթաբազմությունը բաց է $X$-ում։ Ուստի $f$-ը անընդհատ է ըստ \hyperref[թեորեմ $3$]{թեորեմ $3$}-ի \red{ref}։ \qed

\par Այժմ բերենք արտապատկերումների անընդհատության մի քանի հայտանիշ ենթաբազմությունների փակման գործողության տերմիններով։
\par Ենթագիտակցաբար, $f:X \rightarrow Y$ անընդհատ արտապատկերումը մենք պատկերացնում ենք որպես այնպիսի արտապատկերում, որը $X$-ի ցանկացած երկու «բավականաչափ մոտիկ» ($X$-ի տոպոլոգիայի իմաստով) $x_1$ և $x_2$ կետերի համապատասխանեցնում է $Y$-ի «բավականաչափ մոտիկ» $f(x_1)$ և $f(x_2)$ կետեր ($Y$-ի տոպոլոգիայի իմաստով)։ Մյուս կողմից, ենթաբազմության հպման կետը մենք պատկերացնում ենք որպես այդ ենթաբազմությանը «շատ մոտիկ», կամ նրան կպած կետ այն իմաստով, որ նրան հնարավոր չէ անջատել ենթաբազմությունից մի որևէ բաց շրջակայքով։ Ուստի ճշմարիտ է թվում հետևյալ պնդումը․ եթե $x \in X$ կետը հպման կետ է $A\subset X$ ենթաբազմության համար, ապա $f(x)$ կետը հպման կետ է $f(A)\subset Y$ ենթաբազմության համար։ \red{tvum?}

\begin{theorem}[(անընդհատության հայտանիշ փակման գործողության տերմիններով)] \label{թեորեմ $5$}
\par $f:X \rightarrow  Y$ արտապատկերումը անընդհատ է այն և միայն այն դեպքում, երբ ցանկացած $A\subset X$ ենթաբազմության դեպքում $f( \widebar{A} )\subset \widebar{f(A)}$։
\end{theorem}

\begin{proof}
 \textbf{Պայմանի անհրաժեշտությունը։} Դիցուք $f$-ը անընդհատ է։ Վերցնենք կամայական $y\in f(\widebar{A})$ կետ և ցույց տանք, որ $y\in \widebar{f(A)}$։ Իրոք, գոյություն ունի $x\in \widebar{A}$ կետ, որ $f(x)=y$։ Դիտարկենք $y$ կետի կամայական $V$ շրջակայք։ Ըստ $x$ կետում $f$-ի անընդհատության, գոյություն ունի $x$-ի $U$ շրջակայք, որ $f(U)\subset V$։ Քանի որ $U\cap A\neq \varnothing $, ուստի $f(U)\cap f(A)\neq \varnothing \Rightarrow V  \cap f(A)\neq  \varnothing \Rightarrow y\in  \widebar{f(A)}$։ Այսպիսով՝ $f(\bar{A}) \subset  \widebar{f(A)}$։ 
 \par
 \textbf{Պայմանի բավարարությունը։} Դիտարկենք կամայական $B\subset Y$ փակ ենթաբազմություն՝ $B=\widebar{B}$ և ապացուցենք, որ $A=f^{-1}(B)$  ենթաբազմությունը փակ է $X$-ում (դրանից կհետևի, որ $f$-ը անընդհատ է ըստ \hyperref[թեորեմ $4$]{թեորեմ $4$}-ի) \red{ref}։

\par Դրա համար ցույց տանք, որ $A$-ն պարունակում է իր բոլոր հպման կետերը։ Եթե $x\in \widebar{A}$, ապա ըստ պայմանի $\left(f(x)\in f(\widebar{A})\subset \widebar{f(A)}\subset \widebar{B}=B\right)$: Ուստի $\left(x \in {f^{-1}(B)}=A\right)$, որից էլ ստանում ենք $\left(\widebar{A}=A\right)$։ Ուրեմն $A$-ն փակ է $X$-ում։
\end{proof}

\par Գոյություն ունի նաև արտապատկերման անընդհատության հայտանիշ ենթա\-բազ\-մության ներքինի տերմիններով։
\begin{theorem} 
$f:X  \rightarrow Y$ արտապատկերումը անընդհատ է այն և միայն այն դեպքում, երբ կամայական $B\subset Y$ ենթաբազմության համար $f^{-1} (\inter B)\subset \inter (f^{-1} (B))$։
\end{theorem}

\begin{proof}
    ա) Ենթադրենք $f$-ը անընդհատ է։ Վերցնենք կամայական $x\in f^{-1} (\inter B)$ կետ և ցույց տանք, որ $x\in \inter(f^{-1} (B))$։ Ունենք՝ $f(x)\in \inter B\subset B$, ուրեմն $x\in f^{-1} (\inter B)\subset f^{-1} (B)$։ Քանի որ, ըստ \hyperref[թեորեմ $3$]{թեորեմ $3$}-ի \red{ref..}, $f^{-1} (\inter B)$-ն բաց ենթաբազմություն է $X$-ում, ուստի $x$ կետը ներքին կետ է $f^{-1} (B)$-ի համար  $\Rightarrow  x\in \inter(f^{-1} (B))$։
    \par բ) Հակառակ պնդումը ապացուցելու համար վերցնենք $\forall  V\subset Y$ բաց ենթա\-բազ\-մություն և ցույց տանք, որ $f^{-1}$ $(V)$-ն բաց է $X$-ում։ Շարունակությունը թողնում ենք ընթերցողին։
\end{proof}

\begin{definition}
Արտապատկերումը կոչվում է \textbf{խզվող}, եթե այն անընդհատ չէ։ Բերենք խզվող արտապատկերումների օրինակներ՝ հիմնավորումները կատարելով \hyperref[թեորեմ $5$]{թեորեմ $5$}-ում \red{ref} բերված հայտանիշի միջոցով։
\end{definition}

\begin{example}
Դիտարկենք $(\R, \textrm{սովոր․})$ տարածությունը և $f:\R \rightarrow \R$ արտապատ\-կե\-րումը, որտեղ $f(x)$-ը $x\in \R$ թվի ամբողջ մասն է։ Ըստ սահմանման, եթե $x\in [n,n+1),\ n\in \Z$, ապա $f(x)=n$։ Մասնավորապես $f(n)=n,\forall n \in \Z$։ Վերցնելով $A=[n-1,n)$՝ կունենանք $f(A)=\{n-1\}$ և $\widebar A =[n-1; n]$։ Բացի այդ $n\in \widebar{A} \Rightarrow f(n)\in f(\widebar{A})$։ Ենթադրելով, որ $f$-ը անընդհատ է $n\in Z$ կետում, կստանանք՝ $f(n)\in f(\widebar{A})\subset \widebar{f(A)} =\widebar{\{n-1\}}=\{n-1\}$։ Ստացանք $f(n)=n-1$ (հակասություն)։ Հետևաբար $f$-ը խզվող է, անընդհատ չէ $\R$-ի $n\in \Z$ կետերում։
\end{example}
\begin{example} Մաթեմատիկական անալիզի դասընթացում բերվում է թվային ֆունկցիայի օրինակ (Դիրիխլեի ֆունկցիան), որն անընդհատ չէ $\R$-ի բոլոր կետերում։ Այն սահմանվում է որպես $f:\R \rightarrow \R$ արտապատկերում, որտեղ $f(x)=0$, երբ $x$-ը ռացիոնալ է՝ $x\in \Q$ և $f(x)=1$, երբ $x$-ը իռացիոնալ է՝ $x\in I$։ Նկատենք, որ $\R=\Q\cup I$, $\Q\cap I=\varnothing$, $\widebar{\Q}=\widebar{I}=\R$։ Այդ արտապատկերումը ընդհանրացվում է հետևյալ տեսքով։ \red{chem jokum inch a uzum}

\par Դիցուք $X$-ը տոպոլոգիական տարածություն է, ընդ որում գոյություն ունեն $X$-ի այնպիսի $A$ և $B$ ենթաբազմություններ, որ $X=A\cup B$, $A\cap B=\varnothing$, $\widebar{A}=\widebar{B}=X$։ Ապա $f:X \rightarrow \R$ արտապատկերումը, որտեղ $f(x)=0$, երբ $x\in A$ և $f(x)=1$, երբ $x\in B$, անընդհատ չէ $X$-ի բոլոր կետերում։ Ապացուցելու համար վերցնենք որևէ $x_0\in A$ կետ։ Ունենք $f(x_0)=0$։ Մյուս կողմից, քանի որ $X=\widebar{B}$, ստանում ենք՝  $x_0\in \widebar{B}$ որից հետևում է $f(x_0)\in f(\widebar{B})\subset  \widebar{f(B)}=\widebar{\{1\}}=\{1\}$։ Ստացանք $f(x_0)=1$ (հակասություն)։ Նման ձևով ապացուցված է, որ $f$-ը անընդհատ չէ նաև $B$-ի բոլոր կետերում։
\end{example}
\end{document}