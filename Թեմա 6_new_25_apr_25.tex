\documentclass[./main.tex]{subfiles}
\DeclareMathOperator{\exter}{ext}
 
\begin{document}
\onehalfspacing


% ISSUES:
% - Տերմիններ․  ներքին -> ներքնամաս, արտաքին -> արտաքնամաս, փակույթ -> փակում


\section{Կետի դիրքը ենթաբազմության նկատմամբ․ ենթաբազմության ներքին և արտաքին կետեր, ենթաբազմության ներքինն ու արտաքինը, ենթաբազմության եզրային կետեր և ենթաբազմության եզր։ Փակման գործողություն բազմության վրա, Կուրատովսկու թեորեմը։}\label{sec:6}

Հիշեցնենք, որ թվային ուղղի $X$ ենթաբազմության $x_0$ կետը կոչվում է նրա ներքին կետ, եթե գոյություն ունի այդ կետի որևէ $U(x_0, \varepsilon) = (x_0 - \varepsilon,\ x_0 + \varepsilon)$ $\varepsilon$-շրջակայք, որ $U(x_0, \varepsilon) \subset X$: Նույնպիսի հասկացություն, հետևյալ ընդհանուր տեսքով, ներմուծվում է բոլոր տոպոլոգիական տարածություններում։

Դիցուք $(X, \tau)$-ն տոպոլոգիական տարածություն է, $A$-ն $X$-ի որևէ ենթաբազմու\-թյուն է։

\begin{definition}
$x \in X$ կետը կոչվում է $A$ \textbf{ենթաբազմության ներքին կետ}, եթե գոյություն ունի $x$-ի որևէ $U_x$ շրջակայք, որ $x\in U_x \subset A$։ Տվյալ $A$ ենթաբազմության բոլոր ներքին կետերի բազմությունը կոչվում է $A$-ի \textbf{ներքինը} և նշանակվում է $\inter A$։
\end{definition}

Նկատենք, որ ներքին կետի սահմանման մեջ կարելի էր պահանջել, որ $U_x$-ը լինի $x$ կետի բաց շրջակայք։ Այն, որ սահմանման այդ երկու տարբերակները համարժեք են միմյանց, հետևում է կետի շրջակայքի սահմանումից: (հիմնավորե՛ք)։
%\par \vspace{4.5pt} \textbf{Ներքնամասի հատկությունները․}
\paragraph*{Ենթաբազմության ներքինի հատկությունները․}
\begin{enumerate}
    \item $\inter A \subset A$ և $\inter A$-ն $X$-ի բաց ենթաբազմություն է։ Իրոք, ըստ ներքին կետի համարժեք սահմանման՝ ամեն մի $x \in \inter A$ կետի համար գոյություն ունի $U_x$ բաց շրջակայք, որ $x \in U_x \in \inter A$։ Այստեղից ստանում ենք՝ $\inter A = \bigcup U_x$, որտեղ միավորումը կատարվում է ըստ բոլոր $x \in \inter A$ կետերի։ Հետևաբար $\inter A$-ն բաց ենթաբազմություն է՝ որպես $U_x$ բաց ենթաբազմությունների միա\-վորում։
    
    \item $\inter A$-ն $A$-ի մեջ պարունակվող բոլոր բաց ենթաբազմություններից ամենա\-մեծն է հետևյալ իմաստով․ եթե $V$-ն $A$-ի որևէ բաց ենթաբազմություն է, ապա $V \subset \inter A$։ Իրոք, $V$-ի բոլոր կետերը ներքին կետեր են $V$-ի համար, հետևա\-բար ներքին կետեր են նաև $A$-ի համար, ուստի $V \subset \inter A$։
    
    \item $A \mapsto \inter A$ համադրումը գործողություն է որոշված $X$-ի բոլոր ենթաբազմու\-թյուն\-ների բազմության վրա։ Մասնավորապես $ \inter\varnothing=\varnothing,\ \inter X=X$։ Այդ գործողությունը բնութագրում է $X$-ի բաց ենթաբազմությունները հետևյալ \linebreak իմաստով․ \textbf{$X$-ի որևէ ենթաբազմություն բաց ենթաբազմություն է այն և միայն այն դեպքում, երբ այն համընկնում է իր ներքինի հետ։} Իրոք, եթե $A$-ն բաց ենթաբազմություն է, ապա $A=\inter A$ ըստ ներքինի սահմանման։ Հակառակը՝ եթե $A= \inter A$, ապա $A$-ն բաց ենթաբազմություն է ըստ հատկու\-թյուն 1-ի։
\end{enumerate}

\begin{nexamples}
$(X,\textrm{դիսկր․})$ տարածությունում ցանկացած $A \subset X$ ենթաբազմու\-թյան ներքինը $A$-ն է։ Իսկ $(X, \textrm{անտիդ․})$-ում $\inter A=X$, երբ $A=X$, և $\inter A=\varnothing$, երբ $A \neq X$։ Այնուհետև, $(\R, \textrm{սովոր․})$ տարածությունում $(a,b),\ [a,b),\ (a,b],\ [a,b]$ ենթաբազմությունների ներքինները $(a,b)$-ն է, իսկ ամբողջ թվերի $\Z$, ռացիոնալ թվերի $\Q$, իռացիոնալ թվերի $I$ բազմությունների ներքինները $\varnothing$ բազմությունն է, $(\R, \textrm{հաշվ․ լր․})$ տարածությունում $\inter \Z= \inter \Q=\varnothing$, իսկ $\inter I=I$ (հիմնավորե՛ք)։
\end{nexamples}

\paragraph*{Ենթաբազմության ներքինի հատկությունները (շարունակություն)․}
% \textbf{Ներքնամասի հատկությունները} (շարունակություն)։ 
% rip ներքնամաս :praying-hands-emoji:
\begin{enumerate}
    \item[4.] Եթե $A \subset B \subset X$, ապա $\inter A \subset \inter B$ (հետևում է սահմանումից)։ 
    
    \item[5.] Ցանկացած $A, B \subset X$ ենթաբազմությունների դեպքում $ \inter(A \cap B)=\inter A \cap \inter B$։
    \begin{proof} 
    Ունենք՝ $A \cap B \subset A \ \Rightarrow\ \inter(A\cap B)\subset \inter A$։ Նման ձևով $\inter(A\cap B)\subset \inter B\ \Rightarrow\ \inter(A\cap B)\subset(\inter A)\cap( \inter B)$։ Հակառակը՝ 
    \begin{equation*}
    \begin{aligned}
    x\in(\inter A)\cap( \inter B) &\Rightarrow \begin{cases} 
    x\in \inter A \\ 
    x\in \inter B
    \end{cases}\ \Rightarrow\ \begin{cases} 
    \exists U_x\in\tau,\textrm{ որ } x\in U_x\subset A\\ 
    \exists V_x\in\tau,\textrm{ որ } x\in V_x\subset B
    \end{cases}\ \Rightarrow \\ 
    &\Rightarrow \begin{cases} 
    (U_x\cap V_x)\in\tau \\ 
    x\in (U_x\cap V_x)\subset A\cap B
    \end{cases}\ \Rightarrow\ x\in \inter(A \cap B)։
    \end{aligned}
    \end{equation*}
    \end{proof}
    
    \item[6.] Ցանկացած $A,B\subset X$ ենթաբազմությունների դեպքում $(\inter A)\cup(\inter B)\subset \inter(A\cup B)$։
    \begin{proof} 
    Ունենք՝ $A \subset A \cup B,\ B \subset A \cup B \Rightarrow (\inter A,\ \inter B \subset \inter (A \cup B)) \Rightarrow( \inter A) \cup (\inter B) \subset \inter(A \cup B)$։ Նկատենք, որ ընդհանուր դեպքում $(\inter A) \cup (\inter B) \neq \inter(A \cup B)$։ Որպես օրինակ՝ $(\R,\textrm{ սովոր․})$-ում դիտարկենք $A=\Q$ և $B=I$ ենթաբազմությունները։ Մի կողմից՝ $\inter \Q= \inter I= \varnothing \Rightarrow \inter \Q \cup \inter I=\varnothing$, իսկ մյուս կողմից $\inter(\Q \cup I)= \inter \R=\R$։
    \end{proof}
\end{enumerate}

Ենթաբազմության ներքին կետ, ներքինը հասկացությունների նմանությամբ \linebreak կամայական տոպոլոգիական տարածությունում ներմուծվում են ենթաբազմության արտաքին կետ, արտաքինը, ինչպես նաև ենթաբազմության եզրային կետ և եզր հասկացությունները։ 

Որևէ $x \in X$ կետ կոչվում է տվյալ $A \subset X$ ենթաբազմության \textbf{արտաքին կետ}, եթե գոյություն ունի $x$-ի որևէ $U_x$ շրջակայք, որ $U \subset X\setminus A$: Տվյալ $A$ ենթաբազմության բոլոր արտաքին կետերի բազմությունը կոչվում է \textbf{$A$-ի արտաքինը} և նշանակվում է $\exter A$: Նույն ձևով, ինչպես ներքինի դեպքում, ապացուցվում է, որ ենթաբազմության արտաքինը $X$-ի բաց ենթաբազմություն է (տե՛ս \hyperref[հատկություն 1]{հատկություն 1}-ի ապացույցը)։

Որևէ $x\in X$ կետ կոչվում է $A$ ենթաբազմության \textbf{եզրային կետ}, եթե նրա ցան\-կացած շրջակայք ունի ոչ դատարկ հատում $A$-ի և ներքինի, և արտաքինի հետ։ Տվյալ $A$ ենթաբազմության բոլոր եզրային կետերի բազմությունը կոչվում է \textbf{$A$-ի եզր} և նշանակվում է $\partial A$:

Սահմանումներից հետևում է՝
\[
\inter A \cap \exter A = \inter A \cap \partial A = \exter A \cap \partial A = \varnothing,
\]
\[
X = \inter A \cup \exter A \cup \partial A
\]

Այժմ $\partial A = X \setminus ({\inter A}  \cup {\exter A})$ նույնությունից հետևում է՝ $X$-ի ցանկացած ենթա\-բազմության եզրը փակ ենթաբազմություն է։

\begin{example} \label{Օրինակ 1։}
Դիցուք $A$-ն հետևյալ $(0,1),[0,1), (0,1], [0,1]$ ենթաբազմություններից որևէ մեկն է թվային ուղղի սովորական տոպոլոգիայում։ Ապա $ \inter A = (0, 1)$, $\exter A = (-\infty, 0) \cup (1, +\infty)$, $ \partial A = \{0, 1\}$ (հիմնավորե՛ք)։
\end{example}

\begin{example} \label{Օրինակ 2։}
Դիցուք $A$-ն նախորդ օրինակում բերված ենթաբազմություններից որևէ մեկն է թվային ուղղի աջից կիսաբաց ինտերվալների տոպոլոգիայում։ Դրան\-ցից $A = (0,1)$ դեպքում $\inter A = (0,1)$, $\exter A = (-\infty,0) \cup [1,+\infty)$, $ \partial A = \{0\}$, իսկ $A=[0,1]$ դեպքում  $\inter A = [0,1)$, $\exter A = (-\infty,0) \cup [1,+\infty)$, $\partial A = \varnothing $։

Մյուս երկու դեպքերը որպես խնդիր թողնում ենք ընթերցողին։
\end{example}

Այժմ անդրադառնանք տոպոլոգիական տարածության փակ ենթաբազմության հասկացությանը։

Նախորդ՝ թեմա $5$-ում փակ ենթաբազմությունները սահմանվեցին բաց ենթա\-բազմությունների միջոցով։ Այժմ ցույց տանք, թե ինչպես են նկարագրվում փակ ենթաբազմությունները կետերի շրջակայքերի տերմիններով։

\begin{definition} 
$(X,\tau)$ տոպոլոգիական տարածությունում $x\in X$ կետը կոչվում է $M\subset X$ \textbf{ենթաբազմության հպման կետ}, եթե այդ կետի ցանկացած շրջակայք ունի ոչ դատարկ հատում $M$-ի հետ։ 
\end{definition}

\begin{note}
Մաթեմատիկական անալիզի դասընթացներում գործածվում է նաև \textbf{ենթաբազմության սահմանային կետ} հասկացությունը։ 

Այս դեպքում պահանջվում է, որպեսզի $x$ կետի ցանկացած շրջակայք ունենա ոչ դատարկ հատում $M\setminus \{x\}$ ենթաբազմության հետ։ Պարզ է, որ ենթաբազմության ամեն մի սահմանային կետ նաև նրա հպման կետ է։ Հակառակը ընդհանուր դեպքում ճիշտ չէ։ Օրինակ՝ $\forall (X,\tau)$ տարածությունում ցանկացած $x$ կետ հպման կետ է $\{x\}$ ենթաբազմության համար, բայց սահմանային կետ չէ նրա համար։
\end{note}

\begin{definition} 
$M\subset X$ ենթաբազմության բոլոր հպման կետերի բազմությունը կոչվում է այդ \textbf{ենթաբազմության փակույթ} և նշանակվում է $\widebar{M}$։

Անցումը $M$ ենթաբազմությունից $\widebar{M}$-ին, այսինքն $M\mapsto \widebar{M}$ համադրումը կոչվում է \textbf{փակման գոր\-ծո\-ղու\-թյուն տոպոլոգիական տարածությունում}։
\end{definition}

\begin{example}
    ա) $(X,\textrm{անտիդ․})$ տարածությունում $X$-ի ցանկացած $x$ կետ հպման կետ է ցանկացած $M\subset X,\ M\neq \varnothing$ ենթաբազմության համար, և ուրեմն $\widebar{M}=X$։
    
    \par բ) $(X,\textrm{դիսկր․})$-ում $M$ ենթաբազմության համար հպման կետեր են միայն և միայն $M$-ի կետերը։ Այսինքն՝ $\widebar{M} = M, \forall M\subset X$ ենթաբազմության դեպքում։
    
    \par գ) $(\R,\textrm{սովոր․})$ տարածությունում $(a,b),\ (a,b],\ [a,b),\ [a,b]$ ենթաբազմություն\-նե\-րից յուրաքանչյուրի փակույթը $[a,b]$-ն է։ Ամբողջ թվերի $\Z \subset \R$ ենթաբազմության փակույթը ինքը՝ $\Z$-ն է, իսկ բոլոր ռացիոնալ, կամ բոլոր իռացիոնալ թվերից կազմը\-ված ենթաբազմությունների փակույթները $\R$-ն է։
    
    \par դ) $(\R, \textrm{վերջ․ լր․})$ տարածությունում $(a,b],\ [a,b]$ ենթաբազմություններից յուրա\-քանչյուրի փակույթը $\R$-ն է։ Ընդհանրապես, այս տարածությունում ցանկացած $M\subset \R$ անվերջ ենթաբազմության (օրինակ՝ $\Z$-ի) փակույթը $\R$-ն է (հիմնավորե՛ք)։
\end{example}

\paragraph{Փակույթի հատկությունները․}\ Ցանկացած $(X,\tau)$ տոպոլոգիական տարածությունում՝
\begin{enumerate}
    \item $\widebar{\varnothing} =\varnothing,\ \widebar{X}=X$;
    \item $ M\subset \widebar{M}$ ցանկացած $M$ ենթաբազմության դեպքում;
    \item եթե $M\subset N$, ապա $\widebar{M} \subset \widebar{N}$;
    \item ցանկացած $M\subset X$ ենթաբազմության դեպքում $\widebar{\widebar{M}}=\widebar{M}$ (այստեղ $\widebar{\widebar{M}}$ սիմվոլով նշանակված է $\widebar{M}$-ի փակույթը, այսինքն $\widebar{M} = \widebar{(\widebar{M})}$)։
\end{enumerate}

Սրանցից 1-ը, 2-ը և 3-ը ակնհայտ են, ապացուցենք 4-ը։ Ըստ 2-ի՝ $\widebar{M}\subset \widebar{\widebar{M}}$, ուստի մնում է ցույց տալ, որ $\widebar{\widebar{M}}\subset \widebar{M}$։ Դիտարկենք կամայական $x\in \widebar{\widebar{M}}$ կետ։ Այն հպման կետ է $\widebar{M}$-ի համար։ Ցույց տանք, որ $x$-ը հպման կետ է նաև $M$-ի համար։ Եթե $U$-ն $x$-ի որևէ բաց շրջակայք է, ապա $U\cap \widebar{M} \neq \varnothing$, և հետևաբար գոյություն ունի $y\in X$ կետ, որ $y\in U$ և $y\in \widebar{M}$։ Ստացվում է, որ $y$-ը հպման կետ է $M$-ի համար։ Քանի որ $U$-ն նաև $y$-ի շրջակայք է, ուստի $U\cap M\neq \varnothing$, և ուրեմն $x\in \widebar{M}$։ Այսպիսով $\widebar{\widebar{M}} \subset \widebar{M}$։

\begin{note} 
Բերված ապացույցում մենք վերցրինք $x$ կետի ոչ թե կամայա\-կան $U$ շրջակայք, այլ կամայական $U$ \textbf{բաց} շրջակայք։ Առաջարկում ենք ընթերցողին նախ պարզել, թե ինչը ստիպեց այդպես վարվել, և ապա հիմնավորել, որ դրանով չի խախտվել ապացույցի լիարժեքությունը։
\end{note}

\begin{theorem}
$X$ տոպ․ տարածության $M$ ենթաբազմությունը փակ ենթաբազմու\-թյուն է այն և միայն այն դեպքում, երբ $M$-ը պարունակում է իր բոլոր հպման կետերը։ 
\end{theorem}

\begin{proof}
    ա) Եթե $M$-ը փակ է $\Rightarrow (X \setminus M)$-ը բաց է $\Rightarrow (X \setminus M)$-ում չկան $M$-ի հպման կետեր $\Rightarrow M$-ի հպման կետերը $M$-ում են $\Rightarrow \widebar{M} \subset M$։

    \par բ) Եթե $\widebar{M} \subset M\ \Rightarrow\ (X \setminus M)$-ում $M$-ի հպման կետեր չկան $\Rightarrow \forall x \in X \setminus M$ կետ հպման կետ չէ $M$-ի համար $\Rightarrow \forall x \in X \setminus M$ կետի համար $\exists x$-ի $U_{x}$ (բաց) շրջակայք, որ $U_{x} \subset (X \setminus M)$։ Այսպիսով $(X \setminus M)$-ը շրջակայք է իր բոլոր կետերի համար $\Rightarrow (X \setminus M)$-ը բաց ենթաբազմություն է $\Rightarrow M$-ը փակ ենթաբազմություն է։
\end{proof}

\begin{hetevanq_counter} 
$M \subset X$ ենթաբազմությունը փակ է այն և միայն այն դեպքում, երբ համընկնում է իր փակույթի հետ՝ $M = \widebar{M}$։
\end{hetevanq_counter}

\begin{hetevanq_counter} 
Ցանկացած $M \subset X$ ենթաբազմության $\widebar{M}$ փակույթը $X$-ի փակ ենթաբազմություն է։ Իրոք, քանի որ $\widebar{\widebar{M}} = \widebar{M}$ ըստ հատկություն 4-ի, ուստի $\widebar{M}$-ը փակ ենթաբազմություն է համաձայն հետևանք 1-ի։
\end{hetevanq_counter}

\begin{theorem} 
Ցանկացած $M \subset X$ ենթաբազմության $\widebar{M}$ փակույթը համընկնում է $M$-ը ընդգրկող բոլոր փակ ենթաբազմությունների հատման հետ։ Այսինքն ${\widebar{M} = \bigcap F}$, որտեղ հատումը կատարվում է ըստ այն բոլոր $F \subset X$ փակ ենթաբազմությունների, որ $M \subset F$։
\end{theorem}

\begin{proof} 
Մի կողմից՝ եթե $F$-ը փակ է և $M \subset F$, ապա $\widebar{M} \subset \widebar{F} = F$, ուստի $\widebar{M} \subset \bigcap F$։ Մյուս կողմից՝ քանի որ $M \subset \widebar{M}$ և $\widebar{M}$-ը փակ է, ուստի $F$ փակ ենաբազմություններից մեկը $\widebar{M}$-ն է։ Հետևաբար $\bigcap F \subset \widebar{M}$, և ուրեմն $\widebar{M} = \bigcap F$։
\end{proof}

\begin{theorem}
\label{թեորեմ 3}
Ցանկացած $M, N \subset X$ ենթաբազմությունների դեպքում տեղի ունի
$\widebar{M \cup N} = \widebar{M} \cup \widebar{N}$ համընկում։
\end{theorem}

\begin{proof} 
Ցույց տանք, որ $\widebar{M \cup N}$ և $\widebar{M} \cup \widebar{N}$ բազմություններից յուրաքանչ\-յուրը մյուսի ենթաբազմություն է։ Իրոք, 
\[
M \subset \widebar{M},\ N \subset \widebar{N} \ \Rightarrow\ M \cup N \subset \widebar{M} \cup \widebar{N} \ \Rightarrow\ \widebar{M \cup N} \subset \widebar{\widebar{M} \cup \widebar{N}} = \widebar{M} \cup \widebar{N}.
\]
Այստեղ օգտվեցինք նրանից, որ $\widebar{M} \cup \widebar{N}$-ը փակ է որպես երկու փակ ենթաբազմու\-թյուն\-ների միավորում։ Մյուս կողմից՝
\[
M \subset M \cup N,\ N \subset M \cup N \ \Rightarrow\ \widebar{M} \subset \widebar{M \cup N},\ \widebar{N} \subset \widebar{M \cup N} \ \Rightarrow\ \widebar{M} \cup \widebar{N} \subset \widebar{M \cup N}. \qedhere
\]
\end{proof}

\par Ավարտելով փակույթի հատկությունները՝ նկատենք, որ $\forall M, N \subset X$ ենթաբազ\-մությունների դեպքում տեղի ունի $\overline{M \cap N} \subset \overline{M} \cap \overline{N}$ ներդրում (հիմնավորե՛ք և բերե՛ք օրինակ, երբ $\overline{M \cap N} \neq \overline{M} \cap \overline{N}$)։

\par Այս ամենից հետո պատրաստ ենք ծանոթանալու կամայական բազմության վրա տոպոլոգիա սահմանելու ևս մի հիմնական եղանակի հետ։

\par Վերը $(X, \tau)$ տոպոլոգիական տարածության ամեն մի $M \subset X$ ենթաբազմության համար սահմանվեց $\widebar{M}$ փակ ենթաբազմություն ($M$-ի փակույթը $\tau$ տոպոլոգիայում)։ Այդ $M \mapsto \widebar{M}$ համադրումը կոչվում է \textbf{փակման գործողություն տոպոլոգիական տարածությունում}։ Այն օժտված է վերը դիտարկված որոշակի հատկություններով։

% Այդ $M\mapsto\widebar{M}$ համադրումը կոչվում է \textbf{փակման գործողություն տոպո\-լոգիական տա\-րա\-ծութ\-յու\-նում}։ Այն օժտված է վերը դիտարկված որոշակի հատկություններով։

\par Այժմ գնալու ենք հակառակ ուղղությամբ․ վերացարկելով այդ հատկություննե\-րից հիմնականները՝ սահմանվում է գործողություն, որի միջոցով որոշվում է տոպո\-լոգիա բազմության վրա։ 

\begin{definition}
Դիցուք ինչ-որ եղանակով $X$ բազմության ամեն մի $M$ ենթաբազ\-մության համադրված է $X$-ի ինչ-որ ենթաբազմություն, որը նշանակվում է $\cl M$ ($\cl$-ն ֆրանսերեն clôture — փակում բառի կրճատն է), այնպես, որ բավարարվում են հետևյալ 4 պահանջները․
\begin{itemize}
    \item[K1.] $\cl \varnothing = \varnothing$;
    \item[K2.] $M \subset \cl M$;
    \item[K3.] $\cl(\cl M) = \cl M$;
    \item[K4.] $\cl(M \cup N) = \cl M \cup \cl N$, ցանկացած $M, N \subset X$ ենթաբազմությունների դեպքում։
\end{itemize}
\par Ամեն մի այսպիսի գործողություն կոչվում է Կուրատովսկու \textbf{փակման գործո\-ղու\-թյուն բազմության վրա}։
\end{definition}

\begin{lemma}
1-4 աքսիոմներից հետևում է՝ 
\begin{itemize}
    \item[ա)] $\cl X = X$;
    \item[բ)] եթե $A \subset B \subset X$, ապա $\cl A \subset \cl B$։
\end{itemize}
\end{lemma}

\begin{proof}
ա) Մի կողմից, ըստ K2-ի $X \subset \cl X$, իսկ մյուս կողմից ակնհայտ է, որ $ \cl X \subset X$։ Ուստի $ \cl X = X$;

\par բ) $A \subset B\ \Rightarrow\ (B = A \cup (B \setminus A))\ \Rightarrow\ (\cl B = \cl A \cup \cl(B \setminus A))\ \Rightarrow\ (\cl A \subset \cl B)$։
\end{proof}

\begin{theorem}[(K. Kuratowski, 1922)]
$X$ բազմության վրա տրված Կուրատովսկու փակման գործողությունը որոշում է տոպոլոգիա $X$-ի վրա։ Ընդ որում՝ ցանկացած $M \subset X$ ենթաբազմության $\widebar{M}$ փակույթը այդ տոպոլոգիայում համընկնում է $\cl M$-ի հետ՝ $\widebar{M} = \cl M$։
\end{theorem}

\begin{proof} 
Սահմանենք $\sigma$ տոպոլոգիա $X$ բազմության վրա՝ հիմնվելով փակ ենթաբազմությունների վրա (տե՛ս թեորեմ 3-ը թեմա 5-ում)։ $M \subset X$ ենթաբազմու\-թյունը համարելու ենք փակ $\sigma$ տոպոլոգիայում, եթե $M = \cl M$։ Այսպիսով 
\[
{M \in \sigma} \ \Leftrightarrow\ {M = \cl M}.
\]
Ստուգենք $\sigma$-ի համար տոպոլոգիայի 1*-3* աքսիոմները (տե՛ս թեորեմ 3-ը թեմա 5-ում)։ Դրանցից 1*-ը հետևում է K1-ից և լեմմայից։ Միավորման 2* աքսիոմը բավարարվում է շնորհիվ K4-ի։ Ստուգենք 3*-ը։ Դիցուք ունենք փակ ենթաբազմու\-թյունների որևէ $\left\{ M_i \subset X, i \in I \mid \cl M_i = M_i \right\}$ ընտանիք։ Ցույց տանք, որ նրանց \linebreak$F = \bigcap\limits_{i \in I} M_{i}$ հատումը պատկանում է $\sigma$-ին (դա համարժեք է նրան, որ ցույց տանք $\cl F = F$)։ Մի կողմից ունենք $F \subset \cl F$ ըստ K2-ի։ Մյուս կողմից՝ $F \subset M_i\ \Rightarrow\ (\cl F \subset \cl M_i )$, և քանի որ $\cl M_i = M_i$, ուստի $\cl F \subset \bigcap\limits_{i \in I} M_i = F$։ Այսպիսով $\cl F = F$, և ուրեմն $\sigma$-ն տոպոլոգիա է $X$ բազմության վրա։

\par Այժմ ապացուցենք թեորեմի երկրորդ մասը․ ցույց տանք, որ $\forall M \subset X$ ենթա\-բազ\-մության համար $\widebar{M} = \cl M$։ Ըստ \hyperref[թեորեմ 3]{թեորեմ 3}-ի ունենք՝ $\widebar{M} = \bigcap F$, որտեղ հատումը տարվում է ըստ բոլոր այն $F \subset X$ ենթաբազմությունների, որ $\cl F = F$ և $M \subset F$։ Նրանից, որ $M \subset F \ \Rightarrow\ (\cl M \subset \cl F = F)\ \Rightarrow\ (\cl M \subset F)\ \Rightarrow\ (\cl M \subset \bigcap F = \widebar{M})$։ Մյուս կողմից՝ $M \subset \cl M\ \Rightarrow\ \widebar{M} \subset \widebar{\cl M}$։ Քանի որ $\cl(\cl M) = \cl M$ ըստ K3-ի, ուստի $\cl M \in \sigma$ ըստ $\tau$ տոպոլոգիայի սահմանման։ Քանի որ $\cl M$-ը փակ ենթաբազմություն է $\sigma$ տոպոլոգիայում, հետևաբար $\widebar{\cl M} = \cl M$, որից հետևում է, որ $\widebar{M} = \cl M$։
\end{proof}

\bigskip
\bigskip
\subsubsection*{Խնդիրներ և հարցեր թեմա 6-ի վերաբերյալ}

\begin{enumerate}[label=\thesection.\arabic*.]
% 6.1
\item Իրական թվերի սովորական տոպոլոգիայում գտեք անվերջ քանակով փակ ենթաբազմությունների որևէ ընտանիք, որի տարրերի միավորումը փակ չէ։

% 6.2
\item Ճի՞շտ է արդյոք հետևյալ պնդումը․ ցանկացած տոպոլոգիական տարածությու\-նում (բացառությամբ դիսկրետ տարածությունների) գոյություն ունի անվերջ քանակով փակ ենթաբազմությունների ընտանիք, որի տարրերի միավորումը փակ չէ։

\begin{hint}
Դիտարկեք $\left\{ [x,1);\;  x > 0 \right\}$ ընտանիքը $(\mathbb{R}, \mapsto)$ տարածությունում։
\end{hint}

% 6.3
\item Թվային ուղղի սովորական տոպոլոգիայում գտեք $A$ ենթաբազմության փա\-կույթը, եթե
\begin{itemize}
    \item[ա)] $A$-ն ամբողջ թվերի $\Z$ ենթաբազմությունն է,
    \item[բ)] $ A = \left\{ \dfrac{(-1)^{n}}{n}; \; n \in \mathbb{N} \right\} $,
    \item[գ)] $A$-ն բոլոր ռացիոնալ թվերի ենթաբազմությունն է,
    \item[դ)] $A$-ն բոլոր իռացիոնալ թվերի ենթաբազմությունն է։
\end{itemize}

% 6.4
\item Թվային ուղղի հաշվելի լրացումների տոպոլոգիայում գտեք $A$ ենթաբազմու\-թյան փակույթը նախորդ խնդրում թվարկված դեպքերում։

% 6.5
\item Դիտարկենք $\mathbb{R}$ թվային ուղղի ենթաբազմությունների $\Phi$ ընտանիքը՝ կազմված $\varnothing$, $\mathbb{R}$ և բոլոր $(-r, r)$, $r > 0$ ենթաբազմություններից։ Ապացուցեք, որ $\Phi$-ն որոշում է տոպոլոգիա $\mathbb{R}$-ի վրա։ Ամեն մի $r \in \mathbb{R}$ դեպքում գտեք $X(r)= (-\infty, -r) \cup (r, +\infty)$ 
ենթաբազմության ներքինը, արտաքինը, եզրը և փակույթը։

% 6.6
\item Դիտարկենք $0$ սկզբնակետով $\mathbb{R}^2$ կոորդինատային հարթության ենթաբազմու\-թյունների $\Psi$ ընտանիքը՝ կազմված $\varnothing$, $\mathbb{R}^2$ և $0$ կենտրոնով բոլոր $\mathcal{D}(r) = \{(x,y) \mid x^2 + y^2 < r^2; \; r>0\}$ շրջաններից։ Ապացուցեք, որ $\Psi$-ն որոշում է տոպոլոգիա $\mathbb{R}^2$-ի վրա (կոչվում է \textbf{համակենտրոն տոպոլոգիա})։ Ապացուցեք, որ ամեն մի $r>0$ դեպքում $A(r)=\mathbb{R}^2 \setminus  \mathcal{D}(r)$ ենթաբազմության ներքինը, արտաքինը, եզրը և փակույթը հետևյալն են՝
\[
\inter A(r)=\varnothing, \quad \exter A(r) = \mathcal{D}(r), \quad \partial A(r) = A(r), \quad \widebar{A(r)} = A(r):
\]

% 6.7
\item Ապացուցեք, որ $X$ տոպոլոգիական տարածության կամայական $A$ ենթաբազ\-մու\-թյան և $X \setminus A$ ենթաբազմության եզրերը նույնն են՝ $\partial A = \partial (X \setminus A)$:

% 6.8
\item Ապացուցեք, որ տոպոլոգիական տարածության կամայական երկու չհատվող փակ ենթաբազմություններ չունեն որևէ ընդհանուր եզրային կետ։

% 6.9
\item Ապացուցեք․ եթե $X$ տոպոլոգիական տարածության $Y$ ենթաբազմությունն այնպիսին է, որ $Y \subset F \subset X$, որտեղ $F$-ը փակ ենթաբազմություն է, ապա $\widebar{Y} \subset F$։

% 6.10
\item Ապացուցեք․ $X$ տոպոլոգիական տարածության $A$ ենթաբազմությունը փակ է այն և միայն այն դեպքում, երբ $\partial A \subset A$։

% 6.11
\item Ապացուցեք, որ $X$ տոպոլոգիական տարածության $Y$ ենթաբազմության $\partial Y$ եզրը դատարկ ենթաբազմություն է այն և միայն այն դեպքում, երբ $Y$-ը բաց է և փակ է։

% 6.12
\item Ապացուցեք․ տոպոլոգիական տարածության կամայական երկու չհատվող բաց ենթաբազմություններից յուրաքանչյուրի փակումը չի հատվում մյուս ենթա\-բազ\-մության հետ։
\end{enumerate}




\end{document}