\documentclass[./main.tex]{subfiles}
\usepackage{array}
\newcolumntype{C}[1]{>{\centering\arraybackslash$}p{#1}<{$}}

\begin{document}
\onehalfspacing
\section{Վերջավոր և անվերջ բազմություններ, բազմությունների քանակական համեմատումը, թեորեմներ հաշվելի բազմությունների վերաբերյալ։ Բազմության հզորության հասկացությունը, Կանտորի թեորեմը։ Պարադոքսները բազմությունների նաիվ տեսությունում։
}\label{sec:3}

\par Դիցուք $A$-ն որևէ ոչ դատարկ բազմություն է, նրանում ընտրենք որևէ $a_1 \in A$ տարր։ Եթե $A \setminus  \{a_1\} \neq \varnothing$, ապա կարող ենք ընտրել ևս մի $a_2 \in A$ տարր։ Եթե $A \setminus \{a_1, a_2\} \neq \varnothing$, ապա կարող ենք ընտրել ևս մի $a_3 \in A$ տարր և այսպես շարունակ։ Հնարավոր է երկու դեպք․ կա՛մ այս պրոցեսը ինչ-որ $n$-րդ քայլում կընդհատվի, և ուստի ${A=\{a_1,\\ a_2,  \dots, a_n\}}\textrm{-ը}$ \textbf{վերջավոր բազմություն} է, կա՛մ էլ այս պրոցեսը վերջ չունի։ Այս երկրորդ դեպքում $A$-ն կոչվում է \textbf{անվերջ բազմություն}։ 
 
\begin{example}
Վերցնենք $1$ թիվը, նրան հաջորդում է $1+1=2$ թիվը, դրան հաջոր\-դում է $2+1=3$ թիվը և այլն։ Այս պրոցեսն անվերջ է։ Ստացվող $\N = \{1, 2, 3, \dots\}$  բազ\-մու\-թյունը կոչվում է բնական թվերի բազմություն։ Կարո՞ղ ենք այն համարել լիովին կազմավորված ավարտուն բազմություն։ Մաթեմատիկայում սա վիճելի հարց է։
\end{example}
\par Անվերջ բազմությունները լինում են երկու տեսակի, կապված անվերջության ընկալման հետ՝ \textbf{պոտենցիալ} (\textbf{շարունակական}) \textbf{անվերջություն} և \textbf{ակտուալ} \linebreak (\textbf{ա\-վար\-տուն}) \textbf{անվերջություն}։ Այսօր մաթեմատիկոսների մեծամասնությունը \linebreak տար\-բերութ\-յուն չի դնում այս երկու անվերջությունների միջև՝ համարելով դրանք միա\-տեսակ ան\-վեր\-ջութ\-յուն\-ներ։ Օրինակ՝ և՛ բնական թվերի բազմությունը (որպես պո\-տեն\-ցիալ անվերջ բազմություն), և՛ կամայական ուղիղ  (որպես հարթության կե\-տե\-րից կազմված ավար\-տուն անվերջ բազմություն) համարվում են պարզապես ան\-վերջ բազմություններ։
\par Անցնենք բազմությունների քանակական համեմատմանը։
\begin{example}
Ենթադրենք ունենք բավականաչափ մեծ երկու քարակույտեր։ Խըն\-դիր\-ն է պարզել, թե դրանցից որում ավելի շատ քար կա։ Խնդիրը կարող ենք լուծել փորձելով հաշվել կույտերում քարերի քանակները։ Այս մեթոդի անհարմարությունը նրանում է, որ մեզ հայտնի (գործածական) թվերը կարող են չբավականացնել։ Ուստի վարվենք այսպես. վերցնենք մի քար առաջին կույտից, մի քար էլ երկրորդ կույտից և այս երկուսը դնենք մի կողմ։ Այնուհետև նորից վերցնենք մի քար առաջին կույտից, մի քար էլ երկրորդ կույտից և նորից դնենք մի կողմ։ Շարունակելով այսպես՝ որոշ քայլերից հետո կպարզվի, որ դրանցից մեկում քարերն ավելի քիչ են, քան մյուսում, կամ երկուսում էլ քարերի քանակները նույնն են։
\end{example}
\par Այս օրինակը հիմք է ծառայում հետևյալ սահմանման համար․ երկու $A$ և $B$ բազ\-մութ\-յուն\-ներ (վերջավոր կամ անվերջ) կոչվում են \textbf{քանակապես համարժեք} կամ \textbf{հա\-վա\-սա\-րա\-զոր} (նշանակվում է $A \sim B$), եթե նրանց տարրերի միջև կարելի է հաստատել մեկը-մեկի կամ փոխ\-միարժեք համապատասխանություն։
% page 24
\begin{example}
Չնայած, որ զույգ թվերի բազմությունը կազմում է մաս ամբողջ թվերի բազմության՝ $2\Z \subset \Z$, դրանք հավասարազոր բազմություններ են։ Իրոք, $n \mapsto 2n$ համապատասխանությունը որոշում է բիյեկտիվ (փոխմիարժեք) $\Z \to 2\Z$ արտապատկերում։
\par Միմյանց հավասարազոր են նաև կետերի $(-1, 1)$ և $(-\infty, +\infty )$ բազմությունները (հիմնավորումը տե՛ս թեմա $1$-ի օրինակ $9$-ում)՝ չնայած, որ $(-1, 1)$ ինտերվալը կազ\-մում է թվային ուղղի մաս՝ չհամընկնելով նրա հետ։ Ինչպես կտեսնենք հետագայում, այն իրավիճակը, երբ բազմությունը հավասարազոր է իր սեփական մասին (այսինքն՝ իր հետ չհամընկնող են\-թա\-բազ\-մութ\-յանը), բնութա\-գրում է բոլոր անվերջ բազմու\-թյուն\-ները։
\end{example}
\par Բերենք այդպիսի, ավելի ընդհանուր բնույթի օրինակ։
\begin{ntheorem}
Ցանկացած $X$ անվերջ բազմություն հավասարազոր է իր $X \times X$ քա\-ռա\-կու\-սուն։
\end{ntheorem}
\par Չապացուցելով թեորեմը՝ բավարարվենք նրա որոշ մեկնաբանությամբ։ Սևեռե\-լով որևէ $x_0 \in X$ կետ՝ դիտարկենք $f:X \rightarrow X\times X,\ f(x)=(x_0,x)$ արտապատկերումը։ Պարզ է, որ $f$-ը ինյեկտիվ արտապատկերում է և թույլ է տալիս համարել $X$-ը ներդրված $X\times X$-ում՝ որպես $X\times X$-ի սեփական ենթաբազմություն։ Իրոք, $(X\times X) \setminus  f(X) \neq \varnothing$, քանի որ $(x, x_0)$-ն չի պատկանում $f(X)$-ին, եթե $x\neq x_0$։ Այսպիսով՝ այս դեպքում ևս ըստ թեորեմի ունենք իրավիճակ, երբ $X\times X$ անվերջ բազմությունը հավասարազոր է իր $f(X)$ սեփական ենթաբազմությանը։
\par Վերադառնալով երկու բազմությունների հավասարազորության սահմանմանը՝ նկատենք, որ ցանկացած երկու $A$ և $B$ բազմությունների դեպքում տեսականորեն կարող է տեղի ունենալ հետևյալ $5$ հնարավորություններից որևէ մեկը։
\begin{enumerate}
    \item $A$-ն և $B$-ն հավասարազոր են։ Այս դեպքում ասում են, որ $A$-ն և $B$-ն ունեն \textbf{հավասար քանակով տարրեր}։
    \item $A$-ն հավասարազոր է $B$-ի որևէ մասին, բայց $B$-ն հավասարազոր չէ $A$-ի որևէ մասին (հետևաբար նաև $A$-ին)։ Այս դեպքում ասում են, որ $A$-ի տարրերի քանակը \textbf{քիչ է} $B$-ի տարրերի քանակից։
    \item $B$-ն հավասարազոր է $A$-ի որևէ մասին, բայց $A$-ն հավասարազոր չէ $B$-ի որևէ մասին (հետևաբար նաև $B$-ին)։ Այս դեպքում ասում են նաև, որ $A$-ի տարրերի քանակը \textbf{շատ է} $B$-ի տարրերի քանակից։
    \item $A$-ն հավասարազոր չէ $B$-ի որևէ մասի, և $B$-ն էլ հավասարազոր չէ $A$-ի որևէ մասի։
    \par Սա իհարկե հնարավոր չէ վերջավոր բազմությունների դեպքում։ Ապացուցվում է, որ այն հնարավոր չէ նաև անվերջ բազմությունների դեպքում (Կանտոր-Բերնշտեյնի թեորեմ)։ Ապացույցը բարդ է, և մենք այն այստեղ չենք բերում։
% page 25
    \item $A$-ն հավասարազոր է $B$-ի որևէ մասին, և $B$-ն էլ հավասարազոր է $A$-ի որևէ մասին։
    \par Ապացուցվում է (Շրյոդեր-Բերնշտեյնի թեորեմ), որ այս դեպքում $A$-ն և $B$-ն հավասարազոր են (5-րդ դեպքում տեղի ունի 1-ը)։
\end{enumerate}
\par Այսպիսով՝ փաստացի հնարավոր են միայն 1-3 դեպքերը։ Իսկ դա նշանակում է, որ \textbf{ցանկացած երկու բազմություններ քանակապես համեմատելի են}։

Մինչև վերոհիշյալ՝ 5-րդ դեպքի կիրառության մի օրինակի քննարկումը հի\-շեց\-նենք. մաթեմատիկական անալիզի դասընթացից գիտենք՝ ցանկացած իրական թիվ ներ\-կա\-յաց\-վում է անվերջ տասնորդական կոտորակի տեսքով։ Ընդ որում՝ իռացիոնալ թվերը ներկայացվում են միակ ձևով՝ ոչ պարբերական անվերջ տասնորդական կոտորակների տեսքով։ Իսկ ռացիոնալ թվերը ներկայացվում են պարբերական կո\-տո\-րակների տեսքով, ընդ որում՝ որոշ թվերի դեպքում այդպիսի ներկայացումը կարող է միակը չլինել։ Օրինակ՝ $\dfrac{1}{8}$ թվի համար ունենք երկու տարբեր տեսքերի ներկայացում\-ներ՝ $0{,}12499\dots$ և $0{,}12500\dots$ ։ Ամեն մի այդպիսի դեպքում ո\-րո\-շա\-կիութ\-յան համար կդիտարկենք միայն առաջին տեսքի ներկայացումը։ Ապացուցվում է. երկու իրական թվեր, գրված տասնորդական կոտորակների տեսքով, հավասար են այն և միայն այն դեպքում, երբ հավասար են նրանց ամբողջ մասերը և հավասար են ստորակետից հետո նույն դիրքերում գրված թվանշանները։
\begin{example}
Ցույց տանք, որ $(0,1)$ ինտերվալը և նրա $(0,1)\times(0,1)$ քառակուսին հավասարազոր են։ Իհարկե, դա հետևում է վերը բերված ընդհանուր թեորեմից, բայց մենք կտանք ուղղակի ապացույց՝ հիմնվելով Շրյոդեր-Բերնշտեյնի թեորեմի վրա։ 

Վերցնենք որևէ $(r_1, r_2) \in (0,1)\times(0,1)$ թվազույգ, որտեղ $r_1= 0{,}\alpha_1\alpha_2\dots$, իսկ $r_2= 0{,}\beta_1\beta_2\dots$ և $(r_1;r_2)$ թվազույգին համադրենք $r=0{,}\alpha_1\beta_1\alpha_2\beta_2\dots$ թիվը $(0,1)$-ից։ Գրեթե ակնհայտ է, որ ${(r_1;r_2) \mapsto r}$ համադրումը որոշում է ինչ-որ $(0,1)\times(0,1)\rightarrow (0,1)$ ինյեկտիվ արտապատկերում (հիմնավորե՛լ)։ Այսպիսով՝ $(0,1)\times(0,1)$-ը հավա\-սա\-րազոր է $(0,1)$-ի ինչ-որ մասին։ Մյուս կողմից էլ ունենք, որ $(0,1)$-ը  հավասարազոր է $(0,1)\times(0,1)$-ի որոշ մասին (տե՛ս վերը բերված թեորեմի մեկնաբանությունը)։ Այժմ Շրյոդեր-Բերնշտեյնի թեորեմից հետևում է, որ $(0,1)$-ը և $(0,1)\times(0,1)$-ը հավասարա\-զոր են։
\end{example}

\par Մինչև այժմ մենք դիտարկում էինք հավասարազոր անվերջ բազմությունների օրինակներ։ Հարց է առաջանում․ իսկ կա՞ն ոչ հավասարազոր անվերջ բազմություն\-ներ։ Պատասխանը դրական է և կտրվի ավելի ուշ (\hyperref[թեորեմ 6]{թեորեմ 6}-ում)։
\begin{definition}
Բազմությունը, որը հավասարազոր է բնական թվերի $\N$ բազ\-մութ\-յանը կամ նրա որևէ մասին, կոչվում է \textbf{հաշվելի բազմություն}։ Այսպիսով՝ հաշվելի բազ\-մութ\-յունը կարող է լինել վերջավոր բազմություն կամ անվերջ բազմություն։
\end{definition}
\begin{equivdefinition}
բազմությունը կոչվում է հաշվելի բազմություն, եթե նրա բոլոր տարրերը կարելի է ինդեքսավորել (համարակալել) բոլոր բնական թվերով կամ դրա որևէ $\{1, 2,\dots, n\}$ մասով։ 
\end{equivdefinition}
\begin{example}
Հաշվելի բազմություններ են բոլոր վերջավոր բազմությունները, $\N\textrm{-ը}$, $\Z$-ը, $2\Z$-ը։ %, բոլոր պարզ թվերի $P =\{2, 3, 5, 7, 11,\dots \}$ բազմությունը և այլն։
\end{example}
\begin{theorem}
\label{թեորեմ 1}
Հաշվելի բազմության ցանկացած ենթաբազմություն հաշվելի բազմու\-թյուն է (վերջավոր կամ անվերջ)։
\end{theorem}
\begin{proof}
Դիցուք $A$-ն որևէ հաշվելի բազմություն է, և $B \subset A$։ Կարող ենք համարել, որ $A$-ի տարրերը համարակալված են՝ $A=\{ a_1,a_2,\dots \}$։ Դիտարկենք այդ  հաջորդականության այն առաջին անդամը, որը պատկանում է $B$-ին։ Դիցուք դա $a_{n_1}$-ն է։ Եթե $B\neq \{a_{n_1}, a_{n_2}\}$, ապա նշանակենք $a_{n_3}$-ով $a_{n_2+1}, a_{n_2+2}, \dots$ հաջոր\-դա\-կա\-նու\-թյան այն առաջին անդամը, որը պատկանում է $B$-ին։ Արդյունքում ստանում ենք կա՛մ $B=\{a_{n_1}, a_{n_2}, a_{n_3}, \dots, a_{n_k}\}$, վերջավոր բազմու\-թյուն, որի տարրերը համարա\-կալ\-ված են $n_1, n_2, n_3, \dots, n_k$ թվերով, կամ որ նույնն է՝  $1, 2, 3, \dots, k$ բնական թվերով, կա՛մ $B=\{ a_{n_1}, a_{n_2},  a_{n_3},\dots \}$ անվերջ բազմություն, որի տարրերը համարակալվում են ${n_1<n_2<n_3<\dots}$ բնական թվերով, կամ որ նույնն է՝ բոլոր $1, 2, 3, \dots$ բնական թվերով։ Ուստի $B$ ենթա\-բազմությունը հաշվելի բազմություն է։
Որպես հետևանք թեորեմ \ref{թեորեմ 1}-ից և Էվկլիդեսի թեորեմից՝ ստանում ենք․ բոլոր պարզ թվերի բազմությունը հաշվելի անվերջ բազմություն է։
\end{proof}

\begin{theorem} \label{թեորեմ 2}
Ցանկացած $X$ անվերջ բազմություն պարունակում է հաշվելի անվերջ ենթաբազմություն։
\begin{proof}
Ընտրենք որևէ $a_1\in X$ տարր։ Քանի որ $X\setminus \{a_1\} \neq \varnothing$, կարող ենք ընտրել որևէ $a_2\in X\setminus \{a_1\}$ տարր։ Քանի որ $X\setminus \{a_1,a_2\} \neq  \varnothing$, կարող ենք ընտրել որևէ $a_3\in X\setminus \{a_1,a_2\}$ տարր և այդպես շարունակ։ Այս պրոցեսն անվերջ է, և արդյունքում ստանում ենք $X$-ի հաշվելի անվերջ $A=\{a_1,a_2,a_3,\dots\}$ ենթաբազմություն։ 
\end{proof}
\end{theorem}
\begin{hetevanq}
Ցանկացած անվերջ բազմության տարրերի քանակը մեծ կամ հավա\-սար է հաշվելի անվերջ բազմության տարրերի քանակից (այստեղ տեղին է նկատել, որ այդուհանդերձ մենք դեռ չենք սահմանել կամայական բազմության համար նրա տարրերի քանակ հասկացությունը)։
\end{hetevanq}
\par \hyperref[թեորեմ 2]{Թեորեմ 2}-ի հետ կապված առաջանում է հարց․ որևէ անվերջ բազմություն իր մեջ որքա՞ն զույգ առ զույգ չհատվող հաշվելի անվերջ ենթաբազմություններ կարող է պարունակել։
\par Պատասխանը հետևյալն է․ ցանկացած անվերջ բազմություն իր մեջ պարունա\-կում է հաշվելի անվերջ թվով զույգ առ զույգ չհատվող հաշվելի անվերջ ենթաբազմու\-թյուններ։
\par Բերենք համապատասխան օրինակ, որից և \hyperref[թեորեմ 2]{թեորեմ 2}-ից հետևում է պատասխանը։
\begin{example}
\label{օրինակ 6}
Դիտարկենք բոլոր պարզ թվերի $P=\{2,3,5,\dots\}$ հաշվելի անվերջ բազմությունը, և ամեն մի $k$ բնական թվի համար՝ $P_k=\{2^k,3^k,5^k,\dots\}$ բազմությունը։ Պարզ է, որ ${P_1=P}$, ${P_k\cap P_l=\varnothing}$, եթե $k\neq l$ և $\bigcup\limits_k P_k \subset \N$։ Այսպիսով՝ բնական թվերի բազմությունն իր մեջ պարունակում է հաշվելի անվերջ թվով զույգ առ զույգ չհատվող հաշվելի անվերջ ենթաբազմութ\-յուն\-ներ։
\end{example}
\par Այժմ որպես հետևանք \ref{օրինակ 6}-ից և \ref{թեորեմ 2}-ից՝ ստանում ենք. ցանկացած անվերջ բազմու\-թյուն իր մեջ պարունակում է հաշվելի անվերջ թվով զույգ առ զույգ չհատվող հաշվելի անվերջ ենթաբազմություններ։

\begin{theorem}
Հաշվելի թվով հաշվելի բազմությունների միավորումը հաշվելի բազ\-մու\-թյուն է։
\par Նկատենք, որ թեորեմի պայմանն իր մեջ պարունակում է 4 հնարավորություն․
\begin{enumerate}
    \item[ա)] վերջավոր թվով վերջավոր բազմությունների միավորում,
    \item[բ)] վերջավոր թվով հաշվելի անվերջ բազմությունների միավորում,
    \item[գ)] հաշվելի անվերջ քանակով վերջավոր բազմությունների միավորում,
    \item[դ)] հաշվելի անվերջ քանակով հաշվելի անվերջ բազմությունների միավորում։
\end{enumerate}
\end{theorem}
\par Քննարկենք թեորեմը դ) դեպքում։ Դիցուք ունենք հաշվելի անվերջ քանակով $A_1,A_2,\dots$ հաշվելի անվերջ բազմություններ։Պարզության համար ենթադրենք, որ այդ բազմությունները զույգ առ զույգ չեն հատվում։ Յուրաքանչյուր $A_i$ բազմության տարրերը համարակալենք բոլոր բնական թվերով՝ $A_i=\{a_1^i,a_2^i,a_3^i,\dots\}$ և դասավո\-րենք դրանք հետևյալ ուղղանկյուն անվերջ աղյուսակի տողերի տեսքով՝\\
% \[
% \begin{matrix}
%     a_1^1 & \rightarrow & a_2^1 & & a_3^1 & \rightarrow & a_4^1 & \dots\\
%     & \swarrow &  & \nearrow & & \swarrow \\
%     a_1^2 & & a_2^2 & & a_3^2 & & a_4^2 & \dots \\
%     \downarrow & \nearrow & & \swarrow & & \nearrow \\
%     a_1^3 & & a_2^3 & & a_3^3 & & a_4^3 & \dots \\
%     & \swarrow &  & \nearrow & & \swarrow \\
%     a_1^4 & & a_2^4 & & a_3^4 & & a_4^4 & \dots \\
%     \dots & \dots & \dots & \dots & \dots & \dots & \dots & \dots
% \end{matrix}
% \]
\begin{center}
\begin{tikzcd}[row sep=small, column sep = small]
a_1^1 \arrow{r} & a_2^1 \arrow{dl} & a_3^1 \arrow{r} & a_4^1\arrow{dl} & \dots \\
a_1^2 \arrow{d} & a_2^2 \arrow{ur} & a_3^2 \arrow{dl} & a_4^2 & \dots\\
a_1^3 \arrow{ur} & a_2^3 \arrow{dl}& a_3^3 & a_4^3 & \dots\\
\dots^{\textcolor{white}{1}} & \dots & \dots & \dots & \dots
\end{tikzcd}
\end{center}
\par Այժմ կարող ենք $\bigcup A_i$ միավորման տարրերը համարակալել բոլոր բնական թվերով՝ շարժվելով անկյունագծերով։ Արդյունքում ստանում ենք, որ $\bigcup\limits_i A_i=\{b_1, b_2, b_3, \dots\}$ բազմությունը հաշ\-վե\-լի անվերջ բազմություն է, որտեղ $b_1=a_1^1,\, b_2=a_2^1,\, b_3=a_1^2,\, b_4=a_1^3,\, b_5=a_2^2$ և այլն (ապացուցման այս եղանակը հայտնի է «Կանտորի նշանավոր անկյունագծային մեթոդ» անվանումով)։
% \begingroup
% \renewcommand\proofname{\hspace{15pt} \textbf{Թեորեմ 3-ի ապացուցումը}}
\renewcommand*{\proofname}{\hspace{18pt}\textbf{Թեորեմ 3-ի ապացուցումը։}\nopunct}
\begin{proof} Դիցուք ունենք հաշվելի (վերջավոր կամ անվերջ) քանակով հաշվելի (վերջավոր կամ անվերջ) $A_i,\ i\in I$ բազմություններ, որտեղ ինդեքսների $I$ բազմությունը կա՛մ բնական թվերի $\N$ բազմությունն է, կա՛մ նրա որևէ վերջավոր $\{1,2,\dots,n\}$ մասն է։ Սկզբում դիտարկենք մասնավոր դեպք, երբ $A_i\cap   A_j =\varnothing$, ցանկացած $i\neq j$ ինդեքսների դեպքում։ Թեորեմի պայմանից հետևում է, որ գոյություն ունի փոխմիարժեք համապատասխանություն $A_1$ բազմության և պարզ թվերի $P_1=\{2,3,5,\dots\}$ բազմության ինչ-որ $B_1$ ենթաբազմության տարրերի միջև ($B_1=P_1$ համընկնումը չի բացառվում և տեղի ունի, երբ $A_1$-ը հաշվելի անվերջ բազմություն է)։
\par
Ասվածը գրառենք կարճ՝ $\exists A_1\leftrightarrow B_1\subset P_1$ տեսքով։ Նույն դատողություններով գոյություն ունեն
\begin{align*}
&A_2 \leftrightarrow B_2 \subset P_2=\{2^2,3^2,5^2,\dots\}\\
&\dots\dots\dots\dots\dots\dots\dots\dots\dots\dots\\
&A_k\leftrightarrow B_k\subset P_k=\{2^k,3^k,5^k,\dots\}\\
&\dots\dots\dots\dots\dots\dots\dots\dots\dots\dots
\end{align*}
փոխմիարժեք համապատասխանություններ։

\par Արդյունքում ստանում ենք $\bigcup\limits_i A_i \leftrightarrow \bigcup\limits_i B_i$ փոխմիարժեք համապատասխանություն բոլոր $A_i,\ i\in I$ բազմությունների $\bigcup\limits_i A_i$ միավորման և բոլոր $B_i$ բազմությունների $\bigcup\limits_i B_i$ միավորման տարրերի միջև։ Ըստ թեորեմ 1-ի՝ $\bigcup\limits_i P_i$-ն հաշվելի բազմություն է որպես բնական թվերի բազմության ենթաբազմություն։ Նույն պատճառով $\bigcup\limits_i B_i$-ն հաշվելի բազմություն է որպես $\bigcup\limits_i P_i$ հաշվելի բազմության ենթաբազմություն։ Ուստի $\bigcup\limits_i A_i$ բազմությունը հաշվելի բազմություն է։
\par Այժմ դիտարկենք ընդհանուր դեպքը, երբ որոշ $A_i\cap A_j,\ i\neq j$ հատումներ կարող են դատարկ չլինել։ Անցնենք նոր՝ $A_1',A_2',\dots$ բազմությունների՝ սահմանելով դրանք (ինդուկտիվ եղանակով) հետևյալ բանաձևերով․
\[
A_1'=A_1,\ A_2'=A_2\setminus A_1', \ A_3'=A_3\setminus (A_1' \cup A_2'),\ \dots,\ A_n'=A_n\setminus (A_1'\cup A_2'\cup \dots \cup A_{n-1}')։
\]
\par Պարզ է, որ յուրաքանչյուր $A_i'$ հաշվելի բազմություն է՝ որպես $A_i$ հաշվելի բազմութ\-յան ենթաբազմություն։ Բացի այդ, $A_i' \cap A_j'=\varnothing$, երբ $i\neq j$։ Ուստի $\bigcup\limits_i A_i'$-ը հաշվելի բազմություն է՝ համաձայն վերը ապացուցված մասնավոր դեպքի։ Ցույց տանք, որ $\bigcup\limits_i A_i =\bigcup\limits_i A_i'$, որից էլ կհետևի, որ $\bigcup\limits_i A_i$ հաշվելի բազմություն է։

\par Քանի որ $A_i' \subset A_i$, ուստի $\bigcup\limits_i A_i' \subset \bigcup\limits_i A_i$։ Միավորելով $A_n'=A_n\setminus {(A_1' \cup  A_2' \cup \dots \cup  A_{n-1}')}$ հավասարության երկու կողմերը $A_1'\cup  A_2'\cup \dots \cup  A_{n-1}'$ բազմության հետ՝ կստանանք $\bigcup\limits_{i=1}^n A_i'=A_n\cup \left(\bigcup\limits_{i=1}^{n-1} A_i'\right)$։ Ուստի $A_n\subset \bigcup\limits_{i=1}^n A_i'$, և ուրեմն $\bigcup\limits_i A_i \subset \bigcup\limits_i A_i'$։ Այսպիսով՝ $\bigcup\limits_i A_i=\bigcup\limits_i A_i'$։
\end{proof}
\renewcommand*{\proofname}{\hspace{18pt}\textbf{Ապացուցում։}\nopunct}
\begin{hetevanq3}
\begin{enumerate}
    \item[ա)] $\Z\times \Z$-ը հաշվելի անվերջ բազմություն է,
    \item[բ)] ռացիոնալ թվերի $\Q$ բազմությունը հաշվելի անվերջ բազմություն է։
\end{enumerate}
\end{hetevanq3}
\begin{proof}
\begin{enumerate}
    \item[ա)] Սևեռելով որևէ $a\in \Z$ ամբողջ թիվ՝ դիտարկենք $\{a\}\times\Z$ ենթաբազմությունը $\Z\times \Z$-ում։ Պարզ է, որ այն հաշվելի անվերջ բազմություն է։ Այժմ կարող ենք $\Z\times \Z$-ը ներկայացնել որպես հաշվելի անվերջ թվով, հաշվելի անվերջ բազմությունների միավորում՝ $\Z\times \Z= \bigcup\limits (\{a\}\times \Z)$, որտեղ միավորումը կատարվում է ըստ $a \in \Z$ փոփոխականի։ Ուստի $\Z\times\Z$-ը հաշվելի անվերջ բազմություն է՝ ըստ թեորեմ 3-ի։
    \item[բ)] Ցանկացած $r\in \Q$, $r\neq 0$ ռացիոնալ թիվ կարող է միակ ձևով ներկայացվել որպես չկրճատվող $\dfrac{p}{q}$ կոտորակ, որի $q$ հայտարարը դրական ամբողջ թիվ է, իսկ $p$ համարիչը ամբողջ թիվ է։ Կառուցենք ինյեկտիվ $f:\Q\rightarrow \Z\times\Z$ արտապատկե\-րում $f(r)=(p,q)$, $f(0)=(0,1)$ բանաձևով։ Ըստ թեորեմ 1-ի՝ $f(\Q)$-ն հաշվելի անվերջ բազմություն է որպես $\Z\times \Z$ հաշվելի անվերջ բազմության անվերջ ենթաբազմու\-թյուն։ Հետևաբար $\Q$-ն ևս հաշվելի անվերջ բազմություն է։ \qedhere
\end{enumerate}
\end{proof}

\begin{theorem}
Ցանկացած $X$ անվերջ բազմությունից կարելի է զատել հաշվելի անվերջ ենթաբազմություն այնպես, որ մնացորդն իր մեջ դարձյալ պարունակի հաշվելի անվերջ ենթաբազմություն։
\end{theorem}
\begin{proof}
Ընտրենք որևէ $a_1\in X$ տարր, իսկ $X\setminus \{a_1\}$-ում ընտրենք որևէ $b_1$ տարր։ Այնուհետև $X\setminus \{a_1,b_1\}$-ում ընտրենք որևէ $a_2$ տարր, իսկ $X\setminus\{a_1,b_1,a_2\}$-ում ընտրենք որևէ $b_2$ տարր և այսպես շարունակ։ Կստանանք երկու չհատվող $A=\{a_1,a_2,\dots\}$ և $B=\{b_1,b_2,\dots\}$ հաշվելի անվերջ ենթաբազմություններ։ Նշանակենք $X\setminus (A\cup B)=Y$։ Քանի որ $A\cap B = \varnothing$, ուստի $X=A\cup  B\cup  Y$ և $A\subset X,\ B\subset X\setminus A$։
\end{proof}

\begin{theorem}
Բազմությունը անվերջ է այն և միայն այն դեպքում, երբ այն պարունա\-կում է իրեն հավասարազոր, իրենից տարբեր ենթաբազմություն։
\begin{proof}
Պայմանի \textbf{անհրաժեշտությունը}։ Դիցուք $X$-ը անվերջ բազմություն է։ Ըստ թեորեմ 4-ի՝ գոյություն ունեն $A$ և $B$ հաշվելի անվերջ ենթաբազմություններ այնպես, որ $X=A\cup B \cup Y$ և $X\setminus A=B\cup Y$։ Քանի որ $(A\cup  B)$-ն (ըստ թեորեմ 3-ի) և $B$-ն հաշվելի անվերջ բազմություններ են, ուստի գոյություն ունի $(A\cup  B)\leftrightarrow B$ փոխմիարժեք համապատասխանություն նրանց տարրերի միջև։ Այդ համապա\-տաս\-խանության երկու կողմերը միավորելով $Y\leftrightarrow Y$ նույնական համապատասխանու\-թյան հետ՝ կստա\-նանք $X\leftrightarrow (X\setminus A)$ փոխմիարժեք համապատասխանություն, ընդ որում $(X\setminus A)\subset X$ և $X\setminus A\neq X$։
\par Պայմանի \textbf{բավարարությունը}։ Դիցուք $X$-ը որևէ բազմություն է, $A\subset X,\ A\neq X$, և գոյություն ունի $X\leftrightarrow A$ փոխմիարժեք համապատասխանություն։ Պարզ է, որ $X$-ը վերջավոր բազմություն լինել չի կարող, ուստի այն անվերջ բազմություն է։
\end{proof}
\end{theorem}

\begin{theorem}
Իրական թվերի բազմությունը ոչ հաշվելի բազմություն է։
\end{theorem}
\begin{proof}
Նախ նկատենք, որ ցանկացած $(a,b)$ ինտերվալ հավասարազոր է $(0,1)$ ինտերվալին։ Իրոք, $g:(a,b)\rightarrow (0,1),\ g(x)=\dfrac{x-a}{b-a}$ արտապատկերումը փոխմիարժեք արտապատկերում է (հիմնավորե՛լ)։ Այնուհետև քանի որ $(-\infty,+\infty)$-ը հավասարազոր է $(-1,1)$-ին (տե՛ս օրինակ 9-ը թեմա 1-ում), ուստի $(-\infty,+\infty)$-ը հավասարազոր է $(0,1)$ ինտերվալին։ Հետևաբար թեորեմի ապացուցման համար բավական է ցույց տալ, որ $(0,1)$ ինտերվալը ոչ հաշվելի բազմություն է։ Կատարենք հակասող ենթադրու\-թյուն․ դիցուք գոյություն ունի $f:\N \rightarrow (0,1)$ փոխմիարժեք համապատասխանություն։ Սա նույնն է, թե $(0,1)$ ինտերվալի բոլոր թվերը կարելի է համարակալել բոլոր $1,2,\dots$ բնական թվերով։ Ներկայացնելով $f(i)$ թիվը միակ $0{,}\alpha_{i1} \alpha_{i2} \alpha_{i3}\dots$ անվերջ տասնորդական կոտորակի տեսքով, որտեղ $\alpha_{ij}$-ն $0,1,2,\dots,9$ թվանշաններից որևէ մեկն է, կունենանք՝
\begin{align*}
&f(1)=0{,}\alpha_{11} \alpha_{12} \alpha_{13} \dots\\
&f(2)=0{,}\alpha_{21} \alpha_{22} \alpha_{23} \dots \\
&\dots\dots\dots\dots\dots\dots\dots.\\
&f(n)=0{,}\alpha_{n1} \alpha_{n2} \alpha_{n3} \dots\\
&\dots\dots\dots\dots\dots\dots\dots.
\end{align*}
\par Օգտվելով այս շարքից՝ կազմենք մի $b=0{,}\beta_1\beta_2\beta_3\dots$ թիվ, որտեղ $\beta_{1}$-ը $\alpha_{11}$ թվանշանից տարբեր, $1$-ից մինչև $8$ կամայական թվանշան է, $\beta_{2}$-ը $\alpha_{22}$ թվանշանից տարբեր, $1$-ից մինչև $8$ կամայական թվանշան է, $\dots$, $\beta_{n}$-ը $\alpha_{nn}$ թվանշանից տարբեր, $1$-ից մինչև $8$ կամայական թվանշան է, $\dots$ և այսպես շարունակ։
% $b \in (0,1)$ թիվ՝ $b=0{,}\beta_{1} \beta_{2} \beta_{3}\dots$, որտեղ ${\beta_{i}=\begin{cases} 1,\textrm{ եթե } \alpha_{ii} \neq 1 \\ 2, \textrm{ եթե } \alpha_{ii}=1 \end{cases}\hspace{-1em}}$։ Օրինակ՝ եթե ${f(1)=0{,}03451\dots},\ {f(2)=0{,}91407\dots},\\ {f(3)=0{,}70496\dots},\ {f(4)=0{,}42017\dots},\ {f(5)=0{,}70496\dots}$, ապա $b=0{,}12121\dots$։
\par Պարզ է, որ $b \neq f(n)$ ցանկացած բնական $n$-ի դեպքում ($b$ և $f(n)$ թվերի $n$-րդ տասնորդական թվանշանները տարբեր են)։ Իսկ դա նշանակում է, որ մի կողմից ${b \in (0,1)}$, մյուս կողմից $b$ թիվը բացակայում է համարակալված թվերի շարքում, այսինքն՝ $b \not\in (0,1)$։ Ստացանք հակասություն։
\end{proof}

\begin{note} Ուշադիր ուսանողը հավանաբար նկատեց, որ $b$ թիվը կազմելիս նրա թվանշանները ընտրվեցին յուրատեսակ անկյունագծային եղանակով։
\end{note}

\begin{hetevanq6}
\begin{enumerate}
    \item[ա)] Բոլոր իռացիոնալ թվերի $I$ բազմությունը ոչ հաշվելի բազմություն է։ Իրոք, $\R= I \cup \Q,\ I \cap \Q=\varnothing$։ Եթե $I$-ն լիներ հաշվելի բազմություն, ապա $\R$-ը նույնպես կլիներ հաշվելի բազմություն՝ համաձայն թեորեմ 3-ի, ինչը հակասում է թեորեմ 6-ին։
    \item[բ)] Իռացիոնալ թվերը քանակապես շատ են ռացիոնալ թվերից։ Սա հետևում է թեորեմ 2-ի հետևանքից և թեորեմ 6-ից։
\end{enumerate}
\end{hetevanq6}
Մինչև այժմ մենք քանակապես համեմատում էինք բազմությունները՝ առանց հստակեցնելու, թե ինչ է \textbf{բազմության տարրերի քանակությունը}։ Հիշեցնենք, որ երկու $X$ և $Y$ բազմություններ կոչվում են հավասարազոր (կնշանակենք $X\sim Y$), եթե գոյություն ունի փոխմիարժեք համապատասխանություն նրանց տարրերի միջև։ Սա երկտեղ հարաբերություն է սահմանված և՛ վերջավոր, և՛ անվերջ բազմություն\-ների դեպքում։ Ցույց տանք, որ այն համարժեքության հարաբերություն է։
\begin{enumerate}
    \item $X\sim X$ (իրոք, $X$-ի նույնական $\nuynakan_X: X \rightarrow X$ արտապատկերումը ինքն իր վրա փոխմիարժեք է)։
    \item Եթե $X\sim Y$, ապա $Y\sim X$ (եթե գոյություն ունի ${f:X\rightarrow Y}$ փոխմիարժեք արտապատկերում, ապա գոյություն ունի նաև նրան հակադարձ ${f^{-1}:Y\rightarrow X}$ փոխմիարժեք արտապատկերում)։
    \item Եթե $X\sim Y$ և $Y\sim Z$, ապա $X\sim Z$ (եթե գոյություն ունեն ${f:X\rightarrow Y},\ {g:Y\rightarrow Z}$ փոխմիարժեք արտապատկերումներ, ապա պարզ է, որ ${g\circ f:X\rightarrow Z}$ արտա\-պատ\-կե\-րումը նույնպես փոխմիարժեք է)։
\end{enumerate}
\par Հետևաբար դիտարկվող բազմությունները տրոհվում են համարժեքության դա\-սե\-րի՝ ըստ այդ համարժեքության հարաբերության։ Ստացված ֆակտոր-բազմության տար\-րերը կոչվում են \textbf{հզորություններ}։ Ամեն մի $X$ բազմության համար $X$-ը պարու\-նակող $\big[X\big]$ համարժեքության դասը կոչվում է $X$-ի հզորություն և նշանակվում է $\abs{X}$։ Վերջավոր բազմությունների դեպքում բազմության հզորությունը կարող է նույնացվել նրա տարրերի քանակի հետ։
\par Ընդհանուր դեպքում բազմությունների հզորությունները կոչվում են \textbf{կարդինալ թվեր} կամ \textbf{պարզապես կարդինալներ}։ Բոլոր հաշվելի անվերջ բազմություններն ունեն նույն հզորությունը կամ նույն կարդինալը, որը նշանակվում է $\omega$։ Իրական թվերի բազմության հզորությունը կոչվում է \textbf{կոնտինում} և նշանակվում է $\mathfrak{c}$։ Ըստ բազմությունների քանակական համեմատման՝ ունենք $\omega<\mathfrak{c}$։
\par Այսպիսով՝ անվերջ բազմությունների դեպքում առայժմ ունենք որակապես տար\-բեր երկու հզորություններ՝ $\omega$ և $\mathfrak{c}$։ Իսկ այդ երկուսից բացի գոյություն ունե՞ն արդյոք այլ հզորություններ։ Հարցին պատասխան է տալիս Կանտորի հետևյալ թեորեմը։
\begin{theorem} \label{թեորեմ 7}
Ցանկացած $X$ բազմության բոլոր ենթաբազմություններից կազմված $\Ens X$ բազմության հզորությունը մեծ է $X$-ի հզորությունից։
\end{theorem}
\begin{proof}
Նախ դիտարկենք այն դեպքը, երբ $X$-ը վերջավոր բազմություն է՝ $X=\{x_1,x_2,\dots,x_n\}$։ Ինչպես գիտենք, այս դեպքում $\Ens X$ բազմության տարրերի քանակը $C_n^0+C_n^1+\dots+C_n^n=2^n$ թիվն է, և $2^n>n$, երբ $n\geq 1$։ Ուստի $\abs{\Ens X}>\abs{X}$։
\par Դիցուք՝ այժմ $X$-ը անվերջ բազմություն է։ Համապատասխանեցնելով $X$-ի ամեն մի $x$ տարրին $\{x\}$ տարրը $\Ens X$-ում՝ ստանում ենք, որ $\abs{X}\leq \abs{\Ens X}$։ Ցույց տանք, որ իրականում $\abs{X}<\abs{\Ens X}$։
\par Կատարենք հակասող ենթադրություն, այն է՝  $\abs{X}=\abs{\Ens X}$, այսինքն գոյու\-թյուն ունի $f:X\rightarrow \Ens X$ փոխմիարժեք համապատասխանություն։ Ցանկացած $x\in X$ տարրի դեպքում կա՛մ $x\in f(x)$, կա՛մ $x\not\in f(x)$։ Դիտարկենք $X$-ի $A$ ենթաբազմու\-թյունը կազմված $X$-ի այն բոլոր $a$ տարրերից, որ $a\not \in f(a)$։ Կարճ՝ $A=\{a\in X\mid a\not\in f(a)\}$։ Ցույց տանք, որ $A$-ն ոչ դատարկ բազմություն է։ Իրոք, ըստ ենթադրության, գոյություն ունի $x_0\in X$ տարր, որ $f(x_0)=\varnothing$ և իհարկե $x_0\not \in f(x_0)$։ Ուստի $x_0\in A$, և ուրեմն $A\neq \varnothing$: Քանի որ $A\in \Ens X$, ուստի գոյություն ունի $a_0\in X$ տարր, որ $f(a_0)=A$ և $a_0\neq x_0$։
\par Պարզ է, որ կա՛մ ա) $a_0\in A$, կա՛մ բ) $a_0\not\in A$։ Քննարկենք այս երկու դեպքերն առանձին-առանձին։
\begin{enumerate}
    \item[ա)] Եթե $a_0\in A$, ապա $a_0\not\in f(a_0)$։ Նշանակում է $a_0\not\in A$ (հակասություն)։
    \item[բ)] Եթե $a_0\not\in A$, ապա $a_0\in f(a_0)$։ Նշանակում է $a_0\in A$ (հակասություն)։
\end{enumerate}
\par Այսպիսով մեր ենթադրությունը, որ գոյություն ունի փոխմիարժեք հա\-մա\-պա\-տաս\-խա\-նութ\-յուն $X\leftrightarrow \Ens X$, բերեց աբսուրդի։ Հետևաբար $\abs{X} \neq \abs{\Ens X}$, և ուրեմն $\abs{X} < \abs{\Ens X}$։\qedhere
\end{proof}
% page 32
\begin{hetevanq7}
Վերցնելով որևէ $X$ անվերջ բազմություն և հաջորդաբար կիրառե\-լով \hyperref[թեորեմ 7]{թեորեմ 7}-ը՝ կստանանք $\abs{X}<\abs{\Ens X}<\abs{\Ens (\Ens X)}<\dots$
\par Այսինքն՝ գոյություն ունեն անթիվ քանակով միմյանց զույգ առ զույգ ոչ հավա\-սա\-րա\-զոր անվերջ բազմություններ, ուստի և անթիվ քանակով կարդինալ թվեր։
\end{hetevanq7}
\subsection*{Բազմությունների տեսության պարադոքսների մասին}
Բազմությունների այսպես կոչված կանտորյան կամ նաիվ տեսության հետ կապ\-ված առաջին պարադոքսները հայտնվեցին դեռ $1897$ թվին։ Դրանցից մեկը հայտ\-նա\-բերեց ինքը՝ Կանտորը։
\par Դիցուք $\Omega$-ն բոլոր հնա\-րա\-վոր բազմությունների բազմությունն է։ Ըստ թեորեմ ${ 7 \textrm{-ի} }$՝ $\abs{\Omega}<\abs{\Ens \Omega}$։ Բայց $\Ens \Omega$ ինքը բազմություն է, որի տարրերը (որպես բազմութ\-յուն\-ներ) պարունակվում են $\Omega$-ում։ Այսինքն՝ $\Ens \Omega \subset \Omega$, և հետևաբար $\abs{\Ens \Omega}\leq \abs{\Omega}$։ Ստացանք հակասություն։
\par Բերենք ևս մի պարադոքս (Ռասսելի պարադոքսը)։ Համարենք, որ բոլոր հնա\-րա\-վոր բազմությունները տրված են միաժամանակ։ Նշանակենք $Q$-ով  այն բոլոր բազմութ\-յուն\-ների բազմությունը, որոնք չեն պարունակում իրենց որպես տարր։ Հարց․ պարու\-նա\-կո՞ւմ է արդյոք $Q$-ն իրեն որպես տարր։ Եթե ոչ, այսինքն $Q \not\in Q$, ապա ըստ $Q$-ի սահմանման $Q\in Q$ (հակասություն)։ Իսկ եթե այո, այսինքն՝ $Q\in Q$, ապա նորից ըստ $Q$-ի սահմանման $Q$-ն չպետք է լինի տարր իր համար, ուստի $Q\not\in Q$ (հակասություն)։
\par Որպես կանոն, այդպիսի պարադոքսների առաջացման պատճառը բազմություն, բազմության տարր հասկացությունների կամայական (ոչ կանոնակարգված) ընկա\-լումն է կամ մեկնաբանումը։
\par Եղել են տարբեր մոտեցումներ՝ ինչպես խուսափել պարադոքսներից։ Հիմնական ելքը բազմությունների տեսության կառուցումն է աքսիոմատիկ եղանակով: Կան մի քանի այդպիսի համակարգեր, որոնցից ամենահայտնին այսպես կոչված Ցերմելո-Ֆրենկելի աքսիոմատիկ համակարգն է (տե՛ս \cite{Frenkel}-ում)։
\par Գոյություն ունի մեկ ուրիշ, գործնականում պարզ և հարմար եղանակ՝ խուսա\-փե\-լու պարադոքսներց, այդուհանդերձ մնալով բազմությունների կանտորյան (նաիվ) տեսության շրջանակներում։ Այդ մոտեցման հիմքում ընկած են երկու հիմնական սկզբունքներ։
\par\textbf{Սկզբունք A․} չի կարելի բոլոր հնարավոր բազմությունները համարել տրված միաժամանակ:
\par\textbf{Սկզբունք B․} ոչ մի բազմություն չի կարող լինել տարր ինքն իր համար:
\par Թվում է, թե սկզբունք A-ն ենթադրում է դիտարկվող բազմությունների տեսա\-կա\-նին կամ քանակությունը սահմանափակել բազմությունների նախապես տրված մի ինչ-որ $M$ ընտանիքով։ Դա այնքան էլ այդպես չէ։ Յուրաքանչյուր կոնկրետ մաթեմա\-տիկա\-կան հետազոտության դեպքում նպատակահարմար է տվյալ պահին մեր տրա\-մա\-դրության տակ գտնվող բազմությունների $M$ ընտանիքը համարել սահմանափակ։ Բայց եթե ընթացքում հարկ է լինում, ըստ նպատակահարմարության, հետազոտութ\-յան մեջ ներգրավել նոր բազմություններ, այսինքն ընդլայնել $M$-ը, ապա առաջարկ\-վող համակարգը այդպիսի հնարավորություն տալիս է։ Այս և այլ դետալների հետ ավելի մանրամասն կարելի է ծանոթանալ \cite{Arxangelski}-ում։


% \newpage
% Զուտ թեմա երկու խնդիրներն եմ քոփի արել ու համարյա բան չեմ փոխել
\bigskip
\bigskip
\subsubsection*{Խնդիրներ և հարցեր թեմա 3-ի վերաբերյալ}

\begin{enumerate}[label=\thesection.\arabic*.]
% 3.1
\item Թեմա $1$-ի օրինակ $9$-ի նմանությամբ կառուցեք փոխմիարժեք $h$ արտապատ\-կերում $\{(x,y)| x^2 + y^2 < 1\}$ անեզր շրջանի և $\mathbb{R}^2$ հարթության կետերի միջև։ 

\begin{hint}
    Տեղադրելով $1$ շառավղով անեզր կիսասֆերան այնպես, որ շոշափի $\mathbb{R}^2$ հարթությունը $O$ կետում, նախ կառուցեք փոխմիարժեք համապատասխանություն շրջանի և կիսասֆերայի միջև, և ապա՝ կիսասֆերայի և հարթության կետերի միջև՝ ինչպես ցույց է տրված նկարում։
\end{hint}

% \textbf{Ցուցում։} Տեղադրեք $1$ շառավղով անեզր կիսասֆերան այնպես, որ շոշափի $/mathbb{R}^2$ հարթությունը $O$ կետում։
\begin{figure}
    \centering
    \includegraphics[width=0.5\linewidth]{images/Screenshot 2024-10-03 173106.png}
    \caption{Խնդիր 3.1}
    % \label{fig:enter-label}
\end{figure}

% 3.2
\item Ապացուցեք հետևյալ հավասարազորությունները․
\[
(0,1) \sim (0,1] \sim [0,1) \sim [0,1]
\]

\begin{hint}
    Նախ այդ բազմությունները ներկայացրեք որպես միայն ռացիոնալ թվերից և միայն իռացիոնալ թվերից կազմված երկու ենթաբազմությունների միավորում։ Օրինակ՝ $(0,1) = \mathbb{Q} \cup I$ և $[0,1]=(\mathbb{Q} \cup {0,1}) \cup I$, որտեղ $\mathbb{Q}$-ն և $\mathbb{I}$-ն $(0,1)$ ինտերվալի ռացիոնալ և իռացիոնալ թվերի բազմություններն են։ Այժմ կարող ենք հաստատել $(0,1) \sim [0;1]$ հավասարազորություն՝ լրացնելով $\mathbb{Q} \sim \mathbb{Q} \cup {0,1}$ հավասարազորությունը (որը տեղի ունի ըստ թեորեմ $3$-ի) նույնական $I \rightarrow I$ արտապատկերումով։ \red{vstah chem vor petq er es mathb-n hanel I-ic, mi ban bayc cher havanel hastat}
\end{hint}

% \textbf{Ցուցում:} Նախ այդ բազմությունները ներկայացրեք որպես միայն ռացիոնալ թվերից և միայն իռացիոնալ թվերից կազմված երկու ենթաբազմությունների միավորում։ Օրինակ, $(0;1) = \mathbb{Q} \cup \mathbb{I}$ և $[0;1]=(\mathbb{Q} \cup {0;1}) \cup \mathbb{I}$, որտեղ $\mathbb{Q}$-ն և $\mathbb{I}$-ն $(0;1)$ ինտերվալի ռացիոնալ և իռացիոնալ թվերի բազմություններն են։ Այժմ կարող ենք հաստատել $(0;1) \sim [0;1]$  լրացնելով $\mathbb{Q} \sim \mathbb{Q} \cup {0,1}$ հավասարազորությունը (որը տեղի ունի ըստ թեորեմ $3$-ի) նույնական $\mathbb{I} \rightarrow \mathbb{I}$ արտապատկերումով։

% 3.3
\item Ապացուցեք, որ թվային ուղղի $(-\infty, a), (-\infty,b], (c, +\infty), [d, +\infty], , (a,b), [c,d], [m,n), [p,q] $ տեսքի ցանկացած երկու ենթաբազմություն հավասարազոր են։

\begin{hint}
    Դիտարկենք հետևյալ արտապատկերումները․
\begin{enumerate}
    \item[ա)] $f_1 : (a,b) \rightarrow (0,1), f_2 : [a,b] \rightarrow [0,1], f_3 : (a,b] \rightarrow (0;1], f_4 : [a,b) \rightarrow [0;1)$, որոնք սահմանվում են $x \mapsto \dfrac{x-a}{b-a}$ համադրումով,
    \item[բ)]  $g_1 : (-\infty; 0) \rightarrow (0,1), g_2 : (-\infty; 0] \rightarrow (0,1]$, որոնք սահմանվում են $x \mapsto \frac{1}{1-x}$ համադրումով,
    \item[գ)]  $h_1 : (0; + \infty) \rightarrow  (0,1), h_2 : [0; \infty) \rightarrow  [0,1]$ որոնք սահմանվում են $x \mapsto \dfrac{x}{x+1}$ համա\-դրումով,
    \item[դ)]  $u_1 : (- \infty; a) \rightarrow (- \infty; 0 ), u_2 : (-\infty; a] \rightarrow (-\infty; 0]$, որոնք սահմանվում են $x \mapsto x-a $ համադրումով,
    \item[ե)] $v_1 : [a;+ \infty) \rightarrow ( 0;+ \infty), u_2 : ( a;+\infty] \rightarrow ( 0;+\infty]$, որոնք սահմանվում են $x \mapsto x-a $ համադրումով,
    \item[զ)] $w : ( - \infty;0) \rightarrow ( 0;+ \infty)$, սահմանվում է՝ $w(x) = -x$: 

\end{enumerate}

Հիմնավորելով սրանցից յուրաքանչյուրի փոխմիարժեքությունը՝ օգտվեք դրանց համադրույթներից և խնդիր $3.2$-ից։
\end{hint}

% 3.4
\item Ապացուցեք, որ հաշվելի $A$ բազմությունից նրա որևէ $B$ վերջավոր ենթաբազ\-մություն անջատելիս ստացված $A\setminus B$ մնացորդը դարձյալ հաշվելի բազմու\-թյուն է։

\begin{hint}
     Դիտարկեք $A$-ի տարրերի որևէ ${a_1, a_2, \dots}$ համարակալում և երկու դեպք՝ $A$-ն վերջավոր է, $A$-ն անվերջ է։ Երկրորդ դեպքում օգտվեք թեորեմ $1$-ից \red{do the ref magic here bitte}։ 
\end{hint}

% 3.5
\item Ապացուցեք. եթե $A_1, A_2, \dots, A_n$, որտեղ $n \geq 1$, բազմությունները հաշվելի են, ապա նրանց $A_1 \times A_2 \times \dots \times A_n$ ուղիղ արտադրյալը նույնպես հաշվելի բազմություն է։

\begin{hint}
    Դիտարկեք  $A_1 \times A_2 \times \dots \times A_n = \bigcup\limits_{i} A_1 \times A_2 \times \dots \times A_{n-1} \times \{ b_{i} \} $  ներկայացումը, որտեղ $A_n = \{b_1, b_2, \dots \}$, և օգտվելով թեորեմ $3$-ից՝ կիրառեք ինդուկցիա ըստ $n$-ի։
\end{hint}
 
% 3.6
\item Ապացուցեք․ հաշվելի բազմության տարրերով կազմված բոլոր վերջավոր հա\-ջոր\-դականությունների բազմությունը հաշվելի է։

\begin{hint}
    Հաշվելի $A = \{a_1, a_2, \dots \}$ բազմության տարրերով կազմված $n$ երկարության ամեն մի $a_{i_{1}}, a_{i_{2}}, \dots, a_{i_{n}}$ \red{asum a indexnery mecacnel, bayc axer ay dzer cavy tanem, vonc, petq a a-ern el hety mecacnel uremn, esim e, esim} հաջորդականություն կարելի է դիտել որպես $A \times A \times  \dots \times A$ ուղիղ արտադրյալի տարր և հակառակը։ Օգտվելով խնդիր $3.5$-ից՝ կիրառեք ինդուկցիա ըստ $n$-ի։
\end{hint}

% 3.7
\item Ունենք բազմությունների որևէ $f : A \rightarrow B$ սյուրյեկտիվ արտապատկերում։ Ապացուցեք՝ եթե $A$-ն հաշվելի է, ապա $B$-ն նույնպես հաշվելի է։

\begin{hint}
    Ամեն մի $b \in B$ տարրի համար ընտրենք որևէ $a \in f^{-1}(B)$ տարր և սահմանենք $g : B \rightarrow A$, $g(b)=a$ արտապատկերում։
Ցույց տվեք, որ $g$-ն ինյեկտիվ արտապատկերում է, և օգտվեք թեորեմ $1$-ից։
\end{hint}

% 3.8
\item Ապացուցեք, որ հաշվելի $A = \{a_1, a_2, \dots \}$ բազմության բոլոր վերջավոր ենթա\-բազմությունների բազմությունը հաշվելի է։

\begin{hint}
    Տարրերի ամեն մի $a_{i_{1}}, a_{i_{2}}, \dots , a_{i_{n}}$ վերջավոր հաջորդականություն համադրենք $A$ բազմության $\{a_{i_{1}}, a_{i_{2}}, \dots , a_{i_{n}}\}$ ենթաբազմությունը։
Արդյունքում ստանում ենք որոշակի $f$ արտապատկերում $A$ բազմության տարրերով կազմված բոլոր վեր\-ջավոր հաջորդականությունների բազմությունից դեպի $A$-ի բոլոր վերջավոր ենթաբազմություններով կազմված բազմության մեջ ցույց տվեք, որ $f$-ը սյուր\-յեկտիվ արտապատկերում է, և օգտվեք նախորդ երկու խնդիրներից։
\end{hint}

% 3.9
\item Ապացուցեք․ եթե $A$-ն անվերջ բազմություն է, իսկ $B$-ն հաշվելի բազմություն է, ապա $A \cup B$-ն հավասարազոր է $A$-ին։

\begin{hint}
    Օգտվեք հետևյալ երկու պնդումներից՝ հիմնավորելով դրանք․
\begin{enumerate}
    \item [ա)] եթե $A_1$-ը $A$ բազմության որևէ հաշվելի ենթաբազմություն է, ապա տեղի ունի $A_1 \cup B \sim A_1$ հավասարազորությունը,
    \item [բ)] $A \cup B=\left(A \setminus A_1\right) \cup\left(A_1 \cup B\right) \sim \left(A \setminus A_1\right) \cup A_1 = A_1$։
\end{enumerate}
\end{hint}


% 3.10
\item Ապացուցեք․ եթե $A$-ն անվերջ և ոչ հաշվելի բազմություն է, իսկ $B$-ն հաշվելի է, ապա $A \setminus B \sim A$։

\begin{hint}
    Օգտվեք $A=(A \setminus B) \cup B$ նույնությունից և նախորդ խնդրից։
\end{hint}

% 3.11
\item Ի՞նչ հզորություն ունի բոլոր իռացիոնալ թվերի բազմությունը։ 

\begin{hint}
    Օգտվեք խնդիր $3.9$-ից \red{screw the ref-s, ageed? pls} ՝ վերցնելով որպես $A$ բոլոր իռացիոնալ թվե\-րի, իսկ որպես $B$՝ բոլոր ռացիոնալ թվերի բազմությունները։
\end{hint}


% 3.12
\item Ճի՞շտ է արդյոք, որ բոլոր իռացիոնալ թվերի բազմությունը և $(-1, 1)$ ին\-տեր\-վալի իռացիոնալ թվերի բազմությունը ունեն նույն հզորությունը։
\end{enumerate}



\end{document}