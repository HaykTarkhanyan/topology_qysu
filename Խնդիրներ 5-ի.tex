
\subsubsection*{Խնդիրներ և հարցեր թեմա 5-ի վերաբերյալ}
\begin{enumerate}
    \item[5.1] Հանդիսանու՞մ է արդյոք ենթաբազմությունների $\{[a,b;a\leq b]\}$ ընտանիքը թվային ուղղի որևէ տոպոլոգիայի բազա։
    \item[5.2] Ապացուցեք, որ թեորեմ թ-ի երկրորդ պայմանը կարելի է փոխարինել հետևյալ համարժեք պայմանով․ ցանկացած  $W_i, W_j \in B$ տարրերի և ամեն մի $x \in W_i \cap W_j$ տարրի համար գոյություն ունի $W_k \in B$ տարր, որ $x \in W_k$ և $W_k \subset W_i \cap W_j$: 
    \item[5.3] Դիտարկենք $R^2$ կոորդինատային հարթության ենթաբազմություններ $\Phi_1$ և $\Phi_2$ ընտանիքներ կազմված բոլոր այնպիսի անեզր քառակուսիներից (կողմերն ու գագաթները հեռացված են), որ առաջին ընտանիքում քառակուսիների կողմերը զուգահեռ են կոորդինատային առանցքներին, իսկ երկրորդ ընտանիքում քառակուսիների անկյունագծերն են զուգահեռ կոորդինատային առանցքներին: \\
    %Here can be your graph
    Ապացուցեք․ $\Phi_1$-ը և $\Phi_2$-ը ծառայում են որպես բազա $\R^2$-ի ինչ-որ $\tau_1$ և $\tau_2$ տոպոլոգիաների համար։\\
    \underline{Ցուցում։} Դիտարկենք որևէ երկու հատվող քառակուսի առաջին ընտանիքից և օգտվելով խնդիր 5․2-ից ստուգենք թեորեմ 2-ի երկրորդ պայմանը(առաջին պայմանը բավարարում է ակնհայտորեն)։ Երկրորդ ընտանիքի դեպքը բերվում է առաջին ընտանիքի դեպքին կատարելով պտույտ կոորդինատների սկզբնակետի շուրջը $45^\circ$-ով։
    \item[5.4] Դիտարկենք $\R^2$ հարթության բոլոր անեզր շրջանների(եզրային շրջանագծերը հեռացված են) $\Phi_3$ ընտանիքը։ Ապացուցենք, որ $\Phi_3$-ը բազա է $\R^2$-ի ինչ-որ $\tau_3$ տոպոլոգիայի համար։
    \item[5.5] Ճի՞խտ է արդյոք, որ $\R^2$ հարթության բոլոր եզրով(փակ) շրջանների ընտանիքը կազմում է բազա $\R^2$-ի ինչ-որ տոպոլոգիայի համար։
    \item[5.6]Ապացուցենք, որ 5․3 և 5․4 խնդիրներում նկարագրված $\Phi_1, \Phi_2, \Phi_3$ բազաները որոշում են թվային ուղղի նույն տոպոլոգիան։\\
    \underline{Ցուցում}։ Ցույց տվեք, որ յուրաքանչյուր $\Phi_i$ բազայի ($i=1,2,3$) ցանկացած տարր կարող է ներկայացվել որպես $\Phi_j, j \not= i$ բազայի անվերջ քանակությամբ որոշ տարրերի միավորում:
    \item[5.7] Ապացուցենք, որ բնական թվերից կազմված բոլոր անվերջ թվաբանական պրոգրեսիաների համախմբությունը բոլոր բնական թվերի բազմության ինչ-որ տոպոլոգիայի բազա է։\\
    \underline{Ցուցում}։ Դիցուք ունենք բնական թվերից կազմված $A=\{a_1,a_2,...\}$ և $B=\{b_1,b_2,...\}$ երկու անվերջ թվաբանական պրոգրեսիա համապատասխանաբար $d_1$ և $d_2$ տարբերություններով։ Դիցուք $c_1$-ը $A\cap B$ բազմության փոքրագույն տարրն է։ Ցույց տվեք, որ $C=A \cap B=\{c_1,c_2,...\}$ բազմությունը անվերջ թվաբանական պրոգրեսիա է $d_3=[d_1,d_2]$ տարբերությունով, որտեղ $[d_1,d_2]$-ը $d_1$ և $d_2$ թվերի ամենափոքր ընդհանուր բազմապատիկն է։
    \item[5.8] Ապացուցեք․ ցանկացած $T_1$ տարածության ամեն մի վերջավոր ենթաբազմություն փակ բազմություն է։
    \item[5.9] Ապացուցեք, որ աջից կիսաբաց ինտերվալների $(\R, \longmapsto)$ տոպոլոգիական տարածությունում
    \item[ա)]ամեն մի $[a, \infty)$ ենթաբազզմություն և բաց է, և փակ է;
    \item[բ)]ամեն մի $(a, b]$ ենթաբազզմություն ոչ բաց է, ոչ էլ փակ է;
    \item[5.10] Ապացուցեք․ խնդիր 5․4-ում դիտարկված $(\R^2, \tau_3)$ տարածությունը Հաուսդորֆյան տարածություն է։
    \item[5.11] Պարզեք՝ ստորև բերված տարածություններից որո՞նք են Հաուսդորֆյան տարածություն․
    \item[ա)] դիսկրետ տարածություն;
    \item[բ)] անտիդիսկրետ տարածություն;
    \item[գ)] աջից կիսաբաց ինտերվալների տարածություն։

\end{enumerate}