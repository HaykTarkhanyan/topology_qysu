1
Ապացուցեք․ ․․ տարածությունը բավարարում է հաշվելիության առաջին աքսիոմին։

Ցուցում։ Դիտարկելով կամայական ․․ կետ, ցույց տվեք, որ աջից կիսաբաց ինտերվալների ․․ ընտանիքը ․․ կետի շրջակայքի հաշվելի բազա է։ 

2
Ճի՞շտ է արդյոք, որ ցանկացած ․․ վերջ․ լր․ տարածություն, որտեղ ․․-ը ոչ հաշվելի բազմություն է, չի բավարարում անջատելիության առաջին աքսիոմին։

Ցուցում։ Օգտվեք օրինակ ․․-ում վերված ապացուցման ընթացքից։

3
Ապացուցեք, ․․ վերջ․ լրվ տարածությունը, որտեղ ․-ը ամողջ թվերի բազմությունն է, բավարարում է հաշվելիության առաջին աքսիոմին։

ցուցում։ Օգտվելով թեմա ․․-ի ․․ խնդրի նախ ցույց տվեք, որ ․․ վերջ․ լր․ տարածությունը բավարարում է հաշվելիություն երկրորդ աքիսումին։

4
Ճի՞շտ է արդյոք, որ՝ ․․ հաշվ․ լր․ տարածություն բավարարում է հաշվելիության առաջին աքսիոմին

Ցուցում։ Տես օրինակ ․․-ը։
5
Գտեք մետրիկական տարածություն, որտ չի բավարարում հաշվելիության առաջին աքսիոմին։
6
Ճի՞շտ է արդյոք, որ՝ գոյություն ունի ցանկացած հզորություն ․․ տոպոլոգիական տարածություն, որի յուրաքանչյուր մի կետանոց ենթբազմություն ամենուրեք խիտ է ․․-ում։
7
Ապացուցեք․ որևէ ․․ տարածության տոպոլոգիան դիսկրետ տոպոլոգիան է այն և միայն դեպքում, երբ ․․ճում գոյություն ունի միայն մի ամենուրեք խիտ ենթաբազմություն, և դա ինքը՝ ․․-ն է։

8
Ճի՞շտ է արդյոք, որ՝ կամայական տոպոլոգիական տարածությունում ցանկացած երկու ամենուրեք խիտ ենթաբազմությունների ա․․ միավորումը, ․․ հատումը ամենուրեք խիտ է։
9
Դիցուք ․․ ենթաբազմությունը ամենուրեք խիտ է ․․-ում։ Ապացուցեք․ ցանկացած ․․ բաց ենթաբազմության դեպքում ․․ ենթաբազմության փակումը ․․-ում համընկնում է ․․-ի փակման հետ՝ ․․․

Ցուցում։ Դիտարկենք կամայական ․․ կետ և այդ կետի ցանկացած ․․ բաց շրջակայք։ Հիմնվելով թեորեմ ․-ի վրա ցույց տվեք, որ ․․ դրանից կհետևի որ ․․ 

10
Ապացուցեք․ կամայական տոպոլոգիական տարածության վերջավոր քանակով ամենուրեք խիտ բաց ենթաբազմությունների հատումը նույնպես ամենուրեք խիտ ենթաբազմություն է։
Ցուցում։ Դիցուք ․․ ենթաբազմություններն ամենուրեք խիտ են ․․-ում։ Կիրառեք խնդիր ․․ճը, վերցնելով 
