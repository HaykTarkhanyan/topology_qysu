\bigskip
\bigskip
\subsubsection*{Խնդիրներ և հարցեր թեմա 4-ի վերաբերյալ}

\begin{enumerate}[label=\thesection.\arabic*.]
    \item Ապացուցեք, որ թվային ուղղի ամեն մի ինտերվալ կարող է ներկայացվել որպես $[a, b]$ տեսքի հատվածների միավորում։
    
    \item Ապացուցեք, որ թվային ուղղի վերջավոր քանակով կամայական ինտերվալ\-ների հատումը կամ դատարկ է, կամ ինտերվալ է։
    
    \item Ապացուցեք, որ թվային ուղղի անվերջ քանակով կամայական ինտերվալների հատումը կարող է լինել $ \varnothing$, $\{a\}$,  $(a,b)$, $[a,b)$, $(a,b]$, $[a,b]$ տեսքերի ենթաբազմու\-թյուններից որևէ մեկը։ Այդ 6 տեսքերից յուրաքանչյուրի համար կառուցեք այդ\-պիսի հատումների մեկական օրինակ։
    
    \item Դիտարկենք երեք տարրից կազմված որևէ $X=\{a,b,c\}$ բազմություն և նրա ենթաբազմությունների հետևյալ ընտանիքները․
    \begin{align*}
    &\Phi_1 = \{ \varnothing, \{c\}, \{a,b\},X \}, \\
    &\Phi_2 = \{ \varnothing, \{a\}, \{b\},\{a,c\},\{b,c\},X \}, \\ 
    &\Phi_3 = \{ \varnothing, \{a\}, \{b\},\{a,b,c\}\},\\ &\Phi_4 = \{ \varnothing, \{c\}, \{a,c\},\{a,b,c\} \}:
    \end{align*}

    \begin{enumerate}
        \item[ա)] Դրանցից որո՞նք են որոշում $X$ բազմության տոպոլոգիա։
        
        \item[բ)] Գրեք $X$-ի վրա իրարից տարբեր բոլոր տոպոլոգիաները։
    \end{enumerate}


    \item Որոշու՞մ է արդյոք տոպոլոգիա բոլոր բնական թվերի $\N$ բազմության վրա նրա ենթաբազմությունների հետևյալ ընտանիքը․
        \begin{enumerate}
        \item[ա)] $\{\varnothing, \{V_n \mid V_n \subset \N,\ n \geq 1\}\}$, որտեղ $V_n=\{n,n+1,\dots\}$,
        
        \item[բ)] $\{\N, \{W_n \mid W_n \subset \N,\ n \geq 1\}\}$, որտեղ $W_n=\{x\mid x \in \N \text{ և } x<n\}$:
    \end{enumerate}
    
    \item Ճի՞շտ է արդյոք, որ թվային ուղղի սովորական տոպոլոգիայում ամեն մի ոչ դատարկ բաց ենթաբազմություն կարող է ներկայացվել որպես $[a,b]$ տեսքի հատվածների միավորում։
    
    \item Ապացուցեք, որ $\R$ թվային ուղղի ենթաբազմությունների հետևյալ համա\-խմբու\-թյուններից ոչ մեկը տոպոլոգիա չէ $\R$-ի համար․
    \begin{enumerate}
        \item[ա)] $\{\varnothing,\ \R,\ \{(-\infty,x]\text{, որտեղ $x$-ը կամայական թիվ է $\R$-ում}\}$,
        
        \item[բ)] $\{\varnothing,\ \R,\ \{(a, b)\text{, որտեղ $a, b\in\R$ կամայական են և } a<b\}\}$:
    \end{enumerate}
    
    \item $\R$ թվային ուղղի ենթաբազմությունների հետևյալ ընտանիքներից որո՞նք են որոշում տոպոլոգիա $\R$-ի վրա․
    \begin{enumerate}
        \item[ա)] $\{\varnothing,\ \R,\ \{(-\infty,x) \mid x\in\R \text{ կամայական թիվ է}\}\}$,
        
        \item[բ)] $\{\varnothing,\ \R,\ \{(-\infty,x) \mid x\in\Q \text{ կամայական ռացիոնալ թիվ է}\}\}$,

        \item[գ)] $\{\varnothing,\ \{(-\infty,-x]\cup(x,+\infty) \mid x\ge 0 \text{ կամայական թիվ է}\}\}$,

        \item[դ)] $\{\varnothing,\ \R,\ \{[-x,x) \mid x>0 \text{ կամայական իռացիոնալ թիվ է}\}$։
    \end{enumerate}

    \item Դիցուք $X$-ը որևէ ոչ հաշվելի բազմություն է։ Ապացուցեք, որ օրինակ 5-ում սահմանված ենթաբազմությունների $\tau$ ընտանիքը որոշում է տոպոլոգիա $X$-ի վրա։
    
    \item Ապացուցեք, որ $\R$ թվային ուղղի հաշվելի լրացումների տոպոլոգիան ուժեղ է $\R$-ի վերջավոր լրացումների տոպոլոգիայից։
    
    \item Ապացուցեք, որ $\R$ թվային ուղղի հաշվելի լրացումների տոպոլոգիան համեմա\-տելի չէ $\R$-ի սովորական տոպոլոգիայի հետ։
    
    \item Հանդիսանու՞մ է արդյոք շրջակայք ուղղի $\sqrt{2}$ կետի համար բոլոր իռացիոնալ թվերից կազմված ենթաբազմությունը
        \begin{enumerate}
        \item[ա)] ($\R$, սովոր․) տարածությունում,
        
        \item[բ)] ($\R$, վերջ․ լր․) տարածությունում,
        
        \item[գ)] ($\R$, հաշվ․ լր․) տարածությունում։
    \end{enumerate}
\end{enumerate}