\documentclass[./main.tex]{subfiles}

\begin{document}
\onehalfspacing
\section{Տոպոլոգիական տարածությունների ուղիղ արտադրյալը, օրինակներ, ուղիղ արտադրյալի հատկությունները։ Ենթաբազմությունների ուղիղ արտադրյալի փակումն ու ներքնամասը: Ուղիղ արտադրյալի ժառանգական հատկություններ։ Որոշ երկրաչափական հասկացություններ $\mathbb{R}^n$ էվկլիդյան տարածություններում:}\label{sec:14}

\par Դիցուք ունենք $(X,\tau)$ և $(Y,\sigma)$ տոպոլոգիական տարածություններ։ Մեր խնդիրն է, որ\-պես ելակետ ընդունելով $\tau$ և $\sigma$ տոպոլոգիաները, սահմանել տոպոլոգիա բազմությունների $X\times Y$ դե\-կարտ\-յան արտադրյալի համար։ Նշանակվում է այն $\tau \times \sigma$։ Բնական կլինի $X\times Y$-ում բաց բազ\-մութ\-յուն\-ներ հայտարարել նրա $U \times V$ տեսքի ենթաբազմությունները, որտեղ $U$-ն և $V$-ն բաց ենթաբազմություններ են համապատասխանաբար $X$-ում և $Y$-ում։ Ստուգենք տո\-պո\-լո\-գիա\-յի 1-3 աքսիոմները։
\begin{enumerate}
    \item Պարզ է, որ $\varnothing \times \varnothing =\varnothing \in \tau \times \sigma$, և $X \times Y \in \tau \times \sigma$ (քանի որ $X\in \tau,\ Y\in \sigma$)։
    \item Միավորման աքսիոմը տեղի չունի, քանի որ ընդհանուր դեպքում $U_i\times V_j$ և $U_k\times V_l$ տեսքերի ենթաբազմությունների միավորումը կարող է չունենալ նույնպիսի տեսք (տես թեորեմ 1-ը թեմա 2-ում)։
\end{enumerate}
\par Այսպիսով $U \times V,\ U \in \tau,\ V \in \sigma$ ենթաբազմությունները չեն կազմում տոպոլոգիա $X\times Y$-ի համար։ Բայց կազմում են տոպոլոգիայի բազա։ Իրոք, բավարարվում են թեմա $5$-ում թեորեմ $2$-ի 1-2 պայմանները․
\begin{enumerate}
    \item $\bigcup\limits_{i,\, j}(U_i\times V_j)=X\times Y$;
    \item $(U_i\times V_j )\cap(U_k\times V_l)=(U_i\cap U_k)\times (V_j\cap V_l)$։
\end{enumerate}
\par Այդ բազայով ծնված $\tau \times \sigma$ տոպոլոգիան կոչվում է $X \times Y$ \textbf{ուղիղ արտադրյալի տոպոլոգիա}, իսկ $(X\times Y,\tau \times \sigma)$ զույգը կոչվում է $(X,\tau)$ և $(Y,\sigma)$ \textbf{տոպոլոգիական տարածությունների ուղիղ արտադրյալ}։ Այսպիսով $(X\times Y,\tau \times \sigma)$ տարածությունում բաց բազմություններ են համարվում բոլոր $U \times V$, որտեղ $U\in \tau,\ V \in \sigma$ ենթա\-բազ\-մութ\-յուններն ու նրանց բոլոր հնարավոր միավորումները։
\par Նման ձևով սահմանվում է տոպոլոգիական տարածությունների ուղիղ ար\-տադ\-րյալ կամայական վերջավոր քանակով $(X_1,\tau_1),\ (X_2,\tau_2),\dots,(X_n,\tau_n)$ տարածութ\-յուն\-ների դեպքում: Ստացվող $(X_1\times X_2\times\dots\times X_n,\tau_1\times\tau_2\times\dots\times\tau_n)$  տարածության տոպոլոգիայի բազա են կազմում բոլոր $U_1\times\ U_2\times\dots\times\ U_n,\ 
U_i\in\tau_i$ տեսքի են\-թա\-բազ\-մութ\-յունները:

\begin{example}
Դիտարկենք $\mathbb{R}$ թվային ուղիղը սովորական մետրական տոպոլո\-գիա\-յով՝ $(\mathbb{R},\textrm{սովոր․})$, և $(\mathbb{R}, \textrm{սովոր․}) \times (\mathbb{R},\textrm{սովոր․})$ ուղիղ արտադրյալը։ Այս տարածությունը $\mathbb{R}\times \mathbb{R}=\mathbb{R}^2$ հարթության կետերի բազմությունն է, որի ${\textrm{սովոր․} \times \textrm{սովոր․}}$ տոպոլոգիայի համար բազա է ծառայում բոլոր $(a,b)\times (c,d)$ տեսքի անեզր ուղղանկ\-յուն\-ների բազ\-մու\-թ\-յունը։
\import{tikz/uniform/}{31.tex}
%\begin{center}
%\includegraphics[scale=1]{images/id31.png}
%\end{center}
\par Մյուս կողմից՝ $\mathbb{R}^2$ բազմության համար ունենք նաև սովորական մետրական տո\-պո\-լո\-գիա, որի համար բազա է ծառայում բոլոր անեզր $\mathcal{D}(x,\mathbb{R})$ շրջանների բազմութ\-յունը։
\par Ցույց տանք, որ այս երկու բազաները որոշում են $\mathbb{R}^2$ հարթության միևնույն տոպոլոգիան, և հետևաբար $(\mathbb{R}^2,\textrm{սովոր․ մետր․ տոպ․})$ և $(\mathbb{R}^2,\textrm{սովոր․} \times \textrm{սովոր․})$ տոպո\-լո\-գիա\-կան տա\-րա\-ծութ\-յունները նույնն են։ Դրա համար բավական է ցույց տալ, որ ամեն մի անեզր ուղղանկյուն կարելի է ներկայացնել որպես անեզր շրջանների միավորում, և հակա\-ռակը՝ ամեն մի անեզր շրջան կարելի է ներկայացնել որպես անեզր ուղղանկյունների միավորում։
\import{tikz/uniform/}{32.tex}
%\begin{center}
%\includegraphics[scale=1]{images/id32.png}
%\end{center}
\par Իրոք, (տե՛ս գծագիրը) անեզր ուղղանկյան ամեն մի $x$ կետի համար կարելի է ընտրել $r>0$ թիվ այնպես, որ $\mathcal{D}(x,r)$ շրջանն ամբողջությամբ պարունակվի ուղղանկյան ներսում։ Նաև հակառակը՝ անեզր շրջանի ամեն մի $y$ կետի համար գոյություն ունի $y$ կենտրոնով անեզր ուղղանկյուն, որն ամբողջությամբ գտնվում է շրջանի ներսում։
\par Այս օրինակը ընդհանրացվում է․ $(\R,\textrm{սովոր․})$ թվային ուղղի $n$ կրկօրինակների $(\underbrace{\mathbb{R}\times \mathbb{R}\times \dots \times \mathbb{R}}_\text{$n$ հատ}; \underbrace{ \textrm{սովոր․} \times \textrm{սովոր․} \times \dots \times \textrm{սովոր․}}_\text{$n$ հատ})$ ուղիղ արտադրյալը նույն տոպո\-լո\-գիական տարածությունն է, ինչ $\mathbb{R}^n$ էվկլիդյան տարածությունը, վերցված սովո\-րա\-կան մետրական տոպոլոգիայով։
\end{example} %օրինակ 1
\label{օրինակ 1}
\begin{example}
Դիցուք $S$-ը շրջանագիծ է $\mathbb{R}^2$ հարթությունում՝ օժտված $\mathbb{R}^2$-ի սովո\-րա\-կան մետրական տոպոլոգիայից մակածված $\tau$ տոպոլոգիայով (տես օրինակ $3\textrm{-ը}$ թեմա 12-ում): Դիտարկենք ${S\times S}$ տորը (տես \red{օրինակ $2$-ը թեմա $2$}-ում) $\tau\times \tau$ տոպոլո\-գիայով։ Այս տոպոլոգիայի համար բազա է ծառայում տորի (որպես մակերևույթի) վրա բոլոր կորագիծ անեզր ուղղանկ\-յուն\-ների բազմությունը, որոնք սահմանափակ\-ված են որևէ երկու միջօրեականներով և որևէ երկու զուգահեռականներով (տես գծագիրը)։
\begin{center}
\includegraphics[scale=0.3]{images/id33.jpg}
\end{center}
\end{example} %օրինակ 2
\par Ինչպես \hyperref[օրինակ 1]{օրինակ 1}-ում, նույնանման դատողություններով, կարելի է ցույց տալ, որ տորի $\tau\times \tau$ տոպոլոգիան և $\mathbb{R}^3$-ի սովորական մետրական տոպոլոգիայից տորի վրա մակածված տոպոլոգիան նույնն են։
\par Այսպիսով տորը, որպես տոպոլոգիական տարածություն, կարող է ներկայացվել երեք համարժեք եղանակներով․
\par ա) մակերևույթ, որն ստացվում է շրջանագիծը իրեն չհատվող ուղղի շուրջ $\R^3$ տարածության մեջ պտտումով, իսկ նրա տոպոլոգիան $\R^3$-ից մակածված տոպո\-լո\-գիան է;
\par բ) որպես ուղղանկյան ֆակտոր-բազմություն համապատասխան ֆակտոր տո\-պո\-լոգիայով;
\par գ) որպես երկու շրջանագծերի $S\times S$ ուղիղ արտադրյալ $\tau\times \tau$ տո\-պ\-ոլոգիայով։
\begin{definition}
Դիցուք $(X,\tau)$-ն որևէ տոպոլոգիական տարածություն է, իսկ $I$-ն $[0,1]$ հատվածն է $\mathbb{R}$ թվային ուղղի սովորական տոպոլոգիայից մակածված տոպոլո\-գիա\-յով։ Ապա $(X\times I,\tau \times \tau)$ տոպոլոգիական տարածությունը կոչվում է $X$ \textbf{հիմքով գլան}։ Հիշեցնենք, որ $X\times I$ արտադրյալը որպես բազմություն քննարկվել է \hyperlink{sec:2}{թեմա $2$}-ում (տես օրինակ $4$-ը)։ Այժմ մենք $X\times I$-ն վերածեցինք տոպոլոգիական տարածության։
\end{definition}
Սահմանվում է նաև \textbf{$\boldsymbol{X}$ հիմքով և $\boldsymbol{y}$ գագաթով $\boldsymbol{CX}$ կոնը}՝ որպես $X\times I$ գլանի ֆակտոր տարածություն ըստ հետևյալ համարժեքության հարաբերության․ $(x,t)\sim (x,t)$, երբ $1>t\geq 0$, իսկ $x\in X$ կամայական կետ է, և $(x_1,1)\sim (x_2,1)$ ցանկացած $x_1,x_2\in X$ կետերի դեպքում։

Պատկերավոր ասած, $X\times I$ գլանի վերին $X\times \{1\}$ հիմքի բոլոր $(x,1)$ կետերը «սոսնձվում են» միմյանց հետ՝ առաջացնելով $CX$ կոնի $y$ գագաթը։

Որպես օգտակար խնդիր՝ ընթերցողին առաջարկում ենք նկարագրել կոնի $y$ գագաթի բաց շրջակայքերը $CX=\faktor{X\times I}{\sim} $ ֆակտոր տարածությունում:
\subsubsection*{Ուղիղ արտադրյալի հատկություններ}
\begin{theorem}
\label{14:թեորեմ 1}
Կամայական $(X,\tau)$ և $(Y,\sigma)$ տոպոլոգիական տարածությունների դեպքում $(X\times Y,\tau\times  \sigma)$ ուղիղ արտադրյալի 
\densequation
\[ P_X:X\times Y \rightarrow X,\ P_X(x,y )=x \quad \textrm{ և }\quad P_Y:X\times Y\rightarrow Y,\ P_Y (x,y)=y\]
կանոնական պրոյեկցիաները բաց արտապատկերումներ են։ Բացի այդ $\tau\times \sigma$-ն ամենաթույլ տոպոլոգիան է $X\times Y$ բազմության վրա, որի դեպքում $P_X$ և $P_Y$ պրոյեկ\-ցիա\-ները անընդհատ են։
\end{theorem} %թեորեմ 1

\begin{proof}
Եթե $U\in \tau,\ V\in \sigma$, ապա պարզ է, որ $P_X^{-1} (U)=U\times Y$ և $P_Y^{-1} (V)=X\times V$ ենթաբազմությունները բաց են $X\times Y$-ում, ուստի $P_X$-ը և $P_Y$-ը անընդհատ են։ Այժմ դիտարկենք կամայական $U=\bigcup\limits_{i,j}(U_i\times V_j),\ U_i\in \tau,\ V_j\in \sigma$ բաց բազմություն $X\times Y$-ում։ Ունենք՝
\densequation
\[ P_X (U)=P_X \bigg(\mathsmaller{\bigcup}\limits_{i,j} U_i\times V_j\bigg)=\mathsmaller{\bigcup}\limits_{i,j}P_X (U_i\times V_j)=\mathsmaller{\bigcup}\limits_i U_i\in \tau\]
\densequation
\[ P_Y (U)=P_Y\bigg(\mathsmaller{\bigcup}\limits_{i,j}U_i\times V_j\bigg)=\mathsmaller{\bigcup}\limits_{i,j}P_Y(U_i\times V_j)=\mathsmaller{\bigcup}\limits_j V_j\in \sigma \]
ուստի $P_X$-ը և $P_Y$-ը բաց արտապատկերումներ են։

\par Ապացուցենք թեորեմի երկրորդ մասը։ Դիցուք $\varkappa$-ն որևէ այնպիսի տոպոլոգիա է $X\times Y$ բազմության վրա, որ $P_X$ և $P_Y$ կանոնական պրոյեկցիաները անընդհատ են։ Նշանակում է՝ կամայական $U \in \tau$ և $V\in \sigma$ ենթաբազմությունների $P_X^{-1}(U)=U\times Y$ և $P_Y^{-1}(V)=X\times V$ նախակերպարները բաց բազմություններ են $X\times Y$-ում։ Ուստի բաց է նաև նրանց $P_X^{-1}(U)\cap P_Y^{-1}(V)=(U\times Y)\cap (X\times V)=U\times V$ հատումը։ Հետևաբար $\varkappa$ տոպոլոգիան իր մեջ պարունակում է $\tau\times \sigma$ տոպոլոգիան։ Ուստի $\tau\times \sigma$-ն ամենաթույլ տոպոլոգիան է $X\times Y$ բազմության վրա, որի դեպքում $P_X$-ը և $P_Y$-ը անընդհատ են։
\end{proof}

\begin{hetevanq1_sa}
 $(X \times Y,\tau\times \sigma)$ տարածությունում $U\times V$ տեսքի ենթա\-բազ\-մությունը բաց է այն և միայն այն դեպքում, երբ $U$-ն բաց է $X$-ում և $V$-ն բաց է $Y$-ում (հիմնավորել)։
\end{hetevanq1_sa}
\begin{theorem}
Դիցուք ունենք տոպոլոգիական տարածությունների $f_1:X_1\rightarrow Y_1,\\ f_2:X_2\rightarrow Y_2$ արտապատկերումներ։ Ապա նրանց $f_1\times f_2:X_1\times X_2\rightarrow Y_1\times Y_2,\\ (f_1\times f_2)(x_1,x_2)=(f_1 (x_1),f_2 (x_2))$ արտադրյալ արտապատկերումն անընդհատ է այն և միայն այն դեպքում, երբ անընդհատ են $f_1$-ը և $f_2$-ը։
\end{theorem}
\begin{proof}
% \renewcommand*{\windowpagestuff}{
% \begin{tikzcd}[row sep=large, column sep = huge]
% X_1\times X_2 \arrow{r}{f_1\times f_2} \arrow{d}[swap]{P_1}    
% & Y_1\times Y_2 \arrow{d}{P_1} \\
% X_1 \arrow{r}{f_1}
% & Y_1 \end{tikzcd}}
ա) Դիցուք $f_1\times f_2$-ը անընդհատ է։ Դիտարկենք հետևյալ կոմու\-տա\-տիվ դիագրամը (տես \hyperlink{sec:2}{թեմա $2$}-ում)։ Դիագրամի կոմուտատիվությունը նշանակում
%\tikzcdset{row sep/normal=\the\baselineskip}.
\begin{minipage}[b]{0.25\linewidth}
\begin{tikzcd}[row sep=large, column sep = huge,every label/.append style={font=\normalsize}]
X_1\times X_2 \arrow{r}{f_1\times f_2} \arrow{d}[swap]{P_{X_1}}    
& Y_1\times Y_2 \arrow{d}{P_{Y_1}} \\
X_1 \arrow{r}{f_1}
& Y_1 \end{tikzcd}
\end{minipage}%   <-- this % is needed
\hspace{0.125\textwidth}
 \begin{minipage}{0.625\linewidth}
  է, որ $P_{Y_1}\circ (f_1\times f_2)$ և $f_1\circ P_{X_1}$ համադրույթները նույն $X_1\times X_2\rightarrow Y_1$ արտապատկերումն են։ Քանի որ $P_{Y_1}\circ (f_1\times f_2)$-ը անընդհատ է, ուստի $f_1\circ P_{X_1}$ անընդհատ է։ Մյուս կողմից՝ ըստ \hyperref[թեորեմ $1$]{թեորեմ $1$}-ի $P_{X_1}$-ը սյուրյեկտիվ
\end{minipage}
   բաց արտապատկերում է, ուստի $f_1$-ը անընդհատ է համաձայն \red{թեմա $11$-ի թեորեմ $4$}-ի։ Նման ձևով ապացուցվում է $f_2$-ի անընդհատությունը։

\par բ) Դիցուք այժմ անընդհատ են $f_1$-ը և $f_2$-ը։ Ցույց տանք, որ $Y_1\times Y_2$-ում բաց կամայական ենթաբազմության նախակերպարը $f_1\times f_2$ արտապատկերման դեպքում բաց ենթաբազմություն է $X_1\times X_2$-ում։ Բավական է դա ցույց տալ $Y_1\times Y_2$ տարա\-ծության տոպոլոգիայի բազային $V_1\times V_2$ տարրերի համար (ինչո՞ւ)։ Քանի որ \\ ${(f_1\times f_2)^{-1}} (V_1\times V_2)=f_1^{-1} (V_1 )\times f_2^{-1} (V_2), $ ուստի $(f_1\times f_2)^{-1} (V_1\times V_2)$ ենթաբազմությունը բաց է $X_1\times X_2\textrm{-ում}$:
\end{proof}

\begin{theorem}
Ունենք տոպոլոգիական տարածությունների $f_1:X\rightarrow Y_1,\ {f_2:X\rightarrow Y_2}$ արտապատկերումներ։ Ապա նրանցով որոշված (տես թեմա $2$-ում)
\densequation
\[ (f_1,f_2):X\rightarrow Y_1\times Y_2,\quad (f_1,f_2)(x)=(f_1(x),f_2(x)), \quad x\in X \]
անկյունագծային արտապատկերումն անընդհատ է այն և միայն այն դեպքում, երբ անընդհատ են $f_1$-ը և $f_2$-ը։
\end{theorem}
\begin{proof} 
    ա) Դիցուք $(f_1,f_2)$-ը անընդհատ է։ Դիտարկենք $P_{Y_1}\circ (f_1,f_2):{X\rightarrow Y_1}$ համադրույթը։ Ունենք՝ 
    \[\densequation P_{Y_1}\circ (f_1,f_2)(x)=P_{Y_1}((f_1,f_2)(x))=P_{Y_1}(f_1(x),f_2(x))=f_1(x),\quad \forall x\in X։\] 
    Նշանակում է $P_{Y_1}\circ (f_1,f_2)=f_1$, և նման ձևով $P_{Y_2}\circ(f_1,f_2)=f_2$: Ուստի $f_1$-ը և $f_2$-ը անընդհատ են, որպես անընդհատ արտապատկերումների համադրույթներ։
    \par բ) Դիցուք անընդհատ են $f_1$-ը և $f_2$-ը։ Ցույց տանք, որ $(f_1,f_2)$-ը անընդհատ է։ Բավական է ցույց տալ, որ $\forall V_1\times V_2\subset Y_1\times Y_2$ բաց ենթաբազմության համար $(f_1,f_2)^{-1} (V_1\times V_2)$-ը բաց է $X$-ում։ Ունենք՝
    \begin{equation*}
    \begin{aligned}
    (f_1,f_2)^{-1} (V_1\times V_2)&=(f_1,f_2)^{-1} ((V_1\times Y_2)\cap (Y_1\times V_2))=(f_1,f_2)^{-1} (P_{Y_1}^{-1} (V_1)\cap P_{Y_2}^{-1}(V_2))\\ &=(f_1,f_2)^{-1}(P_{Y_1}^{-1} (V_1))\cap(f_1,f_2)^{-1}(P_{Y_2}^{-1} (V_2))\\&=(P_{Y_1}\circ(f_1,f_2))^{-1}(V_1)\cap (P_{Y_2}\circ(f_1,f_2))^{-1} (V_2)=f_1^{-1} (V_1)\cap f_2^{-1} (V_2),
    \end{aligned}
    \end{equation*}
    որը բաց է որպես $X$-ում բաց ենթաբազմությունների հատում։
\par Ուստի $(f_1,f_2)$-ը անընդհատ է:
\end{proof}
\begin{theorem}
Ունենք $X,\ Y$ տոպոլոգիական տարածություններ, նրանցում ${A\subset X},\\ B\subset Y$ ենթաբազմություններ։ Դիտարկենք $A\times B\subset X\times Y$ ենթաբազմությունը։ Ապա 
\begin{enumerate}
    \item $\widebar{A\times B}={\widebar{A} \times \widebar {B}}$,
    \item $\inter(A\times B)=\inter A\times \inter B$:
\end{enumerate}
\end{theorem}
\begin{proof}
Նկատենք, որ $1$-ում միևնույն գծիկով մենք նշանակում ենք փակ\-ման գործողությունները երեք տարբեր $X,\ Y,\ X\times Y$ տարածություններում։ Նույնը վերաբերում է ներքնամասի $\inter$ սիմվոլին $2$-ում։ Նախ ապացուցենք $1$-ը։ Դիցուք $(x_0,y_0)\in \widebar{A\times B}$: Կիրառելով \red{թեմա $10$-ի թեորեմ $5$-ը} $P_X:X\times Y\rightarrow X$ (անընդհատ) պրոյեկցիայի և $A\times B\subset X\times Y$ ենթաբազմության նկատմամբ, կունենանք՝ \densequation
\[P_X (\widebar{A\times B})\subset \widebar{P_X (A\times B)}=\widebar{A}\ \Rightarrow\ P_X(x_0,y_0)\in \widebar{A}\ \Rightarrow\ x_0\in \widebar{A}։\]
Նման ձևով $y_0\in \widebar{B}\ \Rightarrow\ (x_0,y_0)\in \widebar{A} \times \widebar{B}$: Ուստի $\widebar{A\times B}\subset \widebar{A}\times \widebar{B}$: Այժմ հակառակը․ դիցուք $(x_0,y_0)\in A \times B$: Ցույց տանք, որ $(x_0,y_0)$-ն հպման կետ է $A\times B$-ի համար։ Դիտարկենք $(x_0,y_0)$-ի որևէ $W\subset X\times Y$ շրջակայք։ Գոյություն ունի $(x_0,y_0)$-ի $U\times V$ տեսքի բաց շրջակայք, որ $(x_0,y_0 )\in U\times V \subset W$: Քանի որ $x_0\in \widebar A$ և $x_0\in U$, ուստի $U\cap A\neq \varnothing$: Նման ձևով՝ $V\cap B\neq \varnothing$: Հետևաբար՝ $(U\times V)\cap (A\times B)\neq \varnothing\ \Rightarrow\ $
$\ \Rightarrow\ W\cap (A\times B)\neq \varnothing\ \Rightarrow\ (x_0,y_0)\in A\times B\ \Rightarrow\ \widebar A \times \widebar B \subset \widebar{(A\times B)}$։

\par Ապացուցենք $2$-ը։ Ունենք՝ $\inter(A\times B)\subset A\times B \Rightarrow P_X (\inter(A\times B))\subset P_X (A\times B)=A$: Ըստ թեորեմ $1$-ի՝ $P_X(\inter(A\times B))$-ն բաց է $X$-ում, ուստի $P_X(\inter(A\times B))\subset \inter A\ \Rightarrow \inter (A\times B)\subset P_X^{-1} (\inter A)=(\inter A)\times Y$: Նման ձևով $\inter (A\times B)\subset P_Y^{-1}(\inter B)=X\times (\inter B)$։ Հետևաբար $\inter (A\times B)\subset ((\inter A)\times Y)\cap (X\times (\inter B))=\inter A\times \inter B$: Հակառակ պնդումը՝ $\inter A\times \inter B\subset \inter (A\times B)$ ապացուցվում է հեշտությամբ և թողնվում է ընթերցողին։
\end{proof}
\begin{hetevanq4}
    1. $A\times B$-ն փակ ենթաբազմություն է $X\times Y$-ում այն և միայն այն դեպքում, երբ $A$-ն փակ է $X$-ում և $B$-ն փակ է $Y$-ում։
    \par 2. $A\times B$-ն բաց ենթաբազմություն է $X\times Y$-ում այն և միայն այն դեպքում, երբ $A$-ն բաց է $X$-ում և $B$-ն բաց է $Y$-ում (սա մենք արդեն ստացել էինք նաև որպես հետևանք \hyperref[թեորեմ $1$]{թեորեմ $1$}-ից)։
\end{hetevanq4}
\par Գոյություն ունեն մի շարք տոպոլոգիական հատկություններ, որ եթե $X$ և $Y$ տարածություններն օժտված են դրանցից մեկով, ապա նրանց $X\times Y$ արտադրյալը նույնպես բավարարում է այդ հատկությանը և հակառակը։ Այդպիսի հատկություններ են օրինակ, անջատելիության $\textrm{T}_0,\textrm{T}_1,\textrm{T}_2$ աքսիոմները, կապակցվածությունը, կոմ\-պակտու\-թյունը։ Ապացուցենք մի այդպիսի թեորեմ։
\begin{theorem}
$X\times Y$ տարածությունը հաուսդորֆյան է այն և միայն այն դեպքում, երբ հաուսդորֆյան տարածություններ են $X$-ը և $Y$-ը։ 
\end{theorem}
\begin{proof}
Ենթադրենք $X\times Y$-ը հաուսդորֆյան է և ցույց տանք, որ $X$-ը (նման ձևով՝ $Y$-ը) հաուսդորֆյան է։ Դիցուք $x_1,x_2\in X$ և $x_1\neq x_2$: Սևեռենք որևէ $y_0\in Y$ կետ և դիտարկենք $(x_1,y_0)$ և $(x_2,y_0)$ կետերը $X\times Y$-ում։ Ըստ պայմանի, քանի որ $(x_1,y_0)\neq (x_2,y_0)$, գոյություն ունեն այդ կետերի չհատվող, բաց $W_1,W_2$ շրջակայքեր $X\times Y$-ում՝ $(x_1,y_0)\in W_1,\ (x_2,y_0)\in W_2$: Համաձայն ուղիղ արտադրյալի տոպոլոգիայի սահմանման, գոյություն ունեն այդ կետերի համապատասխանաբար $U_1\times V_1$ և $U_2\times V_2$ բաց շրջակայքեր $X\times Y$-ում, որ $(x_1,y_0)\in U_1\times V_1\subset W_1,\ (x_2,y_0)\in U_2\times V_2\subset W_2$։ Քանի որ $y_0\in V_1$ և $y_0\in V_2$, ուստի $V_1\cap V_2\neq \varnothing$: Ըստ պայմանի $W_1\cap W_2=\varnothing$, որից հետևում է, որ նաև $(U_1\times V_1 )\cap (U_2\times V_2 )=(U_1\cap U_2 )\times (V_1\cap V_2 )=\varnothing$: Դրանից էլ հետևում է, որ $U_1\cap U_2=\varnothing$: Այսպիսով՝ $X$-ի կամայական $x_1\neq x_2$ կետեր ունեն $U_1,U_2$ չհատվող շրջակայքեր $\Rightarrow X$-ը հաուսդորֆյան է: Նման ձևով ապացուցվում է, որ $Y$-ը ևս հաուսդորֆյան է:
\par Այժմ հակառակը․ ենթադրենք՝ $X$-ը և $Y$-ը հաուդորֆյան տարածություններ են, ցույց տանք, որ $X\times Y$-ը հաուսդորֆյան է։ Դիցուք $(x_1,y_1),\ (x_2,y_2)\in X\times Y $ և $(x_1,y_1)\neq (x_2,y_2)$: Որոշակիության համար ենթադրենք $x_1\neq x_2$: Ըստ պայմանի $X\textrm{-ում}$ գոյություն ունեն $x_1, x_2$ կետերի չհատվող, բաց $U_1,U_2$ շրջակայքեր՝ $x_1\in U_1,\ x_2\in U_2$: Պարզ է, որ $(x_1,y_1)\in U_1\times Y,\ (x_2,y_2)\in U_2\times Y$, իսկ $U_1\times Y$-ը և $U_2\times Y$-ը այդ կետերի չհատվող բաց շրջակայքեր են $X\times Y$-ում
\end{proof}
\begin{note} Վերը բերված 1-5 թեորեմներն ունեն իրենց անալոգները կա\-մա\-յա\-կան վերջավոր քանակով $X_1,X_2,\dots,X_n$ տոպոլոգիական տարածությունների դեպքում (ձևակերպումներն ու ապացուցումները թողնում ենք ընթերցողին):
\end{note}
Այժմ քննարկենք հետևյալ հարցը․ կարելի՞ է արդյոք $X$ (կամ $Y$) տարածությունը ինչ-որ իմաստով դիտարկել որպես $X\times Y$ տարածության ենթատարածություն։
\par Դիցուք ունենք $X$ և $Y$ տոպոլոգիական տարածություններ։ Սևեռենք որևէ $y_0\in Y$ կետ և դիտարկենք $X\times Y$-ի $X\times y_0$ ենթատարածությունը։ 
\begin{theorem} 
$X\times \{y_0\}$ ենթատարածությունը հոմեոմորֆ է $X$-ին։
\end{theorem}
\begin{proof}
Սահմանենք $h:X\rightarrow X\times \{y_0\}$ արտապատկերում՝ $h(x)=(x,y_0),\\ x\in X$ բանաձևով։ Ցույց տանք, որ $h$-ը հոմեոմորֆիզմ է։ Պարզ է, որ $h$-ը բիյեկտիվ արտապատկերում է, ուստի համաձայն \red{թեմա $11$-ում թեորեմ $6$}-ի բավական է ցույց տալ, որ $h$-ը բաց արտապատկերում է։ Այդ նպատակով նախ նկարագրենք ${X\times \{y_0\}}$ տարածության տոպոլոգիան։ Դիտարկենք որևէ $U\times V$ բաց ենթաբազմություն ${X\times Y\textrm{-ում}}$ ($U$-ն բաց է $X$-ում, $V$-ն՝ $Y$-ում) և նրա հատումը $X\times \{y_0\}$-ի հետ (հիշենք, որ $X\times Y\textrm{-ից}$ $X\times \{y_0\}$-ի վրա մակածված տոպոլոգիայում բաց են միայն $(U\times V)\cap (X\times \{y_0\})$ տեսքի ենթաբազմությունները և նրանց միավորումները)։ Ունենք՝ \[(U\times V)\cap (X\times \{y_0\})=(U\cap X)\times (V\cap \{y_0\})=
\begin{sqcases}
\varnothing, \textrm{ երբ } y_0\notin V, \\ U\times \{y_0\}, \textrm{ երբ } y_0\in V։\end{sqcases} \]
\par Այստեղից, վերցնելով $V=Y$, ստանում ենք, որ $h(U)=U\times \{y_0\}$ ենթաբազմությունը բաց բազմություն է $X\times \{y_0\}$ տարածությունում:
\par Բացի այդ $h$-ը անընդհատ է, քանի որ $h^{-1}(U\times \{y_0\})=U$ բաց է $X$-ում։
\end{proof}
\par Նման ձևով ապացուցվում է, որ սևեռված $x_0 \in X$ կետի դեպքում $X \times Y$-ի $X \times \{y_0\}$ ենթատարածությունը հոմեոմորֆ է $Y$-ին։
\subsubsection*{Որոշ երկրաչափական հասկացություններ $\mathbb{R}^n$ էվկլիդյան տարածությունում:}
Ստորև $n$ չափականության $\mathbb{R}^n=\underbrace{\mathbb{R}\times \mathbb{R}\times \dots\times \mathbb{R}}_\text{$n$}$ էվկլիդյան տարածությունը կդիտարկվի ուղիղ արտադրյալի տոպոլոգիայով (յուրաքանչյուր $\mathbb{R}$ արտադրիչը թվա\-յին ուղիղն է սովորական տոպոլոգիայով):
\begin{definition}
Երկու՝ $A=(a_1,a_2,\dots, a_n)$ և $B=(b_1,b_2,\dots, b_n)$ կետերով անցնող \textbf{ուղիղ $\mathbb{R}^n$ տարածությունում} կոչվում է $\{(1-t)A+tB,\ t\in \mathbb{R}\}$ կետերի բազմությունը:
\par Դիտարկելով այս ուղղի $X=(x_1,x_2,\dots x_n)$ փոփոխական կետ՝ ստանում ենք ուղղի այսպես կոչված պարամետրական հավասարումները՝ \densequation
\[ x_i=(1-t)a_i+tb_i,\quad i=1,2,\dots,n \]
\par Սահմանափակելով $t$ պարամետրի որոշման տիրույթը $[0,1]$ հատվածով՝ ստանում ենք ուղղի մաս, որը կոչվում է $A$ և $B$ ծայրակետերով \textbf{հատված $\mathbb{R}^n$-ում} (կնշանակենք $[A,B]$):
\par Մասնավոր՝ $n=1$ դեպքում ունենք $A=(a),\ B=(b)$ կետեր $\mathbb{R}$-ում, իսկ $[A,B]$ հատվածը կարող է նույնացվել թվային ուղղի $(1-t)a+tb,\ t\in [0,1]$ ենթաբազմության հետ: Կարճ կնշանակենք այն $[a,b]$: Այսպիսով ունենք $f:[a,b]\rightarrow \mathbb{R}$ ներդրում ($f$-ը $[a,b]$-ի նույնական հոմեոմորֆիզմն է իր կերպարի վրա):
\end{definition} 
\par Դիցուք այժմ ունենք $[a_1,b_1],\ [a_2,b_2],\dots,[a_n,b_n]$ հատվածներ $\mathbb{R}$-ում, դիտարկենք $f_i:[a_i,b_i]\rightarrow \mathbb{R}$ ներդրումները և դրանց $f_1\times f_2\times \dots\times f_n$ ուղիղ արտադրյալը:
\begin{definition}
\textbf{Ուղղանկյունանիստ $\mathbb{R}^n$-ում} կոչվում է $[a_1,b_1 ]\times [a_2,b_2 ]\times \dots,\times [a_n,b_n ]$ ուղիղ արտադրյալի կերպարը $f_1\times f_2\times \dots\times f_n:[a_1,b_1]\times [a_2,b_2]\times \dots,\times [a_n,b_n]\rightarrow \underbrace{\mathbb{R}\times \mathbb{R}\times \dots\times \mathbb{R}}_\text{$n$}=\mathbb{R}^n$ արտապատկերման դեպքում:
\par $A=(a_1,a_2,\dots, a_n)$ և $B=(b_1,b_2,\dots, b_n)$ կետերը կանվանենք ուղղանկյունանիստի \textbf{հակադիր գագաթներ}, իսկ $[A,B]$  հատվածը՝ ուղղանկյունանիստի \textbf{անկյունագիծ}:
\end{definition} 
\par Այս սահմանումը կարիք ունի լրացման: Այդ նպատակով նկատենք, որ թվային ուղղի $[a,b]$ հատված սահմանելիս մենք չենք պահանջել, որ անպայման $a<b$: Նկա\-տենք նաև, որ $[a,b]$ և $[b,a]$ հատվածները նույնն են: Ուստի, $[a_1,b_1 ]\times [a_2,b_2 ]\times \dots\times [a_n,b_n ]$ արտադրյալում մի քանի $[a_i,b_i]$ արտադրիչներ փոխարինելով $[b_i,a_i]$ արտադրիչներով և կիրառելով ուղղանկյունանիստի սահմանումը, ստանում ենք, որ նոր և հին ուղղանկյունանիստերը նույնն են (հիմնավորել): Այստեղից հետևում է, որ ուղղանկյունանիստն ունի $2^n$ հատ գագաթ և $2^{n-1}$ հատ անկյունագիծ: Պարզաբա\-նելու համար, թե որոնք են այդ գագաթներն ու անկյունագծերը, դի\-տար\-կենք սիմվոլիկ $(x_1,x_2,\dots,x_n)$ հաջորդականություն և նրանում որոշ $x_i$ սիմվոլներ փոխարի\-նենք $a_i$-ներով, իսկ մնացածները՝ $b_i$-ներով։ Ստացված հաջորդականությունը որո\-շում է ուղղանկյունանիստի գագաթ։ Իսկ եթե փոխարինումները կատարենք հակա\-ռակ կարգով, կստանանք նախկինին հակադիր գագաթ և դրանց միացնող անկյունա\-գիծ։ Այստեղից հետևում է, որ ուղղանկյունանիստի բոլոր անկյունագծերն ունեն միև\-նույն՝ $\sqrt{\sum\limits_{i=1}^{n}(a_i-b_i)^2}$ երկարությունը: Բացի այդ $M=\left(\dfrac{a_1+b_1}{2},\dfrac{a_2+b_2}{2},\dots,\dfrac{a_n+b_n}{2}\right)$ կետը պատկանում է բոլոր անկյունագծերին։ Այսպիսով, ուղղանկյունիստի անկ\-յու\-նագծերը հատվում են մի կետում և այդ կետով կիսվում են:
\par Հեշտությամբ ապացուցվում է, որ $(f_1\times f_2\times \dots\times f_n)([a_1,b_1 ]\times [a_2,b_2 ]\times \dots \times [a_n,b_n ])$ ուղղանկյունանիստը պարունակվում է $M$ կենտրոնով և $R=\max\limits_{1\leq i\leq n}\dfrac{a_i+b_i}{2}$ շառավղով $\sum\limits_{i=1}^{n}\left(x_i-\max\limits_{1\leq i\leq n} \dfrac{a_i+b_i}{2}\right)^2\leq R^2$  եզրով գնդում, ինչպես նաև
\setlength{\belowdisplayskip}{5pt}
%\setlength{\belowdisplayshortskip}{3pt}
\setlength{\abovedisplayskip}{5pt}
% \setlength{\abovedisplayshortskip}{3pt}
%\densequation
\[(f_1\times f_2\times \dots\times f_n)\big((a_1,b_1)\times (a_2,b_2)\times \dots\times (a_n,b_n)\big)=f_1 (a_1,b_1)\times f_2 (a_2,b_2)\times \dots\times f_n (a_n,b_n)\] բաց բազմությունը պարունակվում է $\sum\limits_{i=1}^{n}\left(x_i-\max\limits_{1\leq i\leq n} \dfrac{a_i+b_i}{2}\right)^2<R^2$  անեզր գնդում: Ճիշտ է նաև հակառակը․ յուրաքանչյուր $N(c_1,c_2,\dots,c_n)$ կենտրոնով և $R$ շառավղով $\sum\limits_{i=1}^{n}(x_i-c_i)^2\leq R^2$ եզրով գունդ պարունակվում է $f_1 ([c_1-R,c_1+R])\times f_2 {([c_2-R,c_2+R])}\times \times \dots \times f_n([c_n-R,c_n+R])$ փակ ուղղանկյունանիստում, իսկ $\sum\limits_{i=1}^{n}(x_i-c_i)^2<R^2$ անեզր գունդը պարունակվում է $f_1((c_1-R,c_1+R))\times f_2 ((c_2-R,c_2+R))\times f_n ((c_n-R,c_n+R))$ բաց ուղղանկյունանիստում:
%\abovedisplayskip=0.0pt plus 3.0pt
%\belowdisplayshortskip=6.5pt plus 3.5pt minus 3.0pt
\belowdisplayskip=12.0pt plus 3.0pt minus 7.0pt
\abovedisplayskip=12.0pt plus 3.0pt minus 7.0pt
\par Այստեղից ստանում ենք մի քանի կարևոր հետևանք:
\begin{hetevanq}
$\mathbb{R}^n$-ի ուղիղ արտադրյալի տոպոլոգիական և մետրական տոպոլոգիան նույնն են:
\end{hetevanq}
\begin{definition}
$\mathbb{R}^n$-ի որևէ ենթաբազմություն կոչվում է \textbf{սահմանափակ ենթա\-բազ\-մու\-թյուն}, եթե այն պարունակվում է որևէ ուղղանկյունանիստում:
\end{definition}
\begin{hetevanq}
Ենթաբազմության սահմանափակությունը համարժեք է նրան, որ այն պարունակվի որևէ գնդում:
\end{hetevanq}
\begin{definition}
$\mathbb{R}^n$-ի ենթաբազմությունը կոչվում է \textbf{ուռուցիկ ենթաբազմություն}, եթե այն իր կամայական երկու կետերի հետ միասին պարունակում է դրանց միացնող հատվածը:
\end{definition}

\par Վերոհիշյալներից հետևում է, որ $\mathbb{R}^n$-ը ինքը ուռուցիկ է (բայց կարող է պարու\-նա\-կել նաև ոչ ուռուցիկ ենթաբազմություններ): Նկատենք նաև, որ ուռուցիկությունը տո\-պո\-լո\-գիական հատկություն չէ (բերել օրինակ):

\end{document}