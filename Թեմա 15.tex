\documentclass[./main.tex]{subfiles}

\begin{document}
\onehalfspacing
\section{Կապակցված տարածություն, կապակցված ենթաբազմություն։ Թեորեմներ, կապակցված ենթաբազմությունների միավորման և կապակցված ենթաբազմության փակման մասին։ Կապակցվածության բաղադրիչ, ոչ հոմեոմորֆ տարածությունների օրինակներ։}\label{sec:15}

% 1 +
\par Եթե ունենք $(X, \tau)$ և $(Y, \sigma)$ տոպոլոգիական տարածություններ, ընդ որում $X \cap Y =\varnothing$, ապա $X \cup Y$ բազմությունը կարելի է վերածել տոպոլոգիական տարածության. կհամարենք $U \subset X \cup Y$ ենթաբազմությունը բաց ենթաբազմություն $X \cup Y$-ում այն և միայն այն դեպքում, երբ $U \cap X$-ը և $U \cap Y$-ը բաց են համապատասխանաբար $X$-ում և $Y$-ում։ Տոպոլոգիայի 1-3 աքսիոմները ստուգվում են հեշտությամբ։ Նկատենք, որ ստացված տոպոլոգիական տարածությունում $X$ և $Y$ ենթաբազմությունները ոչ դատարկ, չհատվող, միաժամանակ բաց և փակ ենթաբազմություններ են։ Այս տարածությունը կոչվում է $X$ և $Y$ տարածությունների \textbf{չկապակցված միավորում}։

% 2 + 
\begin{definition}
Տոպոլոգիական տարածությունը կոչվում է \textbf{կապակցված տարա\-ծու\-թյուն}, եթե այն չի կարող ներկայացվել որպես իր երկու ոչ դատարկ, չհատվող, բաց ենթա\-բազ\-մութ\-յուն\-ների միավորման տեսքով։
% 3 +
\par Համարժեք ձևակերպում․ տարածությունը կապակցված է, եթե այն չի կարող ներկայացվել որպես իր երկու ոչ դատարկ, չհատվող փակ ենթաբազմությունների միավորման տեսքով։
% 4 +
\par Եվս մի համարժեք ձևակերպում. $X$ տոպ. տարածությունը կապակցված է, եթե $X$-ում գոյություն ունի միայն մի ոչ դատարկ, միաժամանակ բաց և փակ ենթա\-բազ\-մութ\-յուն՝ ինքը $X$-ը։
\end{definition}
% 5 +
\begin{definition}
 $X$ տոպոլոգիական տարածության $Y$ ենթաբազմությունը կոչվում է \textbf{$\boldsymbol{X}$-ի կապակցված ենթաբազմություն}, եթե $Y$-ը կապակցված տարածություն է $X$-ից մակածված տոպոլոգիայով։
\end{definition}
% 6 +
\begin{example}
\label{15:օրինակ 1}
ա) Ցանկացած անտիդիսկրետ տարածություն կապակցված տարա\-ծութ\-յուն է։ բ) Մեկից ավելի կետեր պարունակող ամեն մի դիսկրետ տարածություն կապակցված չէ։ գ) $(\R, \textrm{սովոր.})$ թվային ուղղի $(-\infty, 0)\cup(0, +\infty)$, $(0,1)\cup(1,2)$ և $\Z=\{0, \pm 1, \pm2, \dots\}$ ենթաբազմությունները կապակցված չեն (ինչո՞ւ)։
\end{example}

% 7
\begin{theorem}
\label{15:թեորեմ1}
Թվային ուղղի $[a,b]$ հատվածը կապակցված տարածություն է ($\R$-ի սովորական մետրական տոպոլոգիայից մակածված տոպոլոգիայով)։
\end{theorem}
% 8
\begin{proof}
Դիցուք $[a,b]=U\cup V$, որտեղ $U$-ն և $V$-ն չհատվող, ոչ դատարկ, բաց (հետևաբար նաև փակ) ենթաբազմություններ են։ Որոշակիության համար ենթադրենք $a\in U$ և դիտար\-կենք $U$-ի ենթաբազմությունը կազմված $U$-ի այն բոլոր տարրերից, որոնք փոքր են $V$-ի բոլոր տարրերից՝ $W=\{u\in U\mid u < v,\, \forall v \in V \}$: Քանի որ $a\in W$, ուստի $W \neq \varnothing$: Նշանակենք $h$-ով $W$ ենթաբազմության ճշգրիտ վերին եզրը՝ $h=\sup W$: Քանի որ $h$-ը հպման կետ է $W$-ի համար, և $W \subset U$, ուստի $h$-ը հպման կետ է նաև $U$-ի համար։ Հետևաբար $h\in \widebar{U} = U$, ուրեմն $h\in W$ (ինչո՞ւ)։ Մյուս կողմից՝ ցանկացած $\varepsilon > 0$ թվի համար $(h-\varepsilon,h+\varepsilon) \cap V \neq \varnothing$։ Հակառակ դեպքում, եթե $(h-\varepsilon_0, h+\varepsilon_0 ) \cap V=\varnothing$ որևէ $\varepsilon_0$-ի համար, ապա կունենանք $[h,h+\varepsilon_0)\subset U$։ Իսկ դրանից կհետևի, որ $h+\dfrac{\varepsilon_0}{2}\in W$, և ուրեմն $h \neq \sup W$։

%$(h-\varepsilon_0,h+\varepsilon_0) \subset U_0\ \Rightarrow\ \left(h+\dfrac{\varepsilon_0}{2}\right) \in W\ \Rightarrow\ h \neq \sup W$):
% 9 +
\par Նշանակում է՝ $h$-ը հպման կետ է $V$-ի համար, ուստի $h \in \overline{V} = V$: Այսպիսով ստացանք $h \in U \cap V$ (հակասություն):
\end{proof}

% 10 + 
\begin{theorem}
\label{15:թեորեմ 2}
Կապակցված տարածության կերպարը անընդհատ արտապատ\-կեր\-ման դեպքում կապակցված տարածություն է։
\end{theorem}
% 11 +
\begin{proof}
Դիցուք $X$-ը կապակցված է, $f:X \rightarrow Y$ անընդհատ է և $f(X)=Y$: Ենթադրենք $Y=U \cup V$, որտեղ $U$-ն և $V$-ն ոչ դատարկ, չհատվող, բաց ենթա\-բազ\-մութ\-յուն\-ներ են $Y$-ում։ Պարզ է, որ $f^{-1}(U),\ f^{-1}(V)$-ն ոչ դատարկ, բաց ենթա\-բազ\-մութ\-յուն\-ներ են $X$-ում և $f^{-1}(U) \cup f^{-1}(V) = X,\ f^{-1}(U) \cap f^{-1}(V)= \varnothing$: Ստացվեց, որ $X$-ը կա\-պակցված չէ (հակասություն):
\end{proof}
% 12 +
\begin{hetevanq}
Կապակցվածությունը տոպոլոգիական հատկություն է (ինչո՞ւ)։
\end{hetevanq}
% 13 +
\begin{theorem}
\label{15:թեորեմ 3}
Դիցուք ունենք $X$ տարածության $Y_i \subset X,\ i \in I$ կապակցված ենթա\-բազ\-մութ\-յուն\-ներ։ Եթե նրանց հատումը դատարկ չէ, ապա նրանց $Y=\bigcup\limits_i Y_i$ միա\-վո\-րումը նույնպես կապակցված է։ 
\end{theorem} 

% 14 +
\begin{proof}
Դիցուք $U$-ն ոչ դատարկ, միաժամանակ բաց և փակ ենթա\-բազ\-մութ\-յուն է $Y$-ում։ Ցույց տանք, որ այն համընկնում է $Y$-ի հետ (դրանից կհետևի, որ $Y$-ը կապակցված է)։ Գոյություն ունի $i_0 \in I$, որ $U \cap Y_{i_{_0}} \neq \varnothing$: Պարզ է, որ $U \cap Y_{i_{_0}}$-ն ոչ դատարկ, միաժամանակ բաց և փակ ենթաբազմություն է $Y_{i_{_0}}$-ում, ուստի $U \cap Y_{i_{_0}}=Y_{i_{_0}}$ (շնորհիվ $Y_{i_{_0}}$-ի կապակցվածության)։ Նշանակում է՝ $Y_{i_{_0}} \subset U$: Կամայական $i \neq i_0$ ինդեքսի դեպքում, քանի որ $Y_{i_{_0}} \cap Y_i \neq \varnothing$, ուստի $U \cap Y_{i} \neq \varnothing$: Այժմ, կատարելով վերը բերված դատողությունները արդեն $Y_i$ և $U$ ենթաբազմությունների համար, ստանում ենք, որ $Y_i \subset U$: Հետևաբար $\bigcup\limits_i Y_i \subset U$, և ուրեմն $U=Y$:
\end{proof} 
% 15 +
\begin{theorem}
\label{15:թեորեմ4}
Տոպոլոգիական տարածությունների $X \times Y$ արտադրյալը կապակցված է այն և միայն այն դեպքում, երբ կապակցված են $X$-ը և $Y$-ը։
\end{theorem}

% 16 
\begin{proof}
 ա) Ենթադրենք $X \times Y$-ը կապակցված է։ Քանի որ $P_1:{X \times Y \rightarrow X}$, $P_2:X \times Y \rightarrow Y$ կանոնական պրոյեկցիաները անընդհատ և սյուրյեկտիվ արտա\-պատ\-կերում\-ներ են, ուստի $X$-ը և $Y$-ը կապակցված են համաձայն \hyperref[15:թեորեմ 2]{թեորեմ 2}-ի։
% 17
\par բ) Դիցուք $X$-ը և $Y$-ը կապակցված են։ Կամայական $(x,y) \in X \times Y$ կետի դեպքում $\{x\} \times Y$ և $X \times \{y\}$ տարածությունները, ըստ \red{թեմա 14-ում թեորեմ 6}-ի՝ հոմեոմորֆ են համապատասխանաբար $Y$ և $X$ տարածություններին: Ուստի նրանք կապակցված տարածություններ են՝ ըստ \hyperref[15:թեորեմ 2]{թեորեմ 2}-ի հետևանքի:
% 18 +
\par Այնուհետև, քանի որ $(x,y) \in (\{x\} \times Y) \cap (X \times \{y\})$, ուստի $(\{x\} \times Y) \cup (X\times \{y\})$ միավորումը կապակցված է համաձայն \hyperref[15:թեորեմ 3]{թեորեմ 3-ի}։ Այժմ, սևեռելով որևէ $y_0 \in Y$ կետ, ամեն մի $x \in X$ կետի համար դիտարկենք $Y_x=(\{x\} \times Y) \cup (X\times \{y_0\})$ ենթաբազմությունը $X \times Y$-ում։ Քանի որ $X \times \{y_0\} \subset Y_x$, ուստի $\bigcap\limits_x Y_x$  հատումը դատարկ չէ։ Մյուս կողմից պարզ է, որ $\bigcup\limits_x Y_x$ միավորումը $X \times Y$-ն է։ Հետևաբար $X \times Y$-ը կապակցված է համաձայն \hyperref[15:թեորեմ 3]{թեորեմ 3}-ի։
\end{proof}
% 19 +
\begin{hetevanq_counter}
$\R$ թվային ուղիղը կապակցված է, քանի որ $\R= \bigcup\limits_{n \in \N} [-n;n]$ և $\bigcap\limits_{n \in \N} [-n;n] \neq \varnothing$, իսկ $[-n;n],\ n \in N$ հատվածները կապակցված են ըստ \hyperref[15:թեորեմ1]{թեորեմ 1}-ի:
\end{hetevanq_counter}
% 20 +
\begin{hetevanq_counter}
Կամայական $(x_0,y_0) \in \R^2$ կետի դեպքում $\R^2 \setminus (x_0,y_0)$ տարածու\-թ\-յունը (որպես $\R^2$-ի ենթատարածություն) կապակցված է։
% 21 +
\par Իրոք, $\R^2 \setminus (x_0,y_0)$-ն կարող ենք ներկայացնել որպես $\R^2$-ի չորս բաց՝ $A, B, C, D$ ենթաբազմությունների միավորում՝ $\R^2 \setminus (x_0,y_0 ) = A \cup B \cup C \cup D=$\\
$=\big(\R \times (y_0,+\infty)\big) \cup \big(\R \times (-\infty, y_0)\big) \cup \big((-\infty, x_0 ) \times \R\big) \cup \big((x_0,+\infty) \times \R\big)$։
% 22 +
\par Դրանցից յուրաքանչյուրը կապակցված է որպես երկու կապակցված ենթա\-բազ\-մութ\-յուն\-ների ուղիղ արտադրյալ։ Քանի որ $A \cap C \neq \varnothing$ և $B \cap D \neq \varnothing$ ուստի $A \cup C$ և $B \cup D$ ենթաբազմությունները կապակցված են ըստ \hyperref[15:թեորեմ 3]{թեորեմ 3}-ի։ Նաև ակնհայտ է, որ $(A \cup C) \cap (B \cup D) \neq \varnothing$, ուստի $\R^2 \setminus (x_0,y_0)$ տարածությունը կապակցված է դարձյալ ըստ \hyperref[15:թեորեմ 3]{թեորեմ 3}-ի։
\end{hetevanq_counter}
% 23 +
\par Կապակցվածությունը տոպոլոգիական հատկություն է և թույլ է տալիս որոշ դեպքերում ապացուցել երկու տարածությունների ոչ հոմեոմորֆությունը։
% 24 +
\par Տոպոլոգիայում տեղի ունի հետևյալ նշանավոր թեորեմը (Բրաուեր, 1911թ․)։ 
% 25 +
\begin{theorem}
\label{15:թեորեմ5}
Եթե $n \neq m$, ապա $(\R^n,\textrm{սովոր․ մետր․ տոպ․})$ և $(\R^m, \textrm{սովոր․ մետր․ տոպ․})$ տարածությունները միմյանց հոմեոմորֆ չեն։
\end{theorem} 
% 26 +
\par Ապացույցը ընդհանուր դեպքում բարդ է, իսկ դրա համար անհրաժեշտ գիտելիքը դուրս է մեր դասընթացի շրջանակներից։
% 27 +
\par Ապացուցենք թեորեմը $n=1,\ m=2$ մասնավոր դեպքում։ Ենթադրենք՝ գոյութ\-յուն ունի $h:\R \rightarrow \R^2$ հոմեոմորֆիզմ։ Դիցուք $h(0)=z_0 \in \R^2$։ Հեռացնելով $\R$-ից 0 կետը, իսկ $\R^2$-ից $z_0$ կետը՝ դիտարկենք $\widebar{h}: \R \setminus 0 \rightarrow \R^2 \setminus z_0$ արտապատկերում՝ սահմանելով $\widebar{h}(t)=h(t),\ t \in \R \setminus 0$։ Պարզ է, որ $\widebar{h}$ արտապատկերումը փոխմիարժեք է։ Նրա անընդհատությունը հետևում է $h$-ի անընդհատությունից (հիմնավորել)։
% 28 +
\par Քանի որ $(\R \setminus 0)$-ն կապակցված չէ, իսկ $(\R^2 \setminus z_0)$-ն կապակցված է, ստանում ենք հակասություն \hyperref[15:թեորեմ 2]{թեորեմ 2}-ի հետ:\qed
% 29 +
\begin{theorem}
\label{15:թեորեմ6}
Տոպոլոգիական տարածության կապակցված ենթաբազմության փա\-կումը նորից կապակցված ենթաբազմություն է։ 
\end{theorem} 
% 30 + 
Սա ապացուցելու նպատակով նախ ապացուցենք։
% 31 +
\begin{lemma}
Դիցուք $X$-ը $Y$ տարածության որևէ կապակցված ենթաբազմություն է։ Եթե $y_0 \in Y$ կետը հպման կետ է $X$-ի համար, ապա $\{y_0\} \cup X$-ը $Y$-ի կապակցված ենթաբազմություն է։
\end{lemma}
% 32 + 
\par Այլ կերպ ասած, կապակցված ենթաբազմությանը նրա որևէ հպման կետ ավե\-լաց\-նելիս դարձյալ ստացվում է կապակցված ենթաբազմություն։
%33 +
\begin{proof}
Դիտարկենք այն դեպքը, երբ $y_0 \notin X$: Ենթադրենք $\{y_0\} \cup X= U \cup V$, որտեղ $U$-ն և $V$-ն ոչ դատարկ, չհատվող, բաց ենթաբազմություններ են $\{y_0\} \cup X$-ում ($Y$-ից մակածված տոպոլոգիայում)։ Դիցուք $y_0 \in U$, և ուրեմն $V \subset X$: Նշանակում է $V$-ն միաժամանակ բաց և փակ ենթաբազմություն է $X$ կապակցված ենթա\-բազ\-մութ\-յունում, որից հետևում է, որ $V=X$ և $U=\{y_0\}$: Այսպիսով՝ $y_0$ կետն ունի $U=\{y_0\}$ բաց շրջակայք և $U \cap X=\{y_0\} \cap X=\varnothing$: Սրանից հետևում է, որ $y_0$-ն հպման կետ չէ $X$-ի համար (հակասություն):
\end{proof}
\renewcommand*{\proofname}{\hspace{18pt}\textbf{Ապացուցենք թեորեմ 6-ը։}\nopunct}
% 34 +
\begin{proof}
Դիցուք $X$-ը ինչ-որ տոպոլոգիական տարածության կապակցված ենթաբազմություն է, իսկ $Y$-ը $X$-ի հպման կետերի բազմությունն է՝ $Y=\widebar{X}$։ Կարող ենք $Y$-ը ներկայացնել $Y=\widebar{X} =X \cup Y= \bigcup\limits_{y \in Y} (X \cup \{y\})$ տեսքով։ Յուրաքանչյուր $X \cup \{y\}$ ենթաբազմություն կապակցված է համաձայն լեմմայի։ Քանի որ $\bigcap\limits_{y \in Y} ( X \cup \{y\})$ հատումը դատարկ չէ, ուստի $Y$-ը կապակցված է համաձայն \hyperref[15:թեորեմ 3]{թեորեմ 3}-ի:
\end{proof}
\renewcommand*{\proofname}{\hspace{18pt}\textbf{Ապացուցում։}\nopunct}
% 35 +
\par Ստորև ներմուծվում է կապակցված տարածությունների մի կարևոր բնութագրիչ։
% 36 +
\subsection*{Տոպոլոգիական տարածության կապակցվածության բաղադրիչները։}
% 37 +
\begin{definition}
Տոպոլոգիական տարածության կապակցվածության բաղադրիչ կոչ\-վում է նրա ամեն մի կապակցված ենթաբազմություն, որը չի պարունակվում տվյալ տարա\-ծութ\-յան մի այլ կապակցված ենթաբազմության մեջ։
\end{definition} 
% 38 +
\begin{example}
\label{15:օրինակ2}
ա) Հասկանալի է, որ յուրաքանչյուր կապակցված տարածություն ունի միայն մի կապակցվածության բաղադրիչ՝ ինքը։ բ) $(-\infty,0) \cup (0, +\infty)$ տարա\-ծութ\-յունը ($(\R,\textrm{սովոր.})$ տարածության տոպոլոգիայից մակածված տոպոլոգիայով) ունի երկու կապակցվածության բաղադրիչ՝ $(-\infty, 0)$ և $(0, +\infty)$ ենթաբազմություն\-ները։ գ) $\Z=\{0, \pm 1, \pm 2, \dots\}$ տարածությունը, որպես $(\R,\textrm{սովոր.})$-ի ենթատարա\-ծու\-թյուն, ունի անթիվ կապակցվածության բաղադրիչներ՝ իր բոլոր կետերը։
\end{example}
% 39 +
\begin{theorem}
\label{15:թեորեմ7}
Տոպոլոգիական տարածության յուրաքանչյուր կետ պատկանում է նրա ճիշտ մի կապակցվածության բաղադրիչի։
\end{theorem}
% 40 +
\begin{proof}
Ակնհայտ է, որ $X$ տարածության ցանկացած մի կետանոց $\{x\}$ ենթաբազմություն $X$-ի կապակցված ենթաբազմություն է։ Դիտարկենք $U(x)= \bigcup\limits_i U_i$ ենթաբազմությունը, որտեղ միավորումը տարվում է ըստ $x \in X$ սևեռված կետն ընդգրկող բոլոր կապակցված $U_i$ ենթաբազմությունների։ Քանի որ $\bigcap U_i \neq \varnothing$, ուստի $U(x)$-ը կապակցված ենթաբազմություն է (ըստ \hyperref[15:թեորեմ 3]{թեորեմ 3}-ի), պարունակում է $x$ կետը և ակնհայտ է, որ $U(x)$-ը չի պարունակվում իրենից տարբեր որևէ կապակցված ենթաբազմությունում:
\end{proof}
% 41 +
\setcounter{hetevanq_counter}{0} 
\begin{hetevanq_counter}
Տվյալ տոպ․ տարածության ցանկացած երկու կապակցվածութ\-յան բաղադրիչ կամ չեն հատվում, կամ համընկնում են։ Ուստի ցանկացած տոպո\-լո\-գիական տարածություն ներկայացվում է որպես իր կապակցվածության բաղադրիչ\-ների չկապակցված միավորում։
\end{hetevanq_counter} 
% 42 +
\begin{hetevanq_counter}[(թեորեմներ 6 և 7-ից)] Տոպոլոգիական տարածության ամեն մի կապակցվածության բաղադրիչ փակ ենթաբազմություն է այդ տարածությունում։
\end{hetevanq_counter}
% 43 +
\begin{example}
Դիտարկենք $\R^2$ հարթությունը իր սովորական մետրական տոպո\-լո\-գիայով և նրանում $C$ կորը, որտեղ $C$-ն ա) շրջանագիծ է, բ) երկու ներքնապես շոշափող շրջանագծերի միավորումն է, գ) երկու արտաքնապես շոշափող շրջանագը\-ծերի միա\-վորումն է։
% 44 +
%%%%%%%%
% nkar 1 %
%%%%%%%%%
\import{tikz/uniform/}{35.tex}
%\begin{center}
%\includegraphics[scale=1]{images/id35.png}
%\end{center}
\par Ապա $\R^2 \setminus C$ ենթատարածությունը ունի ա) դեպքում երկու, իսկ բ) և գ) դեպքերում երեքական կապակցվածության բաղադրիչներ (որո՞նք են դրանք):
% 45 +
\end{example}
\begin{theorem}
\label{15:թեորեմ 8}
Տոպոլոգիական տարածության կապակցվածության բաղադրիչների քանակությունը (ընդհանուր դեպքում որպես բազմության հզորություն) տոպո\-լոգիա\-կան ինվարիանտ է։
% 46 +
\par Մասնավորապես դա նշանակում է, որ եթե ինչ-որ տոպոլոգիական տարածութ\-յուն ունի վերջավոր քանակով՝ ճիշտ $n$ հատ կապակցվա\-ծության բաղադրիչ, ապա նրան հոմեոմորֆ ամեն մի տարածություն նույնպես ունի ճիշտ $n$ հատ կապակցվա\-ծության բաղադրիչ։
\end{theorem}
% 47 +
\begin{proof}
Դիցուք $X$-ը և $Y$-ը հոմեոմորֆ տարածություններ են։ Նշանակում է գոյություն ունեն $f:X \rightarrow Y,\ g:Y \rightarrow X$ անընդհատ արտապատկերումներ, որ $f \circ g=\nuynakan_Y$ և $g \circ f=\nuynakan_X$: Նշանակենք $N(X)$-ով և $N(Y)$-ով համապատասխանաբար $X$-ի և $Y$-ի կապակցվածության բաղադրիչների քանակությունները։
% 48 +
\par Այժմ նկատենք, որ $X$-ի ամեն մի կապակցվածության բաղադրիչի կերպարը $f$ անընդհատ արտապատկերման դեպքում ընկած է $Y$-ի որևէ կապակցվածության բաղադրիչի մեջ։ Իրոք, դիցուք $A \subset X,\ B \subset Y$ ենթաբազմությունները կապակցվա\-ծության բաղադրիչներ են, ընդ որում $f(A) \cap B \neq \varnothing$: Ուստի $f(A) \cup B$-ն $Y$-ի կապակցված ենթաբազմություն է համաձայն \hyperref[15:թեորեմ 2]{թեորեմ 2}-ի և \hyperref[15:թեորեմ 3]{թեորեմ 3}-ի։ Կապակց\-վածության բաղադրիչի սահմանումից հետևում է, որ $f(A) \subset B$: Նշանակում է $N(X)\leq N(Y)$: Կատարելով նույն դատողությունները $g$ արտապատկերման նկատ\-մամբ՝ ստանում ենք $N(Y)\leq N(X)$, ուստի $N(X)=N(Y)$: 
\end{proof}
% 49 +
\par Որպես հետևանք \hyperref[15:թեորեմ 8]{թեորեմ 8}-ից ստանում մենք, որ \hyperref[15:օրինակ 3]{օրինակ 3}-ում ա) տարածութ\-յունը հոմեոմորֆ չէ բ) և գ) տարածություններին։ Իսկ ահա բ) և գ) տարածություն\-ների հոմեոմորֆության վերաբերյալ \hyperref[15:թեորեմ 8]{թեորեմ 8}-ը ոչինչ չի տալիս։
% 50 +
\begin{example}
Դիտարկենք հայոց այբուբենի \inlinegraphics{\import{tikz/Letter A/}{A.tex}} և \inlinegraphics{\import{tikz/Letter A/}{a.tex}} տառերը (գծապատկերները) որպես $(\R^2, \textrm{սովոր.})$ տարածության ենթատարածություններ։ Հարց․ արդյոք հոմեո\-մո՞րֆ են դրանք միմյանց։ 
\end{example}
\label{15:օրինակ 3}
% 51 +
\par Նկատենք, որ եթե հնարավոր է երկու գծապատկերներից մեկը անընդհատ ձևա\-փոխելով, առանց կտրատելու և առանց ինքնահատումների համընկեցնել մյուսի հետ, ապա դրանք հոմեոմորֆ են։ Տվյալ դեպքում հարցի պատասխանը դրական է, և օրինակ հոմեոմորֆիզմ կարելի է կառուցել հետևյալ հաջորդականությամբ՝
% 52
\import{tikz/Letter A/}{Atoa.tex}
\par Այժմ քննարկենք նույն հարցը \inlinegraphics{\import{tikz/Letter A/}{NA.tex}} և
\inlinegraphics{\import{tikz/Letter A/}{a.tex}}
գծապատկերների համար։ Այս դեպ\-քում բոլոր փորձերը նախորդի նմանությամբ մի պատկերից ստանալ մյուսը դատա\-պարտ\-ված են ձախողման։ Պատճառը այդ պատկերների կառուցվածքային էական տար\-բե\-րութ\-յան մեջ է. եթե մենք \inlinegraphics{\import{tikz/Letter A/}{a.tex}} պատկերից հեռացնենք մի որևէ կետ, ապա ստացված պատկերը կարող է լինել ոչ կապակցված տարածություն, որի կապակց\-վա\-ծության բաղադրիչների քանակությունը կարող է լինել ամենաշատը երեք։ Իսկ եթե նման գործողություն կատարենք \inlinegraphics{\import{tikz/Letter A/}{NA.tex}} պատկերի հետ, ապա կախված հեռացվող կետից կարող է ստացվել տարածություն, որն ունի կապակցվածության չորս բա\-ղադրիչ։ Այժմ, օգտագործելով այս հանգամանքը, ապացուցենք, որ այս պատկեր\-ները հոմեմորֆ չեն միմյանց։
% 53
\par Իրոք, ենթադրենք գոյություն ունի $h:$ \inlinegraphics{\import{tikz/Letter A/}{NA.tex}}  $\rightarrow$ \inlinegraphics{\import{tikz/Letter A/}{a.tex}} հոմեոմորֆիզմ։ Հեռացնենք \inlinegraphics{\import{tikz/Letter A/}{Awitha.tex}} պատկերից $a$ կետը, իսկ \inlinegraphics{\import{tikz/Letter A/}{a.tex}} պատկերից $h(a)$ կետը։\\ Ստացված $\widebar{h}:$ \inlinegraphics{\import{tikz/Letter A/}{NA.tex}} $\setminus \{a\} \rightarrow$ \inlinegraphics{\import{tikz/Letter A/}{a.tex}} $\setminus \{h(a)\}$ արտապատկերումը նորից հոմեոմորֆիզմ է (ինչո՞ւ)։ Բայց \inlinegraphics{\import{tikz/Letter A/}{NA.tex}} $\setminus \{a\}$ տարածությունն ունի կապակցվածության 4 բաղադրիչ, մինչդեռ \inlinegraphics{\import{tikz/Letter A/}{a.tex}} $\setminus \{h(a)\}$ տարածության համար դրանց քանակությունը փոքր է 4-ից (հակասություն \hyperref[15:թեորեմ 8]{թեորեմ 8}-ի հետ)։
% 54
\par Անդրադառնալով տարածությունների միջև հոմեոմորֆիզմ կառուցելու խնդրին՝ կատարենք կարևոր դիտողություն։ Եթե մի տարածություն (օրինակ գծապատկեր) հնարավոր չէ առանց կտրատումների և ինքնահատումների համընկեցնել մյուսի հետ, դա դեռ չի նշանակում, որ այդ տարածությունները հոմեոմորֆ չեն։
% 55
\par Օրինակ՝ $\R^3$ տարածությունում հնարավոր չէ առանց ինքնահատումների $x^2+y^2=r^2,\ z=0$ շրջանագիծը անընդհատ ձևափոխումներով համընկեցնել այսպես կոչված «երեքնուկաձև» օվալի հետ։ Մինչդեռ դրանք հոմեոմորֆ են միմյանց։
% 56
\import{tikz/uniform/}{37.tex}
%\begin{center}
%\includegraphics[scale=1]{images/id37.png}
%\end{center}
\par Իրոք, վերցնելով շրջանագծի վրա $x_1,x_2,\dots,x_9$ կետերը, իսկ օվալի վրա $y_1,y_2,\dots,y_9$ կետերը, մենք կարող ենք նախ կառուցել հոմեոմորֆիզմներ այդ պատկերների համա\-պա\-տաս\-խան աղեղների միջև՝ $h_1:x_1 x_2 \rightarrow y_1 y_2,\ h_2:x_2 x_3 \rightarrow y_2 y_3, \dots,\ h_8:x_8 x_9 \rightarrow y_8 y_9, h_9:x_9 x_1 \rightarrow y_9 y_1$: Այնուհետև հաջորդաբար «սոսնձելով» այդ արտապատկերումներից յուրաքանչյուրն իր հաջորդի հետ (\red{թեմա 12-ից թեորեմ 3}-ի իմաստով), կստանանք որոնվող $h$ հոմեոմորֆիզմ։
\end{document}